\documentclass[12pt]{article}    % For submission

\usepackage{authblk}            % For a nice author block
\usepackage[inline]{enumitem}   % For inline enumeration
\usepackage[letterpaper, left=1in, right=1in, top=1in, bottom=1in, footskip=.25in]{geometry} % For better margins.


\title{Executive summary for \\
  Energy, expenditure, and consumption \\
  aspects of rebound,\\
        Part I: A rigorous analytical framework}
\author[1,*]{Matthew Kuperus Heun}
\author[2]{Gregor Semieniuk}
\author[3]{Paul E.\ Brockway}
\affil[1]{Engineering Department, Calvin University, 3201 Burton St. SE, Grand Rapids, MI, 49546}
\affil[2]{Political Economy Research Institute and 
  Department of Economics,
  UMass Amherst}
\affil[3]{Sustainability Research Institute, 
  School of Earth and Environment,
  University of Leeds}
\affil[*]{\normalfont{Corresponding author: \texttt{mkh2@calvin.edu}}}
\renewcommand\Affilfont{\itshape\small}

\date{} % Kill the date

\begin{document}

\maketitle


%++++++++++++++++++++++++++++++
\subsection*{Motivations underlying the research}
\label{sec:motivations}
%++++++++++++++++++++++++++++++

Widespread implementation of energy efficiency
is a key greenhouse gas emissions mitigation measure, 
but rebound can ``take back'' energy savings.
However, the absence of solid analytical foundations hinders
empirical determination of the size of rebound.
Until now, the microeconomic categories of substitution and
income effects provided analytical clarity about how behavior
changes affect energy service consumption,
but it was unclear how they could be used 
for precise numerical rebound calculations. 
Where previous numerical calculations were made, 
they tended approximate the substitution effect
from other goods to the cheaper energy service, 
without maintaining constant utility for the device user.
They also used constant price elasticities
for non-marginal efficiency improvements,
even though constant price elasticities 
provide only approximations of substitution and
income effects for small efficiency changes.
Further, previous analytical studies have stressed the importance of the
cost of buying the new device as well as energy embodied in the device.
Yet, there is no clearly formulated approach for how to incorporate these 
cost and energy components into rebound calculations. 
While recent general equilibrium rebound modeling has led to 
important insights about the effects of changing prices,
dynamic aspects of a macroeconomic rebound have
been neglected by these approaches.
Finally, rebound involves simultaneous changes in energy, expenditure,
and consumption aspects, and keeping an overview of all
aspects is hard, with no approach to our knowledge documenting all 
changes in a straightforward manner.
A new clarity is needed, one that is built upon solid analytical frameworks
involving both economics and energy analysis.


%++++++++++++++++++++++++++++++
\subsection*{Short account of the research performed}
\label{sec:account}
%++++++++++++++++++++++++++++++

In this paper (Part~I of a two-part paper),
we help advance a rigorous analytical framework
that starts at the microeconomic level
and is approachable for both energy analysts and economists. 
We develop the first (to our knowledge)
rebound analysis framework that 
%
\begin{enumerate*}[label={(\roman*)}]
	
  \item clarifies the energy, expenditure, and consumption aspects of rebound,

  \item combines 
        embodied energy effects with
        maintenance and disposal effects
        (under a new ``emplacement effect'' term), and

  \item encompasses non-marginal energy efficiency increases and
        non-marginal energy service price decreases.
  
\end{enumerate*}
%
Furthermore, we develop the first operationalized link between 
rebound effects on microeconomic and macroeconomic levels.

The key contributions of this paper are 
%
\begin{enumerate*}[label={(\roman*)}]
	
	\item a novel and clear explication of interrelated
	      energy, expenditure, and consumption
	      aspects of energy rebound,
	
  \item development of the first (to our knowledge)
        rebound analysis framework that combines 
        embodied energy effects, 
        maintenance and disposal effects, 
        non-marginal energy efficiency increases, and 
        non-marginal energy service price decreases, and

  \item the first operationalized link between 
        rebound effects on microeconomic and macroeconomic levels.
        
\end{enumerate*}

%++++++++++++++++++++++++++++++
\subsection*{Main conclusions}
\label{sec:conclusions}
%++++++++++++++++++++++++++++++

Development of the rigorous analytical rebound framework shows that
it is possible to quantify rebound magnitudes simultaneously
at microeconomic and macroeconomic levels, including 
direct and indirect rebound 
and emplacement, substitution, income, and macro effects.


%++++++++++++++++++++++++++++++
\subsection*{Potential benefits, applications, and policy implications}
\label{sec:benefits}
%++++++++++++++++++++++++++++++

With
careful explication of rebound effects and 
clear derivation of rebound expressions,
we facilitate interdisciplinary understanding of rebound phenomena
toward the goal of enhancing clarity in the field of energy rebound and
enabling more robust rebound calculations
for sound energy and climate policy.


\end{document}