\documentclass[12pt]{article}\usepackage[]{graphicx}\usepackage[]{xcolor}
% maxwidth is the original width if it is less than linewidth
% otherwise use linewidth (to make sure the graphics do not exceed the margin)
\makeatletter
\def\maxwidth{ %
  \ifdim\Gin@nat@width>\linewidth
    \linewidth
  \else
    \Gin@nat@width
  \fi
}
\makeatother

\definecolor{fgcolor}{rgb}{0.345, 0.345, 0.345}
\newcommand{\hlnum}[1]{\textcolor[rgb]{0.686,0.059,0.569}{#1}}%
\newcommand{\hlsng}[1]{\textcolor[rgb]{0.192,0.494,0.8}{#1}}%
\newcommand{\hlcom}[1]{\textcolor[rgb]{0.678,0.584,0.686}{\textit{#1}}}%
\newcommand{\hlopt}[1]{\textcolor[rgb]{0,0,0}{#1}}%
\newcommand{\hldef}[1]{\textcolor[rgb]{0.345,0.345,0.345}{#1}}%
\newcommand{\hlkwa}[1]{\textcolor[rgb]{0.161,0.373,0.58}{\textbf{#1}}}%
\newcommand{\hlkwb}[1]{\textcolor[rgb]{0.69,0.353,0.396}{#1}}%
\newcommand{\hlkwc}[1]{\textcolor[rgb]{0.333,0.667,0.333}{#1}}%
\newcommand{\hlkwd}[1]{\textcolor[rgb]{0.737,0.353,0.396}{\textbf{#1}}}%
\let\hlipl\hlkwb

\usepackage{framed}
\makeatletter
\newenvironment{kframe}{%
 \def\at@end@of@kframe{}%
 \ifinner\ifhmode%
  \def\at@end@of@kframe{\end{minipage}}%
  \begin{minipage}{\columnwidth}%
 \fi\fi%
 \def\FrameCommand##1{\hskip\@totalleftmargin \hskip-\fboxsep
 \colorbox{shadecolor}{##1}\hskip-\fboxsep
     % There is no \\@totalrightmargin, so:
     \hskip-\linewidth \hskip-\@totalleftmargin \hskip\columnwidth}%
 \MakeFramed {\advance\hsize-\width
   \@totalleftmargin\z@ \linewidth\hsize
   \@setminipage}}%
 {\par\unskip\endMakeFramed%
 \at@end@of@kframe}
\makeatother

\definecolor{shadecolor}{rgb}{.97, .97, .97}
\definecolor{messagecolor}{rgb}{0, 0, 0}
\definecolor{warningcolor}{rgb}{1, 0, 1}
\definecolor{errorcolor}{rgb}{1, 0, 0}
\newenvironment{knitrout}{}{} % an empty environment to be redefined in TeX

\usepackage{alltt}    % For submission



%% Package imports %%%%%%%%%%%%%%%%%%%%%%%%%%%%%%%%%%%%%%%%%%%

\usepackage{amsfonts}           % For access to \checkmark.
\usepackage{amsmath}            % For better symbol decorations
\usepackage{authblk}            % For a nice author block
\usepackage{booktabs}           % For nice table rules
\usepackage[official]{eurosym}  % For the euro symbol
\usepackage[figurename=Fig.]{caption} % For changing "Figure" to "Fig." in figure captions
\usepackage{chngcntr}           % To get figure and table numbering correct in the appendices
\usepackage[makeroom]{cancel}   % To show terms go to zero
\usepackage[inline]{enumitem}   % For inline enumeration
\usepackage[letterpaper, left=0.75in, right=0.75in, top=0.75in, bottom=0.75in, footskip=.25in]{geometry} % For better margins.
\usepackage[]{lineno}           % For line numbers
% \modulolinenumbers[5]
\usepackage{microtype}          % For beautiful typesetting
\usepackage{multicol}           % For multi-column layout
\usepackage{multirow}           % For multi-row layout
\usepackage{natbib}
\setlength{\bibsep}{0.0pt}      % To set the separation between entries to 0.
\usepackage{nccmath}            % For the \useshortskip command above equations
\usepackage{parcolumns}         % For multi-column layout
\usepackage{pdflscape}          % For landscape environment \begin{landscape} ... \end{landscape}
\usepackage{setspace}           % For double spacing
\doublespacing
\usepackage{soul}               % For text highlighting
\usepackage[most]{tcolorbox}    % For color highlighting of text
\usepackage{tikz}               % For diagrams
\usetikzlibrary{arrows,positioning}
\usepackage{xcolor}             % For ... colors
\usepackage{xr}                 % For external references to the framework paper
\externaldocument[PartI-]{HSB_framework}

\usepackage{hyperref}           % hyperref should be the last package loaded


%% Macros defined by authors %%%%%%%%%%%%%%%%%%%%%%%%%%%%%%%%%

% The next command tells RStudio to do "Compile PDF" on HSB_framework.Rnw,
% instead of this file, thereby eliminating the need to switch back to HSB_framework.Rnw 
% before building the paper.
%!TEX root = ../HSB_framework.Rnw


%%%%%%%%%%%%%%%%%%%%%%%%%%%%%%%%%%%%%%%%%%%%%%%%%%%%%%%%%%%%%%
% This file contains macros for
% Heun, Semieniuk, Brockway, 
% Toward a Comprehensive, Consumer-sided Energy Rebound Analysis Framework. 
% It is incorporated into the main file by the command
% % The next command tells RStudio to do "Compile PDF" on HSB_framework.Rnw,
% instead of this file, thereby eliminating the need to switch back to HSB_framework.Rnw 
% before building the paper.
%!TEX root = ../HSB_framework.Rnw


%%%%%%%%%%%%%%%%%%%%%%%%%%%%%%%%%%%%%%%%%%%%%%%%%%%%%%%%%%%%%%
% This file contains macros for
% Heun, Semieniuk, Brockway, 
% Toward a Comprehensive, Consumer-sided Energy Rebound Analysis Framework. 
% It is incorporated into the main file by the command
% % The next command tells RStudio to do "Compile PDF" on HSB_framework.Rnw,
% instead of this file, thereby eliminating the need to switch back to HSB_framework.Rnw 
% before building the paper.
%!TEX root = ../HSB_framework.Rnw


%%%%%%%%%%%%%%%%%%%%%%%%%%%%%%%%%%%%%%%%%%%%%%%%%%%%%%%%%%%%%%
% This file contains macros for
% Heun, Semieniuk, Brockway, 
% Toward a Comprehensive, Consumer-sided Energy Rebound Analysis Framework. 
% It is incorporated into the main file by the command
% \input{macros.tex}.
%%%%%%%%%%%%%%%%%%%%%%%%%%%%%%%%%%%%%%%%%%%%%%%%%%%%%%%%%%%%%%


%%%%% Override Energy Economics macros
\renewcommand{\emph}{\textit}



%%%%% Decorations for symbols

\newcommand{\rate}[1]{\dot{#1}}                    % Rate of a quantity

% Create "after" commands
\newcommand{\orig}[1]{{}{#1}^{\scriptscriptstyle \circ}}
\newcommand{\aempl}[1]{{#1}^*}
\newcommand{\asub}[1]{\hat{#1}}
\newcommand{\ainc}[1]{\bar{#1}}
\newcommand{\amacro}[1]{\tilde{#1}}

% Create the "before" commands
\newcommand{\bempl}[1]{\orig{#1}}
\newcommand{\bsub}[1]{\aempl{#1}}
\newcommand{\binc}[1]{\asub{#1}}
\newcommand{\bmacro}[1]{\ainc{#1}}

% Decoration combinations
% Rates after
\newcommand{\rorig}[1]{\orig{\rate{#1}}}
\newcommand{\raempl}[1]{\aempl{\rate{#1}}}
\newcommand{\rasub}[1]{\asub{\rate{#1}}}
\newcommand{\rainc}[1]{\ainc{\rate{#1}}}
\newcommand{\ramacro}[1]{\amacro{\rate{#1}}}

% Rates before
\newcommand{\rbempl}[1]{\rorig{#1}}
\newcommand{\rbsub}[1]{\raempl{#1}}
\newcommand{\rbinc}[1]{\rasub{#1}}
\newcommand{\rbmacro}[1]{\rainc{#1}}

%%%%% Subscript kerning

% \newcommand{\OM}{O\!M}
% \newcommand{\md}{O\!M\!d}
\newcommand{\md}{md}
\newcommand{\macro}{macr\!o}
\newcommand{\life}{li\!f\!e}

%%%%% Expression kerning

\newcommand{\MPC}{M\!PC}


%%%%% Convenient symbols

\newcommand{\Sdot}{\rate{S}_{dev}}
\newcommand{\Mdothatprime}{\rbinc{M}^\prime}


%%%%% Elasticities and income shares

\newcommand{\eqspsUC}{\varepsilon_{\rate{q}_s\!,p_s}}
\newcommand{\eqopsUC}{\varepsilon_{\rate{q}_o\!,p_s}}
\newcommand{\eqspsC}{\varepsilon_{\rate{q}_s\!,p_s\!,c}}
\newcommand{\eqopsC}{\varepsilon_{\rate{q}_o\!,p_s\!,c}}

% originally
\newcommand{\eqspsUCorig}{\varepsilon^\circ_{\rate{q}_s\!,p_s}}
\newcommand{\eqopsUCorig}{\varepsilon^\circ_{\rate{q}_o\!,p_s}}
\newcommand{\eqspsCorig}{\varepsilon^\circ_{\rate{q}_s\!,p_s\!,c}}
\newcommand{\eqopsCorig}{\varepsilon^\circ_{\rate{q}_o\!,p_s\!,c}}
% With hats
\newcommand{\eqspsUChat}{\hat{\varepsilon}_{\rate{q}_s\!,p_s}}
\newcommand{\eqopsUChat}{\hat{\varepsilon}_{\rate{q}_o\!,p_s}}
\newcommand{\eqspsChat}{\hat{\varepsilon}_{\rate{q}_s\!,p_s\!,c}}
\newcommand{\eqopsChat}{\hat{\varepsilon}_{\rate{q}_o\!,p_s\!,c}}

\newcommand{\eqsM}{\varepsilon_{\rate{q}_s\!,\rate{M}}}
\newcommand{\eqoM}{\varepsilon_{\rate{q}_o\!,\rate{M}}}

\newcommand{\fCs}{\bempl{f}_{\rate{C}_s}}


%%%%% Colors

% Original spectrum colours
% \colorlet{emplcolor}{red!25!white}
% \colorlet{subcolor}{orange!25!white}
% \colorlet{inccolor}{green!25!white}
% \colorlet{macrocolor}{blue!25!white}

% New Viridis "plasma" colours
\definecolor{emplcolor}{HTML}{150789}
\definecolor{subcolor}{HTML}{99149F}
\definecolor{inccolor}{HTML}{E76F5A}
\definecolor{macrocolor}{HTML}{F7E225}



%%%%% Coloration of text background

%
% Inline color box around text
% Arguments:
%   [#1]: background color for the box
%   {#2}: text inside the box
%
\newtcbox{\inlinebox}[1][]{on line, 
colback=#1,
colframe=#1,
before upper={\rule[-2pt]{0pt}{10pt}},
boxrule=1pt,
boxsep=0pt,
left=3pt,
right=3pt,
top=2pt,
bottom=2pt}


%%%%% Colored phrases

% Emplacement effect
\newcommand{\empleffect}{\inlinebox[emplcolor]{\textcolor{white}{emplacement effect}}}
\newcommand{\empleffectadj}{\inlinebox[emplcolor]{\textcolor{white}{emplacement-effect}}}
\newcommand{\Empleffect}{\inlinebox[emplcolor]{\textcolor{white}{Emplacement effect}}}
\newcommand{\EmplEffect}{\inlinebox[emplcolor]{\textcolor{white}{Emplacement Effect}}}

% Substitution effect
\newcommand{\subeffect}{\inlinebox[subcolor]{\textcolor{white}{substitution effect}}}
\newcommand{\subeffectadj}{\inlinebox[subcolor]{\textcolor{white}{substitution-effect}}}
\newcommand{\Subeffect}{\inlinebox[subcolor]{\textcolor{white}{Substitution effect}}}
\newcommand{\SubEffect}{\inlinebox[subcolor]{\textcolor{white}{Substitution Effect}}}

% Income effect
\newcommand{\inceffect}{\inlinebox[inccolor]{\textcolor{black}{income effect}}}
\newcommand{\inceffectadj}{\inlinebox[inccolor]{\textcolor{black}{income-effect}}}
\newcommand{\Inceffect}{\inlinebox[inccolor]{\textcolor{black}{Income effect}}}
\newcommand{\IncEffect}{\inlinebox[inccolor]{\textcolor{black}{Income Effect}}}

% Macro effect
\newcommand{\macroeffect}{\inlinebox[macrocolor]{\textcolor{black}{macro effect}}}
\newcommand{\macroeffectadj}{\inlinebox[macrocolor]{\textcolor{black}{macro-effect}}}
\newcommand{\Macroeffect}{\inlinebox[macrocolor]{\textcolor{black}{Macro effect}}}
\newcommand{\MacroEffect}{\inlinebox[macrocolor]{\textcolor{black}{Macro Effect}}}


%%%%% minipage for assumptions and constraints tables
% Arguments:
%   #1: Width (multiple of \linewidth
%   #2: text inside the minipage
%
\newcommand{\mptable}[2]{\begin{minipage}{#1\linewidth} \useshortskip{} \begin{equation} #2 \end{equation} \end{minipage}}


%%%%% Oft-used references

\newcommand{\Ba}[1]{\citeauthor[#1]{Borenstein:2015aa}}
\newcommand{\Bapp}[1]{\citeauthor[#1]{Borenstein:2015aa}'s \citeyearpar{Borenstein:2015aa}}
\newcommand{\Bp}[1]{\citep[#1]{Borenstein:2015aa}}
\newcommand{\Bt}[1]{\citet[#1]{Borenstein:2015aa}}

\newcommand{\Ta}[1]{\citeauthor[#1]{Thomas:2013aa}}
\newcommand{\Tapp}[1]{\citeauthor[#1]{Thomas:2013aa}'s \citeyearpar{Thomas:2013aa}}
\newcommand{\Tp}[1]{\citep[#1]{Thomas:2013aa, Thomas:2013ab}}
\newcommand{\Tpone}[1]{\citep[#1]{Thomas:2013aa}}
\newcommand{\Tptwo}[1]{\citep[#1]{Thomas:2013ab}}
\newcommand{\Tt}[1]{\citet[#1]{Thomas:2013aa, Thomas:2013ab}}
\newcommand{\Ttone}[1]{\citet[#1]{Thomas:2013aa}}
\newcommand{\Tttwo}[1]{\citet[#1]{Thomas:2013ab}}


%%%%% Derivation pages

% Column widths
\newcommand{\derivtextsize}{\footnotesize}
\newcommand{\derivpageleftcolwidth}{0.11\textwidth}
\newcommand{\derivpageenergycolwidth}{0.6\textwidth}
\newcommand{\derivpagefinancialcolwidth}{0.6\textwidth}

% Horizontal rule between sections of derivations

\newcommand{\sectionsep}{\noindent\rule{1.4\textwidth}{0.4pt}}


%
% Derivation section
% Arguments:
%   #1: accounting stage (original, prime, etc.)
%   #2: energy column
%   #3: financial column
%
\newcommand{\derivsection}[3]{%

\derivtextsize{}

\begin{minipage}[t]{\derivpageleftcolwidth}
~\\#1
\end{minipage}
%
%
%
\begin{minipage}[t]{\derivpageenergycolwidth}
#2
\end{minipage}
%
~
%
\begin{minipage}[t]{\derivpagefinancialcolwidth}
#3
\end{minipage}

\normalsize{}

}


%
% Derivation page header
% Arguments:
%   #1: Effect type header text (e.g., Emplacement Effect)
%
\newcommand{\derivheader}[1]{

\begin{center}
  #1
\end{center}

\derivsection{}
{\begin{center}\emph{Energy analysis}\end{center}}
{\begin{center}\emph{Financial analysis}\end{center}}

}


% Equations

% Efficiency ratios
\newcommand{\etaratioinline}{\amacro{\eta}/\bempl{\eta}}
\newcommand{\etaratiostacked}{\frac{\amacro{\eta}}{\bempl{\eta}}}

% Derivative with respect to efficiency ratio
\newcommand{\dbydetaeta}{\frac{\mathrm{d}}{\mathrm{d}(\etaratioinline{})}}


% Original
\newcommand{\Eacctorig}{\rbempl{E} = \rbempl{E}_s + \rbempl{E}_{emb} + (\rbempl{C}_{\md} + \rbempl{C}_o) I_E}
\newcommand{\Macctorig}{\rbempl{M} = p_E \rbempl{E}_s + \rbempl{C}_{cap} + \rbempl{C}_{\md} + \rbempl{C}_o + \rbempl{N}}

% Before emplacement effect (same as original)
\newcommand{\Eacctbempl}{\Eacctorig}      
\newcommand{\Macctbempl}{\Macctorig}      

% After emplacement effect
\newcommand{\Eacctaempl}{\raempl{E} = \raempl{E}_s + \raempl{E}_{emb} + (\raempl{C}_{\md} + \raempl{C}_o) I_E}                  
\newcommand{\Macctaempl}{\raempl{M} = p_E \raempl{E}_s + \raempl{C}_{cap} + \raempl{C}_{\md} + \raempl{C}_o + \raempl{N}}         

% Before substitution effect (same as after emplacement effect)
\newcommand{\Eacctbsub}{\Eacctaempl}
\newcommand{\Macctbsub}{\Macctaempl}

% After substitution effect
\newcommand{\Eacctasub}{\rasub{E} = \rasub{E}_s + \rasub{E}_{emb} + (\rasub{C}_{\md} + \rasub{C}_o) I_E}
\newcommand{\Macctasub}{\rasub{M} = p_E \rasub{E}_s + \rasub{C}_{cap} + \rasub{C}_{\md} + \rasub{C}_o + \rasub{N}}

% Before income effect (same as after substitution effect)
\newcommand{\Eacctbinc}{\Eacctasub}
\newcommand{\Macctbinc}{\Macctasub}

% After income effect
\newcommand{\Eacctainc}{\rainc{E} = \rainc{E}_s + \rainc{E}_{emb} + (\rainc{C}_{\md} + \rainc{C}_o) I_E}
\newcommand{\Macctainc}{\rainc{M} = p_E \rainc{E}_s + \rainc{C}_{cap} + \rainc{C}_{\md} + \rainc{C}_o + \rainc{N}}

% Embodied energy rebound
\newcommand{\Reembeqn}{\frac{\left( \frac{\aempl{E}_{emb}}{\bempl{E}_{emb}}
  \frac{\bempl{t}_{\life}}{\aempl{t}_{\life}} - 1 \right) \rbempl{E}_{emb}}{\Sdot}}
  
% Ops, Maintenance, and disposal energy rebound
\newcommand{\ReOMdeqn}{\frac{\left( \frac{\raempl{C}_{\md}}{\rbempl{C}_{\md}} - 1 \right) \rbempl{C}_{\md} I_E}{\Sdot}}

% Equation for S_dot_dev
% \newcommand{\Sdoteqn}{\left( \etaratiostacked - 1 \right)\!\etaratiostacked \rbempl{E}_s}
\newcommand{\Sdoteqn}{\left( \etaratiostacked - 1 \right)\! 
                            \frac{\bempl{\eta}}{\amacro{\eta}} \rbempl{E}_s}

% Equation for Re_dsub
\newcommand{\Redsubeqn}{\frac{\left( \etaratiostacked \right)^{-\eqspsC} - 1}
                        {\etaratiostacked - 1}}
                        
% Equation for Re_isub
\newcommand{\Reisubeqn}{\frac{{\left( \etaratiostacked  \right)}
                          ^{-\eqopsC} - 1}{\etaratiostacked - 1} \; 
                          \etaratiostacked \; 
                          \frac{\rbempl{C}_o I_E}{\rbempl{E}_s}}
                          
% CES utility equation
\newcommand{\cesutility}{\left[ \fCs \left( \frac{\rate{q}_s}{\rbempl{q}_s} \right)^\rho 
        + (1-\fCs) \left( \frac{\rate{C}_o}{\rbempl{C}_o} \right)^\rho  \right]^{(1/\rho)}}
        
% Equation for q_s_hat/q_s_orig
\newcommand{\qssolution}{\left\{ \fCs + (1-\fCs)
      \left[ \left(  \frac{1-\fCs}{\fCs}  \right) \frac{\amacro{p}_s \rbempl{q}_s}{\rbempl{C}_o}  \right]
                                                  ^{\rho / (1 - \rho)} \right\} ^ {-1/\rho}}

% Equation for C_o_hat/C_o_orig
\newcommand{\Cosolution}{ \left( 1 + \fCs \left\{ \left[ \left( \frac{1-\fCs}{\fCs} \right)
          \frac{\amacro{p}_s \rbempl{q}_s}{\rbempl{C}_o} \right] ^{\rho/(\rho - 1)} - 1 \right\} \right)^{-1/\rho}}

% Equation for Re_dsub for the CES utility model
\newcommand{\RedsubCES}{\frac{\qssolution{} - 1}{\etaratiostacked{} - 1}}

% Equation for Re_isub for the CES utility model
\newcommand{\ReisubCES}{\frac{\Cosolution{} - 1}{\etaratiostacked{} - 1}
                         \etaratiostacked \; 
                          \frac{\rbempl{C}_o I_E}{\rbempl{E}_s}}


% Equation for Re_dinc, approximate method
\newcommand{\Redinceqnapprox}{\frac{ \left( 1 + \frac{\rbinc{N}}{\Mdothatprime} \right) ^{\eqsM} - 1}
              { \etaratiostacked - 1 } \left( \etaratiostacked \right)^{-\eqspsC}}

% Equation for Re_dinc, exact method
\newcommand{\Redinceqnexact}{\frac{ \left( 1 + \frac{\rbinc{N}}{\Mdothatprime} \right) ^{\eqsM} - 1}
              { \etaratiostacked - 1 } \qssolution{} }

% Equation for Re_cap
\newcommand{\Recapeqn}{\frac{\Delta \raempl{C}_{cap} I_E}{\Sdot}}

% Equation for Re_iinc, approximate method
\newcommand{\Reiinceqnapprox}{\frac{\left( 1 + \frac{\rbinc{N}}{\Mdothatprime} \right)^{\eqoM} - 1}{\etaratiostacked - 1} 
              \left( \etaratiostacked \right)^{1 - \eqopsC}
              \frac{\rbempl{C}_o I_E}{\rbempl{E}_s}}

% Equation for Re_iinc, exact method
% \newcommand{\Reiinceqnexact}{\frac{\left( 1 + \frac{\rbinc{N}}{\Mdothatprime} \right)^{\eqoM} - 1}{\etaratiostacked - 1} 
%               \left( \etaratiostacked \right)
%               \left( \frac{\rasub{C}_o}{\rorig{C}_o} \right) 
%               \frac{\rbempl{C}_o I_E}{\rbempl{E}_s}}

\newcommand{\Reiinceqnexact}{\frac{\left( 1 + \frac{\rbinc{N}}{\Mdothatprime} \right)^{\eqoM} - 1}{\etaratiostacked - 1} 
              \left( \etaratiostacked \right)
              \frac{\rbempl{C}_o I_E}{\rbempl{E}_s}
              \Cosolution{}}



% Equation for Re_d (total direct rebound)
\newcommand{\Redeqn}{\frac{ \left( \etaratiostacked \right)^{-\eqspsC}
             \left( 1 + \frac{\rasub{N}}{\rbempl{M}} \right)^{\eqsM}   - 1}
         {\etaratiostacked - 1}}

% Equation for Re_macro
% \newcommand{\Remacroeqn}{k (p_E I_E - Re_{cap} - Re_{\md} - p_E I_E Re_{dsub} - Re_{isub})}
\newcommand{\Remacroeqn}{k (p_E I_E - Re_{cap} - Re_{\md})}

% Equation for Re_tot
% \newcommand{\Retoteqn}{&Re_{emb} - k Re_{cap} + (1-k) Re_{\md}         \nonumber \\
%                        &+ (1 - k p_E I_E) Re_{dsub} + (1 - k) Re_{isub}   \nonumber \\
%                        &+ Re_{dinc} + Re_{iinc} +  k p_E I_E}
\newcommand{\Retoteqn}{&Re_{emb} + k (p_E I_E - Re_{cap}) + (1-k) Re_{\md}   \nonumber \\
                       &+ Re_{dsub} + Re_{isub}                              \nonumber \\
                       &+ Re_{dinc} + Re_{iinc}}
                 

%%%% Income preference equations

% Equation for energy service income preferences
\newcommand{\incprefseqn}{\frac{\rainc{q}_s}{\rbinc{q}_s} = \left( 1 + \frac{\rbinc{N}}{\Mdothatprime}  \right) ^{\eqsM}}

% Equation for other goods income preferences
\newcommand{\incprefoeqn}{\frac{\rainc{q}_o}{\rbinc{q}_o} = \left( 1 + \frac{\rbinc{N}}{\Mdothatprime}  \right) ^{\eqoM}}

% Equation for effective income
% \newcommand{\effinceqn}{\Mdothatprime \equiv \rbempl{M} - \rbempl{C}_{cap} - \rbempl{C}_{\md} 
%                         - \rate{G} + p_E \Delta \rbinc{E}_s + \Delta \rbinc{C}_o}
\newcommand{\effinceqn}{\Mdothatprime \equiv \rorig{M} - \raempl{C}_{cap} - \raempl{C}_{\md} - \rasub{N}}
                        



% Segments and lines
% Arguments:
%   #1: left character
%   #2: line color
%   #3: line thickness (e.g., 0.1 mm)
%   #4: right character
% Note that \raisebox{0.9 mm} moves the line up from the baseline.
% Also, \line(1,0){12} gives a horizontal line with length "12" (unknown units!)
% (1, 0) is the slope (1 unit to right, 0 units up).
\newcommand{\seg}[4]{#1\linethickness{#3}\raisebox{0.88 mm}{\textcolor{#2}{\line(1,0){12}}}#4}

% Construction lines
\newcommand{\iicirc}{\seg{$\bempl{\text{i}}$}{black}{0.3 mm}{$\,\bempl{\text{i}}$}}
\newcommand{\iibar}{\seg{$\bmacro{\text{i}}$}{black}{0.3 mm}{$\,\bmacro{\text{i}}$}}
\newcommand{\rr}{\seg{r}{black}{0.1 mm}{r}}
\newcommand{\circcirc}{\seg{$\circ$}{black}{0.1 mm}{$\circ$}}
\newcommand{\starstar}{\seg{$*$}{black}{0.1 mm}{$*$}}
\newcommand{\hathat}{\seg{$\wedge$}{black}{0.1 mm}{$\wedge$}}
\newcommand{\barbar}{\seg{$-\,$}{black}{0.1 mm}{$\, -$}}

% Line segments
\newcommand{\circa}{\seg{$\circ$}{emplcolor}{0.6 mm}{$a$}}
% \newcommand{\ab}{\seg{a}{emplcolor}{0.6 mm}{b}}
\newcommand{\ab}{$a$\tikz[baseline=-0.6ex]\draw [line width=0.6mm,dotted,emplcolor] (0,0) -- (0.45,0);$b$}
\newcommand{\bstar}{\seg{$b\,$}{emplcolor}{0.6 mm}{$*$}}
\newcommand{\starc}{\seg{$*$}{subcolor}{0.6 mm}{$\,c$}}
\newcommand{\chat}{\seg{$c\,$}{subcolor}{0.6 mm}{$\wedge$}}
\newcommand{\hatd}{\seg{$\wedge$}{inccolor}{0.6 mm}{$\,d$}}
\newcommand{\dbar}{\seg{$d$}{inccolor}{0.6 mm}{$\,-$}}
\newcommand{\hatbar}{\seg{$\wedge$}{inccolor}{0.6 mm}{$\,-$}}
\newcommand{\bartilde}{\seg{$- \,$}{macrocolor}{0.6 mm}{$\, \sim$}}


% Rotated text for tables
% See
% https://tex.stackexchange.com/questions/98388/how-to-make-table-with-rotated-table-headers-in-latex/98439#98439
% for details.
\newcommand{\rot}{\rotatebox{90}}


% A "rating" command for filled circles with tikz.
% See 
% https://tex.stackexchange.com/questions/194955/get-partly-filled-circle-symbol-scale-linearly-with-parameter
% for details.
\newcommand{\rating}[2][0.75ex]{%
  \pgfmathsetmacro\th{asin(#2/50-1)}% (theta angle of polar coordinates)
    \tikz{%
      \fill[black] (\th:#1) arc (\th:-180-\th:#1) -- cycle;
      \draw[black, thin, radius=#1] (0,0) circle;
    }%
}
.
%%%%%%%%%%%%%%%%%%%%%%%%%%%%%%%%%%%%%%%%%%%%%%%%%%%%%%%%%%%%%%


%%%%% Override Energy Economics macros
\renewcommand{\emph}{\textit}



%%%%% Decorations for symbols

\newcommand{\rate}[1]{\dot{#1}}                    % Rate of a quantity

% Create "after" commands
\newcommand{\orig}[1]{{}{#1}^{\scriptscriptstyle \circ}}
\newcommand{\aempl}[1]{{#1}^*}
\newcommand{\asub}[1]{\hat{#1}}
\newcommand{\ainc}[1]{\bar{#1}}
\newcommand{\amacro}[1]{\tilde{#1}}

% Create the "before" commands
\newcommand{\bempl}[1]{\orig{#1}}
\newcommand{\bsub}[1]{\aempl{#1}}
\newcommand{\binc}[1]{\asub{#1}}
\newcommand{\bmacro}[1]{\ainc{#1}}

% Decoration combinations
% Rates after
\newcommand{\rorig}[1]{\orig{\rate{#1}}}
\newcommand{\raempl}[1]{\aempl{\rate{#1}}}
\newcommand{\rasub}[1]{\asub{\rate{#1}}}
\newcommand{\rainc}[1]{\ainc{\rate{#1}}}
\newcommand{\ramacro}[1]{\amacro{\rate{#1}}}

% Rates before
\newcommand{\rbempl}[1]{\rorig{#1}}
\newcommand{\rbsub}[1]{\raempl{#1}}
\newcommand{\rbinc}[1]{\rasub{#1}}
\newcommand{\rbmacro}[1]{\rainc{#1}}

%%%%% Subscript kerning

% \newcommand{\OM}{O\!M}
% \newcommand{\md}{O\!M\!d}
\newcommand{\md}{md}
\newcommand{\macro}{macr\!o}
\newcommand{\life}{li\!f\!e}

%%%%% Expression kerning

\newcommand{\MPC}{M\!PC}


%%%%% Convenient symbols

\newcommand{\Sdot}{\rate{S}_{dev}}
\newcommand{\Mdothatprime}{\rbinc{M}^\prime}


%%%%% Elasticities and income shares

\newcommand{\eqspsUC}{\varepsilon_{\rate{q}_s\!,p_s}}
\newcommand{\eqopsUC}{\varepsilon_{\rate{q}_o\!,p_s}}
\newcommand{\eqspsC}{\varepsilon_{\rate{q}_s\!,p_s\!,c}}
\newcommand{\eqopsC}{\varepsilon_{\rate{q}_o\!,p_s\!,c}}

% originally
\newcommand{\eqspsUCorig}{\varepsilon^\circ_{\rate{q}_s\!,p_s}}
\newcommand{\eqopsUCorig}{\varepsilon^\circ_{\rate{q}_o\!,p_s}}
\newcommand{\eqspsCorig}{\varepsilon^\circ_{\rate{q}_s\!,p_s\!,c}}
\newcommand{\eqopsCorig}{\varepsilon^\circ_{\rate{q}_o\!,p_s\!,c}}
% With hats
\newcommand{\eqspsUChat}{\hat{\varepsilon}_{\rate{q}_s\!,p_s}}
\newcommand{\eqopsUChat}{\hat{\varepsilon}_{\rate{q}_o\!,p_s}}
\newcommand{\eqspsChat}{\hat{\varepsilon}_{\rate{q}_s\!,p_s\!,c}}
\newcommand{\eqopsChat}{\hat{\varepsilon}_{\rate{q}_o\!,p_s\!,c}}

\newcommand{\eqsM}{\varepsilon_{\rate{q}_s\!,\rate{M}}}
\newcommand{\eqoM}{\varepsilon_{\rate{q}_o\!,\rate{M}}}

\newcommand{\fCs}{\bempl{f}_{\rate{C}_s}}


%%%%% Colors

% Original spectrum colours
% \colorlet{emplcolor}{red!25!white}
% \colorlet{subcolor}{orange!25!white}
% \colorlet{inccolor}{green!25!white}
% \colorlet{macrocolor}{blue!25!white}

% New Viridis "plasma" colours
\definecolor{emplcolor}{HTML}{150789}
\definecolor{subcolor}{HTML}{99149F}
\definecolor{inccolor}{HTML}{E76F5A}
\definecolor{macrocolor}{HTML}{F7E225}



%%%%% Coloration of text background

%
% Inline color box around text
% Arguments:
%   [#1]: background color for the box
%   {#2}: text inside the box
%
\newtcbox{\inlinebox}[1][]{on line, 
colback=#1,
colframe=#1,
before upper={\rule[-2pt]{0pt}{10pt}},
boxrule=1pt,
boxsep=0pt,
left=3pt,
right=3pt,
top=2pt,
bottom=2pt}


%%%%% Colored phrases

% Emplacement effect
\newcommand{\empleffect}{\inlinebox[emplcolor]{\textcolor{white}{emplacement effect}}}
\newcommand{\empleffectadj}{\inlinebox[emplcolor]{\textcolor{white}{emplacement-effect}}}
\newcommand{\Empleffect}{\inlinebox[emplcolor]{\textcolor{white}{Emplacement effect}}}
\newcommand{\EmplEffect}{\inlinebox[emplcolor]{\textcolor{white}{Emplacement Effect}}}

% Substitution effect
\newcommand{\subeffect}{\inlinebox[subcolor]{\textcolor{white}{substitution effect}}}
\newcommand{\subeffectadj}{\inlinebox[subcolor]{\textcolor{white}{substitution-effect}}}
\newcommand{\Subeffect}{\inlinebox[subcolor]{\textcolor{white}{Substitution effect}}}
\newcommand{\SubEffect}{\inlinebox[subcolor]{\textcolor{white}{Substitution Effect}}}

% Income effect
\newcommand{\inceffect}{\inlinebox[inccolor]{\textcolor{black}{income effect}}}
\newcommand{\inceffectadj}{\inlinebox[inccolor]{\textcolor{black}{income-effect}}}
\newcommand{\Inceffect}{\inlinebox[inccolor]{\textcolor{black}{Income effect}}}
\newcommand{\IncEffect}{\inlinebox[inccolor]{\textcolor{black}{Income Effect}}}

% Macro effect
\newcommand{\macroeffect}{\inlinebox[macrocolor]{\textcolor{black}{macro effect}}}
\newcommand{\macroeffectadj}{\inlinebox[macrocolor]{\textcolor{black}{macro-effect}}}
\newcommand{\Macroeffect}{\inlinebox[macrocolor]{\textcolor{black}{Macro effect}}}
\newcommand{\MacroEffect}{\inlinebox[macrocolor]{\textcolor{black}{Macro Effect}}}


%%%%% minipage for assumptions and constraints tables
% Arguments:
%   #1: Width (multiple of \linewidth
%   #2: text inside the minipage
%
\newcommand{\mptable}[2]{\begin{minipage}{#1\linewidth} \useshortskip{} \begin{equation} #2 \end{equation} \end{minipage}}


%%%%% Oft-used references

\newcommand{\Ba}[1]{\citeauthor[#1]{Borenstein:2015aa}}
\newcommand{\Bapp}[1]{\citeauthor[#1]{Borenstein:2015aa}'s \citeyearpar{Borenstein:2015aa}}
\newcommand{\Bp}[1]{\citep[#1]{Borenstein:2015aa}}
\newcommand{\Bt}[1]{\citet[#1]{Borenstein:2015aa}}

\newcommand{\Ta}[1]{\citeauthor[#1]{Thomas:2013aa}}
\newcommand{\Tapp}[1]{\citeauthor[#1]{Thomas:2013aa}'s \citeyearpar{Thomas:2013aa}}
\newcommand{\Tp}[1]{\citep[#1]{Thomas:2013aa, Thomas:2013ab}}
\newcommand{\Tpone}[1]{\citep[#1]{Thomas:2013aa}}
\newcommand{\Tptwo}[1]{\citep[#1]{Thomas:2013ab}}
\newcommand{\Tt}[1]{\citet[#1]{Thomas:2013aa, Thomas:2013ab}}
\newcommand{\Ttone}[1]{\citet[#1]{Thomas:2013aa}}
\newcommand{\Tttwo}[1]{\citet[#1]{Thomas:2013ab}}


%%%%% Derivation pages

% Column widths
\newcommand{\derivtextsize}{\footnotesize}
\newcommand{\derivpageleftcolwidth}{0.11\textwidth}
\newcommand{\derivpageenergycolwidth}{0.6\textwidth}
\newcommand{\derivpagefinancialcolwidth}{0.6\textwidth}

% Horizontal rule between sections of derivations

\newcommand{\sectionsep}{\noindent\rule{1.4\textwidth}{0.4pt}}


%
% Derivation section
% Arguments:
%   #1: accounting stage (original, prime, etc.)
%   #2: energy column
%   #3: financial column
%
\newcommand{\derivsection}[3]{%

\derivtextsize{}

\begin{minipage}[t]{\derivpageleftcolwidth}
~\\#1
\end{minipage}
%
%
%
\begin{minipage}[t]{\derivpageenergycolwidth}
#2
\end{minipage}
%
~
%
\begin{minipage}[t]{\derivpagefinancialcolwidth}
#3
\end{minipage}

\normalsize{}

}


%
% Derivation page header
% Arguments:
%   #1: Effect type header text (e.g., Emplacement Effect)
%
\newcommand{\derivheader}[1]{

\begin{center}
  #1
\end{center}

\derivsection{}
{\begin{center}\emph{Energy analysis}\end{center}}
{\begin{center}\emph{Financial analysis}\end{center}}

}


% Equations

% Efficiency ratios
\newcommand{\etaratioinline}{\amacro{\eta}/\bempl{\eta}}
\newcommand{\etaratiostacked}{\frac{\amacro{\eta}}{\bempl{\eta}}}

% Derivative with respect to efficiency ratio
\newcommand{\dbydetaeta}{\frac{\mathrm{d}}{\mathrm{d}(\etaratioinline{})}}


% Original
\newcommand{\Eacctorig}{\rbempl{E} = \rbempl{E}_s + \rbempl{E}_{emb} + (\rbempl{C}_{\md} + \rbempl{C}_o) I_E}
\newcommand{\Macctorig}{\rbempl{M} = p_E \rbempl{E}_s + \rbempl{C}_{cap} + \rbempl{C}_{\md} + \rbempl{C}_o + \rbempl{N}}

% Before emplacement effect (same as original)
\newcommand{\Eacctbempl}{\Eacctorig}      
\newcommand{\Macctbempl}{\Macctorig}      

% After emplacement effect
\newcommand{\Eacctaempl}{\raempl{E} = \raempl{E}_s + \raempl{E}_{emb} + (\raempl{C}_{\md} + \raempl{C}_o) I_E}                  
\newcommand{\Macctaempl}{\raempl{M} = p_E \raempl{E}_s + \raempl{C}_{cap} + \raempl{C}_{\md} + \raempl{C}_o + \raempl{N}}         

% Before substitution effect (same as after emplacement effect)
\newcommand{\Eacctbsub}{\Eacctaempl}
\newcommand{\Macctbsub}{\Macctaempl}

% After substitution effect
\newcommand{\Eacctasub}{\rasub{E} = \rasub{E}_s + \rasub{E}_{emb} + (\rasub{C}_{\md} + \rasub{C}_o) I_E}
\newcommand{\Macctasub}{\rasub{M} = p_E \rasub{E}_s + \rasub{C}_{cap} + \rasub{C}_{\md} + \rasub{C}_o + \rasub{N}}

% Before income effect (same as after substitution effect)
\newcommand{\Eacctbinc}{\Eacctasub}
\newcommand{\Macctbinc}{\Macctasub}

% After income effect
\newcommand{\Eacctainc}{\rainc{E} = \rainc{E}_s + \rainc{E}_{emb} + (\rainc{C}_{\md} + \rainc{C}_o) I_E}
\newcommand{\Macctainc}{\rainc{M} = p_E \rainc{E}_s + \rainc{C}_{cap} + \rainc{C}_{\md} + \rainc{C}_o + \rainc{N}}

% Embodied energy rebound
\newcommand{\Reembeqn}{\frac{\left( \frac{\aempl{E}_{emb}}{\bempl{E}_{emb}}
  \frac{\bempl{t}_{\life}}{\aempl{t}_{\life}} - 1 \right) \rbempl{E}_{emb}}{\Sdot}}
  
% Ops, Maintenance, and disposal energy rebound
\newcommand{\ReOMdeqn}{\frac{\left( \frac{\raempl{C}_{\md}}{\rbempl{C}_{\md}} - 1 \right) \rbempl{C}_{\md} I_E}{\Sdot}}

% Equation for S_dot_dev
% \newcommand{\Sdoteqn}{\left( \etaratiostacked - 1 \right)\!\etaratiostacked \rbempl{E}_s}
\newcommand{\Sdoteqn}{\left( \etaratiostacked - 1 \right)\! 
                            \frac{\bempl{\eta}}{\amacro{\eta}} \rbempl{E}_s}

% Equation for Re_dsub
\newcommand{\Redsubeqn}{\frac{\left( \etaratiostacked \right)^{-\eqspsC} - 1}
                        {\etaratiostacked - 1}}
                        
% Equation for Re_isub
\newcommand{\Reisubeqn}{\frac{{\left( \etaratiostacked  \right)}
                          ^{-\eqopsC} - 1}{\etaratiostacked - 1} \; 
                          \etaratiostacked \; 
                          \frac{\rbempl{C}_o I_E}{\rbempl{E}_s}}
                          
% CES utility equation
\newcommand{\cesutility}{\left[ \fCs \left( \frac{\rate{q}_s}{\rbempl{q}_s} \right)^\rho 
        + (1-\fCs) \left( \frac{\rate{C}_o}{\rbempl{C}_o} \right)^\rho  \right]^{(1/\rho)}}
        
% Equation for q_s_hat/q_s_orig
\newcommand{\qssolution}{\left\{ \fCs + (1-\fCs)
      \left[ \left(  \frac{1-\fCs}{\fCs}  \right) \frac{\amacro{p}_s \rbempl{q}_s}{\rbempl{C}_o}  \right]
                                                  ^{\rho / (1 - \rho)} \right\} ^ {-1/\rho}}

% Equation for C_o_hat/C_o_orig
\newcommand{\Cosolution}{ \left( 1 + \fCs \left\{ \left[ \left( \frac{1-\fCs}{\fCs} \right)
          \frac{\amacro{p}_s \rbempl{q}_s}{\rbempl{C}_o} \right] ^{\rho/(\rho - 1)} - 1 \right\} \right)^{-1/\rho}}

% Equation for Re_dsub for the CES utility model
\newcommand{\RedsubCES}{\frac{\qssolution{} - 1}{\etaratiostacked{} - 1}}

% Equation for Re_isub for the CES utility model
\newcommand{\ReisubCES}{\frac{\Cosolution{} - 1}{\etaratiostacked{} - 1}
                         \etaratiostacked \; 
                          \frac{\rbempl{C}_o I_E}{\rbempl{E}_s}}


% Equation for Re_dinc, approximate method
\newcommand{\Redinceqnapprox}{\frac{ \left( 1 + \frac{\rbinc{N}}{\Mdothatprime} \right) ^{\eqsM} - 1}
              { \etaratiostacked - 1 } \left( \etaratiostacked \right)^{-\eqspsC}}

% Equation for Re_dinc, exact method
\newcommand{\Redinceqnexact}{\frac{ \left( 1 + \frac{\rbinc{N}}{\Mdothatprime} \right) ^{\eqsM} - 1}
              { \etaratiostacked - 1 } \qssolution{} }

% Equation for Re_cap
\newcommand{\Recapeqn}{\frac{\Delta \raempl{C}_{cap} I_E}{\Sdot}}

% Equation for Re_iinc, approximate method
\newcommand{\Reiinceqnapprox}{\frac{\left( 1 + \frac{\rbinc{N}}{\Mdothatprime} \right)^{\eqoM} - 1}{\etaratiostacked - 1} 
              \left( \etaratiostacked \right)^{1 - \eqopsC}
              \frac{\rbempl{C}_o I_E}{\rbempl{E}_s}}

% Equation for Re_iinc, exact method
% \newcommand{\Reiinceqnexact}{\frac{\left( 1 + \frac{\rbinc{N}}{\Mdothatprime} \right)^{\eqoM} - 1}{\etaratiostacked - 1} 
%               \left( \etaratiostacked \right)
%               \left( \frac{\rasub{C}_o}{\rorig{C}_o} \right) 
%               \frac{\rbempl{C}_o I_E}{\rbempl{E}_s}}

\newcommand{\Reiinceqnexact}{\frac{\left( 1 + \frac{\rbinc{N}}{\Mdothatprime} \right)^{\eqoM} - 1}{\etaratiostacked - 1} 
              \left( \etaratiostacked \right)
              \frac{\rbempl{C}_o I_E}{\rbempl{E}_s}
              \Cosolution{}}



% Equation for Re_d (total direct rebound)
\newcommand{\Redeqn}{\frac{ \left( \etaratiostacked \right)^{-\eqspsC}
             \left( 1 + \frac{\rasub{N}}{\rbempl{M}} \right)^{\eqsM}   - 1}
         {\etaratiostacked - 1}}

% Equation for Re_macro
% \newcommand{\Remacroeqn}{k (p_E I_E - Re_{cap} - Re_{\md} - p_E I_E Re_{dsub} - Re_{isub})}
\newcommand{\Remacroeqn}{k (p_E I_E - Re_{cap} - Re_{\md})}

% Equation for Re_tot
% \newcommand{\Retoteqn}{&Re_{emb} - k Re_{cap} + (1-k) Re_{\md}         \nonumber \\
%                        &+ (1 - k p_E I_E) Re_{dsub} + (1 - k) Re_{isub}   \nonumber \\
%                        &+ Re_{dinc} + Re_{iinc} +  k p_E I_E}
\newcommand{\Retoteqn}{&Re_{emb} + k (p_E I_E - Re_{cap}) + (1-k) Re_{\md}   \nonumber \\
                       &+ Re_{dsub} + Re_{isub}                              \nonumber \\
                       &+ Re_{dinc} + Re_{iinc}}
                 

%%%% Income preference equations

% Equation for energy service income preferences
\newcommand{\incprefseqn}{\frac{\rainc{q}_s}{\rbinc{q}_s} = \left( 1 + \frac{\rbinc{N}}{\Mdothatprime}  \right) ^{\eqsM}}

% Equation for other goods income preferences
\newcommand{\incprefoeqn}{\frac{\rainc{q}_o}{\rbinc{q}_o} = \left( 1 + \frac{\rbinc{N}}{\Mdothatprime}  \right) ^{\eqoM}}

% Equation for effective income
% \newcommand{\effinceqn}{\Mdothatprime \equiv \rbempl{M} - \rbempl{C}_{cap} - \rbempl{C}_{\md} 
%                         - \rate{G} + p_E \Delta \rbinc{E}_s + \Delta \rbinc{C}_o}
\newcommand{\effinceqn}{\Mdothatprime \equiv \rorig{M} - \raempl{C}_{cap} - \raempl{C}_{\md} - \rasub{N}}
                        



% Segments and lines
% Arguments:
%   #1: left character
%   #2: line color
%   #3: line thickness (e.g., 0.1 mm)
%   #4: right character
% Note that \raisebox{0.9 mm} moves the line up from the baseline.
% Also, \line(1,0){12} gives a horizontal line with length "12" (unknown units!)
% (1, 0) is the slope (1 unit to right, 0 units up).
\newcommand{\seg}[4]{#1\linethickness{#3}\raisebox{0.88 mm}{\textcolor{#2}{\line(1,0){12}}}#4}

% Construction lines
\newcommand{\iicirc}{\seg{$\bempl{\text{i}}$}{black}{0.3 mm}{$\,\bempl{\text{i}}$}}
\newcommand{\iibar}{\seg{$\bmacro{\text{i}}$}{black}{0.3 mm}{$\,\bmacro{\text{i}}$}}
\newcommand{\rr}{\seg{r}{black}{0.1 mm}{r}}
\newcommand{\circcirc}{\seg{$\circ$}{black}{0.1 mm}{$\circ$}}
\newcommand{\starstar}{\seg{$*$}{black}{0.1 mm}{$*$}}
\newcommand{\hathat}{\seg{$\wedge$}{black}{0.1 mm}{$\wedge$}}
\newcommand{\barbar}{\seg{$-\,$}{black}{0.1 mm}{$\, -$}}

% Line segments
\newcommand{\circa}{\seg{$\circ$}{emplcolor}{0.6 mm}{$a$}}
% \newcommand{\ab}{\seg{a}{emplcolor}{0.6 mm}{b}}
\newcommand{\ab}{$a$\tikz[baseline=-0.6ex]\draw [line width=0.6mm,dotted,emplcolor] (0,0) -- (0.45,0);$b$}
\newcommand{\bstar}{\seg{$b\,$}{emplcolor}{0.6 mm}{$*$}}
\newcommand{\starc}{\seg{$*$}{subcolor}{0.6 mm}{$\,c$}}
\newcommand{\chat}{\seg{$c\,$}{subcolor}{0.6 mm}{$\wedge$}}
\newcommand{\hatd}{\seg{$\wedge$}{inccolor}{0.6 mm}{$\,d$}}
\newcommand{\dbar}{\seg{$d$}{inccolor}{0.6 mm}{$\,-$}}
\newcommand{\hatbar}{\seg{$\wedge$}{inccolor}{0.6 mm}{$\,-$}}
\newcommand{\bartilde}{\seg{$- \,$}{macrocolor}{0.6 mm}{$\, \sim$}}


% Rotated text for tables
% See
% https://tex.stackexchange.com/questions/98388/how-to-make-table-with-rotated-table-headers-in-latex/98439#98439
% for details.
\newcommand{\rot}{\rotatebox{90}}


% A "rating" command for filled circles with tikz.
% See 
% https://tex.stackexchange.com/questions/194955/get-partly-filled-circle-symbol-scale-linearly-with-parameter
% for details.
\newcommand{\rating}[2][0.75ex]{%
  \pgfmathsetmacro\th{asin(#2/50-1)}% (theta angle of polar coordinates)
    \tikz{%
      \fill[black] (\th:#1) arc (\th:-180-\th:#1) -- cycle;
      \draw[black, thin, radius=#1] (0,0) circle;
    }%
}
.
%%%%%%%%%%%%%%%%%%%%%%%%%%%%%%%%%%%%%%%%%%%%%%%%%%%%%%%%%%%%%%


%%%%% Override Energy Economics macros
\renewcommand{\emph}{\textit}



%%%%% Decorations for symbols

\newcommand{\rate}[1]{\dot{#1}}                    % Rate of a quantity

% Create "after" commands
\newcommand{\orig}[1]{{}{#1}^{\scriptscriptstyle \circ}}
\newcommand{\aempl}[1]{{#1}^*}
\newcommand{\asub}[1]{\hat{#1}}
\newcommand{\ainc}[1]{\bar{#1}}
\newcommand{\amacro}[1]{\tilde{#1}}

% Create the "before" commands
\newcommand{\bempl}[1]{\orig{#1}}
\newcommand{\bsub}[1]{\aempl{#1}}
\newcommand{\binc}[1]{\asub{#1}}
\newcommand{\bmacro}[1]{\ainc{#1}}

% Decoration combinations
% Rates after
\newcommand{\rorig}[1]{\orig{\rate{#1}}}
\newcommand{\raempl}[1]{\aempl{\rate{#1}}}
\newcommand{\rasub}[1]{\asub{\rate{#1}}}
\newcommand{\rainc}[1]{\ainc{\rate{#1}}}
\newcommand{\ramacro}[1]{\amacro{\rate{#1}}}

% Rates before
\newcommand{\rbempl}[1]{\rorig{#1}}
\newcommand{\rbsub}[1]{\raempl{#1}}
\newcommand{\rbinc}[1]{\rasub{#1}}
\newcommand{\rbmacro}[1]{\rainc{#1}}

%%%%% Subscript kerning

% \newcommand{\OM}{O\!M}
% \newcommand{\md}{O\!M\!d}
\newcommand{\md}{md}
\newcommand{\macro}{macr\!o}
\newcommand{\life}{li\!f\!e}

%%%%% Expression kerning

\newcommand{\MPC}{M\!PC}


%%%%% Convenient symbols

\newcommand{\Sdot}{\rate{S}_{dev}}
\newcommand{\Mdothatprime}{\rbinc{M}^\prime}


%%%%% Elasticities and income shares

\newcommand{\eqspsUC}{\varepsilon_{\rate{q}_s\!,p_s}}
\newcommand{\eqopsUC}{\varepsilon_{\rate{q}_o\!,p_s}}
\newcommand{\eqspsC}{\varepsilon_{\rate{q}_s\!,p_s\!,c}}
\newcommand{\eqopsC}{\varepsilon_{\rate{q}_o\!,p_s\!,c}}

% originally
\newcommand{\eqspsUCorig}{\varepsilon^\circ_{\rate{q}_s\!,p_s}}
\newcommand{\eqopsUCorig}{\varepsilon^\circ_{\rate{q}_o\!,p_s}}
\newcommand{\eqspsCorig}{\varepsilon^\circ_{\rate{q}_s\!,p_s\!,c}}
\newcommand{\eqopsCorig}{\varepsilon^\circ_{\rate{q}_o\!,p_s\!,c}}
% With hats
\newcommand{\eqspsUChat}{\hat{\varepsilon}_{\rate{q}_s\!,p_s}}
\newcommand{\eqopsUChat}{\hat{\varepsilon}_{\rate{q}_o\!,p_s}}
\newcommand{\eqspsChat}{\hat{\varepsilon}_{\rate{q}_s\!,p_s\!,c}}
\newcommand{\eqopsChat}{\hat{\varepsilon}_{\rate{q}_o\!,p_s\!,c}}

\newcommand{\eqsM}{\varepsilon_{\rate{q}_s\!,\rate{M}}}
\newcommand{\eqoM}{\varepsilon_{\rate{q}_o\!,\rate{M}}}

\newcommand{\fCs}{\bempl{f}_{\rate{C}_s}}


%%%%% Colors

% Original spectrum colours
% \colorlet{emplcolor}{red!25!white}
% \colorlet{subcolor}{orange!25!white}
% \colorlet{inccolor}{green!25!white}
% \colorlet{macrocolor}{blue!25!white}

% New Viridis "plasma" colours
\definecolor{emplcolor}{HTML}{150789}
\definecolor{subcolor}{HTML}{99149F}
\definecolor{inccolor}{HTML}{E76F5A}
\definecolor{macrocolor}{HTML}{F7E225}



%%%%% Coloration of text background

%
% Inline color box around text
% Arguments:
%   [#1]: background color for the box
%   {#2}: text inside the box
%
\newtcbox{\inlinebox}[1][]{on line, 
colback=#1,
colframe=#1,
before upper={\rule[-2pt]{0pt}{10pt}},
boxrule=1pt,
boxsep=0pt,
left=3pt,
right=3pt,
top=2pt,
bottom=2pt}


%%%%% Colored phrases

% Emplacement effect
\newcommand{\empleffect}{\inlinebox[emplcolor]{\textcolor{white}{emplacement effect}}}
\newcommand{\empleffectadj}{\inlinebox[emplcolor]{\textcolor{white}{emplacement-effect}}}
\newcommand{\Empleffect}{\inlinebox[emplcolor]{\textcolor{white}{Emplacement effect}}}
\newcommand{\EmplEffect}{\inlinebox[emplcolor]{\textcolor{white}{Emplacement Effect}}}

% Substitution effect
\newcommand{\subeffect}{\inlinebox[subcolor]{\textcolor{white}{substitution effect}}}
\newcommand{\subeffectadj}{\inlinebox[subcolor]{\textcolor{white}{substitution-effect}}}
\newcommand{\Subeffect}{\inlinebox[subcolor]{\textcolor{white}{Substitution effect}}}
\newcommand{\SubEffect}{\inlinebox[subcolor]{\textcolor{white}{Substitution Effect}}}

% Income effect
\newcommand{\inceffect}{\inlinebox[inccolor]{\textcolor{black}{income effect}}}
\newcommand{\inceffectadj}{\inlinebox[inccolor]{\textcolor{black}{income-effect}}}
\newcommand{\Inceffect}{\inlinebox[inccolor]{\textcolor{black}{Income effect}}}
\newcommand{\IncEffect}{\inlinebox[inccolor]{\textcolor{black}{Income Effect}}}

% Macro effect
\newcommand{\macroeffect}{\inlinebox[macrocolor]{\textcolor{black}{macro effect}}}
\newcommand{\macroeffectadj}{\inlinebox[macrocolor]{\textcolor{black}{macro-effect}}}
\newcommand{\Macroeffect}{\inlinebox[macrocolor]{\textcolor{black}{Macro effect}}}
\newcommand{\MacroEffect}{\inlinebox[macrocolor]{\textcolor{black}{Macro Effect}}}


%%%%% minipage for assumptions and constraints tables
% Arguments:
%   #1: Width (multiple of \linewidth
%   #2: text inside the minipage
%
\newcommand{\mptable}[2]{\begin{minipage}{#1\linewidth} \useshortskip{} \begin{equation} #2 \end{equation} \end{minipage}}


%%%%% Oft-used references

\newcommand{\Ba}[1]{\citeauthor[#1]{Borenstein:2015aa}}
\newcommand{\Bapp}[1]{\citeauthor[#1]{Borenstein:2015aa}'s \citeyearpar{Borenstein:2015aa}}
\newcommand{\Bp}[1]{\citep[#1]{Borenstein:2015aa}}
\newcommand{\Bt}[1]{\citet[#1]{Borenstein:2015aa}}

\newcommand{\Ta}[1]{\citeauthor[#1]{Thomas:2013aa}}
\newcommand{\Tapp}[1]{\citeauthor[#1]{Thomas:2013aa}'s \citeyearpar{Thomas:2013aa}}
\newcommand{\Tp}[1]{\citep[#1]{Thomas:2013aa, Thomas:2013ab}}
\newcommand{\Tpone}[1]{\citep[#1]{Thomas:2013aa}}
\newcommand{\Tptwo}[1]{\citep[#1]{Thomas:2013ab}}
\newcommand{\Tt}[1]{\citet[#1]{Thomas:2013aa, Thomas:2013ab}}
\newcommand{\Ttone}[1]{\citet[#1]{Thomas:2013aa}}
\newcommand{\Tttwo}[1]{\citet[#1]{Thomas:2013ab}}


%%%%% Derivation pages

% Column widths
\newcommand{\derivtextsize}{\footnotesize}
\newcommand{\derivpageleftcolwidth}{0.11\textwidth}
\newcommand{\derivpageenergycolwidth}{0.6\textwidth}
\newcommand{\derivpagefinancialcolwidth}{0.6\textwidth}

% Horizontal rule between sections of derivations

\newcommand{\sectionsep}{\noindent\rule{1.4\textwidth}{0.4pt}}


%
% Derivation section
% Arguments:
%   #1: accounting stage (original, prime, etc.)
%   #2: energy column
%   #3: financial column
%
\newcommand{\derivsection}[3]{%

\derivtextsize{}

\begin{minipage}[t]{\derivpageleftcolwidth}
~\\#1
\end{minipage}
%
%
%
\begin{minipage}[t]{\derivpageenergycolwidth}
#2
\end{minipage}
%
~
%
\begin{minipage}[t]{\derivpagefinancialcolwidth}
#3
\end{minipage}

\normalsize{}

}


%
% Derivation page header
% Arguments:
%   #1: Effect type header text (e.g., Emplacement Effect)
%
\newcommand{\derivheader}[1]{

\begin{center}
  #1
\end{center}

\derivsection{}
{\begin{center}\emph{Energy analysis}\end{center}}
{\begin{center}\emph{Financial analysis}\end{center}}

}


% Equations

% Efficiency ratios
\newcommand{\etaratioinline}{\amacro{\eta}/\bempl{\eta}}
\newcommand{\etaratiostacked}{\frac{\amacro{\eta}}{\bempl{\eta}}}

% Derivative with respect to efficiency ratio
\newcommand{\dbydetaeta}{\frac{\mathrm{d}}{\mathrm{d}(\etaratioinline{})}}


% Original
\newcommand{\Eacctorig}{\rbempl{E} = \rbempl{E}_s + \rbempl{E}_{emb} + (\rbempl{C}_{\md} + \rbempl{C}_o) I_E}
\newcommand{\Macctorig}{\rbempl{M} = p_E \rbempl{E}_s + \rbempl{C}_{cap} + \rbempl{C}_{\md} + \rbempl{C}_o + \rbempl{N}}

% Before emplacement effect (same as original)
\newcommand{\Eacctbempl}{\Eacctorig}      
\newcommand{\Macctbempl}{\Macctorig}      

% After emplacement effect
\newcommand{\Eacctaempl}{\raempl{E} = \raempl{E}_s + \raempl{E}_{emb} + (\raempl{C}_{\md} + \raempl{C}_o) I_E}                  
\newcommand{\Macctaempl}{\raempl{M} = p_E \raempl{E}_s + \raempl{C}_{cap} + \raempl{C}_{\md} + \raempl{C}_o + \raempl{N}}         

% Before substitution effect (same as after emplacement effect)
\newcommand{\Eacctbsub}{\Eacctaempl}
\newcommand{\Macctbsub}{\Macctaempl}

% After substitution effect
\newcommand{\Eacctasub}{\rasub{E} = \rasub{E}_s + \rasub{E}_{emb} + (\rasub{C}_{\md} + \rasub{C}_o) I_E}
\newcommand{\Macctasub}{\rasub{M} = p_E \rasub{E}_s + \rasub{C}_{cap} + \rasub{C}_{\md} + \rasub{C}_o + \rasub{N}}

% Before income effect (same as after substitution effect)
\newcommand{\Eacctbinc}{\Eacctasub}
\newcommand{\Macctbinc}{\Macctasub}

% After income effect
\newcommand{\Eacctainc}{\rainc{E} = \rainc{E}_s + \rainc{E}_{emb} + (\rainc{C}_{\md} + \rainc{C}_o) I_E}
\newcommand{\Macctainc}{\rainc{M} = p_E \rainc{E}_s + \rainc{C}_{cap} + \rainc{C}_{\md} + \rainc{C}_o + \rainc{N}}

% Embodied energy rebound
\newcommand{\Reembeqn}{\frac{\left( \frac{\aempl{E}_{emb}}{\bempl{E}_{emb}}
  \frac{\bempl{t}_{\life}}{\aempl{t}_{\life}} - 1 \right) \rbempl{E}_{emb}}{\Sdot}}
  
% Ops, Maintenance, and disposal energy rebound
\newcommand{\ReOMdeqn}{\frac{\left( \frac{\raempl{C}_{\md}}{\rbempl{C}_{\md}} - 1 \right) \rbempl{C}_{\md} I_E}{\Sdot}}

% Equation for S_dot_dev
% \newcommand{\Sdoteqn}{\left( \etaratiostacked - 1 \right)\!\etaratiostacked \rbempl{E}_s}
\newcommand{\Sdoteqn}{\left( \etaratiostacked - 1 \right)\! 
                            \frac{\bempl{\eta}}{\amacro{\eta}} \rbempl{E}_s}

% Equation for Re_dsub
\newcommand{\Redsubeqn}{\frac{\left( \etaratiostacked \right)^{-\eqspsC} - 1}
                        {\etaratiostacked - 1}}
                        
% Equation for Re_isub
\newcommand{\Reisubeqn}{\frac{{\left( \etaratiostacked  \right)}
                          ^{-\eqopsC} - 1}{\etaratiostacked - 1} \; 
                          \etaratiostacked \; 
                          \frac{\rbempl{C}_o I_E}{\rbempl{E}_s}}
                          
% CES utility equation
\newcommand{\cesutility}{\left[ \fCs \left( \frac{\rate{q}_s}{\rbempl{q}_s} \right)^\rho 
        + (1-\fCs) \left( \frac{\rate{C}_o}{\rbempl{C}_o} \right)^\rho  \right]^{(1/\rho)}}
        
% Equation for q_s_hat/q_s_orig
\newcommand{\qssolution}{\left\{ \fCs + (1-\fCs)
      \left[ \left(  \frac{1-\fCs}{\fCs}  \right) \frac{\amacro{p}_s \rbempl{q}_s}{\rbempl{C}_o}  \right]
                                                  ^{\rho / (1 - \rho)} \right\} ^ {-1/\rho}}

% Equation for C_o_hat/C_o_orig
\newcommand{\Cosolution}{ \left( 1 + \fCs \left\{ \left[ \left( \frac{1-\fCs}{\fCs} \right)
          \frac{\amacro{p}_s \rbempl{q}_s}{\rbempl{C}_o} \right] ^{\rho/(\rho - 1)} - 1 \right\} \right)^{-1/\rho}}

% Equation for Re_dsub for the CES utility model
\newcommand{\RedsubCES}{\frac{\qssolution{} - 1}{\etaratiostacked{} - 1}}

% Equation for Re_isub for the CES utility model
\newcommand{\ReisubCES}{\frac{\Cosolution{} - 1}{\etaratiostacked{} - 1}
                         \etaratiostacked \; 
                          \frac{\rbempl{C}_o I_E}{\rbempl{E}_s}}


% Equation for Re_dinc, approximate method
\newcommand{\Redinceqnapprox}{\frac{ \left( 1 + \frac{\rbinc{N}}{\Mdothatprime} \right) ^{\eqsM} - 1}
              { \etaratiostacked - 1 } \left( \etaratiostacked \right)^{-\eqspsC}}

% Equation for Re_dinc, exact method
\newcommand{\Redinceqnexact}{\frac{ \left( 1 + \frac{\rbinc{N}}{\Mdothatprime} \right) ^{\eqsM} - 1}
              { \etaratiostacked - 1 } \qssolution{} }

% Equation for Re_cap
\newcommand{\Recapeqn}{\frac{\Delta \raempl{C}_{cap} I_E}{\Sdot}}

% Equation for Re_iinc, approximate method
\newcommand{\Reiinceqnapprox}{\frac{\left( 1 + \frac{\rbinc{N}}{\Mdothatprime} \right)^{\eqoM} - 1}{\etaratiostacked - 1} 
              \left( \etaratiostacked \right)^{1 - \eqopsC}
              \frac{\rbempl{C}_o I_E}{\rbempl{E}_s}}

% Equation for Re_iinc, exact method
% \newcommand{\Reiinceqnexact}{\frac{\left( 1 + \frac{\rbinc{N}}{\Mdothatprime} \right)^{\eqoM} - 1}{\etaratiostacked - 1} 
%               \left( \etaratiostacked \right)
%               \left( \frac{\rasub{C}_o}{\rorig{C}_o} \right) 
%               \frac{\rbempl{C}_o I_E}{\rbempl{E}_s}}

\newcommand{\Reiinceqnexact}{\frac{\left( 1 + \frac{\rbinc{N}}{\Mdothatprime} \right)^{\eqoM} - 1}{\etaratiostacked - 1} 
              \left( \etaratiostacked \right)
              \frac{\rbempl{C}_o I_E}{\rbempl{E}_s}
              \Cosolution{}}



% Equation for Re_d (total direct rebound)
\newcommand{\Redeqn}{\frac{ \left( \etaratiostacked \right)^{-\eqspsC}
             \left( 1 + \frac{\rasub{N}}{\rbempl{M}} \right)^{\eqsM}   - 1}
         {\etaratiostacked - 1}}

% Equation for Re_macro
% \newcommand{\Remacroeqn}{k (p_E I_E - Re_{cap} - Re_{\md} - p_E I_E Re_{dsub} - Re_{isub})}
\newcommand{\Remacroeqn}{k (p_E I_E - Re_{cap} - Re_{\md})}

% Equation for Re_tot
% \newcommand{\Retoteqn}{&Re_{emb} - k Re_{cap} + (1-k) Re_{\md}         \nonumber \\
%                        &+ (1 - k p_E I_E) Re_{dsub} + (1 - k) Re_{isub}   \nonumber \\
%                        &+ Re_{dinc} + Re_{iinc} +  k p_E I_E}
\newcommand{\Retoteqn}{&Re_{emb} + k (p_E I_E - Re_{cap}) + (1-k) Re_{\md}   \nonumber \\
                       &+ Re_{dsub} + Re_{isub}                              \nonumber \\
                       &+ Re_{dinc} + Re_{iinc}}
                 

%%%% Income preference equations

% Equation for energy service income preferences
\newcommand{\incprefseqn}{\frac{\rainc{q}_s}{\rbinc{q}_s} = \left( 1 + \frac{\rbinc{N}}{\Mdothatprime}  \right) ^{\eqsM}}

% Equation for other goods income preferences
\newcommand{\incprefoeqn}{\frac{\rainc{q}_o}{\rbinc{q}_o} = \left( 1 + \frac{\rbinc{N}}{\Mdothatprime}  \right) ^{\eqoM}}

% Equation for effective income
% \newcommand{\effinceqn}{\Mdothatprime \equiv \rbempl{M} - \rbempl{C}_{cap} - \rbempl{C}_{\md} 
%                         - \rate{G} + p_E \Delta \rbinc{E}_s + \Delta \rbinc{C}_o}
\newcommand{\effinceqn}{\Mdothatprime \equiv \rorig{M} - \raempl{C}_{cap} - \raempl{C}_{\md} - \rasub{N}}
                        



% Segments and lines
% Arguments:
%   #1: left character
%   #2: line color
%   #3: line thickness (e.g., 0.1 mm)
%   #4: right character
% Note that \raisebox{0.9 mm} moves the line up from the baseline.
% Also, \line(1,0){12} gives a horizontal line with length "12" (unknown units!)
% (1, 0) is the slope (1 unit to right, 0 units up).
\newcommand{\seg}[4]{#1\linethickness{#3}\raisebox{0.88 mm}{\textcolor{#2}{\line(1,0){12}}}#4}

% Construction lines
\newcommand{\iicirc}{\seg{$\bempl{\text{i}}$}{black}{0.3 mm}{$\,\bempl{\text{i}}$}}
\newcommand{\iibar}{\seg{$\bmacro{\text{i}}$}{black}{0.3 mm}{$\,\bmacro{\text{i}}$}}
\newcommand{\rr}{\seg{r}{black}{0.1 mm}{r}}
\newcommand{\circcirc}{\seg{$\circ$}{black}{0.1 mm}{$\circ$}}
\newcommand{\starstar}{\seg{$*$}{black}{0.1 mm}{$*$}}
\newcommand{\hathat}{\seg{$\wedge$}{black}{0.1 mm}{$\wedge$}}
\newcommand{\barbar}{\seg{$-\,$}{black}{0.1 mm}{$\, -$}}

% Line segments
\newcommand{\circa}{\seg{$\circ$}{emplcolor}{0.6 mm}{$a$}}
% \newcommand{\ab}{\seg{a}{emplcolor}{0.6 mm}{b}}
\newcommand{\ab}{$a$\tikz[baseline=-0.6ex]\draw [line width=0.6mm,dotted,emplcolor] (0,0) -- (0.45,0);$b$}
\newcommand{\bstar}{\seg{$b\,$}{emplcolor}{0.6 mm}{$*$}}
\newcommand{\starc}{\seg{$*$}{subcolor}{0.6 mm}{$\,c$}}
\newcommand{\chat}{\seg{$c\,$}{subcolor}{0.6 mm}{$\wedge$}}
\newcommand{\hatd}{\seg{$\wedge$}{inccolor}{0.6 mm}{$\,d$}}
\newcommand{\dbar}{\seg{$d$}{inccolor}{0.6 mm}{$\,-$}}
\newcommand{\hatbar}{\seg{$\wedge$}{inccolor}{0.6 mm}{$\,-$}}
\newcommand{\bartilde}{\seg{$- \,$}{macrocolor}{0.6 mm}{$\, \sim$}}


% Rotated text for tables
% See
% https://tex.stackexchange.com/questions/98388/how-to-make-table-with-rotated-table-headers-in-latex/98439#98439
% for details.
\newcommand{\rot}{\rotatebox{90}}


% A "rating" command for filled circles with tikz.
% See 
% https://tex.stackexchange.com/questions/194955/get-partly-filled-circle-symbol-scale-linearly-with-parameter
% for details.
\newcommand{\rating}[2][0.75ex]{%
  \pgfmathsetmacro\th{asin(#2/50-1)}% (theta angle of polar coordinates)
    \tikz{%
      \fill[black] (\th:#1) arc (\th:-180-\th:#1) -- cycle;
      \draw[black, thin, radius=#1] (0,0) circle;
    }%
}



%% Bibliography style %%%%%%%%%%%%%%%%%%%%%%%%%%%%%%%%%%%%%%%%

\bibliographystyle{apa-good.bst}

% Author info
\title{Energy, expenditure, and consumption aspects of rebound,\\
          Part II: Applications of the framework}
\author[1,2,3,*]{Matthew Kuperus Heun}
\author[4]{Gregor Semieniuk}
\author[2]{Paul E.\ Brockway}
\affil[1]{Engineering Department, Calvin University, 
          3201 Burton St. SE, Grand Rapids, MI, 49546}
\affil[2]{Sustainability Research Institute, 
          School of Earth and Environment, University of Leeds, 
          Woodhouse, Leeds, LS2 9JT, UK}
\affil[3]{School for Public Leadership, Faculty of Economic and Management Science,
          Stellenbosch University,
          Private Bag X1, Matieland, 7602, Stellenbosch, South Africa}
\affil[4]{Political Economy Research Institute and Department of Economics,
          University of Massachusetts Amherst, 
          Amherst, MA, 01003}
\affil[*]{\normalfont{Corresponding author: \texttt{mkh2@calvin.edu}}}
\renewcommand\Affilfont{\itshape\small}

\date{} % Kill the date
\IfFileExists{upquote.sty}{\usepackage{upquote}}{}
\begin{document}

\maketitle


%% Abstract %%%%%%%%%%%%%%%%%%%%%%%%%%%%%%%%%%%%%%%%%%%%%%%%%%

\begin{abstract}
Widespread implementation of energy efficiency
is a key greenhouse gas emissions mitigation measure, 
but rebound can ``take back'' energy savings.
However, the absence of solid analytical foundations hinders
empirical determination of the size of rebound.
In Part~I, we developed foundations of 
a rigorous analytical framework
that is approachable for both energy analysts and economists. 
In this paper (Part~II of two),
we develop energy, expenditure, and consumption planes, 
a novel, mutually consistent, and numerically precise
way to visualize and illustrate rebound.
Further, we operationalize the macro factor
for macroeconomic rebound. 
Using the framework and rebound planes,
we calculate and show total rebound for two examples: 
energy efficiency upgrades of
a car (56.2\%) and
an electric lamp (67.0\%).
We calculate rebound for a producer-sided extension
to the framework, 
namely an energy price rebound effect.
Finally, we provide information about new open source software tools 
for calculating magnitudes and
visualizing rebound effects using the framework.
\end{abstract}

Keywords: Energy efficiency, Energy rebound, Energy services, Microeconomic rebound, Substitution and income effects, Macroeconomic rebound

JEL codes: O13, Q40, Q43




\linenumbers


%%%%%%%%%%%%%%%%%%%%%%%%%%%%%%%%%%%%%%%%%%%%%%%%%%%%%%%%%%%%%%
\section{Introduction}
\label{sec:introduction}
%%%%%%%%%%%%%%%%%%%%%%%%%%%%%%%%%%%%%%%%%%%%%%%%%%%%%%%%%%%%%%

In Part I of this two-part paper, we argued that improved clarity is needed
about energy rebound.
We said that
%
\begin{quote}
  [a] description of rebound [is needed] that is
  %
  \begin{enumerate*}[label={(\roman*)}]

    \item consistent across energy, expenditure, and consumption aspects,

    \item technically rigorous and

    \item approachable from both sides
          (economics and energy analysis). \ldots

  \end{enumerate*}
  %
  In other words,
  the finance and human behavior aspects of rebound need to be presented
  in ways energy analysts can understand.
  And the energy aspects of rebound need to be presented
  in ways economists can understand.
\end{quote}

To help improve clarity in the rebound field,
we developed in Part~I foundations for
a rigorous analytical framework,
one that is tractable for both energy analysts and economists.
Three aspects of rebound are analyzed in the framework:
energy, expenditure, and consumption.
The framework contains both
direct and indirect rebound and
four rebound effects
(emplacement, substitution, income, and macro)
between five stages ($\circ$, $*$, $\wedge$, $-$, and $\sim$).
Rebound terms and symbol decorations are
shown in Fig.~\ref{fig:flowchart}.
(See Table~\ref{PartI-tab:rebound_typology} in Part~I for details.
See Appendix~\ref{sec:nomenclature} for nomenclature.)

\begin{figure}
\centering
  % The file is a Tikz picture.
  % The next command tells RStudio to do "Compile PDF" on HSB.Rnw,
% instead of this file, thereby eliminating the need to switch back to HSB.Rnw 
% before building the paper.
%!TEX root = ../HSB_framework.Rnw

\begin{tikzpicture}%[font=\sffamily]
  % Set up the nodes of the main path
  \node(orig) at (0cm,0cm) {\strut $\circ$};
  \node[right=0.2cm of orig] (empl)[rounded corners, text=empltextcolor, fill=emplcolor] {\strut emplacement};
  \node[right=0.2cm of empl] (star) {$*$};
  \node[right=0.2cm of star] (sub)[rounded corners, text=subtextcolor, fill=subcolor] {\strut substitution};
  \node[right=0.2cm of sub] (hat) {$\wedge$};
  \node[right=0.2cm of hat] (inc)[rounded corners, text=inctextcolor, fill=inccolor] {\strut income};
  \node[right=0.2cm of inc] (bar) {$-$};
  \node[right=0.2cm of bar] (macro)[rounded corners, text=macrotextcolor, fill=macrocolor] {\strut macro};
  \node[right=0.2cm of macro] (tilde) {$\sim$};

  % Draw arrows between main row boxes
  \draw[->] (orig) -- (empl);
  \draw[->] (empl) -- (star);
  \draw[->] (star) -- (sub);
  \draw[->] (sub) -- (hat);
  \draw[->] (hat) -- (inc);
  \draw[->] (inc) -- (bar);
  \draw[->] (bar) -- (macro);
  \draw[->] (macro) -- (tilde);
  % Add nodes for rebound terms beneath nodes
  \node[above=0.1cm of empl] {$Re_{empl}$};
  \node[above=0.1cm of sub] {$Re_{sub}$};
  \node[above=0.1cm of inc] {$Re_{inc}$};
  \node[above=0.1cm of macro] {$Re_{\macro}$};
  % Add nodes for rebound terms above nodes
  \node[below=0.1cm of empl] {\footnotesize $Re_{dempl}$,$Re_{iempl}$};
  \node[below=0.1cm of sub] {\footnotesize $Re_{dsub}$,$Re_{isub}$};
  \node[below=0.1cm of inc] {\footnotesize $Re_{dinc}$,$Re_{iinc}$};
  % Add nodes for spelled-out decorations beneath nodes
  % \node[below=0.715cm of orig] {no decoration};
  \node[below=0.715cm of orig] {orig};
  % \node[below=0.9cm of prime] {prime};
  \node[below=0.9cm of star] {star};
  \node[below=0.9cm of hat] {hat};
  \node[below=0.9cm of bar] {bar};
  \node[below=0.9cm of tilde] {tilde};
\end{tikzpicture}


\caption{Flowchart of rebound effects and decorations.}
\label{fig:flowchart}
\end{figure}

In this paper (Part~II), we make further progress toward the goal of clarity
with five contributions.
First, we develop a new way to communicate
components and mechanisms of rebound
via mutually consistent and numerically precise
visualizations of rebound effects
in energy, expenditure, and consumption planes.
Second, we calculate the macro rebound effect
via a macro factor ($k$) selected to be 3.
Third, we apply the framework to two energy efficiency upgrades (EEUs)
(a car and an electric lamp)
with detailed explication
of numerical results for the examples.
Fourth, we apply the framework to
calculate numerical values
for the producer-sided energy price rebound
extension to the framework.
Finally, we provide information
about new open source software tools
for calculating and visualizing rebound
for any energy efficiency upgrade.

The remainder of this paper is structured as follows.
Section~\ref{sec:methods} describes data for the examples,
our method of visualizing rebound, and
open source software tools for calculating and visualizing rebound.
Section~\ref{sec:results} provides results for two examples:
energy efficiency upgrades to a car and an electric lamp.
Section~\ref{sec:discussion} operationalizes the macro factor ($k$)
and discusses results, and
Section~\ref{sec:conclusion} concludes.


%%%%%%%%%%%%%%%%%%%%%%%%%%%%%%%%%%%%%%%%%%%%%%%%%%%%%%%%%%%%%%
\section{Data and methods}
\label{sec:methods}
%%%%%%%%%%%%%%%%%%%%%%%%%%%%%%%%%%%%%%%%%%%%%%%%%%%%%%%%%%%%%%

This section contains data for the examples
(Section~\ref{sec:data}),
an explication of our new method for visualizing rebound effects and magnitudes
(Section~\ref{sec:path_graphs}), and
a description of new open source software tools
for rebound calculations and visualization
(Section~\ref{sec:software_tools}).


%++++++++++++++++++++++++++++++
\subsection{Data}
\label{sec:data}
%++++++++++++++++++++++++++++++



To demonstrate application of the rebound analysis framework
developed in Part~I,
we analyze two examples:
energy efficiency upgrades to a car and an electric lamp.
The examples are presented with much detail
to support our goal of helping to advance clarity for the process
of calculating the magnitude of rebound effects.
Here, we collect parameter values
for the equations to calculate
nine rebound components:
$Re_{dempl}$,
$Re_{emb}$,
$Re_{\om}$,
$Re_d$,
$Re_{dsub}$,
$Re_{isub}$,
$Re_{dinc}$,
$Re_{iinc}$, and
$Re_{\macro}$.
Total rebound
($Re_{tot}$)
is given by the sum of the above components
or equivalently by Eq.~(\ref{PartI-eq:Re_tot}) of Part~I.


%------------------------------
\subsubsection{Data for car example}
\label{sec:data_car_example}
%------------------------------



For the first example,
we consider the purchase of a more fuel efficient car,
namely a gasoline-electric Ford Fusion Hybrid car,
to replace a conventional gasoline Ford Fusion car.
The cars are matched as closely as possible,
except for the inclusion of an electric battery
in the hybrid car.
The car case study features a larger initial capital investment ($\orig{C}_{cap} < \amacro{C}_{cap}$)
for the long-term benefit of decreased energy service costs ($\rorig{C}_s > \ramacro{C}_s$).

We require three sets of data.
First, basic car parameters are summarized
in Table~\ref{tab:car_parameters}.
Second, we require several general economic parameters,
mainly relating to the U.S.\ economy and personal finances
%of the average U.S.\ citizen,
of a representative U.S.-based user
shown in Table~\ref{tab:car_economic_parameters}.
Third, we require elasticity parameters,
as given in Table~\ref{tab:car_elasticity_parameters}.



% Reduce spacing between table rows henceforth.
\renewcommand{\arraystretch}{0.6}

\begin{landscape}
\begin{table}
\footnotesize
\begin{center}
\caption{Car example: Vehicle parameters.}
\label{tab:car_parameters}
\begin{tabular}{ r c c l }
  \toprule
    Description                  & Ford Fusion & Ford Fusion & Data sources and notes\\
    Parameters [units]                              & (gasoline) & (hybrid EV) &  \\
  \midrule
  Fuel economy                           & 25 & 42 & Combined cycle mpg value taken from \citet{Car_Connection:2020}, \\
  $\bempl{\eta}$, $\amacro{\eta}$ [mpg]             &           &          & for Titanium FWD 2020 model with Intercooled I-4, 2.0 L engine.   \\
                                                    &           &          & Combined cycle mpg value taken from \citet{Car_Connection:2020},\\
                                                    &           &          & for Titanium FWD 2020 model with Gas/Electric I-4, 2.0 L engine. \\
  \midrule
   Undiscounted capital expenditure rate      & 2,533     & 2,518    & Fourteen year annual, averaged capital costs = purchase cost / $t_{\life}$.  \\
   $\rbempl{C}_{cap}$, $\raempl{C}_{cap}$ [\$/yr]   &           &          & Ford Fusion gasoline costs from \citet{Edmunds:2020_fusion_gasoline}. \\
                                                    &           &          & Ford Fusion Hybrid car costs from \citet{Edmunds:2020_fusion_hybrid}. \\
 \midrule
   Lifespan                     & 14  & 14 & Lifetime taken as 14 years, based on 13--17 years for U.S.\ cars \\
   $\bempl{t}_{\life}$, $\aempl{t}_{\life}$ [yr]    &           &          & from \citet{Berla:2016} and 14 years for UK cars from \\
                                                    &           &          & \citet{SMMT:2020}. \\
 \midrule
   Embodied energy             & 34,000    & 40,000   & 34,000 MJ for conventional Ford Fusion gasoline car taken from \\
   $\bempl{E}_{emb}$, $\aempl{E}_{emb}$ [MJ]        &           &          & \citet{Argonne_National_Laboratory:2010}. \\
                                                    &           &          & We assume an additional 6,000 MJ added for Ford Fusion \\
                                                    &           &          & Hybrid Electric Vehicle (HEV) battery, as HEV typically adds 10--25\% \\
                                                    &           &          & to total LCA energy of vehicle manufacture \citep{onat2015conventional}. \\
                                                    &           &          & Battery lifetime assumed same as car lifetime, based on \\
                                                    &           &          & \citet{nordelof2014environmental} and \citet{onat2015conventional}. \\
  \midrule
  Operations and maintenance expenditure rate   & 5,050   & 4,779     & Lifetime (14 year) annual, averaged operation and maintenance \\
  $\rbempl{C}_{\om}$, $\raempl{C}_{\om}$ [\$/yr]                &                  &                                    & (O\&M) costs = sum of insurance, maintenance, repairs, taxes, and \\
                                                                &                  &                                    & fees (excluding financing, depreciation, fuel). 5-year Ford Fusion \\
                                                                &                  &                                    & O\&M costs from \citet{Edmunds:2020_fusion_gasoline}. 5-year Ford Fusion Hybrid \\
                                                                &                  &                                    & O\&M costs from \citet{Edmunds:2020_fusion_hybrid}. Extrapolation of O\&M costs \\
                                                                &                  &                                    & for years 6--14 based on \cite{djokic2015simulation}. \\
 \midrule
  Disposal cost                                     & $-300$  & $-300$        & Salvage value (negative cost) taken from \cite{Junk-Car-Medics:aa} \\
  $\bempl{C}_d$, $\aempl{C}_d$ [\$]                 &                              &                                    &   \\
 \midrule
  Ops., maint., and disposal expenditure rate,  & 5,033  & 4,762    & Sum of annualized operations, maintenance, and disposal costs. \\
  discounted $\rbempl{C}_{\omd}$, $\raempl{C}_{\omd}$ [\$/yr]   &                  &                                    & \\
  \bottomrule
\end{tabular}
\end{center}
\end{table}
\end{landscape}

 
\begin{landscape}
\begin{table}
\footnotesize
\begin{center}
\caption{Car example: Economic parameters (2020).}
\label{tab:car_economic_parameters}
\begin{tabular}{ r c l }
  \toprule
  Description & Value & Data sources and notes \\
  Parameter [units] & & \\
  \midrule
  Distance driven prior to upgrade   & 12,416        & Average U.S.\ vehicle miles/yr, calculated from \citet{Car_Insurance:2019}.  \\
  $\rbempl{q}_s$ [miles/yr]                                 &               & This is slightly lower than the average driver miles/yr (13,476) \\
                                                            &               & \citep{US_DoT:2018}, as there are \\
                                                            &               & more registered U.S.\ vehicles than drivers. \\
  \midrule
  Real median personal income U.S., in 2018                 & 34,317        & Taken from \citet{St_Louis_Fed:2019}.\\
  {} [\$/yr]                                                &               & \\
  \midrule
  U.S.\ 2018 disposable income /                            & 0.88319       & Taken from U.S.\ Bureau of Economic Analysis (BEA) \\
  real income (minus current taxes)                         &               & National and Products Accounts (NIPA) \\
  {} [--]                                                   &               & Table 2.1.\ Personal Income and Its Disposition \\
                                                            &               & \citep{US_BEA:2020}. \\
  \midrule
  Share of savings from 2018 disposable income              & 0.07848       & Taken from U.S.\ Bureau of Economic Analysis (BEA) \\
  {} [--]                                                   &               & National and Products Accounts (NIPA) \\
                                                            &               & Table 2.1.\ Personal Income and Its Disposition \\
                                                            &               & \citep{US_BEA:2020}. \\
  \midrule
  Personal consumption in 2018                  & 27,930      & Calculation: $(\$34,317\mathrm{/yr}) (0.88319) (1 - 0.07848)$ \\
  $\rate{M}$ [\$/yr]                                        &               &                                              \\
  \midrule
  Price of gasoline                       & 2.63       & Source: \citet{US_EIA:2020_gasoline} \\
  $p_E$ [\$/gallon]                                         &               &  \\
  \midrule
  Fractional spend on original energy service      & 0.066      & Calculation: \$1,306 (spend on energy service) \\
  $\fCs$ [--]                                               &               & / [\$19,115 (other goods) + \$1,306 (energy service)] = 0.064, \\
                                                            &               & where spend on energy service = 12,416 miles / 25 mpg \\
                                                            &               & $\times$ \$2.63/gallon = \$1,306. \\
  \midrule
  Real discount rate                                        & 0.03  & Taken from Federal Reserve St.\ Louis for 72 month car loan rate,  \\
  $r$ [1/yr]                                                &                     & which averaged 5\% before the 2022 interest rate raises. \\
                                                            &                     & Subtracting 2\% inflation gives 3\% real interest rate, \\
                                                            &                     & by which we discount. \\
                                                            &                     & \citep{FRED-US:2024aa} \\
  \midrule
  Macro factor                                              & 1.0           & An initial value. See Section~\ref{sec:calculating_k} for additional details. \\
  $k$ [--]                                                  &               &               \\
  \bottomrule
\end{tabular}
\end{center}
\end{table}
\end{landscape}


\begin{landscape}
\begin{table}
\footnotesize
\begin{center}
\caption{Car example: Elasticity parameters.}
\label{tab:car_elasticity_parameters}
\begin{tabular}{ r c l }
  \toprule
  Description & Value & Data sources and notes \\
  Parameter [units] & & \\
  \midrule
  Uncompensated own price elasticity of car use demand  & $-0.2$  & We adopt $-0.2$ as our baseline value, based on U.S.\ studies including \\
  $\eqspsUCorig$ [--]                                           &           &  \citet{Gillingham:2020aa} who estimated a value of $-0.1$, \\
                                                           &           &   \citet{goetzke2018} who estimated values between $-0.05$ and \\
                                                           &          &      $-0.23$, and \citet{Parry2005} who estimated values between \\
                                                             &           &    $-0.1$ and $-0.3$. For comparison, \citet{Borenstein:2015aa} uses values \\
                                                            &           &   of $-0.1$ to $-0.4$ based on \citet{Parry2005}.\\
  \midrule
  Compensated price elasticity of car use demand                   &   $-0.134$   & Calculated via the Slutsky Equation (Eq.~(\ref{PartI-eq:slutsky}) in Part~I).  \\
  $\eqspsCorig$ [--]                                           &                                &   \\
  \midrule
  Compensated cross price elasticity of demand for other goods        &   0.009     & Calculated via Eq.~(\ref{PartI-eq:eqops}) in Part~I.  \\
  $\eqopsCorig$ [--]                                           &                                &   \\
  \midrule
  Income elasticity of demand for car use                 &   1.0      & Follows from CES utility function. \\
  $\eqsM$ [--]                                            &                                &   \\
  \midrule
  Income elasticity of demand for other goods             &   1.0      & Follows from CES utility function. \\
  $\eqoM$ [--]                                            &                                &   \\
  \bottomrule
\end{tabular}
\end{center}
\end{table}
\end{landscape}





%------------------------------
\subsubsection{Data for lamp example}
\label{sec:data_lamp_example}
%------------------------------




For the second example,
we consider purchasing a Light Emitting Diode (LED) electric lamp
to replace a baseline incandescent electric lamp.
Both lamps are matched as closely as possible
in terms of energy service delivery (measured in lumen output per lamp),
the key difference being the energy required to provide that service.
The LED lamp has a low
initial capital investment rate
when spread out over the lifetime of the lamp
(less than the incumbent incandescent lamp)
and a long-term benefit of decreased direct energy expenditures
at approximately the same energy service delivery rate (\lmhr/yr).

Again, three sets of data are required.
First, basic lamp parameters are summarized in Table~\ref{tab:lamp_parameters}.
Second, several general economic parameters,
mainly relating to the U.S.\ economy and
personal finances
of a representative U.S.-based user are
given in Table~\ref{tab:lamp_economic_parameters}.
Third, we require the elasticity parameters,
as shown in Table~\ref{tab:lamp_elasticity_parameters}.


\begin{landscape}
\begin{table}
\footnotesize
\begin{center}
\caption{Lamp example: Electric lamp parameters.}
\label{tab:lamp_parameters}
\begin{tabular}{ r c c l }
  \toprule
    Description                  & Incandescent lamp & LED lamp & Data sources and notes\\
    Parameters [units]                                &  &  & \\
  \midrule
  Lamp efficiency                   & 8.83      & 81.8     & Incandescent: 530 lm output / 60 W energy input. \\
  $\bempl{\eta}$, $\amacro{\eta}$ [\lmhr/\Whr]       &           &          & LED: 450 lm output / 5.5 W energy input. \\
  \midrule
   Undiscounted capital expenditure rate       & 1.044   & 0.121    & Purchase costs: \$1.88 for incandescent lamp from  \\
   $\rbempl{C}_{cap}$, $\raempl{C}_{cap}$ [\$/yr]   &           &          & \citet{Home_Depot:2020_Inc_bulb}, \\
                                                    &           &          & and \$1.21 for LED lamp from \\
                                                    &           &          & \citet{Home_Depot:2020_LED_bulb}. \\
 \midrule
   Lifespan               & 1.8       & 10       & Based on assumed 3 hr/day  \\
   $\bempl{t}_{\life}$, $\aempl{t}_{\life}$ [yr]    &           &          & from \citet{Home_Depot:2020_Inc_bulb} \\
                                                    &           &          & and  \citet{Home_Depot:2020_LED_bulb}. \\
   \midrule
   Life cycle analysis (LCA) embodied energy  & 2.20      & 6.50     & Base document: Table 4.5 Manufacturing Phase Primary  \\
   $\bempl{E}_{emb}$, $\aempl{E}_{emb}$ [MJ]        &           &          & Energy (MJ/20 million \lmhr), contained in \\
                                                    &           &          & U.S.\ DoE Life-cycle assessment of energy and environmental \\
                                                    &           &          & impacts of LED lighting products \citep{US_DoE:2012}.\\
                                                    &           &          & Incandescent lamp: \\
                                                    &           &          & LCA energy = 42.2 MJ/20 million \lmhr.  \\
                                                    &           &          & Lifetime output = 530 lm $\times$ 3 hr/day \\
                                                    &           &          & $\times$ 365 days/yr $\times$ 1.8 yr = 1,044,630 \lmhr.\\
                                                    &           &          & Thus LCA energy / lamp =  42.2 $\times$ 1.0446/20 = 2.20 MJ. \\
                                                    &           &          & LED lamp:  \\
                                                    &           &          & LCA energy = 132 MJ/20 Million \lmhr \\
                                                    &           &          & for pack of 5 LED lamps. \\
                                                    &           &          & Lifetime output = 450 lm $\times$ 3 hr/day  \\
                                                    &           &          & $\times$ 365 days/yr $\times$ 10 yr = 4,926,405 \lmhr. \\
                                                    &           &          & Thus LCA energy / lamp =  132 MJ/5 $\times$ 4.9264/20 = 6.5 MJ. \\
  \midrule
  Operations and maintenance expenditure rate   & 0  & 0   & Lifetime annual, averaged operations and maintenance (O\&M) costs. \\
  $\rbempl{C}_{\om}$, $\raempl{C}_{\om}$ [\$/yr]                &                  &                                   & Once installed assumed 0. Note: O\&M costs exclude fuel \\
                                                                &                  &                                   & (i.e., electricity) costs. \\
 \midrule
  Disposal cost                                     & $0$ & $0$      &  Disposal cost assumed negligible \\
  $\bempl{C}_d$, $\aempl{C}_d$ [\$]                 &                              &                                   & (local/doorstep recycling facility). \\
 \midrule
  Ops., maint., and disposal expenditure rate,  & 0 & 0  &  Sum of annualized operations, maintenance, and disposal costs. \\
  discounted $\rbempl{C}_{\omd}$, $\raempl{C}_{\omd}$ [\$/yr]   &                  &                                    &  \\
  \bottomrule
\end{tabular}
\end{center}
\end{table}
\end{landscape}






\begin{landscape}
\begin{table}
\footnotesize
\begin{center}
\caption{Lamp example: Economic parameters (2020).}
\label{tab:lamp_economic_parameters}
\begin{tabular}{ r c l }
  \toprule
  Description  & Value & Data sources and notes \\
  Parameter [units] & & \\
  \midrule
  Lighting consumption prior to upgrade   & 580,350  & Calculation: (530 lm) (3 hrs/day) (365 days/yr). \\
  $\rbempl{q}_s$ [\lmhr/yr]                                 &               & \\
  \midrule
  Real median personal income U.S.\, in 2018                & 34,317        & Refer to Table~\ref{tab:car_economic_parameters}.\\
  {} [\$/yr]                                                &               & \\
  \midrule
  U.S.\ 2018 disposable income /                            & 0.88319       & Refer to Table~\ref{tab:car_economic_parameters}. \\
  real income (minus current taxes)                         &               &  \\
  {} [--]                                                   &               &  \\
  \midrule
  Share of savings from 2018 disposable income              & 0.07848       & Refer to Table~\ref{tab:car_economic_parameters}. \\
  {} [--]                                                   &               &  \\
  \midrule
  Personal consumption in 2018                   & 27,930     & Calculation: $(\$34,317\mathrm{/yr}) (0.88319) (1 - 0.07848)$. \\
  $\rate{M}$ [\$/yr]                                      &               &                                              \\
  \midrule
  Price of electricity                  & 0.1287        & U.S.\ 2018 average U.S.\ household electricity price \\
  $p_E$ [\$/\kWhr]                                          &               & \citep{US_EIA:2020_electricity}.  \\
  \midrule
  Fractional spend on original energy service    & 0.0003028       & Calculation: \$8.5/yr (spend on energy service) / \\
  $\fCs$ [--]                                               &               & [\$27,920/yr (other goods) + \$8.5/yr (energy service)] =  \\
                                                            &               & 0.0003028, where spend on energy service = 580,350 \lmhr/yr /  \\
                                                            &               & 8.83 lm/W / 1000 W/kW $\times$ \$0.1287/\kWhr{} = \$8.5/yr. \\
                                                            &               & Note: this is energy service from a single lamp. \\
  \midrule
  Real discount rate                                        & 0.03  & Taken from Federal Reserve St.\ Louis for 72 month car loan rate,  \\
  $r$ [1/yr]                                                &                      & which averaged 5\% before the 2022 interest rate raises. \\
                                                            &                      & Subtracting 2\% inflation gives 3\% real interest rate, \\
                                                            &                      & by which we discount. \\
                                                            &                      & \citep{FRED-US:2024aa} \\
  \midrule
  Macro factor                                              & 1.0           & An initial value. See Section~\ref{sec:calculating_k} for additional details. \\
  $k$ [--]                                                  &               &               \\
  \bottomrule
\end{tabular}
\end{center}
\end{table}
\end{landscape}




\begin{landscape}
\begin{table}
\footnotesize
\begin{center}
\caption{Lamp example: Elasticity parameters.}
\label{tab:lamp_elasticity_parameters}
\begin{tabular}{ r c l }
  \toprule
  Description & Value & Data sources and notes \\
  Parameter [units] & & \\
  \midrule
  Uncompensated own price elasticity of lighting demand  &   $-0.4$  & We adopt $-0.4$ as our baseline value, as the average of last 50 years \\
  $\eqspsUCorig$ [--]                                                             &           &   from \citet[Fig.~4]{Fouquet2014}. \\
                                                            &           &  For comparison, \citet{Borenstein:2015aa} uses a range of $-0.4$ to $-0.8$, \\
                                                            &            &     based on \citet{Fouquet2011}. \\
  \midrule
  Compensated own price elasticity of lighting demand                 &  $-0.3997$      & Calculated via the Slutsky Equation (Eq.~(\ref{PartI-eq:slutsky}) in Part~I).  \\
  $\eqspsCorig$ [--]                                           &           &   \\
  \midrule
  Compensated cross price elasticity of demand for other goods        &   0.00012     & Calculated via Eq.~(\ref{PartI-eq:eqops}) in Part~I.  \\
  $\eqopsCorig$ [--]                                           &           &   \\
  \midrule
  Income elasticity of lighting demand                    &   1.0     & Follows from CES utility function. \\
  $\eqsM$ [--]                                            &           &   \\
  \midrule
  Income elasticity of demand for other goods             &   1.0     & Follows from CES utility function. \\
  $\eqoM$ [--]                                            &           &   \\
  \bottomrule
\end{tabular}
\end{center}
\end{table}
\end{landscape}


%++++++++++++++++++++++++++++++
\subsection{Visualization}
\label{sec:path_graphs}
%++++++++++++++++++++++++++++++

A rigorous rebound analysis should track
energy, expenditure, and consumption
aspects of rebound
at the device (direct rebound) and elsewhere in the economy (indirect rebound)
across adjustments for all rebound effects
(emplacement, substitution, income, and macro).
Doing so involves many terms and much complexity.

To date, visualizing the energy, expenditure, and consumption
aspects of rebound phenomena has not been
done in a numerically precise manner
with a set of mutually consistent graphs.
So we introduce \emph{rebound planes}
to help advance clarity of
(direct and indirect) rebound and
adjustments (via emplacement, substitution, income, and macro effects)
across all aspects (energy, expenditure, and consumption).
Each aspect is represented by a path in its own plane,
showing adjustments in response to the EEU.
The order of presentation below is
energy first, followed by expenditure, ending with consumption,
because the EEU triggers rebound
(the topic of this article and visible in the energy plane),
but is caused by expenditures on the EEU and further monetary adjustments
(visible in the expenditure plane),
which are calculated via details about substitution
(visible in the consumption plane).

Axes of the rebound planes
represent direct and indirect effects, with
direct effects shown on the $x$-axes, and
indirect effects shown on the $y$-axes.
Specifically,
%
\begin{enumerate*}[label={(\roman*)}]

  \item direct and indirect energy consumption rates ($\rate{E}_{dir}$, $\rate{E}_{indir}$) are placed on
        the $x$- and $y$-axes of the energy plane, respectively;

  \item direct and indirect expenditure rates ($\rate{C}_{dir}$ and $\rate{C}_{indir}$,
        discounted for beginning of life and end of life costs) are placed on
        the $x$- and $y$-axes of the expenditure plane, respectively; and

  \item the indexed consumption rate of the energy service ($\rate{q}_s$) and
        the indexed expenditure rate of other consumption goods ($\rate{C}_o$) are placed on
        the $x$- and $y$-axes of the consumption plane, respectively.

\end{enumerate*}
%
Paths through energy, expenditure, and consumption planes
consist of segments that represent changes due to the various rebound effects.
Table~\ref{tab:path_graph_segments} provides the key for rebound path segments.
Effects that include both direct and indirect rebound
will show displacement along both axes and
create a path in the $x$-$y$ plane.
See Appendix~\ref{sec:graph_details} for
detailed mathematical descriptions for constructing paths on the rebound planes, and
see Section~\ref{sec:results} for rebound path graphs
for EEU examples of a car and an electric lamp.


\begin{table}
\footnotesize
\centering % Centered table
\caption{Segments in rebound planes.}
\begin{tabular}{r l l c}
  \toprule
  \multicolumn{1}{c}{Segment} & Rebound effect             & Symbol           \\
  \midrule
  \circa{} \hspace{48.8mm}    & Direct emplacement         & $Re_{dempl}$     \\
  \ab{}    \hspace{42.0mm}    & Embodied energy            & $Re_{emb}$       \\
  \bstar{} \hspace{35.2mm}    & Ops.\, maint.\, and disp.\ & $Re_{\omd}$       \\
  \midrule
  \starc{} \hspace{28.8mm}    & Indirect substitution      & $Re_{isub}$      \\
  \chat{}  \hspace{21.6mm}    & Direct substitution        & $Re_{dsub}$      \\
  \midrule
  \hatd{}  \hspace{14.8mm}    & Direct income              & $Re_{dinc}$      \\
  \dbar{}  \hspace{ 7.1mm}    & Indirect income            & $Re_{iinc}$      \\
  \midrule
  \bartilde{}                 & Macro                      & $Re_{macro}$     \\
  \bottomrule
\end{tabular}
\label{tab:path_graph_segments}
\end{table}

Each rebound plane is described in the subsections below.
Reference to the rebound planes in
Figs.~\ref{fig:CarEnergyGraph}--\ref{fig:LampConsGraph}
below will be beneficial.


%------------------------------
\subsubsection{The energy plane}
\label{sec:energy_path_graphs}
%------------------------------

The energy plane
(see Figs.~\ref{fig:CarEnergyGraph} and~\ref{fig:LampEnergyGraph} below)
shows the direct energy consumption rate ($\rate{E}_{dir}$) on the $x$-axis
and
the indirect energy consumption rate ($\rate{E}_{indir}$) on the $y$-axis.%
\footnote{
  A related, notional-only (not quantified as in Section~\ref{sec:results}),
  one-dimensional visualization of direct and indirect energy rebound
  (but not on expenditure or consumption planes)
  can be found in Fig.~1 of \cite{Exadaktylos:2021aa}.}
%
Points $\circ$, $*$, $\wedge$, $-$, and $\sim$ represent stages
between the rebound effects of Fig.~\ref{fig:flowchart}.
Points $a$, $b$, $c$, and $d$ represent intermediate stages.
Lines with negative slope through points
$\circ$, $a$, $*$, $\wedge$, $-$, and $\sim$
indicate energy consumption isoquants
at key points.
Note that segment \bartilde{} appears only in the energy plane,
because the framework tracks energy consumption but not expenditures or consumption
for the macro effect.

In the energy plane,
point $a$ lies on the $Re_{tot} = 0\%$ line
indicating that point $a$ (and the $Re_{tot} = 0\%$ line)
is the point from which all rebound effects
($Re_{empl}$, $Re_{sub}$, $Re_{inc}$, and $Re_{\macro}$)
are measured.
If rebound effects cause
total energy demand to return to the original energy consumption level
(negative sloping line through the $\circ$ point),
all expected energy savings are taken back by rebound effects.
Thus, the line of constant energy consumption through the $\circ$ point is labeled
$Re_{tot} = 100\%$.
The contribution of each rebound effect to total rebound
is represented by the distance that each component's segment
moves across the rebound isoquants.
Total rebound ($Re_{tot}$) is measured linearly between and beyond the
$Re_{tot} = 0\%$ and $Re_{tot} = 100\%$ lines,
with direct rebound in the $x$ direction and
indirect rebound in the $y$ direction.
The region below and to the left of the $Re_{tot} = 0\%$ line
exhibits negative rebound, indicating hyperconservation.
The region above and to the right of the $Re_{tot} = 100\%$ line
shows backfire,
i.e., greater total energy consumption after the EEU than before it.

Segment \starc{} moves in the negative $y$ direction
by definition of the indirect substitution effect,
and segment \chat{} moves in the positive $x$ direction
by the definition of the direct substitution effect.
Both income effect segments (\hatd{} and \dbar{})
show more energy consumption, because net savings are spent
on goods and services that rely on at least some energy consumption.%
\footnote{
  We exclude the case of an inferior good, whose consumption decreases
  as real income increases, but
  we note here the possibility of such behavior.
  This behavior would
  however require a different utility model
  besides the CES utility model,
  which we use throughout this analysis.
}
%
Segment \bartilde{} always moves in the positive $y$ direction,
because macro effects lead to additional indirect energy consumption.


%------------------------------
\subsubsection{The expenditure plane}
\label{sec:expenditure_path_graphs}
%------------------------------

The expenditure plane (see Figs.~\ref{fig:CarCostGraph} and~\ref{fig:LampCostGraph} below) shows
the direct expenditure rate on the energy service ($\rate{C}_{dir}$) on the $x$-axis and
the indirect expenditure rate
($\rate{C}_{indir}$, discounted when appropriate)
on the $y$-axis.
Lines with negative slope through points $\circ$, $a$, $*$, and $\wedge$
indicate expenditure isoquants.
The line through the $\circ$ point is an isoquant for the
cost of purchasing the original consumption bundle
at the original prices.
The line through the $*$ point is an isoquant for the
cost of purchasing the original consumption bundle
at the new prices.
Segments \ab{} and \bstar{}
could both move in the positive $y$ direction,
they could both move in the negative $y$ direction, or
they could move in opposite directions,
depending on the results of the independent analyses for
embodied energy and capital cost rates.
The substitution effect along segments \starc{} and \chat{}
will together, by definition, lead to lower expenditure
due to the energy service price decline and
the budget-reducing compensating variation (CV).
The income effect (segments \hatd{} and \dbar{}) must bring expenditure
back to the original expenditure line
(equal to the budget constraint set by income in dollar or nominal terms)
by assumptions about non-satiation and utility maximization
in the device user's decision function.


%------------------------------
\subsubsection{The consumption plane}
\label{sec:consumption_path_graphs}
%------------------------------

The consumption plane
(Figs.~\ref{fig:CarConsGraph} and~\ref{fig:LampConsGraph} below) shows
the indexed rate of energy service consumption ($\rate{q}_s/\rbempl{q}_s$) on the $x$-axis and
the indexed rate of other goods consumption ($\rate{C}_o/\rbempl{C}_o$) on the $y$-axis.
Iso-expenditure loci of indexed energy service and other goods demand,
i.e. budget constraints,
are shown as lines with negative slope
(lines \circcirc{}, \starstar{}, \hathat{}, and \barbar{}).
Note that budget constraints \circcirc{} and \barbar{}
intersect at the $y$-axis (i.e., where $x$ = 0),
because the prices of other goods and services do not change
as a result of the EEU.
Emplacement (by itself) does not alter consumption patterns, so
the rate of energy service consumption and
the rate of other goods consumption are
unchanged across the emplacement effect
($\rbempl{q}_s = \raempl{q}_s$ and $\rbempl{C}_o = \raempl{C}_o$, respectively).
Thus,
only movements after the $*$ point are visible as a path in the consumption plane, and
points $\circ$, \emph{a}, \emph{b}, and $*$
collapse to the same location in the consumption plane.

Indifference curves for the CES utility model
are denoted by \iicirc{} and \iibar{}
and represent lines of constant utility through the $\circ$ and $-$ points.
Prior to the EEU, the consumption basket (of the energy service and other goods)
is represented by the $\circ$ point.
The budget constraint,
here in real terms, i.e.,
the capacity to purchase either the energy service or other goods and services,
is shown as isoquant \circcirc{}.
The original budget constraint line (\circcirc{})
is tangent to the original indifference curve
(\iicirc{}) at point $\circ$, the optimal consumption
bundle prior to the EEU.
The real budget line \starstar{}
indicates the (higher) capacity to purchase
combinations of energy services and other goods and services
using the same money needed to purchase
the old consumption bundle but at the new, lower price
for the energy service,
thanks to the EEU.

The substitution effect leads to the cheaper, optimal
CES-utility-preserving
consumption bundle at the $\wedge$ point.
The substitution effect is shown by segments
\starc{} (the indirect component, which represents the decrease in other goods consumption) and
\chat{} (the direct component, which represents the increase in energy service consumption).
Although the substitution effect is calculated
in the consumption plane,
its impact
can be seen in
the energy and expenditure planes.

In the consumption plane,
the income expansion path under the CES utility model
is a ray (\rr{}) from the origin through the $\wedge$ point
in the consumption plane.
The pre- and post-income-effect
points ($\wedge$ and $-$, respectively)
lie along the \rr{} ray, due to homotheticity.
The increased consumption rate of the energy service is
represented by segment \hatd{}, and
the increased consumption rate of other goods and services is
represented by segments \dbar{}.

Under non-homothetic utility models,
the income expansion path will be closer to vertical
in the consumption plane,
as the device owner spends more net income ($\rasub{N}$)
on other goods and less on the energy service.
In the limit, consumption of the energy service
is already satiated, so
net income is spent completely on other goods,
resulting in a vertical income expansion path.


%++++++++++++++++++++++++++++++
\subsection{Software tools}
\label{sec:software_tools}
%++++++++++++++++++++++++++++++

We developed an open source \texttt{R} package called \texttt{ReboundTools}
to standardize and distribute the methods for calculating rebound magnitudes in our framework.
\texttt{ReboundTools} can be found at \url{https://github.com/MatthewHeun/ReboundTools}.
(See \citet{Heun:2023aa}.)
\texttt{ReboundTools} provides functions for
%
\begin{enumerate*}[label={(\roman*)}]

  \item reading input data from a spreadsheet,

  \item performing rebound calculations, and

  \item generating rebound tables and rebound planes.

\end{enumerate*}
%
\texttt{ReboundTools} was used for
all calculations and all rebound planes in this paper.

To find the path to an example spreadsheet bundled with the package,
users of \texttt{ReboundTools}
can call the function \texttt{ReboundTools::sample\_eeu\_data\_path()}.
After filling the example spreadsheet with parameters for an EEU,
users can call two functions
(\texttt{ReboundTools::load\_eeu\_data()} and \texttt{ReboundTools::rebound\_analysis()})
to perform all rebound calculations described in this paper.
The function \texttt{ReboundTools::path\_graphs()} creates
rebound paths in the energy, expenditure, and consumption planes.
Extensive documentation for \texttt{ReboundTools}
can be found at \url{https://matthewheun.github.io/ReboundTools/}.

In addition, an Excel workbook that performs identical rebound calculations
using the framework of this paper
**** will be stored at the
Research Data Leeds Repository if this submission is accepted.
The spreadsheet file is included with the submission of this paper. ****
% can be found at the data repository for this paper at
% \url{https://doi.org/10.5518/1201}.
(See \citet{Brockway:2023aa}.)


%%%%%%%%%%%%%%%%%%%%%%%%%%%%%%%%%%%%%%%%%%%%%%%%%%%%%%%%%%%%%%
\section{Results}
\label{sec:results}
%%%%%%%%%%%%%%%%%%%%%%%%%%%%%%%%%%%%%%%%%%%%%%%%%%%%%%%%%%%%%%

In this section we present rebound calculation results for two examples:
energy efficiency upgrades of a car (Section~\ref{sec:car_example}) and
an electric lamp (Section~\ref{sec:lamp_example}).
Univariate sensitivity studies for both examples (car and lamp)
can be found in Appendix~\ref{sec:sensitivity_analyses}.


%++++++++++++++++++++++++++++++
\subsection{Example 1: Purchase of a new car}
\label{sec:car_example}
%++++++++++++++++++++++++++++++

Armed with the data in
Tables~\ref{tab:car_parameters}--\ref{tab:car_elasticity_parameters},
and the equations in Section~\ref{PartI-sec:framework} of Part~I,
we calculate important values at each rebound stage,
as shown in Table~\ref{tab:car_stages_table}.
Note that Table~\ref{tab:car_stages_table} applies to the car user.
Across the macro effect (segment \bartilde{} in Fig.~\ref{fig:CarEnergyGraph}),
changes occur only in the macroeconomy.
For the car user, no changes are recorded across the macro effect.
Thus, the $-$ (bar) and $\sim$ (tilde) columns
of Table~\ref{tab:car_stages_table} are identical.
Rebound components for the car upgrade are shown in Table~\ref{tab:car_results}.
Figs.~\ref{fig:CarEnergyGraph}--\ref{fig:CarConsGraph}
show energy, expenditure, and consumption planes
for the car example.


% % latex table generated in R 4.4.1 by xtable 1.8-4 package
% % Wed Sep 25 14:57:46 2024
% \begin{table}[ht]
% \centering
% \caption{Results for car example with macro factor ($k$) assumed to be 1. There is no change for the consumer across across the macro effect, so the last stage ($\sim$) is not shown.} 
% \label{tab:car_stages_table}
% \begingroup\footnotesize
% \begin{tabular}{rrrrr}
%   \toprule
%   & Original ($\circ$) & After empl ($*$) & After sub ($\wedge$) & After inc ($-$) \\ 
%   \midrule
% $t_{li\!f\!e}$ [yr] & 14 & 14 & 14 & 14 \\ 
%   $R_\alpha$ [--] & 1.203 & 1.203 & 1.203 & 1.203 \\ 
%   $R_\omega$ [--] & 0.796 & 0.796 & 0.796 & 0.796 \\ 
%   $\eta$ [mile/gal] & 25 & 42 & 42 & 42 \\ 
%   $\eta$ [mile/MJ] & 0.197 & 0.332 & 0.332 & 0.332 \\ 
%   $p_s$ [\$/mile] & 0.105 & 0.063 & 0.063 & 0.063 \\ 
%   $\dot{q}_s$ [mile/yr] & 12,416 & 12,416 & 13,323 & 13,899 \\ 
%   $p_E$ [\$/MJ] & 0.0208 & 0.0208 & 0.0208 & 0.0208 \\ 
%   $\dot{E}_s$ [MJ/yr] & 62,885 & 37,432 & 40,167 & 41,903 \\ 
%   $\dot{E}_{emb}$ [MJ/yr] & 2,429 & 2,857 & 2,857 & 2,857 \\ 
%   $\dot{C}_s$ [\$/yr] & 1,306 & 777 & 834 & 870 \\ 
%   $\dot{C}_{cap}$ [\$/yr] & 2,533 & 2,518 & 2,518 & 2,518 \\ 
%   $R_{\alpha}\dot{C}_{cap}$ [\$/yr] & 3,048 & 3,030 & 3,030 & 3,030 \\ 
%   $\dot{C}_{O\!M}$ [\$/yr] & 5,050 & 4,779 & 4,779 & 4,779 \\ 
%   $C_d$ [\$] & $-$300 & $-$300 & $-$300 & $-$300 \\ 
%   $\dot{C}_d$ [\$/yr] & $-$21 & $-$21 & $-$21 & $-$21 \\ 
%   $R_{\omega}\dot{C}_d$ [\$/yr] & $-$17 & $-$17 & $-$17 & $-$17 \\ 
%   $\dot{C}_{O\!M\!d}$ [\$/yr] & 5,033 & 4,762 & 4,762 & 4,762 \\ 
%   $\dot{C}_o$ [\$/yr] & 18,543 & 18,543 & 18,469 & 19,267 \\ 
%   $\dot{N}$ [\$/yr] & 0 & 817 & 835 & 0 \\ 
%   $\dot{M}$ [\$/yr] & 27,930 & 27,930 & 27,930 & 27,930 \\ 
%    \bottomrule
% \end{tabular}
% \endgroup
% \end{table}

% latex table generated in R 4.4.1 by xtable 1.8-4 package
% Mon Oct 21 11:38:05 2024
\begin{table}[ht]
\centering
\caption{Results for car example with macro factor ($k$) assumed to be 1. There is no change for the consumer across across the macro effect, so the last stage ($\sim$) is not shown.  Blanks indicate unchanged values relative to previous or later values in the same row.} 
\label{tab:car_stages_table}
\begingroup\footnotesize
\begin{tabular}{rrrrrr}
  \toprule
   &   & Original ($\circ$) & After empl ($*$) & After sub ($\wedge$) & After inc ($-$) \\ 
  \midrule
   & $\dot{M}$ [\$/yr] & 27,930 & 27,930  & 27,930  & 27,930 \\ 
   & $p_E$ [\$/MJ] & 0.0208 & 0.0208  & 0.0208  & 0.0208 \\ 
   & $t_{li\!f\!e}$ [yr] & 14 & 14 & 14  & 14  \\ 
   & ${\tau}_\alpha$ [--] & 1.203 & 1.203 & 1.203  & 1.203  \\ 
   & ${\tau}_\omega$ [--] & 0.796 & 0.796 & 0.796  & 0.796  \\ 
   & $\eta$ [mile/gal] & 25 & 42 & 42  & 42  \\ 
   & $\eta$ [mile/MJ] & 0.197 & 0.332 & 0.332  & 0.332  \\ 
   & $p_s$ [\$/mile] & 0.105 & 0.063 & 0.063 & 0.063  \\ 
   & $\dot{E}_{emb}$ [MJ/yr] & 2,429 & 2,857 & 2,857  & 2,857 \\ 
   & $\dot{C}_{cap}$ [\$/yr] & 2,533 & 2,518 & 2,518  & 2,518 \\ 
   & ${\tau}_{\alpha}\dot{C}_{cap}$ [\$/yr] & 3,048 & 3,030 & 3,030  & 3,030  \\ 
   & $\dot{C}_{O\!M}$ [\$/yr] & 5,050 & 4,779 & 4,779  & 4,779  \\ 
   & $C_d$ [\$] & $-$300 & $-$300 & $-$300  &  $-$300 \\ 
   & $\dot{C}_d$ [\$/yr] & $-$21 & $-$21 & $-$21  & $-$21  \\ 
   & ${\tau}_{\omega}\dot{C}_d$ [\$/yr] & $-$17 & $-$17 & $-$17  & $-$17  \\ 
   & $\dot{C}_{O\!M\!d}$ [\$/yr] & 5,033 & 4,762 & 4,762  & 4,762 \\ 
   & $\dot{E}_s$ [MJ/yr] & 62,885 & 37,432 & 40,167 & 41,903 \\ 
   & $\dot{C}_s$ [\$/yr] & 1,306 & 777 & 834 & 870 \\ 
   & $\dot{N}$ [\$/yr] & 0 & 817 & 835 & 0 \\ 
   & $\dot{q}_s$ [mile/yr] & 12,416  & 12,416 & 13,323 & 13,899 \\ 
   & $\dot{C}_g$ [\$/yr] & 18,543  & 18,543 & 18,469 & 19,267 \\ 
   \bottomrule
\end{tabular}
\endgroup
\end{table}



% latex table generated in R 4.4.1 by xtable 1.8-4 package
% Wed Sep 25 14:57:46 2024
\begin{table}[ht]
\centering
\caption{Car example: rebound results with macro factor ($k$) assumed to be 1.} 
\label{tab:car_results}
\begingroup\footnotesize
\begin{tabular}{rr}
  \toprule
Rebound term & Value [\%] \\ 
  \midrule
$Re_{dempl}$ & 0.0 \\ 
  $Re_{emb}$ & 1.7 \\ 
  $Re_{O\!M\!d}$ & $-$3.4 \\ 
  $Re_{dsub}$ & 10.7 \\ 
  $Re_{isub}$ & $-$0.9 \\ 
  $Re_{dinc}$ & 6.8 \\ 
  $Re_{iinc}$ & 10.2 \\ 
  $Re_{macro}$ & 10.4 \\ 
   \midrule
$Re_{tot}$ & 35.4 \\ 
   \bottomrule
\end{tabular}
\endgroup
\end{table}










\begin{knitrout}
\definecolor{shadecolor}{rgb}{0.969, 0.969, 0.969}\color{fgcolor}\begin{figure}

{\centering \includegraphics[width=\maxwidth]{figure/CarEnergyGraph-1} 

}

\caption{The energy plane for the car example. Macro factor ($k$) is assumed to be 1. See Table~\ref{tab:path_graph_segments} for meanings of path segments.}\label{fig:CarEnergyGraph}
\end{figure}

\end{knitrout}


\begin{knitrout}
\definecolor{shadecolor}{rgb}{0.969, 0.969, 0.969}\color{fgcolor}\begin{figure}

{\centering \includegraphics[width=\maxwidth]{figure/CarCostGraph-1} 

}

\caption{The expenditure plane for the car example. CV is compensating variation, the increase in consumption of the energy service and decrease in consumption of other goods and services to maintain constant utility. See Table~\ref{tab:path_graph_segments} for meanings of path segments.}\label{fig:CarCostGraph}
\end{figure}

\end{knitrout}


\begin{knitrout}
\definecolor{shadecolor}{rgb}{0.969, 0.969, 0.969}\color{fgcolor}\begin{figure}

{\centering \includegraphics[width=\maxwidth]{figure/CarConsGraph-1} 

}

\caption{The consumption plane for the car example. See Table~\ref{tab:path_graph_segments} for meanings of path segments.}\label{fig:CarConsGraph}
\end{figure}

\end{knitrout}


The \empleffect{} has three components:
the direct emplacement effect,
the embodied energy effect, and
the combined operations, maintenance, and disposal effects.
Rebound from the direct emplacement effect
($Re_{dempl}$) is 0.0\% always,
because energy takeback (and, therefore, rebound)
occurs after the upgraded device is emplaced.
Indirect rebound due to the embodied energy effect
($Re_{emb}$) is 1.7\%,
due to the higher embodied energy rate
($\Delta \raempl{E}_{emb} = 429$~MJ/yr)
stemming from the electric battery in the hybrid EV car.
Rebound due to the operations, maintenance, and disposal effects
($Re_{\omd}$) is small and negative
($-3.4$\%),
because of the slightly lower operations, maintenance, and disposal costs
for the hybrid EV car.

The \subeffect{} has two components:
direct and indirect substitution effect rebound.
Rebound from direct substitution ($Re_{dsub}$) is
positive, as expected (10.7\%).
The car user will, on average, prefer more driving purely
from the change in relative prices because of the fuel economy enhancements
(42~mpg $>$ 25~mpg).
In other words,
due the relative price change,
the more fuel-efficient car is driven 7.3\% further
each year.
Conversely, the indirect substitution effect ($Re_{isub}$)
is slightly negative ($-0.9$\%)
to achieve the same level of utility after increased driving.
Indeed, across the substitution effect,
less money is spent on other goods
($\Delta \rasub{C}_o = -74$~\$/yr).
In Appendix~\ref{sec:price_elasticities_sensitivity} we show how the
displacement along an indifference curve alters the price elasticities,
and in particular, that the uncompensated own price elasticity declines
in magnitude. The decline slows the rate of additional consumption of
energy-intensive driving, and attenuates the microeconomic rebound
relative to assuming constant price elasticities.

The \inceffect{} also has two components:
direct and indirect income effect rebound.
The direct income effect
($Re_{dinc}$) is positive
(6.8\%),
because the car user allocates some
net savings to additional driving.
Rebound from the indirect income effect
($Re_{iinc}$) is positive (10.2\%)
due to higher spending on other goods.
Thus, the net savings after the substitution effect
($\rasub{N} = 835$~\$/yr)
translates into positive direct and indirect income
rebound at the microeconomic level.
Total microeconomic rebound (emplacement, substitution, and income effects)
sums to $Re_{micro} = 25.0$\%.

Finally, in Part~I we noted that
the link between macroeconomic and microeconomic rebound
is largely unexplored,
so we assume a value of $k = 1$ for both examples, initially.
We return to the value for $k$
in the Discussion (Section~\ref{sec:calculating_k}).
With $k$ assumed to be 1,
the \macroeffect{} leads to macroeconomic rebound
($Re_{\macro}$) of 10.4\%
for the car example,
due to economic expansion caused by
productivity enhancements arising from the more-efficient provision of the
energy service (transportation).


%++++++++++++++++++++++++++++++
\subsection{Example 2: Purchase of a new electric lamp}
\label{sec:lamp_example}
%++++++++++++++++++++++++++++++

With the data in
Tables~\ref{tab:lamp_parameters}--\ref{tab:lamp_elasticity_parameters}
and the equations in Section~\ref{PartI-sec:framework} of Part~I in hand,
we calculate important values at each rebound stage,
as shown in Table~\ref{tab:lamp_stages_table}.
Similar to Table~\ref{tab:car_stages_table},
Table~\ref{tab:lamp_stages_table} applies to the lamp user,
so no changes are recorded across the macro effect, and
the $-$ (bar) and $\sim$ (tilde) columns
of Table~\ref{tab:lamp_stages_table} are identical.
Rebound components for the lamp upgrade are shown in Table~\ref{tab:lamp_results}.
Figs.~\ref{fig:LampEnergyGraph}--\ref{fig:LampConsGraph}
show energy, expenditure, and consumption planes
for the lamp example.


% % latex table generated in R 4.4.1 by xtable 1.8-4 package
% % Wed Sep 25 14:57:48 2024
% \begin{table}[ht]
% \centering
% \caption{Results for lamp example with macro factor ($k$) assumed to be 1. There is no change for the consumer across across the macro effect, so the last stage ($\sim$) is not shown.} 
% \label{tab:lamp_stages_table}
% \begingroup\footnotesize
% \begin{tabular}{rrrrr}
%   \toprule
%   & Original ($\circ$) & After empl ($*$) & After sub ($\wedge$) & After inc ($-$) \\ 
%   \midrule
% $t_{li\!f\!e}$ [yr] & 2 & 10 & 10 & 10 \\ 
%   $R_\alpha$ [--] & 1.012 & 1.138 & 1.138 & 1.138 \\ 
%   $R_\omega$ [--] & 0.959 & 0.847 & 0.847 & 0.847 \\ 
%   $\eta$ [\lmhr/\kWhr] & 8,833 & 81,800 & 81,800 & 81,800 \\ 
%   $\eta$ [\lmhr/MJ] & 2,454 & 22,722 & 22,722 & 22,722 \\ 
%   $p_s$ [\$/\lmhr] & 0.00001457 & 0.00000157 & 0.00000157 & 0.00000157 \\ 
%   $\dot{q}_s$ [\lmhr/yr] & 580,350 & 580,350 & 1,412,867 & 1,413,439 \\ 
%   $p_E$ [\$/MJ] & 0.0358 & 0.0358 & 0.0358 & 0.0358 \\ 
%   $\dot{E}_s$ [MJ/yr] & 236.5 & 25.5 & 62.2 & 62.2 \\ 
%   $\dot{E}_{emb}$ [MJ/yr] & 1.222 & 0.650 & 0.650 & 0.650 \\ 
%   $\dot{C}_s$ [\$/yr] & 8.46 & 0.91 & 2.22 & 2.22 \\ 
%   $\dot{C}_{cap}$ [\$/yr] & 1.04 & 0.12 & 0.12 & 0.12 \\ 
%   $R_{\alpha}\dot{C}_{cap}$ [\$/yr] & 1.06 & 0.14 & 0.14 & 0.14 \\ 
%   $\dot{C}_{O\!M}$ [\$/yr] & 0.00 & 0.00 & 0.00 & 0.00 \\ 
%   $C_d$ [\$] & 0.00 & 0.00 & 0.00 & 0.00 \\ 
%   $\dot{C}_d$ [\$/yr] & 0.00 & 0.00 & 0.00 & 0.00 \\ 
%   $R_{\omega}\dot{C}_d$ [\$/yr] & 0.00 & 0.00 & 0.00 & 0.00 \\ 
%   $\dot{C}_{O\!M\!d}$ [\$/yr] & 0.00 & 0.00 & 0.00 & 0.00 \\ 
%   $\dot{C}_o$ [\$/yr] & 27,920 & 27,920 & 27,916 & 27,927 \\ 
%   $\dot{N}$ [\$/yr] & 0.00 & 8.46 & 11.30 & 0.00 \\ 
%   $\dot{M}$ [\$/yr] & 27,930 & 27,930 & 27,930 & 27,930 \\ 
%    \bottomrule
% \end{tabular}
% \endgroup
% \end{table}



% latex table generated in R 4.4.1 by xtable 1.8-4 package
% Mon Oct 21 11:38:06 2024
\begin{table}[ht]
\centering
\caption{Results for lamp example with macro factor ($k$) assumed to be 1. There is no change for the consumer across across the macro effect, so the last stage ($\sim$) is not shown.  Blanks indicate unchanged values relative to previous or later values in the same row.} 
\label{tab:lamp_stages_table}
\begingroup\footnotesize
\begin{tabular}{rrrrrr}
  \toprule
   &   & Original ($\circ$) & After empl ($*$) & After sub ($\wedge$) & After inc ($-$) \\ 
  \midrule
   & $\dot{M}$ [\$/yr] & 27,930 & 27,930 & 27,930 & 27,930 \\ 
   & $p_E$ [\$/MJ] & 0.0358 & 0.0358 & 0.0358  & 0.0358  \\ 
   & $t_{li\!f\!e}$ [yr] & 2 & 10 & 10 & 10  \\ 
   & ${\tau}_\alpha$ [--] & 1.012 & 1.138 & 1.138 & 1.138  \\ 
   & ${\tau}_\omega$ [--] & 0.959 & 0.847 & 0.847 & 0.847 \\ 
   & $\eta$ [\lmhr/\kWhr] & 8,833 & 81,800 & 81,800  & 81,800  \\ 
   & $\eta$ [\lmhr/MJ] & 2,454 & 22,722 & 22,722  & 22,722 \\ 
   & $p_s$ [\$/\lmhr] & 0.00001457 & 0.00000157 & 0.00000157  & 0.00000157  \\ 
   & $\dot{E}_{emb}$ [MJ/yr] & 1.222 & 0.650 & 0.650  & 0.650 \\ 
   & $\dot{C}_{cap}$ [\$/yr] & 1.04 & 0.12 & 0.12  & 0.12  \\ 
   & ${\tau}_{\alpha}\dot{C}_{cap}$ [\$/yr] & 1.06 & 0.14 & 0.14  & 0.14 \\ 
   & $\dot{C}_{O\!M}$ [\$/yr] & 0.00 & 0.00 & 0.00  & 0.00 \\ 
   & $C_d$ [\$] & 0.00 & 0.00 & 0.00  & 0.00 \\ 
   & $\dot{C}_d$ [\$/yr] & 0.00 & 0.00 & 0.00  & 0.00  \\ 
   & ${\tau}_{\omega}\dot{C}_d$ [\$/yr] & 0.00 & 0.00 & 0.00  & 0.00 \\ 
   & $\dot{C}_{O\!M\!d}$ [\$/yr] & 0.00 & 0.00 & 0.00  & 0.00  \\ 
   & $\dot{E}_s$ [MJ/yr] & 236.5 & 25.5 & 62.2 & 62.2 \\ 
   & $\dot{C}_s$ [\$/yr] & 8.46 & 0.91 & 2.22 & 2.22 \\ 
   & $\dot{N}$ [\$/yr] & 0.00 & 8.46 & 11.30 & 0.00 \\ 
   & $\dot{q}_s$ [\lmhr/yr] & 580,350 & 580,350 & 1,412,867 & 1,413,439 \\ 
   & $\dot{C}_g$ [\$/yr] & 27,920  & 27,920 & 27,916 & 27,927 \\ 
   \bottomrule
\end{tabular}
\endgroup
\end{table}



% latex table generated in R 4.4.1 by xtable 1.8-4 package
% Wed Sep 25 14:57:48 2024
\begin{table}[ht]
\centering
\caption{Lamp example: rebound results with macro factor ($k$) assumed to be 1.} 
\label{tab:lamp_results}
\begingroup\footnotesize
\begin{tabular}{rr}
  \toprule
Rebound term & Value [\%] \\ 
  \midrule
$Re_{dempl}$ & 0.0 \\ 
  $Re_{emb}$ & $-$0.3 \\ 
  $Re_{O\!M\!d}$ & 0.0 \\ 
  $Re_{dsub}$ & 17.4 \\ 
  $Re_{isub}$ & $-$6.4 \\ 
  $Re_{dinc}$ & 0.0 \\ 
  $Re_{iinc}$ & 17.3 \\ 
  $Re_{macro}$ & 13.0 \\ 
   \midrule
$Re_{tot}$ & 41.1 \\ 
   \bottomrule
\end{tabular}
\endgroup
\end{table}






The \empleffect{} rebound components start with
the direct emplacement effect ($Re_{dempl}$),
which is always $0.0$\%.
Indirect rebound due to the embodied energy effect
($Re_{emb}$) is $-0.3$\%.
Although the LED lamp has higher embodied energy
($\aempl{E}_{emb} = 6.50$~MJ)
than the incandescent lamp
($\bempl{E}_{emb} = 2.20$~MJ),
the LED lamp has a much longer lifetime,
meaning that the LED embodied energy \emph{rate}
($\raempl{E}_{emb} = 0.65$~MJ/yr)
is less than the incandescent embodied energy rate
($\rbempl{E}_{emb} = 1.22$~MJ/yr).
Thus, the change in embodied energy rate ($\Delta \raempl{E}_{emb}$)
is $-0.57$~MJ/yr,
and embodied energy rebound is negative
($Re_{emb} = -0.3$\%).
Rebound due to the combined operations, maintenance, and disposal effects
($Re_{\omd}$) is 0.0\%,
because we assume no difference in operations, maintenance, or
disposal costs between
the incandescent lamp and the LED lamp.%
\footnote{
  Maintenance cost rates for both incandescent and LED lamps are likely to be equal
  and negligible;
  lamps are usually installed and forgotten.
  Real-world disposal cost differences between the incandescent and LED technologies
  are also likely to be negligible.
  However, if ``disposal'' includes recycling processes,
  cost rates may be different between the two technologies
  due to the wide variety of materials in LED lamps compared to incandescent lamps.
}

Direct \subeffect{} rebound
($Re_{dsub}$) is 17.4\%
due to the much higher LED lamp efficiency
($\amacro{\eta} = 81.8$~lm/W)
compared to the incandescent lamp
($\bempl{\eta} = 8.83$~lm/W),
leading to increased demand for lighting
(from $\rbsub{q}_s = 580,350$~\lmhr/yr
to $\rasub{q}_s = 1,412,867$~\lmhr/yr)
as shown by segment \chat{} in Fig.~\ref{fig:LampConsGraph}.
To maintain constant utility,
consumption of other goods is reduced
($\Delta \rasub{C}_o = -4.15$~\$/yr),
as shown by segment \starc{} in Fig.~\ref{fig:LampConsGraph},
yielding indirect substitution effect rebound
($Re_{isub}$) of $-6.4$\%.

\Inceffect{} rebound arises from spending net energy cost savings
associated with converting from the incandescent lamp to the LED lamp
($\rbinc{N} = 11.30$~\$/yr).
Direct income effect rebound
($Re_{dinc}$) is 0.01\%,
positive but small,
as the lamp user allocates
some of the net savings to additional demand for lighting.
The indirect income effect rebound is large
($Re_{iinc} = 17.3$\%),
due to the energy implications of increased spending on other goods.
Total microeconomic level rebound (emplacement, substitution, and income effects)
sums to $Re_{micro} = 28.1$\%.

Finally, \macroeffect{} rebound
($Re_{\macro}$) is 13.0\%
with $k$ assumed to be 1,
due to economic expansion caused by
productivity enhancements arising from the more-efficient provision of the
energy service (lighting).


\begin{knitrout}
\definecolor{shadecolor}{rgb}{0.969, 0.969, 0.969}\color{fgcolor}\begin{figure}

{\centering \includegraphics[width=\maxwidth]{figure/LampEnergyGraph-1} 

}

\caption{The energy plane for the lamp example. Macro factor ($k$) is assumed to be 1. See Table~\ref{tab:path_graph_segments} for meanings of path segments.}\label{fig:LampEnergyGraph}
\end{figure}

\end{knitrout}


\begin{knitrout}
\definecolor{shadecolor}{rgb}{0.969, 0.969, 0.969}\color{fgcolor}\begin{figure}

{\centering \includegraphics[width=\maxwidth]{figure/LampCostGraph-1} 

}

\caption{Expenditure plane for the lamp example. CV is compensating variation, the increase in consumption of the energy service and decrease in consumption of other goods and services to maintain constant utility. See Table~\ref{tab:path_graph_segments} for meanings of path segments.}\label{fig:LampCostGraph}
\end{figure}

\end{knitrout}


\begin{knitrout}
\definecolor{shadecolor}{rgb}{0.969, 0.969, 0.969}\color{fgcolor}\begin{figure}

{\centering \includegraphics[width=\maxwidth]{figure/LampConsGraph-1} 

}

\caption{Consumption plane for the lamp example. See Table~\ref{tab:path_graph_segments} for meanings of path segments.}\label{fig:LampConsGraph}
\end{figure}

\end{knitrout}


%%%%%%%%%%%%%%%%%%%%%%%%%%%%%%%%%%%%%%%%%%%%%%%%%%%%%%%%%%%%%%
\section{Discussion}
\label{sec:discussion}
%%%%%%%%%%%%%%%%%%%%%%%%%%%%%%%%%%%%%%%%%%%%%%%%%%%%%%%%%%%%%%


%++++++++++++++++++++++++++++++
\subsection{A first attempt at calculating macro rebound}
\label{sec:calculating_k}
%++++++++++++++++++++++++++++++



Few previous studies explored the link
between microeconomic and macroeconomic rebound.
Inspired by \citet{Borenstein:2015aa} and others,
the framework developed in Section~\ref{PartI-sec:framework} of Part~I
links macroeconomic rebound to microeconomic rebound
via the macro factor~($k$) that scales
magnitudes in the microeconomic portion of the framework.
(See Section~\ref{PartI-sec:macro_effect_main_paper} of Part~I.)

For the results presented in Section~\ref{sec:results} above,
we assumed a placeholder value of $k = 1$,
meaning that every \$1 of spending by the device user in the income effect
generates only \$1 of additional economic activity in the broader economy.
There are no estimates of $k$, which ultimately traces the aggregate growth
effects of a single, device-specific technical enhancement and is likely
to differ between EEUs. However, using recent empirical estimates of
sectoral multipliers we can ascertain ourselves that $k$ should be different
from 1 and choose a different value in line with those estimates.

Sectoral multipliers capture the impact of sectoral revenue increases
into aggregate demand or GDP growth.
While the idea of scale economies
from larger markets for particular products have a long
history in economic thought dating back at least to \citet{Smith1776},
data from input-output tables and recent advances in network theory
allowed formalization of the spill-overs from sectoral to aggregate growth.
First results show that U.S. aggregate output growth
may have been up to 3 times as large as the growth of the sector in which
growth originated \citep{Foerster2022}.
And industrial policy to
encourage technology adoption in certain sectors was found to pay back
up to 5 times its cost in India, but with wide variation
across sectors \citep{Buera2024}.
Since our problem also concerns technology adoption,
one that features energy augmenting technical change,
we take the value from the \citet{Buera2024} study,
where the majority of multipliers cluster around 3.
Thus, we adopt the value of $k = 3$,
fully aware that this can only be a first approximation.

After setting $k = 3$,
we can recalculate all rebound components in our framework.
Emplacement ($Re_{empl}$), substitution ($Re_{sub}$), and income ($Re_{inc}$) rebound
magnitudes are unchanged after setting $k = 3$.
However, we see that choosing a placeholder value of $k = 1$
resulted in a low value for $Re_{\macro}$ and, therefore, $Re_{tot}$
in Section~\ref{sec:results}.
In Figs.~\ref{fig:CarEnergyGraph} and~\ref{fig:LampEnergyGraph},
the macro effect segments (\bartilde{})
should be three times longer than they appear.
In Tables~\ref{tab:car_results} and~\ref{tab:lamp_results},
the values of macro rebound ($Re_{\macro}$) should triple to
31.2\% and
39.0\%, and
the values of total rebound ($Re_{tot}$) should increase to
56.2\% and
67.0\%
for the car and lamp examples,
respectively.
For the remainder of this paper,
we use $k = 3$.


%++++++++++++++++++++++++++++++
\subsection{Comparison between the car and lamp case studies}
\label{sec:case_study_comparison}
%++++++++++++++++++++++++++++++



Tables~\ref{tab:car_results} and~\ref{tab:lamp_results}
and selection of $k = 3$ in Section~\ref{sec:calculating_k}
enable fuller comparisons between the car and lamp examples.
Several points can be made.

First,
the magnitude of every rebound effect is different between the two examples,
the exception being direct emplacement rebound ($Re_{dempl}$)
which is always 0.0\% by definition.
The implication is that every EEU needs to be analyzed separately.
The magnitudes of the rebound effects
for one EEU should never be assumed to apply to a different EEU.

Second,
one cannot know \emph{a-priori} which rebound effects
will be large and which will be small
for a given EEU.
Furthermore, some rebound effects are dependent upon economic parameters,
such as energy intensity ($I_E$).
Thus, it is important to calculate the magnitude of all rebound effects
for each EEU in each economy.

Third,
the two examples illustrate the fact that
embodied energy rebound ($Re_{emb}$) can be positive or negative,
as discussed in Section~\ref{PartI-sec:empl_effect_main_paper} of Part~I.
The car's embodied energy rebound is positive ($1.7$\%)
because of the high embodied energy of the hybrid's battery
relative to the internal combustion engine vehicle.
Although the LED lamp's embodied energy is larger than the
incandescent lamp's embodied energy,
the LED lamp's embodied energy rebound is
small but negative ($-0.3$\%),
due to the longer life of the LED lamp compared to the incandescent lamp.
Thus, each EEU should be analyzed independently for its embodied energy rebound.

Fourth,
macro effect rebound is different between the two examples,
owing to differences in net income ($\raempl{N}$) relative to expected savings ($\Sdot$).
(For the car, $Re_{macro}$ is $31.2$\%.
For the lamp, $Re_{macro}$ is $39.0$\%.)
The efficiency gain for the lamp is far greater than the efficiency gain for the car,
leading to much different rates of net income ($\raempl{N}$) and
different macro rebound values.


%++++++++++++++++++++++++++++++
\subsection{Comparison to previous rebound estimates}
\label{sec:comparison_to_other_rebound_estimates}
%++++++++++++++++++++++++++++++

Tables~\ref{tab:rebound_car_comparisons} and~\ref{tab:rebound_lamp_comparisons}
compare car and lamp results (with $k = 3$) to
results from previous studies.
% For punctuation of "rather" here, see https://linguaholic.com/linguablog/comma-before-rather/.
The suite of comparison studies is neither comprehensive nor
definitive of car and lamp EEUs;
rather, they are examples that show the sort of calculations and estimations
carried out in the general literature using a variety of methods.
That said, many of the studies are highly cited,
thereby carrying sufficient academic weight for our purposes.
Tables~\ref{tab:rebound_car_comparisons}
and~\ref{tab:rebound_lamp_comparisons}
and their associated references
enable two types of observations, comparing
%
\begin{enumerate*}[label={(\roman*)}]

  \item coverage of rebound components and

  \item magnitudes and associated calculation or estimation methods.

\end{enumerate*}


% The next command tells RStudio to do "Compile PDF" on HSB_results.Rnw,
% instead of this file, thereby eliminating the need to switch back to HSB_results.Rnw 
% before building the paper.
%!TEX root = ../HSB_results.Rnw


\begin{landscape}
\begin{table}
\footnotesize
\begin{center}
\caption{Rebound magnitude comparisons for the car example. All numbers in \%.
         Note that 
         $Re_{tot} = Re_{empl} + Re_{sub} + Re_{inc} + Re_{macro}$, 
         $Re_{tot} = Re_{micro} + Re_{macro}$, and 
         $Re_{tot} = Re_{dir} + Re_{indir}$.}
\label{tab:rebound_car_comparisons}
\begin{tabular}{ c l l l c c c c @{\hspace*{10mm}} c c @{\hspace*{10mm}} c }
\toprule
  &               &          &                 & \multicolumn{3}{c}{$Re_{micro}$}      & $Re_{macro}$ & $Re_{dir}$ & $Re_{indir}$ & $Re_{tot}$ \\ 
  & Rebound study & Coverage & Analysis method & $Re_{empl}$ & $Re_{sub}$ & $Re_{inc}$ &              &            &              &            \\ 
\midrule
 & This paper & U.S., & Energy, expenditure, and  & $-1.8$
                                                  & $9.8$
                                                  & $17.0$
                                                  & $31.2$
                                                  & $17.6$
                                                  & $38.6$
                                                  & $56.2$  \\
 & (2024)     & 2020  & consumption planes        & & & & & & &   \\
\midrule
1 & \citeauthor{Small:2007aa}  & U.S.,      & Elasticity of VMT w.r.t.\ & & & & & 4.5 (short run, &  &  \\
  & \citeyearpar{Small:2007aa} & 1967--2001 & fuel cost per mile        & & & & & 1967--2001)     &  &  \\
  &                            &            &                           & & & & & 22.2 (long run, &  &  \\
  &                            &            &                           & & & & & 1967--2001)     &  &  \\
  &                            &            &                           & & & & & 2.2 (short run, &  &  \\
  &                            &            &                           & & & & & 1997--2001)     &  &  \\
  &                            &            &                           & & & & & 10.7 (long run, &  &  \\
  &                            &            &                           & & & & & 1997--2001)     &  &  \\
\midrule
2 & \citeauthor{Greene2012}  & U.S.,      & Elasticities of transport   & & & & &  4 (short run) & & \\
  & \citeyearpar{Greene2012} & 1966--2007 & fuel w.r.t.\ price \&       & & & & & 16 (long run)  & & \\
  &                          &            & efficiency                  & & & & &                & & \\
\midrule
3 & \citeauthor{Koesler:2013aa}  & Germany, & Static CGE model,        & & & & & $\le$ 64 & $\le$ 16 & 56  \\
  & \citeyearpar{Koesler:2013aa} & 2009     & 10\% efficiency shock    & & & & &          &          &     \\
\midrule
4 & \citeauthor{Thomas:2013ab}  & U.S., & Expenditure/cross price   & & & & & 10 & 6 &  \\
  & \citeyearpar{Thomas:2013ab} & 2004  & elasticities of personal  & & & & &    &   &  \\
  &                             &       & transport fuels, using    & & & & &    &   &  \\
  &                             &       & household spending        & & & & &    &   &  \\
  &                             &       & survey data               & & & & &    &   &  \\
\midrule
5 & \citeauthor{Borenstein:2015aa}  & U.S., & Microeconomic &  & 13      & 11 & & & &  \\
  & \citeyearpar{Borenstein:2015aa} & 2012  & framework     &  & (6--28) &    & & & &  \\
\midrule
6 & \citeauthor{Chitnis:2015}  & UK,                       & Estimated own/cross price  & & 72 & 5 & & 55 & 23 & 86 \\
  & \citeyearpar{Chitnis:2015} & 1964--2014                & elasticities of transport  & &    &   & &    &    &    \\
  &                            &                           & fuels, uses household      & &    &   & &    &    &    \\
  &                            &                           & spending survey data       & &    &   & &    &    &    \\
\midrule
7 & \citeauthor{Gillingham:2015aa}  & Pennsylvania, & Estimation of gasoline      & & & & & 10 (short run) & &  \\
  & \citeyearpar{Gillingham:2015aa} & 2000--2010    & price elasticity of driving & & & & &                & &  \\
  &                                 &               & demand, from dataset        & & & & &                & &  \\
  &                                 &               & of 75 million vehicle       & & & & &                & &  \\
  &                                 &               & inspection records,         & & & & &                & &  \\
  &                                 &               & including odometer data     & & & & &                & &  \\
\midrule
8 & \citeauthor{Stapleton:2016}  & UK         & Elasticity of VMT w.r.t.\ & & & & & 9--36 & & \\
  & \citeyearpar{Stapleton:2016} & 1970--2011 & fuel cost/prices          & & & & &       & & \\
\midrule
9 & \citeauthor{Moshiri2017}  & Canada,    & Price elasticity of transport & & & & & 82--88 & & \\
  & \citeyearpar{Moshiri2017} & 1997--2009 & fuel, using household         & & & & &        & & \\
  &                           &            & spending survey data          & & & & &        & & \\
\midrule
10 & \citeauthor{Duarte:2018aa}  & Spain,     & Dynamic CGE model, & & & & & & & 26          \\
   & \citeyearpar{Duarte:2018aa} & 2010--2030 & efficiency shock   & & & & & & & (short run) \\
   &                             &            &                    & & & & & & & 52          \\
   &                             &            &                    & & & & & & & (long run)  \\
\bottomrule
\end{tabular}
\end{center}
\end{table}
\end{landscape}


% The next command tells RStudio to do "Compile PDF" on HSB_results.Rnw,
% instead of this file, thereby eliminating the need to switch back to HSB_results.Rnw 
% before building the paper.
%!TEX root = ../HSB_results.Rnw


\begin{landscape}
\begin{table}
\footnotesize
\begin{center}
\caption{Rebound magnitude comparisons for the lamp example. All numbers in \%.
         Note that 
         $Re_{tot} = Re_{empl} + Re_{sub} + Re_{inc} + Re_{macro}$, 
         $Re_{tot} = Re_{micro} + Re_{macro}$, and 
         $Re_{tot} = Re_{dir} + Re_{indir}$.}
\label{tab:rebound_lamp_comparisons}
\begin{tabular}{ c l l l c c c c @{\hspace*{10mm}} c c @{\hspace*{10mm}} c }
\toprule
  &               &          &                 & \multicolumn{3}{c}{$Re_{micro}$}      & $Re_{macro}$ & $Re_{dir}$ & $Re_{indir}$ & $Re_{tot}$ \\ 
  & Rebound study & Coverage & Analysis method & $Re_{empl}$ & $Re_{sub}$ & $Re_{inc}$ &              &            &              &            \\ 
\midrule
 & This paper & U.S., & Energy, expenditure, and  & $-0.3$
                                                  & $11.0$
                                                  & $17.4$
                                                  & $39.0$
                                                  & $17.4$
                                                  & $49.7$
                                                  & $67.0$  \\
 & (2024)     & 2020  & consumption planes        & & & & & & &   \\
\midrule
1 & \citeauthor{Guertin:2003aa}  & Canada, & Econometric residential  & & & & & 32--49 & &  \\
  & \citeyearpar{Guertin:2003aa} & 1993    & energy demand model      & & & & &        & &  \\
  &                              &         & based on Canadian house- & & & & &        & &  \\
  &                              &         & hold data                & & & & &        & &  \\
\midrule
2 & \citeauthor{Freire-Gonzalez:2011aa}  & Catalonia, & Input-output based      & & & & & 49 & 16 &  \\
  & \citeyearpar{Freire-Gonzalez:2011aa} & Spain,     & energy model, utilising & & & & &    &    &  \\
  &                                      & 2000--2008 & expenditure/cross price & & & & &    &    &  \\
  &                                      &            & elasticities            & & & & &    &    &  \\
\midrule
3 & \citeauthor{Thomas:2013ab}  & U.S., & Expenditure/cross price   & & & & & 10 & 10 &  \\
  & \citeyearpar{Thomas:2013ab} & 2004  & elasticities of home      & & & & &    &    &  \\
  &                             &       & electricity use, using    & & & & &    &    &  \\
  &                             &       & household spending        & & & & &    &    &  \\
  &                             &       & survey data               & & & & &    &    &  \\
\midrule
4 & \citeauthor{Schleich2014}  & Germany, & Survey of electricity & & & & & 6 & &  \\
  & \citeyearpar{Schleich2014} & 2012     & consumption in 6409   & & & & &   & &  \\
  &                            &          & German households     & & & & &   & &  \\
\midrule
5 & \citeauthor{Borenstein:2015aa}  & U.S., & Microeconomic &  & 14      & 6 & & & &  \\
  & \citeyearpar{Borenstein:2015aa} & 2012  & framework     &  & (6--37) &   & & & &  \\
\midrule
6 & \citeauthor{Chitnis:2015}  & UK,        & Estimated own/cross price  & & 14 & 35 & & 41 & 8 & 49 \\
  & \citeyearpar{Chitnis:2015} & 1964--2014 & elasticities of transport  & &    &    & &    &   &    \\
  &                            &            & fuels, uses household      & &    &    & &    &   &    \\
  &                            &            & spending survey data       & &    &    & &    &   &    \\
\midrule
7 & \citeauthor{Duarte:2018aa}  & Spain,     & Dynamic CGE model, & & & & & & & 12          \\
  & \citeyearpar{Duarte:2018aa} & 2010--2030 & efficiency shock   & & & & & & & (short run) \\
  &                             &            &                    & & & & & & & 51          \\
  &                             &            &                    & & & & & & & (long run)  \\
\midrule
8 & \citeauthor{Barkhordar:2019aa}  & Iran,      & Dynamic CGE model & & & & & 28        & & 43        \\
  & \citeyearpar{Barkhordar:2019aa} & 2018--2040 &                   & & & & & (average) & & (average) \\
\midrule
9 & \citeauthor{Chitnis:2020aa}  & UK,            & Household demand analysis & & & & & 95 & $-41$ & 54  \\
  & \citeyearpar{Chitnis:2020aa} & 1964--2015     & via Linear approximation  & & & & &    &       &     \\
  &                              &                & to the Almost Ideal       & & & & &    &       &     \\
  &                              &                & Demand System (LAIDS)     & & & & &    &       &     \\
\midrule
10 & \citeauthor{Shojaeddini:2022aa}  & U.S., & Price elasticity of lighting & & & & & 18--29 & & \\
   & \citeyearpar{Shojaeddini:2022aa} & 2009  & from cross sectional data    & & & & &        & & \\
   &                                  &       & from the 2009 Residential    & & & & &        & & \\
   &                                  &       & Energy Consumption           & & & & &        & & \\
   &                                  &       & Survey (RECS)                & & & & &        & & \\
\bottomrule
\end{tabular}
\end{center}
\end{table}
\end{landscape}

First, we see that none of the comparison studies
report all rebound effects considered in this paper.
Also, no previous studies report either emplacement rebound
($Re_{empl} = Re_{emb} + Re_{\omd}$)
or include all of direct and indirect, substitution and
income microeconomic rebound effect combinations.
In addition, none of the other studies report macro rebound ($Re_{macro}$) by itself.
In fact, only 4 and 5 of the 10 studies
in each category (car and lamp, respectively) report total rebound ($Re_{tot}$).
Therefore, by carefully including all rebound components
in the framework and
elucidating all rebound components in Part~II, we are
%
\begin{enumerate*}[label={(\roman*)}]

  \item helping to advance conceptual clarity
        in the field of energy rebound, which

  \item may enable future studies to estimate a broader range of rebound components.

\end{enumerate*}

We also observe that studies which provide total rebound
are based on a top-down calculation of overall, economy-wide rebound,
rather than the bottom-up ``sum-of-components'' approach
that we employ.
That finding is instructive.
It supports the view that
a rigorous analysis framework that sets out individual rebound components
has been missing, which
informed the objective for Part~I of this paper.
Further, the finding means that
comparisons between top-down estimations or calculations
of total, economy-wide rebound
may also be of limited value,
because the rebound effects included or excluded may not be clear,
giving an appearance of a ``black box'' calculation approach.%
\footnote{
  That said, without the top-down approaches, 
  we would have few values to compare with our total rebound ($Re_{tot}$).
}

Second, helpful insights can be gained
from comparison of rebound magnitudes and calculation methods.
Greatest alignment between our values and earlier values
appears within the direct (microeconomic) rebound ($Re_{dir}$) column
in Tables~\ref{tab:rebound_car_comparisons} and~\ref{tab:rebound_lamp_comparisons}.
Our car ($17.6$\%) and lamp ($17.4$\%) values
are in the lower half of the comparison studies
for both cases (10\% to 49\% for the car and 10\% to 55\% for the lamp).
This alignment may be due to the easier determination
of direct rebound, from either empirical data
(e.g., \citet{Small:2007aa}) or
via own price elasticities (e.g., \citet{Chitnis:2015}).%
\footnote{
  Also worthy of note is that direct (microeconomic) rebound
  of personal transport may be the most-studied subfield
  in the rebound literature and likely the only topic with enough studies
  to enable meta-reviews such as
  \citet{Sorrell:2009aa},
  \citet{Dimitropoulos:2018aa}, and
  \citet{Gillingham:2020aa}.
}


For indirect rebound ($Re_{indir}$),
there is little agreement on the magnitude of rebound effects.
Our values for
car ($38.6$\%) and
lamp ($49.7$\%) indirect rebound magnitudes
are higher than those found in the comparison studies
for either the car (6\% to 23\%) or the lamp (8\% to 16\%) cases.
The most likely cause of our larger indirect rebound values
is that we include both micro and macro rebound levels,
whereas the comparison studies
focus mainly on microeconomic rebound only
(commonly via cross price elasticities).
In other words,
comparisons of our indirect rebound values with the studies
in Tables~\ref{tab:rebound_car_comparisons} and~\ref{tab:rebound_lamp_comparisons}
may be too simple and not very meaningful,
as we (alone) include macro-level effects in indirect rebound.
If we exclude $Re_{macro}$ from $Re_{indir}$,
our indirect microeconomic rebound values become
7.5\% (car) and
10.7\% (lamp),
which fit within the ranges reported by the
car ($6$\% to $23$\%) and lamp ($-41$\% to $16$\%)
comparison studies.

For total rebound ($Re_{tot}$),
our values of
$56.2$\% (car) and
$67.0$\% (lamp)
are close to those in the comparison studies for both the car (49\% to 51\%) and
lamp (43\% to 51\%) examples.
Beyond that, comparisons (as noted earlier) are inhibited
by methodological differences between previous studies (top-down methods)
and our bottom-up approach for calculating total rebound.


%++++++++++++++++++++++++++++++
\subsection{Comparison of CES with satiated and
            constant price elasticity (CPE) utility models}
\label{sec:utility_comparison}
%++++++++++++++++++++++++++++++







In Section~\ref{PartI-sec:inc_effect_main_paper} of Part~I,
we showed income-effect rebound expressions
under the limiting condition of already-satiated
consumption of the energy service such that the
income expansion path is a vertical line
in the consumption plane of
Figs.~\ref{fig:CarConsGraph} and~\ref{fig:LampConsGraph}.
Here, we discuss the numerical impact
of the different utility models.

Table~\ref{tab:utility_model_comparison}
compares income-effect rebound under the CES utility model,
the bounding condition of satiated consumption
of the energy service, and
the constant price elasticity (CPE) utility model.%
\footnote{
  The constant price elasticity (CPE) utility model 
  in Table~\ref{tab:utility_model_comparison}
  follows \citet{Borenstein:2015aa} who holds income constant, 
  not utility, 
  when calculating the substitution effect.
  Furthermore, \citeauthor{Borenstein:2015aa}'s income effect
  assumes all post-emplacement freed cash ($\raempl{N}$)
  is spent at the energy intensity of the economy ($I_E$).
}

In the car example, income effect rebound ($Re_{inc}$)
reduces from 17.0\% to
10.6\% when moving from the
CES utility model to the bounding condition of already-satiated
consumption of the energy service.
Total rebound ($Re_{tot}$) goes from
56.2\% to
49.8\%.
On the other hand,
the lamp example shows negligible change
in total rebound ($Re_{tot}$),
moving
from 67.04\%
to 67.03\%.

The reason for the nearly unchanged value
for total rebound ($Re_{tot}$)
in the lamp example is evident in the consumption plane
of Fig.~\ref{fig:LampConsGraph}.
In the CES (homothetic) utility model
shown in Fig.~\ref{fig:LampConsGraph},
there is almost no income-effect spending
on more of the energy service.
Almost all spending of net income ($\rasub{N}$)
is on other goods.
The path between the $\wedge$ and $-$ points
is nearly vertical already.
In contrast, the path from $\wedge$ to $-$ in the car example
(Fig.~\ref{fig:CarConsGraph})
is decidedly \emph{not} vertical and a reduction in
income-effect rebound ($Re_{inc}$) is observed when
moving from the CES utility model to the bounding condition
of already satiated energy service consumption.
Reality is probably somewhere in between.

\begin{table}
\footnotesize
\centering
\caption{Comparison of substitution effects
         ($Re_{dsub}$, $Re_{isub}$, $Re_{sub}$),
         income effects,
         ($Re_{dinc}$, $Re_{iinc}$, and $Re_{inc}$), and
         total ($Re_{tot}$) rebound
         for
         the CES utility model,
         satiated consumption of the energy service, and
         the CPE utility model
         for both car and lamp examples.}
\label{tab:utility_model_comparison}
\begin{tabular}{r r r r r r r}
\toprule
                 & \multicolumn{3}{c}{Car example}
                                                & \multicolumn{3}{c}{Lamp example} \\
Rebound term     & CES & Satiated & CPE         & CES & Satiated & CPE             \\
\midrule
$Re_{dsub}$ [\%] & 10.7
                       & 10.7
                                  & 
                                                & 17.37
                                                      & 17.37
                                                                 &  \\
$Re_{isub}$ [\%] & $-0.9$
                       & $-0.9$
                                  & 
                                                & $-6.37$
                                                      & $-6.37$
                                                                 &  \\
\midrule
$Re_{sub}$ [\%]  & 9.8
                       & 9.8
                                  & 15.0
                                                & 11.00
                                                      & 11.00
                                                                 & 15.37 \\
\midrule
$Re_{dinc}$ [\%] & 6.8
                       & 0.0
                                  & 
                                                & 0.01
                                                      & 0.00
                                                                 &  \\
$Re_{iinc}$ [\%] & 10.2
                       & 10.6
                                  & 
                                                & 17.35
                                                      & 17.35
                                                                 &  \\
\midrule
$Re_{inc}$ [\%] & 17.0
                       & 10.6
                                  & 10.4
                                                & 17.36
                                                      & 17.35
                                                                 & 12.99 \\
\midrule
$Re_{tot}$ [\%] & 56.2
                       & 49.8
                                  & 
                                                & 67.04
                                                      & 67.03
                                                                 &  \\
\bottomrule
\end{tabular}
\end{table}

Calculation of substitution rebound under the 
constant price elasticity (CPE) utility model, 
which approximates the substitution and income effects 
using only the uncompensated own price elasticity 
of energy service consumption ($\eqspsUC$),
systematically overestimates substitution effect rebound
and underestimates income effect rebound.
That's by construction since income, 
not utility, 
is held constant when calculating substitution
of the energy service for other goods consumption
with the CPE model. 
And the income effect, in the CPE utility model,
allows spending on other goods only, 
which leads to a lower income rebound 
than in the satiated model,
since the absence of compensating variation 
leaves less income to spend.
Once again, 
the specific case determines 
the deviation of the sum of substitution and income rebound
from an exact (in our case, CES) model.
Table~\ref{tab:utility_model_comparison}
shows that while the sum of substitution and income effects 
is 1.4\% smaller for the 
CPE utility model relative to the CES utility model, 
they are nearly the same in the lamp example.


%++++++++++++++++++++++++++++++
\subsection{Energy price rebound}
\label{sec:price_effect_results}
%++++++++++++++++++++++++++++++

Appendix~\ref{PartI-sec:price_effect_appendix}
of Part~I, 
Section~\ref{PartI-sec:notes_price_effect}, and
Eq.~(\ref{PartI-eq:Re_p_E})
provide an extension to the framework
involving energy price rebound ($Re_{p_E}$).
This section quantifies energy price rebound
for the car and lamp examples.

To quantify energy price rebound,
data are needed for
personal consumption ($\rorig{E}$)
of the type of energy used by the device,
including energy for the upgraded device and all other devices.
For the car example,
there is typically little other household gasoline consumption
besides for cars,
so we assume $\rorig{E}$ equal to
$\rorig{E}_s/0.95$.
For the lamp example,
a median U.S.\ household consumes about
10,000 kW$\cdot$hr/yr of electricity
\citep{U.S.-Energy-Information-Agency:2023aa}.
Given that there are 2.5 persons per U.S.\ household
\citep{Statista:2024aa},
an individual consumes electricity at a rate of
$\rorig{E} = 4000$~kW$\cdot$hr/yr.

We also need data for the price elasticity
of energy supply ($\varepsilon_{\rate{Q}_E,p_E}$).
For the car case,
we take
the price elasticity of gasoline supply to be 0.29
from \citet{Coyle:2012aa}.
For the lamp case, we adopt the value of
0.33 from \citet[Table~3]{Ghoddusia:2017aa}.

\begin{knitrout}
\definecolor{shadecolor}{rgb}{0.969, 0.969, 0.969}\color{fgcolor}\begin{figure}
\includegraphics[width=\maxwidth]{figure/PriceEffectGraph-1} \caption{Energy price rebound ($Re_{p_{E}}$) as a function of the fraction of all devices replaced by higher-efficiency versions ($f_{EEU}$). Black lines represent the nominal energy price elasticity of energy supply ($\varepsilon_{\rate{Q}_E,p_E}$). Gray bands provide $\pm 0.1$ range in $\varepsilon_{\rate{Q}_E,p_E}$.}\label{fig:PriceEffectGraph}
\end{figure}

\end{knitrout}


Parameterizing on the fraction of all devices in the economy
that are upgraded ($f_{EEU}$) and
the energy price elasticity of energy supply
($\varepsilon_{\rate{Q}_E,p_E}$) yields
Figure~\ref{fig:PriceEffectGraph}.
As expected,
price-effect rebound ($Re_{p_E}$)
grows as more devices are upgraded,
i.e., as $f_{EEU}$ increases.
Furthermore, inelastic energy supply (smaller $\varepsilon_{\rate{Q}_{E,p_E}}$)
leads to higher price-effect rebound.

In these examples, the car upgrade yields
little additional freed cash beyond the
(slightly) cheaper fuel for the car,
so there is limited spending on other goods and services
and little additional indirect energy demand.
In contrast, the upgrade of the electric lamp
is much more likely to provide energy price rebound,
because electricity for the upgraded lamp is
a small fraction of total electricity consumption
by the consumer.
\emph{All} electricity purchased by the consumer becomes cheaper
when the price of electricity falls due to
widespread lamp upgrades throughout the economy,
leading to freed cash spent on other goods and services
which, themselves, demand energy at the energy intensity
of the economy.



At 100\% penetration of LED lamps ($f_{EEU} = 1$) and
at the nominal energy price elasticity of supply
($\varepsilon_{\rate{Q}_{E},p_E} = 0.33$),
energy price rebound is $Re_{p_E} =
60.9$\%.
Combined with consumer sided rebound of
67.0\%
from Section~\ref{sec:calculating_k},
the sum of consumer-sided and supply-side rebound is
127.9\%,
demonstrating that backfire could occur under conditions
of full penetration of the lamp EEU.


%%%%%%%%%%%%%%%%%%%%%%%%%%%%%%%%%%%%%%%%%%%%%%%%%%%%%%%%%%%%%%
\section{Conclusions}
\label{sec:conclusion}
%%%%%%%%%%%%%%%%%%%%%%%%%%%%%%%%%%%%%%%%%%%%%%%%%%%%%%%%%%%%%%

In this paper (Part~II), we help to advance clarity in the field of energy rebound
by
%
\begin{enumerate*}[label={(\roman*)}]

  \item developing mutually consistent and numerically precise
        visualizations of rebound effects
        in energy, expenditure, and consumption planes,

  \item operationalizing the macro factor,

  \item documenting in detail new calculations of rebound
        for car and lighting
        upgrades,

  \item showing the extensibility of our framework
        by applying it to estimate energy price rebound, and

  \item providing information about new open source software tools
        for calculating and visualizing rebound
        for any energy efficiency upgrade.

\end{enumerate*}
%
We encourage energy analysts and economists to use visualizations
like the energy, expenditure, and consumption planes
to document rebound calculations going forward.
Our hope is that additional clarity will
%
\begin{enumerate*}[label={(\roman*)}]

  \item narrow the gap between economists and energy analysts,

  \item lead to deeper interdisciplinary understanding
        of rebound phenomena, and

  \item enable energy and climate policy
        that takes full account of rebound.

\end{enumerate*}

From the application of the framework in Part~II,
we draw two important conclusions.
First, the car and lamp examples (Section~\ref{sec:results}) show that
        the framework enables
        quantification of rebound magnitudes
        at microeconomic and macroeconomic levels, including
        energy, expenditure, and consumption aspects of
        direct and indirect rebound
        for emplacement, substitution, income, and macro effects.
Second, the examples show that magnitudes of all rebound effects
        vary with the type of EEU performed.
Thus, values for rebound effects
for one EEU should never be assumed to apply to a different EEU, and
it is important to calculate the magnitude of all rebound effects
for each EEU in each economy.

Further work could be pursued in several areas.
%
\begin{enumerate*}[label={(\roman*)}]

  \item Additional empirical studies could be performed
        to calculate the magnitude of different
        rebound effects for a variety of real-life EEUs.

  \item Deeper study of macro rebound is needed,
        including improved determination
        of the value of the macro factor ($k$).

  \item The framework could be used to study the
        distribution of rebound values
        across socioeconomic and demographic groups~\citep{Carroll2017}.

  \item The rebound effects of fossil-energy taxes
        could be studied,
        especially for the web of interconnected dynamic effects
        among rebound components that are functions
        of the energy intensity of the economy ($I_E$).

  \item Sensitivities of rebound components to model parameters
        could be investigated more fully
        than in Appendix~\ref{sec:sensitivity_analyses},
        although this will be challenging work because
        many rebound parameters are covariant.
        For example, post-EEU efficiency ($\amacro{\eta}$)
        is unlikely to be independent of
        post-EEU capital cost ($\amacro{C}_{cap}$).

  \item The framework could be extended to encompass
        fuel-switching EEUs, such as the move from
        a gasoline car to an all-electric car.

  \item This framework could be embedded
        in energy-economy models to better include rebound effects
        in discussions of macro energy modeling, energy policy, and
        CO$_2$ emissions mitigation.

\end{enumerate*}


%%%%%%%%%%%%%%%%%%%%%%%%%%%%%%%%%%%%%%%%%%%%%%%%%%%%%%%%%%%%%%
\section*{Competing interests}
\label{sec:competing_interests}
%%%%%%%%%%%%%%%%%%%%%%%%%%%%%%%%%%%%%%%%%%%%%%%%%%%%%%%%%%%%%%

Declarations of interest: none.


%%%%%%%%%%%%%%%%%%%%%%%%%%%%%%%%%%%%%%%%%%%%%%%%%%%%%%%%%%%%%%
\section*{Author contributions}
\label{sec:author_contributions}
%%%%%%%%%%%%%%%%%%%%%%%%%%%%%%%%%%%%%%%%%%%%%%%%%%%%%%%%%%%%%%

Author contributions for this paper (Part~II of the two-part paper) are shown in Table~\ref{tab:credit2}.

\begin{table}[h]
\footnotesize
\begin{center}
\caption{Author contributions.}
\begin{tabular}{r c c c}
  \toprule
                              & MKH          & GS           & PEB          \\
  \midrule
  Conceptualization           & \rating{100} & \rating{100} &              \\
  Methodology                 & \rating{100} & \rating{100} & \rating{100} \\
  Software                    & \rating{100} &              & \rating{100} \\
  Validation                  & \rating{100} &              & \rating{100} \\
  Formal analysis             & \rating{100} & \rating{100} &              \\
  Investigation               & \rating{100} & \rating{100} & \rating{100} \\
  Resources                   & \rating{100} & \rating{100} & \rating{100} \\
  Data curation               &              &              & \rating{100} \\
  Writing--original draft     & \rating{100} & \rating{100} &              \\
  Writing--review \& editing  & \rating{100} & \rating{100} & \rating{100} \\
  Visualization               & \rating{100} & \rating{100} &              \\
  Supervision                 & \rating{100} &              &              \\
  Project administration      & \rating{100} &              &              \\
  Funding acquisition         &              &              & \rating{100} \\
\bottomrule
\end{tabular}
\label{tab:credit2}
\end{center}
\end{table}


%%%%%%%%%%%%%%%%%%%%%%%%%%%%%%%%%%%%%%%%%%%%%%%%%%%%%%%%%%%%%%
\section*{Data repository}
\label{sec:data_repository}
%%%%%%%%%%%%%%%%%%%%%%%%%%%%%%%%%%%%%%%%%%%%%%%%%%%%%%%%%%%%%%

% Data and example calculations in spreadsheet format are stored at the
% Research Data Leeds Repository
% (\url{https://doi.org/10.5518/1201}).

**** Data and example calculations in spreadsheet format will be stored at the
Research Data Leeds Repository if this submission is accepted.
The spreadsheet file is included with the submission of this paper. ****




%%%%%%%%%%%%%%%%%%%%%%%%%%%%%%%%%%%%%%%%%%%%%%%%%%%%%%%%%%%%%%
\section*{Acknowledgements}
\label{sec:acknowledgements}
%%%%%%%%%%%%%%%%%%%%%%%%%%%%%%%%%%%%%%%%%%%%%%%%%%%%%%%%%%%%%%

%List funding sources in this standard way to facilitate compliance to funder's requirements: Funding: This work was supported by the National Institutes of Health [grant numbers xxxx, yyyy]; the Bill \& Melinda Gates Foundation, Seattle, WA [grant number zzzz]; and the United States Institutes of Peace [grant number aaaa].

%It is not necessary to include detailed descriptions on the program or type of grants and awards. When funding is from a block grant or other resources available toa university, college, or other research institution, submit the name of the institute or organization that provided the funding.

%If no funding has been provided for the research, please include the following sentence: This research did not receive any specific grant from funding agencies  in the public, commercial, or not-for-profit sectors.

Paul Brockway’s time was funded by the UK Research and Innovation (UKRI)
Council, supported under EPSRC Fellowship award EP/R024254/1.
The authors benefited from discussions with
Daniele Girardi (University of Massachusetts at Amherst) and
Christopher Blackburn (Bureau of Economic Analysis).
The authors are grateful for comments from internal reviewers
Becky Haney and Jeremy Van Antwerp (Calvin University);
Nathan Chan (University of Massachusetts at Amherst); and
Zeke Marshall (University of Leeds).
The authors appreciate the many constructive comments
on a working paper version of this article from
Jeroen C.J.M.\ van den Bergh (Vrije Universiteit Amsterdam),
Harry Saunders (Carnegie Institution for Science), and
David Stern (Australian National University).
Finally, the authors thank the students of MKH's Fall 2019
Thermal Systems Design course (ENGR333) at Calvin University
who studied energy rebound for many energy conversion devices
using an early version of this framework.

% No need for this section label, because BibTeX already provides a "References" heading.
% \section*{References}
%
{
\scriptsize
\bibliography{HSB_refs}
}


\clearpage


\appendix
\counterwithin{figure}{section}
\counterwithin{table}{section}

% Gives a helpful "Appendices" label
\section*{Appendices}

% Labels each appendix as A, B, C, etc. instead of Appendix A, Appendix B, Appendix C, etc.
\renewcommand{\thesection}{\Alph{section}}


%%%%%%%%%%%%%%%%%%%%%%%%%%%%%%%%%%%%%%%%%%%%%%%%%%%%%%%%%%%%%%
\section{Nomenclature}
\label{sec:nomenclature}
%%%%%%%%%%%%%%%%%%%%%%%%%%%%%%%%%%%%%%%%%%%%%%%%%%%%%%%%%%%%%%

% The next command tells RStudio to do "Compile PDF" on HSB_results.Rnw,
% instead of this file, thereby eliminating the need to switch back to HSB_results.Rnw 
% before building the paper.
%!TEX root = ../HSB_results.Rnw

Presentation of the rigorous analytical framework is aided by a 
nomenclature that describes energy stages and rebound effects.
Table~\ref{tab:symbols} shows symbols and abbreviations, their meanings, and example units.
Table~\ref{tab:greek} shows Greek letters, their meanings, and example units.
Table~\ref{tab:abbreviations} shows abbreviations and acronyms.
Table~\ref{tab:decorations} shows symbol decorations and their meanings.
Table~\ref{tab:subscripts} shows subscripts and their meanings.


%%%%%%%%%%%%%%%%%%%%%%%%%%%%%%%%%%%%%%%%%%%%%%%%%%%%%%%%%%%%%%
% Symbols
%%%%%%%%%%%%%%%%%%%%%%%%%%%%%%%%%%%%%%%%%%%%%%%%%%%%%%%%%%%%%%

\begin{table}
\centering % Centered table
\caption{Symbols and abbreviations.}
\begin{tabular}{r l}
  \toprule
  Symbol & Meaning [example units] \\
  \midrule
  \multirow{2}{*}{$a$} & a point in the emplacement effect in rebound planes or \\
                       & the share parameter in the CES utility model [--] \\
  $b$  & a point in the emplacement effect in rebound planes \\
  $C$  & cost [\$] \\
  $c$  & a point in the substitution effect in rebound planes \\
  $d$  & a point in the income effect on rebound planes \\
  $E$  & final energy [MJ] \\
  $f$  & expenditure share [--] \\
  $G$  & freed cash [\$] \\
  $I$  & energy intensity of economic activity [MJ/\$] \\
  $k$  & macro factor [--] \\
  $M$  & income [\$] \\
  $N$  & net savings [\$] \\
  $p$  & price [\$] \\
  $q$  & quantity [--] \\
  $Re$ & rebound [--] \\
  $S$  & energy cost savings [\$] \\
  $t$  & energy conversion device lifetime [yr] \\
  $u$  & utility [utils] \\
  $x$  & the abscissa coordinate \\
  $y$  & the ordinate coordinate \\
  \bottomrule
\end{tabular}
\label{tab:symbols}
\end{table}


%%%%%%%%%%%%%%%%%%%%%%%%%%%%%%%%%%%%%%%%%%%%%%%%%%%%%%%%%%%%%%
% Greek
%%%%%%%%%%%%%%%%%%%%%%%%%%%%%%%%%%%%%%%%%%%%%%%%%%%%%%%%%%%%%%

\begin{table}
\centering % Centered table
\caption{Greek letters.}
\begin{tabular}{r l}
  \toprule
  Greek letter & Meaning [example units] \\
  \midrule
  $\Delta$       & difference (later quantity less earlier quantity, see Fig.~\ref{fig:flowchart}) \\
  $\varepsilon$  & price or income elasticity [--] \\
  $\eqsM$        & income ($\dot{M}$) elasticity of energy service demand ($\dot{q}_s$) [--] \\
  $\eqoM$        & income ($\dot{M}$) elasticity of other goods demand ($\dot{q}_o$) [--] \\
  $\eqspsUC$     & uncompensated energy service price ($p_s$) elasticity of energy service demand ($\dot{q}_s$) [--] \\
  $\eqopsUC$     & uncompensated energy service price ($p_s$) elasticity of other goods demand ($\dot{q}_o$) [--] \\
  $\eqspsC$      & compensated energy service price ($p_s$) elasticity of energy service demand ($\dot{q}_s$) [--] \\
  $\eqopsC$      & compensated energy service price ($p_s$) elasticity of other goods demand ($\dot{q}_o$) [--] \\
  $\eta$         & final-energy-to-service efficiency [vehicle-km/MJ] \\
  $\rho$         & exponent in the CES utility function, $\rho \equiv (\sigma - 1) / \sigma$ [--] \\
  $\sigma$       & elasticity of substitution between the energy service ($\rbempl{q}_s$) and other goods ($\rbempl{q}_o$) [--] \\
  \bottomrule
\end{tabular}
\label{tab:greek}
\end{table}


%%%%%%%%%%%%%%%%%%%%%%%%%%%%%%%%%%%%%%%%%%%%%%%%%%%%%%%%%%%%%%
% Abbreviations
%%%%%%%%%%%%%%%%%%%%%%%%%%%%%%%%%%%%%%%%%%%%%%%%%%%%%%%%%%%%%%

\begin{table}
\centering % Centered table
\caption{Abbreviations.}
\begin{tabular}{r l}
  \toprule
  Abbreviation & Meaning \\
  \midrule
  APF    & aggregate production function \\
  CES    & constant elasticity of substitution \\
  CGE    & computable general equilibrium \\
  CPE    & constant price elasticity \\
  CV     & compensating variation \\
  EEU    & energy efficiency upgrade \\
  EPSRC  & engineering and physical sciences research council \\
  EV     & electric vehicle \\
  GDP    & gross domestic product \\
  LAIDS  & linear approximation to almost ideal demand system \\
  LED    & light emitting diode \\
  MPC    & marginal propensity to consume \\
  mpg    & miles per U.S.\ gallon \\
  RECS   & residential energy consumption survey \\
  UK     & United Kingdom \\
  UKRI   & UK research and innovation \\
  U.S.\  & United States \\
  VMT    & vehicle miles traveled \\
  w.r.t. & with respect to \\
  \bottomrule
\end{tabular}
\label{tab:abbreviations}
\end{table}


%%%%%%%%%%%%%%%%%%%%%%%%%%%%%%%%%%%%%%%%%%%%%%%%%%%%%%%%%%%%%%
% Decorations
%%%%%%%%%%%%%%%%%%%%%%%%%%%%%%%%%%%%%%%%%%%%%%%%%%%%%%%%%%%%%%

\begin{table}
\centering % Centered table
\caption{Decorations.}
\begin{tabular}{r l}
  \toprule
  Decoration & Meaning [example units] \\
  \midrule
  $\orig{X}$ & $X$ originally (before the \empleffect{}) \\
  $\aempl{X}$  & $X$ after the \empleffect{} (before the \subeffect{}) \\
  $\asub{X}$ & $X$ after the \subeffect{} (before the \inceffect{}) \\
  $\ainc{X}$ & $X$ after the \inceffect{} (before the \macroeffect{}) \\
  $\amacro{X}$ & $X$ after the \macroeffect{} \\
  $\rate{X}$ & rate of $X$ [units of X/yr] \\
  $M^\prime$ & effective income [\$] \\
  \bottomrule
\end{tabular}
\label{tab:decorations}
\end{table}


%%%%%%%%%%%%%%%%%%%%%%%%%%%%%%%%%%%%%%%%%%%%%%%%%%%%%%%%%%%%%%
% Subscripts
%%%%%%%%%%%%%%%%%%%%%%%%%%%%%%%%%%%%%%%%%%%%%%%%%%%%%%%%%%%%%%

\begin{table}
\centering
\caption{Subscripts.}
\begin{tabular}{r l}
  \toprule
  Subscript & Meaning \\
  \midrule
  $0$      & quantity at an initial time \\
  $1$      & a specific point on the consumption plane \\
  $c$      & compensated \\
  $cap$    & capital costs \\
  $dev$    & device \\
  $dempl$  & direct emplacement effect \\
  $dinc$   & direct income effect \\
  $dir$    & direct effects (at the energy conversion device) \\
  $dsub$   & direct substitution effect \\
  $E$      & energy \\
  $emb$    & embodied \\
  $empl$   & emplacement effect \\
  $iempl$  & indirect emplacement effects \\
  $iinc$   & indirect income effect \\
  $inc$    & income effect \\
  $indir$  & indirect effects (beyond the energy conversion device) \\
  $isub$   & indirect substitution effect \\
  $\life$  & lifetime \\
  $\macro$ & macro effect \\
  $\md$    & maintenance and disposal \\
  $o$      & other expenditures (besides energy) by the device user \\
  $own$    & ownership duration \\
  $s$      & service stage of the energy conversion chain \\
  $sub$    & substitution effect \\
  $tot$    & sum of all rebound effects in the framework \\
  \bottomrule
\end{tabular}
\label{tab:subscripts}
\end{table}


Differences are indicated by the Greek letter $\Delta$ and always
signify subtraction of a quantity at an earlier stage of Fig.~\ref{fig:flowchart}
from the same quantity at the next later stage of Fig.~\ref{fig:flowchart}.
E.g.,
$\Delta \ainc{X} \equiv \ainc{X} - \binc{X}$, and
$\Delta \amacro{X} \equiv \amacro{X} - \bmacro{X}$.
Lack of decoration on a difference term indicates a difference that spans all stages of Fig.~\ref{fig:flowchart}.
E.g., $\Delta X \equiv \amacro{X} - \orig{X}$.
$\Delta X$ is also the sum of differences across each stage in Fig.~\ref{fig:flowchart},
as shown below.

\begin{align}
\Delta X &= \Delta \amacro{X} + \Delta \ainc{X} + \Delta \asub{X} + \Delta \aempl{X} \nonumber \\
\Delta X &= (\amacro{X} - \bmacro{X}) + (\ainc{X} - \binc{X})
            + (\asub{X} - \bsub{X}) + (\aempl{X} - \bempl{X}) \nonumber \\
\Delta X &= (\amacro{X} - \cancel{\bmacro{X}}) + (\cancel{\ainc{X}} - \cancel{\binc{X}})
            + (\cancel{\asub{X}} - \cancel{\bsub{X}}) + (\cancel{\aempl{X}} - \bempl{X}) \nonumber \\
\Delta X &= \amacro{X} - \orig{X}
\end{align}



%%%%%%%%%%%%%%%%%%%%%%%%%%%%%%%%%%%%%%%%%%%%%%%%%%%%%%%%%%%%%%
\section{Mathematical details of rebound planes}
\label{sec:graph_details}
%%%%%%%%%%%%%%%%%%%%%%%%%%%%%%%%%%%%%%%%%%%%%%%%%%%%%%%%%%%%%%

% The next command tells RStudio to do "Compile PDF" on HSB.Rnw,
% instead of this file, thereby eliminating the need to switch back to HSB.Rnw 
% before building the paper.
%!TEX root = ../HSB_results.Rnw

Rebound path graphs show the impact of direct and indirect rebound effects
in energy space, expenditure space, and consumption space.
Notional rebound path graphs can be found in 
Figs.~\ref{fig:ExampleEnergyPathGraph}--\ref{fig:ExampleConsPathGraph}.
Rebound path graphs for the car example can be found in 
Figs.~\ref{fig:CarEnergyGraph}--\ref{fig:CarConsGraph}.
Graphs for the lamp example can be found in
Figs.~\ref{fig:LampEnergyGraph}--\ref{fig:LampConsGraph}.

This appendix shows the mathematical details of rebound path graphs,
specifically derivations of equations for lines and curves 
shown in Table~\ref{tab:lines_and_curves}.
The lines and curves enable construction of numerically accurate
rebound path graphs 
as shown in Figs.~\ref{fig:CarEnergyGraph}--\ref{fig:LampConsGraph}.

\begin{table}
\centering
\caption{Lines and curves for rebound path graphs.}
\label{tab:lines_and_curves}
\begin{tabular}{rl}
\toprule
Rebound path graph           & Lines and curves                        \\ 
\midrule
\multirow{2}{*}{Energy}      & Constant total energy consumption lines \\
                             & 0\% and 100\% rebound lines             \\
\midrule
Expenditure                  & Constant expenditure lines              \\
\midrule
\multirow{3}{*}{Consumption} & Constant expenditure lines              \\
                             & Rays from origin to $\wedge$ point      \\
                             & Indifference curves                     \\
\bottomrule
\end{tabular}
\end{table}


%++++++++++++++++++++++++++++++
\subsection{Energy path graphs}
\label{sec:energy_path_graph_details}
%++++++++++++++++++++++++++++++

Energy path graphs show direct (on the $x$-axis) and indirect (on the $y$-axis)
energy consumption associated with the energy conversion device 
and the device owner.
Lines of constant total energy consumption comprise a 
scale for total rebound.
For example, the 0\% and 100\% rebound lines are constant total energy consumption
lines which pass through the original point ($\circ$) and
the post-direct-emplacement-effect point ($a$) 
on an energy path graph.

The equation of a constant total energy consumption line is derived from 

\begin{equation}
  \rate{E}_{tot} = \rate{E}_{dir} + \rate{E}_{indir}
\end{equation}
%
at any rebound stage.
Direct energy consumption is energy consumed by the energy conversion device
($\rate{E}_s$), and 
indirect energy consumption is the sum of embodied energy, 
energy associated with maintenanace and disposal, and energy associated 
with expenditures on other goods
($\rate{E}_{emb} + (\rate{C}_{\md} + \rate{C}_o) I_E$).

For the energy path graph, 
direct energy consumption is placed on the $x$-axis and 
indirect energy consumption is placed on the $y$-axis.
To derive the equation of a constant energy consumption line, 
we first rearrange to put the $y$ coordinate on the left of the equation:

\begin{equation}
  \rate{E}_{indir} = - \rate{E}_{dir} + \rate{E}_{tot} \; .
\end{equation}
%
Next, we substitute $y$ for $\rate{E}_{indir}$,
$x$ for $\rate{E}_{dir}$, and 
$\rate{E}_s + \rate{E}_{emb} + (\rate{C}_{\md} + \rate{C}_o) I_E$ for $\rate{E}_{tot}$
to obtain

\begin{equation}
  y = -x + \rate{E}_s + \rate{E}_{emb} + (\rate{C}_{\md} + \rate{C}_o) I_E \; ,
\end{equation}
%
where all of $\rate{E}_s$, $\rate{E}_{emb}$, $\rate{C}_{\md}$, and $\rate{C}_o$
apply at the same rebound stage.

The constant total energy consumption line 
that passes through the original point ($\circ$)
shows 100\% rebound:

\begin{equation}
  y = -x + \rbempl{E}_s + \rbempl{E}_{emb} + (\rbempl{C}_{\md} + \rbempl{C}_o) I_E \; .
\end{equation}

The 0\% rebound line is the constant total energy consumption line 
that accounts for expected energy savings ($\Sdot$) only:

\begin{equation}
  y = -x + (\rbempl{E}_s - \Sdot)
          + \rbempl{E}_{emb} + (\rbempl{C}_{\md} + \rbempl{C}_o) I_E \; .
\end{equation}
%
The above line passes through the $a$ point on an energy path graph.


%++++++++++++++++++++++++++++++
\subsection{Expenditure path graphs}
\label{sec:expenditure_path_graph_details}
%++++++++++++++++++++++++++++++

Expenditure path graphs show direct (on the $x$-axis) and indirect (on the $y$-axis)
expenses associated with the energy conversion device 
and the device owner.
Lines of constant expenditure are important, 
because they provide budget constraints for the device owner.

The equation of a constant total expenditure line is derived from 
the budget constraint

\begin{equation}
  \rate{C}_{tot} = \rate{C}_{dir} + \rate{C}_{indir}
\end{equation}
%
at any rebound stage.
For the expenditure path graph,
indirect expenditures are placed on the $y$-axis
and direct expenditures on energy for the energy conversion device are place on the $x$-axis.
Direct expenditure is the cost of energy consumed by the energy conversion device
($\rate{C}_s = p_E \rate{E}_s$), and 
indirect expenses are the sum of capital costs, 
maintenanace and disposal costs, and 
expenditures on other goods
($\rate{C}_{cap} + \rate{C}_{\md} + \rate{C}_o$).
Rearranging to put the $y$-axis variable on the left side of the equation gives

\begin{equation}
  \rate{C}_{indir} = - \rate{C}_{dir} + \rate{C}_{tot} \; .
\end{equation}

Substituting $y$ for $\rate{C}_{indir}$, 
$x$ for $\rate{C}_{dir}$, and 
$\rate{C}_s + \rate{C}_{cap} + \rate{C}_{\md} + \rate{C}_o$ for $\rate{C}_{tot}$
gives

\begin{equation}
  y = -x + \rate{C}_s + \rate{C}_{cap} + \rate{C}_{\md} + \rate{C}_o \; ,
\end{equation}
%
where all of $\rate{C}_s$, $\rate{C}_{cap}$, $\rate{C}_{\md}$, and $\rate{C}_o$
apply at the same rebound stage.

The constant total expenditure line 
that passes through the original point ($\circ$)
shows the budget constraint for the device owner:

\begin{equation}
  y = -x + \rbempl{C}_s + \rbempl{C}_{cap} + \rbempl{C}_{\md} + \rbempl{C}_o \; ,
\end{equation}
%
into which Eq.~(\ref{PartI-eq:M_acct_orig}) of Part~I can be substituted with 
$\rorig{C}_s = p_E \rorig{E}_s$ and 
$\rorig{N} = 0$ to obtain

\begin{equation}
  y = -x + \rbempl{M} \; .
\end{equation}

The constant total expenditure line 
that accounts for expected energy savings ($\Sdot$) 
and freed cash ($\rate{G} = p_E \Sdot$) only 
is given by:

\begin{equation}
  y = -x + (\rbempl{C}_s - \rate{G}) + \rbempl{C}_{cap} + \rbempl{C}_{\md} + \rbempl{C}_o \; ,
\end{equation}
%
or

\begin{equation}
  y = -x + \rbempl{M} - \rate{G}\; .
\end{equation}
%
The line given by the above equation
passes through the $a$ point on an expenditure path graph.


%++++++++++++++++++++++++++++++
\subsection{Consumption path graphs}
\label{sec:cons_path_graph_details}
%++++++++++++++++++++++++++++++

Consumption path graphs show expenditures in 
$\rate{C}_o/\rbempl{C}_o$ vs.\ $\rate{q}_s/\rbempl{q}_s$ space
to accord with the utility model.
(See Appendix~\ref{sec:utility_and_elasticities}.)
Consumption path graphs include 
%
\begin{enumerate*}[label={(\roman*)}]
	
  \item constant expenditure lines given prices,
  
  \item a ray from the origin through the $\wedge$ point, and 
  
  \item indifference curves.
    
\end{enumerate*}
%
Derivations for each are shown in the following subsections.


%------------------------------
\subsubsection{Constant expenditure lines} 
\label{sec:pref_graph_constant_expenditure_lines}
%------------------------------

There are four constant expenditure lines on the consumption path graphs of
Figs.~\ref{fig:ExampleConsPathGraph}, \ref{fig:CarConsGraph}, and \ref{fig:LampConsGraph}.
The constant expenditure lines pass through 
the original point (line \circcirc{}), 
the post-emplacement point (line \starstar{}), 
the post-substitution point (line \hathat{}), and 
the post-income point (line \barbar{}).
Like the expenditure path graph, 
lines of constant expenditure on a consumption path graph 
are derived from the budget constraint of the device owner
at each of the four points.

Prior to the EEU, the budget constraint is given by Eq.~(\ref{PartI-eq:M_acct_orig}) of Part~I.
Substituting $\orig{p}_s \rorig{q}_s$ for $p_E \rorig{E}_s$ and 
recognizing that there is no net savings before the EEU
($\rorig{N} = 0$) gives

\begin{equation}
  \rorig{M} = \orig{p}_s \rorig{q}_s + \rorig{C}_{cap} + \rorig{C}_{\md} + \rorig{C}_o \; .
\end{equation}

To create the line of constant expenditure on the consumption path graph, 
we allow $\rorig{q}_s$ and $\rorig{C}_o$ to vary in a compensatory manner:
when one increases, the other must decrease.  
To show that variation along the constant expenditure line, 
we remove the notation that ties $\rorig{q}_s$ and $\rorig{C}_o$
to the original point ($\circ$) to obtain

\begin{equation}
  \rbempl{M} = \bempl{p}_s \rate{q}_s + \rbempl{C}_{cap} + \rbempl{C}_{\md} + \rate{C}_o \; , 
\end{equation}
%
where all of $\rbempl{M}$, $\bempl{p}_s$, $\rbempl{C}_{cap}$, and $\rbempl{C}_{\md}$
apply at the same rebound stage, 
namely the original point ($\circ$) in this instance.

To derive the equation of the line representing the original budget constraint 
in $\rate{C}_o/\rbempl{C}_o$ vs.\ $\rate{q}_s/\rbempl{q}_s$ space
(the \circcirc{} line through the $\circ$ point
in consumption path graphs), 
we solve for $\rate{C}_o$ to obtain

\begin{equation}
  \rate{C}_o = - \bempl{p}_s \rate{q}_s + \rbempl{M} - \rbempl{C}_{cap} - \rbempl{C}_{\md} \; .
\end{equation}
%
Multiplying judiciously by $\rbempl{C}_o/\rbempl{C}_o$ and $\rbempl{q}_s/\rbempl{q}_s$ gives

\begin{equation}
  \frac{\rate{C}_o}{\rbempl{C}_o} \rbempl{C}_o
       = - \bempl{p}_s \frac{\rate{q}_s}{\rbempl{q}_s} \rbempl{q}_s 
         + \rbempl{M} - \rbempl{C}_{cap} - \rbempl{C}_{\md} \; .
\end{equation}
%
Dividing both sides by $\rbempl{C}_o$ yields

\begin{equation}
  \frac{\rate{C}_o}{\rbempl{C}_o}
       = - \frac{\bempl{p}_s \rbempl{q}_s}{\rbempl{C}_o} \frac{\rate{q}_s}{\rbempl{q}_s}
         + \frac{1}{\rbempl{C}_o} (\rbempl{M} - \rbempl{C}_{cap} - \rbempl{C}_{\md}) \; .
\end{equation}
%
Noting that  
$\rate{q}_s/\rbempl{q}_s$ and 
$\rate{C}_o/\rbempl{C}_o$ are
the $x$-axis and $y$-axis, respectively,
on a consumption path graph gives

\begin{equation} \label{eq:orig_line_prefs}
  y = - \frac{\bempl{p}_s \rbempl{q}_s}{\rbempl{C}_o} x
         + \frac{1}{\rbempl{C}_o} (\rbempl{M} - \rbempl{C}_{cap} - \rbempl{C}_{\md}) \; .
\end{equation}

A similar procedure can be employed to derive the equation of the
\starstar{} line through the $*$ point
after the emplacement effect.
The starting point is the budget constraint at the $*$ point
(Eq.~(\ref{PartI-eq:M_acct_aemp}) of Part~I)
with $\rorig{M}$ replacing $\raempl{M}$, 
$\amacro{p}_s \rate{q}_s$ replacing $p_E \raempl{E}_s$, and
$\rate{C}_o$ replacing $\raempl{C}_o$.

\begin{equation}
  \rbempl{M} = \amacro{p}_s \rate{q}_s + \raempl{C}_{cap} + \raempl{C}_{\md} + \rate{C}_o + \raempl{N} \; .
\end{equation}
%
Substituting Eq.~(\ref{PartI-eq:N_dot_star_empl}) of Part~I for $\raempl{N}$,
substituting Eq.~(\ref{PartI-eq:G_dot}) of Part~I to obtain $\rate{G}$,
multiplying judiciously by $\rbempl{C}_o/\rbempl{C}_o$ and $\rbempl{q}_s/\rbempl{q}_s$, 
rearranging, and noting that 
$\rate{q}_s/\rbempl{q}_s$ is the $x$-axis and 
$\rate{C}_o/\rbempl{C}_o$ is the $y$-axis gives

\begin{equation} \label{eq:star_line_prefs}
  y = - \frac{\amacro{p}_s \rbempl{q}_s}{\rbempl{C}_o} x
         + \frac{1}{\rbempl{C}_o} (\rbempl{M} - \rbempl{C}_{cap} - \rbempl{C}_{\md} - \rate{G}) \; .
\end{equation}
%
Note that the slope of Eq.~(\ref{eq:star_line_prefs}) is less negative
than the slope of Eq.~(\ref{eq:orig_line_prefs}), 
because $\amacro{p}_s < \bempl{p}_s$.
The $y$-intercept of Eq.~(\ref{eq:star_line_prefs}) is less than the 
$y$-intercept of Eq.~(\ref{eq:orig_line_prefs}),
reflecting freed cash.
Both effects are seen in
consumption path graphs 
(Figs.~\ref{fig:ExampleConsPathGraph}, \ref{fig:CarConsGraph}, and \ref{fig:LampConsGraph}).
The \circcirc{} and \starstar{} lines intersect at the coincident $\circ$ and $*$ points.

A similar derivation process can be used to find the equation of 
the line representing the budget constraint
after the substitution effect (the \hathat{} line through the $\wedge$ point).
The starting point is Eq.~(\ref{PartI-eq:M_acct_asub}) of Part~I, and 
the equation for the constant expenditure line is

\begin{equation} \label{eq:hat_line_prefs}
  y = - \frac{\amacro{p}_s \rbempl{q}_s}{\rbempl{C}_o} x
         + \frac{1}{\rbempl{C}_o} (\rbempl{M} - \rbempl{C}_{cap} - \rbempl{C}_{\md} 
                                   - \rate{G} + \amacro{p}_s \Delta \rasub{q}_s + \Delta \rasub{C}_o) \; .
\end{equation}

Note that the \hathat{} line (Eq.~(\ref{eq:hat_line_prefs})) has the same slope as 
the \starstar{} line (Eq.~(\ref{eq:star_line_prefs}))
but a lower $y$-intercept.

Finally, the corresponding derivation
for the equation of the constant expenditure line through the 
$-$ point (line \barbar{}) starts with Eq.~(\ref{PartI-eq:M_acct_ainc}) of Part~I and ends with 

\begin{equation} \label{eq:bar_line_prefs}
  y = - \frac{\amacro{p}_s \rbempl{q}_s}{\rbempl{C}_o} x
        + \frac{1}{\rbempl{C}_o} (\rbempl{M} - \rbempl{C}_{cap} - \rbempl{C}_{\md} 
                                   - \Delta \raempl{C}_{cap} - \Delta \raempl{C}_{\md}) \; .
\end{equation}


%------------------------------
\subsubsection{Ray from the origin to the $\wedge$ point} 
\label{sec:pref_graph_ray}
%------------------------------

On consumption path graphs, 
the ray from the origin to the $\wedge$ point 
(line \rr{})
defines the path along which the income effect
(lines \hatd{} and \dbar{})
operates.
The ray from the origin to the $\wedge$ point
has slope $(\rasub{C}_o/\rbempl{C}_o) / (\rasub{q}_s/\rbempl{q}_s)$
and a $y$-intercept of 0.
Therefore, the equation of line \rr{} is

\begin{equation} \label{eq:ray_cons}
  y = \frac{\rasub{C}_o/\rbempl{C}_o}{\rasub{q}_s/\rbempl{q}_s} \, x \; .
\end{equation}


%------------------------------
\subsubsection{Indifference curves} 
\label{sec:cons_graph_indifference_curves}
%------------------------------

On a consumption path graph, 
indifference curves represent lines of constant utility
for the energy conversion device owner.
In $\rate{C}_o/\rbempl{C}_o$ vs.\ $\rate{q}_s/\rbempl{q}_s$ space, 
any indifference curve 
is given by 
Eq.~(\ref{PartI-eq:utility_Co_form}) of Part~I
with $\fCs$ replacing the share parameter $a$, 
as shown in Appendix~\ref{PartI-sec:utility_and_elasticities} of Part~I.
Recognizing that 
$\rate{C}_o/\rbempl{C}_o$ is on the $y$-axis and 
$\rate{q}_s/\rbempl{q}_s$ is on the $x$-axis
leads to substitution of 
$y$ for $\rate{C}_o/\rbempl{C}_o$ and 
$x$ for $\rate{q}_s/\rbempl{q}_s$ to obtain

\begin{equation} \label{eq:utility_line_form}
  y = \left[ \frac{1}{1 - \fCs} \left( \frac{\rate{u}}{\rbempl{u}} \right)^\rho 
            - \frac{\fCs}{1 - \fCs} (x)^\rho \right]^{(1/\rho)} \; .
\end{equation}

At any point in 
$\rate{C}_o/\rbempl{C}_o$ vs.\ $\rate{q}_s/\rbempl{q}_s$ space,
namely ($\rate{q}_{s,1}/\rbempl{q}_s$, $\rate{C}_{o,1}/\rbempl{C}_o$),
indexed utility ($\rate{u}_1/\rbempl{u}$) is given by Eq.~(\ref{PartI-eq:ces_utility}) of Part~I as

\begin{equation} \label{eq:utility_ratio_1_point}
  \frac{\rate{u}_1}{\rbempl{u}} =
        \left[ \fCs \left( \frac{\rate{q}_{s,1}}{\rbempl{q}_s} \right)^\rho
        + (1-\fCs) \left( \frac{\rate{C}_{o,1}}{\rbempl{C}_o} \right)^\rho  \right]^{(1/\rho)} \; .
\end{equation}
%
Substituting Eq.~(\ref{eq:utility_ratio_1_point}) into Eq.~(\ref{eq:utility_line_form})
for $\rate{u}/\rorig{u}$
and simplifying exponents gives

\begin{equation}
  y = \left\{ \frac{1}{1 - \fCs} \left[ \fCs \left( \frac{\rate{q}_{s,1}}{\rbempl{q}_s} \right)^\rho 
        + (1-\fCs) \left( \frac{\rate{C}_{o,1}}{\rbempl{C}_o} \right)^\rho   \right] 
            - \frac{\fCs}{1 - \fCs} (x)^\rho \right\}^{(1/\rho)}  .
\end{equation}
%
Simplifying further yields
the equation of an indifference curve passing through point 
($\rate{q}_{s,1}/\rbempl{q}_s$, $\rate{C}_{o,1}/\rbempl{C}_o$):

\begin{equation} \label{eq:indiff_curve_eqn}
  y = \left\{ \left( \frac{\fCs}{1 - \fCs} \right) \left[ \left( \frac{\rate{q}_{s,1}}{\rbempl{q}_s} \right)^\rho 
                                                          - (x)^\rho  \right]
        + \left( \frac{\rate{C}_{o,1}}{\rbempl{C}_{o}} \right)^\rho \right\}^{(1/\rho)} \; .
\end{equation}
%
Note that if $x$ is $\rate{q}_{s,1}/\rbempl{q}_s$,
$y$ becomes $\rate{C}_{o,1}/\rbempl{C}_o$,
as expected.



%%%%%%%%%%%%%%%%%%%%%%%%%%%%%%%%%%%%%%%%%%%%%%%%%%%%%%%%%%%%%%
\section{Univariate sensitivity analyses}
\label{sec:sensitivity_analyses}
%%%%%%%%%%%%%%%%%%%%%%%%%%%%%%%%%%%%%%%%%%%%%%%%%%%%%%%%%%%%%%t


% The next command tells RStudio to do "Compile PDF" on HSB_results.Rnw,
% instead of this file, thereby eliminating the need to switch back to HSB_results.Rnw 
% before building the paper.
%!TEX root = ../HSB_results.Rnw

  
Sensitivity analyses show the effect of 
independently varied parameters on total rebound and rebound components.
In the context of this framework,
sensitivity analyses
can show important trends, tendencies, and relationships
between rebound parameters and rebound magnitudes.
Key rebound parameters include 
post-EEU efficiency ($\amacro{\eta}$),
post-EEU capital cost ($\amacro{C}_{cap}$),
energy price ($p_E$),
pre-EEU uncompensated price elasticity of energy service demand ($\eqspsUCorig$), 
the macro factor ($k$), and
post-EEU energy service price ($\amacro{p}_s$).
Univariate sensitivity analyses 
(the kind shown here)
should be interpreted carefully,
because some rebound parameters are not expected to be 
independent from others.


%++++++++++++++++++++++++++++++
\subsection{Effect of post-EEU efficiency ($\amacro{\eta}$) on rebound terms} 
\label{sec:effect_of_efficiency}
%++++++++++++++++++++++++++++++


  
Fig.~\ref{fig:eta_tilde_takeback_Sdot_sens_graph} shows that
both the energy takeback rate and expected energy savings ($\Sdot$)
increase with post-EEU efficiency ($\amacro{\eta}$), 
but the relationship is asymptotic.
Each unit increase of fuel economy or lighting efficiency is less effective than
the previous unit increase of fuel economy or lighting efficiency
for saving energy.
At very high levels of fuel economy or lighting efficiency, 
a unit increase leads to almost no additional energy savings.
Thus, we can say there are diminishing returns of fuel economy and lighting efficiency,
leading to saturation of energy savings at very high levels of fuel economy and lighting efficiency.
A simple example illustrates.
A $\bempl{\eta} = 25$ mpg car drives $\bempl{q}_s = 100$ miles 
using $\bempl{E}_s = 4$ gallons of gasoline.
A more-efficient car ($\amacro{\eta} = 30$ mpg) is expected to use
$\aempl{E}_s = 3.33$ gallons to drive the same distance,
a savings of $\Sdot = 0.67$ gallons.
Another 5 mpg boost in efficiency (to $\amacro{\eta} = 35$ mpg)
will use $\aempl{E}_s = 2.86$ gal to drive 100 miles, 
a further expected savings of only $\Sdot = 0.47$ gallons.
Each successive 5~mpg boost in fuel economy 
saves less energy than the previous 5~mpg boost in fuel economy.


\begin{knitrout}
\definecolor{shadecolor}{rgb}{0.969, 0.969, 0.969}\color{fgcolor}\begin{figure}

{\centering \includegraphics[width=\maxwidth]{figure/eta_tilde_takeback_Sdot_sens_graph-1} 

}

\caption[Expected energy savings rate ($\Sdot$, solid line) and takeback rate (dashed line) sensitivity to post-EEU efficiency ($\amacro{\eta}$)]{Expected energy savings rate ($\Sdot$, solid line) and takeback rate (dashed line) sensitivity to post-EEU efficiency ($\amacro{\eta}$). The macro factor, $k = 3$. (Note different $x$- and $y$-axis scales.)}\label{fig:eta_tilde_takeback_Sdot_sens_graph}
\end{figure}

\end{knitrout}
  
  
Saturation can be seen mathematically, too.
Taking the limit as $\amacro{\eta} \rightarrow \infty$ 
in Eq.~(\ref{PartI-eq:Sdot}) of Part~I gives $\Sdot = \rbempl{E}_s$, 
not $\infty$. 
Thus, efficiency saturation must occur.
Fig.~\ref{fig:eta_tilde_takeback_Sdot_sens_graph}
shows that this framework correctly replicates
expected efficiency saturation trends.

Saturation is especially noticeable in the lamp example
compared to the car example,
the difference being that 
the LED lamp is already much more efficient than the incandescent lamp
(9.26$\times$),
whereas the hybrid car is only 
1.68$\times$ more efficient than the conventional gasoline car. 
Thus, at $\amacro{\eta} = 81.8$ \lmhr/\Whr, 
the energy efficient LED
is far closer to efficiency saturation than the hybrid vehicle 
(at $\amacro{\eta} = 42$ mpg).
As a result, further increases in the LED lamp's efficiency 
are less effective than further increases in the hybrid car's efficiency.

That said, actual savings is the difference between the expected energy savings line (solid line)
and the takeback line (dashed line) in Fig.~\ref{fig:eta_tilde_takeback_Sdot_sens_graph}.
Because the gap between the lines grows, 
higher efficiency yields greater energy savings,
even after accounting for rebound effects.
But the actual savings are always less than expected savings, due to takeback.

Fig.~\ref{fig:eta_tilde_takeback_Sdot_sens_graph} shows that
expected energy savings ($\Sdot$) increase faster than takeback 
as $\amacro{\eta}$ increases.
Thus, total rebound ($Re_{tot}$, the ratio of 
takeback rate to expected energy savings rate in Eq.~(\ref{PartI-eq:Re_takeback}) of Part~I),
decreases as efficiency grows.
The lamp exhibits a relatively smaller rebound decline with efficiency,
because the lamp example is closer to saturation than the car example.

Fig.~\ref{fig:all_Re_terms_eta_graph} shows the variation of all rebound components
with post-EEU efficiency ($\amacro{\eta}$).
In the car and lamp examples, 
direct substitution rebound ($Re_{dsub}$) is the 
rebound component 
most sensitive to changes in post-EEU efficiency ($\amacro{\eta}$).

Note that the sensitivity analysis on post-upgrade efficiency 
($\amacro{\eta}$, Fig.~\ref{fig:all_Re_terms_eta_graph})
is the only sensitivity analysis that requires careful explication
of both the numerator and denominator of Eq.~(\ref{PartI-eq:Re_takeback}) in Part~I,
as in Fig.~\ref{fig:eta_tilde_takeback_Sdot_sens_graph}, 
because both the numerator and denominator of Eq.~(\ref{PartI-eq:Re_takeback}) in Part~I
change when post-upgrade efficiency ($\amacro{\eta}$) changes.
The denominator of Eq.~(\ref{PartI-eq:Re_takeback}) in Part~I doesn't change for
the sensitivity analyses of Figs.~\ref{fig:all_Re_terms_C_cap_graph}--\ref{fig:all_Re_terms_k_graph}.
Thus, for the remaining sensitivity analyses, 
when the rebound percentage increases (decreases), 
the energy takeback rate in the numerator of Eq.~(\ref{PartI-eq:Re_takeback}) in Part~I
increases (decreases) proportionally,
and the actual energy savings rate decreases (increases) accordingly.


\begin{knitrout}
\definecolor{shadecolor}{rgb}{0.969, 0.969, 0.969}\color{fgcolor}\begin{figure}

{\centering \includegraphics[width=\maxwidth]{figure/all_Re_terms_eta_graph-1} 

}

\caption[Sensitivity of rebound components to post-EEU efficiency ($\amacro{\eta}$)]{Sensitivity of rebound components to post-EEU efficiency ($\amacro{\eta}$). The macro factor, $k = 3$.}\label{fig:all_Re_terms_eta_graph}
\end{figure}

\end{knitrout}


%++++++++++++++++++++++++++++++
\subsection{Effect of capital cost ($\amacro{C}_{cap}$) on rebound terms} 
\label{sec:effect_of_capital_cost}
%++++++++++++++++++++++++++++++

The sensitivity of energy rebound
to capital cost ($\amacro{C}_{cap}$) is shown
in Fig.~\ref{fig:all_Re_terms_C_cap_graph}.
All other things being equal,
as capital cost of the EEU rises, 
less net savings result from the emplacement effect,
leading to smaller income, macro, and total rebound.
The same effects would be observed
with increasing maintenance and disposal cost rate ($\ramacro{C}_{\md}$).

\begin{knitrout}
\definecolor{shadecolor}{rgb}{0.969, 0.969, 0.969}\color{fgcolor}\begin{figure}

{\centering \includegraphics[width=\maxwidth]{figure/all_Re_terms_C_cap_graph-1} 

}

\caption[Sensitivity of rebound components to capital cost ($\amacro{C}_{cap}$)]{Sensitivity of rebound components to capital cost ($\amacro{C}_{cap}$). The macro factor, $k = 3$.}\label{fig:all_Re_terms_C_cap_graph}
\end{figure}

\end{knitrout}


%++++++++++++++++++++++++++++++
\subsection{Effect of energy price ($p_E$) on rebound terms} 
\label{sec:effect_of_energy_price}
%++++++++++++++++++++++++++++++




The effect of energy price on rebound is shown in Fig.~\ref{fig:all_Re_terms_p_E_graph}.
Increasing energy prices lead to larger total rebound ($Re_{tot}$),
because higher energy prices lead to more net savings ($\rasub{N}$)
to be spent by the device user.
All other things being equal, more net savings leads to 
more spending on other goods and services that demand energy.

Fig.~\ref{fig:all_Re_terms_p_E_graph} also 
shows the effect of energy price ($p_E$)
on all rebound components.
Most rebound components increase with energy price, 
with the car and lamp examples exhibiting different sensitivities. 
Substitution effects ($Re_{dsub}$ and $Re_{isub}$)
are the only rebound components that decrease with energy price ($p_E$).
Substitution effects decrease with energy price, because 
at high energy price, less behavior adjustment is needed to re-equilibrate 
after emplacement of the efficient device.



In Fig.~\ref{fig:all_Re_terms_p_E_graph}, German energy prices%
\footnote{
  For the car example,
  the gasoline price in Germany is taken as 1.42 \euro{}/liter for the average ``super gasoline'' (95 octane) 
  price in 2018~\citep{finanzen}.
  For the lamp example,
  the electricity price in Germany is taken as 0.3 \euro{}/\kWhr for the 2018 price of a household using 3.5~MWh/yr,
  an average value for German households~\citep{bundesministerium}.
  Converting currency (at 1 \euro{} = \$1.21) and
  physical units gives 
  6.5~\$/US~gallon and 
  0.363~\$/\kWhr.}
%
are shown as vertical lines,
providing an indication of possible energy price variations.
All other things being equal, 
if U.S.\ residents paid Germany's energy prices,
total energy rebound ($Re_{tot}$) would be 
$93.0$\%
for the car example and 
$148.0$\%
for the lamp example.

\begin{knitrout}
\definecolor{shadecolor}{rgb}{0.969, 0.969, 0.969}\color{fgcolor}\begin{figure}

{\centering \includegraphics[width=\maxwidth]{figure/all_Re_terms_p_E_graph-1} 

}

\caption[Sensitivity of rebound components to energy price ($p_E$)]{Sensitivity of rebound components to energy price ($p_E$). German energy prices denoted by vertical lines. The macro factor, $k = 3$.}\label{fig:all_Re_terms_p_E_graph}
\end{figure}

\end{knitrout}
  
  
%++++++++++++++++++++++++++++++
\subsection{Effect of original uncompensated own price elasticity ($\eqspsUC^\circ$) on rebound terms} 
\label{sec:effect_of_elasticity}
%++++++++++++++++++++++++++++++



\begin{knitrout}
\definecolor{shadecolor}{rgb}{0.969, 0.969, 0.969}\color{fgcolor}\begin{figure}

{\centering \includegraphics[width=\maxwidth]{figure/all_Re_terms_eps_graph-1} 

}

\caption[Sensitivity of rebound components to uncompensated own price elasticity of energy service demand ($\eqspsUCorig$)]{Sensitivity of rebound components to uncompensated own price elasticity of energy service demand ($\eqspsUCorig$). The macro factor, $k = 3$. (Note reversed $x$-axis scale.)}\label{fig:all_Re_terms_eps_graph}
\end{figure}

\end{knitrout}
  
  


Fig.~\ref{fig:all_Re_terms_eps_graph} shows the variation of total rebound ($Re_{tot}$)
with the original uncompensated price elasticity of energy service demand ($\eqspsUCorig$).
The effect is exponential, and
total rebound increases with larger negative values of $\eqspsUCorig$, as expected.
The lamp example also shows stronger exponential variation than the car example.
The main reason that total rebound values 
are different between the two examples
is the larger absolute value of original uncompensated own price elasticity ($\eqspsUCorig$) 
for the lamp ($-0.4$) compared to the car ($-0.2$).
Were the car to have the same original uncompensated own price elasticity
as the lamp (i.e., $-0.4$), 
total rebound would be closer for both examples 
($73.4$\% for the car and 
$67.0$\% for the lamp).
Fig.~\ref{fig:all_Re_terms_eps_graph} shows that direct substitution rebound 
($Re_{dsub}$) is the most sensitive rebound component to changes in $\eqspsUC^\circ$.
For the lamp example, indirect income rebound ($Re_{iinc}$) also increases
substantially with $\eqspsUC^\circ$, 
because net savings increases substantially with $\eqspsUC^\circ$. 

  
%++++++++++++++++++++++++++++++
\subsection{Effect of macro factor ($k$) on rebound terms} 
\label{sec:effect_of_macro_factor}
%++++++++++++++++++++++++++++++

The sensitivity of energy rebound 
to the macro factor ($k$) is shown 
in Fig.~\ref{fig:all_Re_terms_k_graph}.
The macro factor has a linear effect on total rebound ($Re_{tot}$)
through the macro rebound component ($Re_{\macro}$).
All other rebound components are constant when $k$ is varied independently.

\begin{knitrout}
\definecolor{shadecolor}{rgb}{0.969, 0.969, 0.969}\color{fgcolor}\begin{figure}

{\centering \includegraphics[width=\maxwidth]{figure/all_Re_terms_k_graph-1} 

}

\caption[Sensitivity of rebound components to the macro factor ($k$)]{Sensitivity of rebound components to the macro factor ($k$).}\label{fig:all_Re_terms_k_graph}
\end{figure}

\end{knitrout}
  
  
%++++++++++++++++++++++++++++++
\subsection{Effect of discount rate ($r$) on rebound terms} 
\label{sec:effect_of_discount_rate}
%++++++++++++++++++++++++++++++

The effect of discount rate on rebound is shown in Fig.~\ref{fig:all_Re_terms_r_graph}.
Discounting has little effect on rebound terms
compared to other parameters such as
upgraded efficiency ($\amacro{\eta}$, Fig.~\ref{fig:all_Re_terms_eta_graph}), 
capital cost ($\amacro{C}_{cap}$, Fig.~\ref{fig:all_Re_terms_C_cap_graph}), 
energy price ($p_E$, Fig.~\ref{fig:all_Re_terms_p_E_graph}), and 
own price elasticity of energy service demand 
($\eqspsUCorig$, Fig.~\ref{fig:all_Re_terms_eps_graph}).

\begin{knitrout}
\definecolor{shadecolor}{rgb}{0.969, 0.969, 0.969}\color{fgcolor}\begin{figure}

{\centering \includegraphics[width=\maxwidth]{figure/all_Re_terms_r_graph-1} 

}

\caption[Sensitivity of rebound components to discount rate ($r$)]{Sensitivity of rebound components to discount rate ($r$).}\label{fig:all_Re_terms_r_graph}
\end{figure}

\end{knitrout}


%++++++++++++++++++++++++++++++
\subsection{Effect of energy service price ($\amacro{p}_s$) on price elasticities ($\asub{\varepsilon}$)}
\label{sec:price_elasticities_sensitivity}
%++++++++++++++++++++++++++++++

The sensitivity of post-substitution effect price elasticities ($\asub{\varepsilon}$)
to post-upgrade energy service price ($\amacro{p}_s$)
is shown in Fig.~\ref{fig:PriceElasticitySensGraph}
for the CES utility model described 
in Section~\ref{PartI-sec:sub_effect_main_paper} 
and Appendix~\ref{PartI-sec:utility_and_elasticities}
of Part~I.
Note that the left side of each graph ($\amacro{p}_{s} = 0$)
represents unattainable infinite efficiency ($\amacro{\eta}_s \rightarrow \infty$), 
i.e., delivery of the energy service without energy consumption.

First, note the sign of the elasticities. As expected, both of the
uncompensated price elasticities ($\eqspsUChat$ and $\eqopsUChat$,
dashed lines in Fig.~\ref{fig:PriceElasticitySensGraph}) 
are negative, regardless of the energy service price ($\amacro{p}_s$):
a lower price means more consumption of both
goods, all other things being equal.
The compensated own price elasticity ($\eqspsChat$) is negative
and the compensated cross price elasticity ($\eqopsChat$) is positive.
As $\amacro{p}_s$ declines, the consumers substitutes the energy
service for other goods.

\begin{knitrout}
\definecolor{shadecolor}{rgb}{0.969, 0.969, 0.969}\color{fgcolor}\begin{figure}

{\centering \includegraphics[width=\maxwidth]{figure/PriceElasticitySensGraph-1} 

}

\caption{Sensitivity of post substitution effect price elasticities ($\asub{\varepsilon}$) to post-EEU energy service price ($\amacro{p}_{s}$) for the CES utility model. This graph is a visualization of Eqs.~\ref{PartI-eq:eqopsC_exact}, \ref{PartI-eq:eqopsUC_exact}, \ref{PartI-eq:eqspsC_exact}, and \ref{PartI-eq:eqspsUC_exact} of Part~I. The solid vertical line indicates the original energy service price ($\orig{p}_s$), and the dashed vertical line indicates the upgraded energy service price ($\aempl{p}_s = \asub{p}_s = \ainc{p}_s = \amacro{p}_s$) for the two examples. See Tables~\ref{tab:car_stages_table} and \ref{tab:lamp_stages_table} for $p_s$ in different units.}\label{fig:PriceElasticitySensGraph}
\end{figure}

\end{knitrout}


Second, the magnitude of price elasticities varies.
Fig.~\ref{fig:PriceElasticitySensGraph} shows that the car example exhibits
more variation of price elasticities ($\asub{\varepsilon}$) with energy service price ($\amacro{p}_s$)
than the lamp example, 
because the expenditure share ($\fCs$) for the lamp example is very small
compared to the car example.
Using the constant price elasticity (CPE) utility model may be a
good enough approximation in the lamp example. 
However, for the car example, using the CES utility function will be necessary
to eliminate errors that will be present in the CPE approximation.
This result is an important finding that should encourage 
analysts implementing analytical rebound calculations
with substitution and income effects
to prefer the CES utility model over the CPE approximation.

Fig.~\ref{fig:PriceElasticitySensGraph} shows that
as efficiency increases (and $\amacro{p}_s$ decreases), 
the absolute value of the uncompensated price elasticities 
($\eqspsUChat$ and $\eqopsUChat$) decreases, 
a change that exceeds the slightly increasing (in absolute value terms)
compensated own price elasticity ($\eqspsChat$). 
Thus, direct rebound is attenuated as efficiency increases,
relative to a constant price elasticity model.
(See also the patterns of lines of Fig.~\ref{fig:all_Re_terms_eta_graph},
which show a declining trend.)


\end{document}
