% The next command tells RStudio to do "Compile PDF" on HSB_framework.Rnw,
% instead of this file, thereby eliminating the need to switch back to HSB_framework.Rnw 
% before building the paper.
%!TEX root = ../HSB_framework.Rnw


%%%%%%%%%%%%%%%%%%%%%%%%%%%%%%%%%%%%%%%%%%%%%%%%%%%%%%%%%%%%%%
% This file contains macros for
% Heun, Semieniuk, Brockway, 
% Toward a Comprehensive, Consumer-sided Energy Rebound Analysis Framework. 
% It is incorporated into the main file by the command
% % The next command tells RStudio to do "Compile PDF" on HSB_framework.Rnw,
% instead of this file, thereby eliminating the need to switch back to HSB_framework.Rnw 
% before building the paper.
%!TEX root = ../HSB_framework.Rnw


%%%%%%%%%%%%%%%%%%%%%%%%%%%%%%%%%%%%%%%%%%%%%%%%%%%%%%%%%%%%%%
% This file contains macros for
% Heun, Semieniuk, Brockway, 
% Energy, expenditure, and consumption aspects of rebound,\\
% Part I: A rigorous analytical framework
% and
% Energy, expenditure, and consumption aspects of rebound,\\
% Part II: Applications of the framework
% 
% It is incorporated into the main file by the command
% % The next command tells RStudio to do "Compile PDF" on HSB_framework.Rnw,
% instead of this file, thereby eliminating the need to switch back to HSB_framework.Rnw 
% before building the paper.
%!TEX root = ../HSB_framework.Rnw


%%%%%%%%%%%%%%%%%%%%%%%%%%%%%%%%%%%%%%%%%%%%%%%%%%%%%%%%%%%%%%
% This file contains macros for
% Heun, Semieniuk, Brockway, 
% Energy, expenditure, and consumption aspects of rebound,\\
% Part I: A rigorous analytical framework
% and
% Energy, expenditure, and consumption aspects of rebound,\\
% Part II: Applications of the framework
% 
% It is incorporated into the main file by the command
% % The next command tells RStudio to do "Compile PDF" on HSB_framework.Rnw,
% instead of this file, thereby eliminating the need to switch back to HSB_framework.Rnw 
% before building the paper.
%!TEX root = ../HSB_framework.Rnw


%%%%%%%%%%%%%%%%%%%%%%%%%%%%%%%%%%%%%%%%%%%%%%%%%%%%%%%%%%%%%%
% This file contains macros for
% Heun, Semieniuk, Brockway, 
% Energy, expenditure, and consumption aspects of rebound,\\
% Part I: A rigorous analytical framework
% and
% Energy, expenditure, and consumption aspects of rebound,\\
% Part II: Applications of the framework
% 
% It is incorporated into the main file by the command
% \input{macros.tex}.
%%%%%%%%%%%%%%%%%%%%%%%%%%%%%%%%%%%%%%%%%%%%%%%%%%%%%%%%%%%%%%


%%%%% Override Energy Economics macros
\renewcommand{\emph}{\textit}


%%%%% Units

\newcommand{\kWhr}{kW$\cdot$hr}
\newcommand{\lmhr}{lm$\cdot$hr}
\newcommand{\passkm}{pass$\cdot$km}
\newcommand{\Whr}{W$\cdot$hr}


%%%%% Decorations for symbols

\newcommand{\rate}[1]{\dot{#1}}                    % Rate of a quantity

% Create "after" commands
\newcommand{\orig}[1]{{}{#1}^{\scriptscriptstyle \circ}}
\newcommand{\aempl}[1]{{#1}^*}
\newcommand{\asub}[1]{\hat{#1}}
\newcommand{\ainc}[1]{\bar{#1}}
\newcommand{\amacro}[1]{\tilde{#1}}

% Create the "before" commands
\newcommand{\bempl}[1]{\orig{#1}}
\newcommand{\bsub}[1]{\aempl{#1}}
\newcommand{\binc}[1]{\asub{#1}}
\newcommand{\bmacro}[1]{\ainc{#1}}

% Decoration combinations
% Rates after
\newcommand{\rorig}[1]{\orig{\rate{#1}}}
\newcommand{\raempl}[1]{\aempl{\rate{#1}}}
\newcommand{\rasub}[1]{\asub{\rate{#1}}}
\newcommand{\rainc}[1]{\ainc{\rate{#1}}}
\newcommand{\ramacro}[1]{\amacro{\rate{#1}}}

% Rates before
\newcommand{\rbempl}[1]{\rorig{#1}}
\newcommand{\rbsub}[1]{\raempl{#1}}
\newcommand{\rbinc}[1]{\rasub{#1}}
\newcommand{\rbmacro}[1]{\rainc{#1}}

%%%%% Subscript kerning

\newcommand{\om}{O\!M}
\newcommand{\omd}{O\!M\!d}
\newcommand{\md}{md}
\newcommand{\macro}{macr\!o}
\newcommand{\life}{li\!f\!e}
\newcommand{\EEU}{E\!EU}

%%%%% Expression kerning

\newcommand{\MPC}{M\!PC}


%%%%% Convenient symbols

\newcommand{\Sdot}{\rate{S}_{dev}}
\newcommand{\Mdothatprime}{\rbinc{M}^\prime}


%%%%% Elasticities and income shares

\newcommand{\eqspsUC}{\varepsilon_{\rate{q}_s\!,p_s}}
\newcommand{\eqopsUC}{\varepsilon_{\rate{q}_o\!,p_s}}
\newcommand{\eqspsC}{\varepsilon_{\rate{q}_s\!,p_s\!,c}}
\newcommand{\eqopsC}{\varepsilon_{\rate{q}_o\!,p_s\!,c}}

% originally
\newcommand{\eqspsUCorig}{\varepsilon^\circ_{\rate{q}_s\!,p_s}}
\newcommand{\eqopsUCorig}{\varepsilon^\circ_{\rate{q}_o\!,p_s}}
\newcommand{\eqspsCorig}{\varepsilon^\circ_{\rate{q}_s\!,p_s\!,c}}
\newcommand{\eqopsCorig}{\varepsilon^\circ_{\rate{q}_o\!,p_s\!,c}}
% With hats
\newcommand{\eqspsUChat}{\hat{\varepsilon}_{\rate{q}_s\!,p_s}}
\newcommand{\eqopsUChat}{\hat{\varepsilon}_{\rate{q}_o\!,p_s}}
\newcommand{\eqspsChat}{\hat{\varepsilon}_{\rate{q}_s\!,p_s\!,c}}
\newcommand{\eqopsChat}{\hat{\varepsilon}_{\rate{q}_o\!,p_s\!,c}}

\newcommand{\eqsM}{\varepsilon_{\rate{q}_s\!,\rate{M}}}
\newcommand{\eqoM}{\varepsilon_{\rate{q}_o\!,\rate{M}}}

\newcommand{\fCs}{\bempl{f}_{\rate{C}_s}}
\newcommand{\fCshat}{\asub{f}_{\rate{C}_s}}

\newcommand{\eQEpE}{\varepsilon_{\rate{Q}_E,p_E}}


%%%%% Colors

% Original spectrum colours
% \colorlet{emplcolor}{red!25!white}
% \colorlet{subcolor}{orange!25!white}
% \colorlet{inccolor}{green!25!white}
% \colorlet{macrocolor}{blue!25!white}

% New Viridis "plasma" colours
\definecolor{emplcolor}{HTML}{150789}
\definecolor{subcolor}{HTML}{99149F}
\definecolor{inccolor}{HTML}{E76F5A}
\definecolor{macrocolor}{HTML}{F7E225}



%%%%% Coloration of text background

%
% Inline color box around text
% Arguments:
%   [#1]: background color for the box
%   {#2}: text inside the box
%
\newtcbox{\inlinebox}[1][]{on line, 
colback=#1,
colframe=#1,
before upper={\rule[-2pt]{0pt}{10pt}},
boxrule=1pt,
boxsep=0pt,
left=3pt,
right=3pt,
top=2pt,
bottom=2pt}


%%%%% Colored phrases

% Emplacement effect
\newcommand{\empleffect}{\inlinebox[emplcolor]{\textcolor{white}{emplacement effect}}}
\newcommand{\empleffectadj}{\inlinebox[emplcolor]{\textcolor{white}{emplacement-effect}}}
\newcommand{\Empleffect}{\inlinebox[emplcolor]{\textcolor{white}{Emplacement effect}}}
\newcommand{\EmplEffect}{\inlinebox[emplcolor]{\textcolor{white}{Emplacement Effect}}}

% Substitution effect
\newcommand{\subeffect}{\inlinebox[subcolor]{\textcolor{white}{substitution effect}}}
\newcommand{\subeffectadj}{\inlinebox[subcolor]{\textcolor{white}{substitution-effect}}}
\newcommand{\Subeffect}{\inlinebox[subcolor]{\textcolor{white}{Substitution effect}}}
\newcommand{\SubEffect}{\inlinebox[subcolor]{\textcolor{white}{Substitution Effect}}}

% Income effect
\newcommand{\inceffect}{\inlinebox[inccolor]{\textcolor{black}{income effect}}}
\newcommand{\inceffectadj}{\inlinebox[inccolor]{\textcolor{black}{income-effect}}}
\newcommand{\Inceffect}{\inlinebox[inccolor]{\textcolor{black}{Income effect}}}
\newcommand{\IncEffect}{\inlinebox[inccolor]{\textcolor{black}{Income Effect}}}

% Macro effect
\newcommand{\macroeffect}{\inlinebox[macrocolor]{\textcolor{black}{macro effect}}}
\newcommand{\macroeffectadj}{\inlinebox[macrocolor]{\textcolor{black}{macro-effect}}}
\newcommand{\Macroeffect}{\inlinebox[macrocolor]{\textcolor{black}{Macro effect}}}
\newcommand{\MacroEffect}{\inlinebox[macrocolor]{\textcolor{black}{Macro Effect}}}


%%%%% minipage for assumptions and constraints tables
% Arguments:
%   #1: Width (multiple of \linewidth
%   #2: text inside the minipage
%
\newcommand{\mptable}[2]{\begin{minipage}{#1\linewidth} \useshortskip{} \begin{equation} #2 \end{equation} \end{minipage}}


%%%%% Oft-used references

\newcommand{\Ba}[1]{\citeauthor[#1]{Borenstein:2015aa}}
\newcommand{\Bapp}[1]{\citeauthor[#1]{Borenstein:2015aa}'s \citeyearpar{Borenstein:2015aa}}
\newcommand{\Bp}[1]{\citep[#1]{Borenstein:2015aa}}
\newcommand{\Bt}[1]{\citet[#1]{Borenstein:2015aa}}

\newcommand{\Ta}[1]{\citeauthor[#1]{Thomas:2013aa}}
\newcommand{\Tapp}[1]{\citeauthor[#1]{Thomas:2013aa}'s \citeyearpar{Thomas:2013aa}}
\newcommand{\Tp}[1]{\citep[#1]{Thomas:2013aa, Thomas:2013ab}}
\newcommand{\Tpone}[1]{\citep[#1]{Thomas:2013aa}}
\newcommand{\Tptwo}[1]{\citep[#1]{Thomas:2013ab}}
\newcommand{\Tt}[1]{\citet[#1]{Thomas:2013aa, Thomas:2013ab}}
\newcommand{\Ttone}[1]{\citet[#1]{Thomas:2013aa}}
\newcommand{\Tttwo}[1]{\citet[#1]{Thomas:2013ab}}


%%%%% Derivation pages

% Column widths
\newcommand{\derivtextsize}{\footnotesize}
\newcommand{\derivpageleftcolwidth}{0.11\textwidth}
\newcommand{\derivpageenergycolwidth}{0.6\textwidth}
\newcommand{\derivpagefinancialcolwidth}{0.6\textwidth}

% Horizontal rule between sections of derivations

\newcommand{\sectionsep}{\noindent\rule{1.4\textwidth}{0.4pt}}


%
% Derivation section
% Arguments:
%   #1: accounting stage (original, prime, etc.)
%   #2: energy column
%   #3: financial column
%
\newcommand{\derivsection}[3]{%

\derivtextsize{}

\begin{minipage}[t]{\derivpageleftcolwidth}
~\\#1
\end{minipage}
%
%
%
\begin{minipage}[t]{\derivpageenergycolwidth}
#2
\end{minipage}
%
~
%
\begin{minipage}[t]{\derivpagefinancialcolwidth}
#3
\end{minipage}

\normalsize{}

}


%
% Derivation page header
% Arguments:
%   #1: Effect type header text (e.g., Emplacement Effect)
%
\newcommand{\derivheader}[1]{

\begin{center}
  #1
\end{center}

\derivsection{}
{\begin{center}\emph{Energy analysis}\end{center}}
{\begin{center}\emph{Financial analysis}\end{center}}

}


% Equations

% Efficiency ratios
\newcommand{\etaratioinline}{\amacro{\eta}/\bempl{\eta}}
\newcommand{\etaratiostacked}{\frac{\amacro{\eta}}{\bempl{\eta}}}

% Derivative with respect to efficiency ratio
\newcommand{\dbydetaeta}{\frac{\mathrm{d}}{\mathrm{d}(\etaratioinline{})}}


% Original
\newcommand{\Eacctorig}{\rbempl{E} = \rbempl{E}_s + \rbempl{E}_{emb} + (\rbempl{C}_{\omd} + \rbempl{C}_o) I_E}
\newcommand{\Macctorig}{\rate{M} = p_E \rbempl{E}_s + \bempl{R}_\alpha \rbempl{C}_{cap} + \rbempl{C}_{\omd} + \rbempl{C}_o + \rbempl{N}}

% Before emplacement effect (same as original)
\newcommand{\Eacctbempl}{\Eacctorig}      
\newcommand{\Macctbempl}{\Macctorig}      

% After emplacement effect
\newcommand{\Eacctaempl}{\raempl{E} = \raempl{E}_s + \raempl{E}_{emb} + (\raempl{C}_{\omd} + \raempl{C}_o) I_E}                  
\newcommand{\Macctaempl}{\rate{M} = p_E \raempl{E}_s + \aempl{R}_\alpha \raempl{C}_{cap} + \raempl{C}_{\omd} + \raempl{C}_o + \raempl{N}}         

% Before substitution effect (same as after emplacement effect)
\newcommand{\Eacctbsub}{\Eacctaempl}
\newcommand{\Macctbsub}{\Macctaempl}

% After substitution effect
\newcommand{\Eacctasub}{\rasub{E} = \rasub{E}_s + \rasub{E}_{emb} + (\rasub{C}_{\omd} + \rasub{C}_o) I_E}
\newcommand{\Macctasub}{\rate{M} = p_E \rasub{E}_s + \asub{R}_\alpha \rasub{C}_{cap} + \rasub{C}_{\omd} + \rasub{C}_o + \rasub{N}}

% Before income effect (same as after substitution effect)
\newcommand{\Eacctbinc}{\Eacctasub}
\newcommand{\Macctbinc}{\Macctasub}

% After income effect
\newcommand{\Eacctainc}{\rainc{E} = \rainc{E}_s + \rainc{E}_{emb} + (\rainc{C}_{\omd} + \rainc{C}_o) I_E}
\newcommand{\Macctainc}{\rate{M} = p_E \rainc{E}_s + \ainc{R}_\alpha \rainc{C}_{cap} + \rainc{C}_{\omd} + \rainc{C}_o + \rainc{N}}

% Embodied energy rebound
\newcommand{\Reembeqn}{\frac{\left( \frac{\aempl{E}_{emb}}{\bempl{E}_{emb}}
  \frac{\bempl{t}_{\life}}{\aempl{t}_{\life}} - 1 \right) \rbempl{E}_{emb}}{\Sdot}}
  
% Ops, Maintenance, and disposal energy rebound
\newcommand{\ReOMdeqn}{\frac{\left( \frac{\raempl{C}_{\omd}}{\rbempl{C}_{\omd}} - 1 \right) \rbempl{C}_{\omd} I_E}{\Sdot}}

% Equation for S_dot_dev
% \newcommand{\Sdoteqn}{\left( \etaratiostacked - 1 \right)\!\etaratiostacked \rbempl{E}_s}
\newcommand{\Sdoteqn}{\left( \etaratiostacked - 1 \right)\! 
                            \frac{\bempl{\eta}}{\amacro{\eta}} \rbempl{E}_s}

% Equation for Re_dsub
\newcommand{\Redsubeqn}{\frac{\left( \etaratiostacked \right)^{-\eqspsUC} - 1}
                        {\etaratiostacked - 1}}
                        
% Equation for Re_isub
\newcommand{\Reisubeqn}{\frac{{\left( \etaratiostacked  \right)}
                          ^{-\eqopsC} - 1}{\etaratiostacked - 1} \; 
                          \etaratiostacked \; 
                          \frac{\rbempl{C}_o I_E}{\rbempl{E}_s}}
                          
% CES utility equation
\newcommand{\cesutility}{\left[ \fCs \left( \frac{\rate{q}_s}{\rbempl{q}_s} \right)^\rho 
        + (1-\fCs) \left( \frac{\rate{C}_o}{\rbempl{C}_o} \right)^\rho  \right]^{(1/\rho)}}
        
% Equation for q_s_hat/q_s_orig
\newcommand{\qssolution}{\left\{ \fCs + (1-\fCs)
      \left[ \left(  \frac{1-\fCs}{\fCs}  \right) \frac{\amacro{p}_s \rbempl{q}_s}{\rbempl{C}_o}  \right]
                                                  ^{\rho / (1 - \rho)} \right\} ^ {-1/\rho}}

% Equation for C_o_hat/C_o_orig
\newcommand{\Cosolution}{ \left( 1 + \fCs \left\{ \left[ \left( \frac{1-\fCs}{\fCs} \right)
          \frac{\amacro{p}_s \rbempl{q}_s}{\rbempl{C}_o} \right] ^{\rho/(\rho - 1)} - 1 \right\} \right)^{-1/\rho}}

% Equation for Re_dsub for the CES utility model
\newcommand{\RedsubCES}{\frac{\qssolution{} - 1}{\etaratiostacked{} - 1}}

% Equation for Re_isub for the CES utility model
\newcommand{\ReisubCES}{\frac{\Cosolution{} - 1}{\etaratiostacked{} - 1}
                         \etaratiostacked \; 
                          \frac{\rbempl{C}_o I_E}{\rbempl{E}_s}}


% Equation for Re_dinc, approximate method
\newcommand{\Redinceqnapprox}{\frac{ \left( 1 + \frac{\rbinc{N}}{\Mdothatprime} \right) ^{\eqsM} - 1}
              { \etaratiostacked - 1 } \left( \etaratiostacked \right)^{-\eqspsC}}

% Equation for Re_dinc, exact method
\newcommand{\Redinceqnexact}{\frac{ \left( 1 + \frac{\rbinc{N}}{\Mdothatprime} \right) ^{\eqsM} - 1}
              { \etaratiostacked - 1 } \qssolution{} }

% Equation for Re_cap
\newcommand{\Recapeqn}{\frac{\Delta \aempl{(R_\alpha \rate{C}_{cap})} I_E}{\Sdot}}

% Equation for Re_iinc, approximate method
\newcommand{\Reiinceqnapprox}{\frac{\left( 1 + \frac{\rbinc{N}}{\Mdothatprime} \right)^{\eqoM} - 1}{\etaratiostacked - 1} 
              \left( \etaratiostacked \right)^{1 - \eqopsC}
              \frac{\rbempl{C}_o I_E}{\rbempl{E}_s}}

% Equation for Re_iinc, exact method
% \newcommand{\Reiinceqnexact}{\frac{\left( 1 + \frac{\rbinc{N}}{\Mdothatprime} \right)^{\eqoM} - 1}{\etaratiostacked - 1} 
%               \left( \etaratiostacked \right)
%               \left( \frac{\rasub{C}_o}{\rorig{C}_o} \right) 
%               \frac{\rbempl{C}_o I_E}{\rbempl{E}_s}}

\newcommand{\Reiinceqnexact}{\frac{\left( 1 + \frac{\rbinc{N}}{\Mdothatprime} \right)^{\eqoM} - 1}{\etaratiostacked - 1} 
              \left( \etaratiostacked \right)
              \frac{\rbempl{C}_o I_E}{\rbempl{E}_s}
              \Cosolution{}}



% Equation for Re_d (total direct rebound)
\newcommand{\Redeqn}{\frac{ \left( \etaratiostacked \right)^{-\eqspsC}
             \left( 1 + \frac{\rasub{N}}{\rbempl{M}} \right)^{\eqsM}   - 1}
         {\etaratiostacked - 1}}

% Equation for Re_macro
% \newcommand{\Remacroeqn}{k (p_E I_E - Re_{cap} - Re_{\md} - p_E I_E Re_{dsub} - Re_{isub})}
\newcommand{\Remacroeqn}{k (p_E I_E - Re_{cap} - Re_{\omd})}

% Equation for Re_tot
% \newcommand{\Retoteqn}{&Re_{emb} - k Re_{cap} + (1-k) Re_{\md}         \nonumber \\
%                        &+ (1 - k p_E I_E) Re_{dsub} + (1 - k) Re_{isub}   \nonumber \\
%                        &+ Re_{dinc} + Re_{iinc} +  k p_E I_E}
% \newcommand{\Retoteqn}{&Re_{emb} + k (p_E I_E - Re_{cap}) + (1-k) Re_{\md}   \nonumber \\
%                        &+ Re_{dsub} + Re_{isub}                              \nonumber \\
%                        &+ Re_{dinc} + Re_{iinc}}
\newcommand{\Retoteqn}{Re_{emb} + k (p_E I_E - Re_{cap}) + (1-k) Re_{\omd} + Re_{dsub} + Re_{isub} + Re_{dinc} + Re_{iinc}}
                 

%%%% Income preference equations

% Equation for energy service income preferences
\newcommand{\incprefseqn}{\frac{\rainc{q}_s}{\rbinc{q}_s} = \left( 1 + \frac{\rbinc{N}}{\Mdothatprime}  \right) ^{\eqsM}}

% Equation for other goods income preferences
\newcommand{\incprefoeqn}{\frac{\rainc{q}_o}{\rbinc{q}_o} = \left( 1 + \frac{\rbinc{N}}{\Mdothatprime}  \right) ^{\eqoM}}

% Equation for effective income
% \newcommand{\effinceqn}{\Mdothatprime \equiv \rbempl{M} - \rbempl{C}_{cap} - \rbempl{C}_{\md} 
%                         - \rate{G} + p_E \Delta \rbinc{E}_s + \Delta \rbinc{C}_o}
\newcommand{\effinceqn}{\Mdothatprime \equiv \rate{M} - \aempl{R}_\alpha \raempl{C}_{cap} - \raempl{C}_{\omd} - \rasub{N}}


%%%% Budget constraint symbols and equation

\newcommand{\tlife}{t_{li\!f\!e}}
\newcommand{\oneyr}{1\,\mathrm{yr}}
\newcommand{\twoyr}{2\,\mathrm{yr}}
\newcommand{\itlife}{i\,t_{li\!f\!e}}
\newcommand{\iyr}{i\,\mathrm{yr}}

\newcommand{\budgetconstraint}{\rate{M} - \orig{R}_\alpha \rorig{C}_{cap} - \rorig{C}_{\omd} = \orig{p}_E \frac{\rorig{q}_s}{\orig{\eta}} + p_o \rorig{q}_o}


%%%% Proof characters
% Equal sign with question mark above
\DeclareRobustCommand{\questionequal}{\stackrel{?}{=}}




% Segments and lines
% Arguments:
%   #1: left character
%   #2: line color
%   #3: line thickness (e.g., 0.1 mm)
%   #4: right character
% Note that \raisebox{0.9 mm} moves the line up from the baseline.
% Also, \line(1,0){12} gives a horizontal line with length "12" (unknown units!)
% (1, 0) is the slope (1 unit to right, 0 units up).
\newcommand{\seg}[4]{#1\linethickness{#3}\raisebox{0.88 mm}{\textcolor{#2}{\line(1,0){12}}}#4}

% Construction lines
\newcommand{\iicirc}{\seg{$\bempl{\text{i}}$}{black}{0.3 mm}{$\,\bempl{\text{i}}$}}
\newcommand{\iibar}{\seg{$\bmacro{\text{i}}$}{black}{0.3 mm}{$\,\bmacro{\text{i}}$}}
\newcommand{\rr}{\seg{r}{black}{0.1 mm}{r}}
\newcommand{\circcirc}{\seg{$\circ$}{black}{0.1 mm}{$\circ$}}
\newcommand{\starstar}{\seg{$*$}{black}{0.1 mm}{$*$}}
\newcommand{\hathat}{\seg{$\wedge$}{black}{0.1 mm}{$\wedge$}}
\newcommand{\barbar}{\seg{$-\,$}{black}{0.1 mm}{$\, -$}}

% Line segments
\newcommand{\circa}{\seg{$\circ$}{emplcolor}{0.6 mm}{$a$}}
% \newcommand{\ab}{\seg{a}{emplcolor}{0.6 mm}{b}}
\newcommand{\ab}{$a$\tikz[baseline=-0.6ex]\draw [line width=0.6mm,dotted,emplcolor] (0,0) -- (0.45,0);$b$}
\newcommand{\bstar}{\seg{$b\,$}{emplcolor}{0.6 mm}{$*$}}
\newcommand{\starc}{\seg{$*$}{subcolor}{0.6 mm}{$\,c$}}
\newcommand{\chat}{\seg{$c\,$}{subcolor}{0.6 mm}{$\wedge$}}
\newcommand{\hatd}{\seg{$\wedge$}{inccolor}{0.6 mm}{$\,d$}}
\newcommand{\dbar}{\seg{$d$}{inccolor}{0.6 mm}{$\,-$}}
\newcommand{\hatbar}{\seg{$\wedge$}{inccolor}{0.6 mm}{$\,-$}}
\newcommand{\bartilde}{\seg{$- \,$}{macrocolor}{0.6 mm}{$\, \sim$}}


% Rotated text for tables
% See
% https://tex.stackexchange.com/questions/98388/how-to-make-table-with-rotated-table-headers-in-latex/98439#98439
% for details.
\newcommand{\rot}{\rotatebox{90}}


% A "rating" command for filled circles with tikz.
% See 
% https://tex.stackexchange.com/questions/194955/get-partly-filled-circle-symbol-scale-linearly-with-parameter
% for details.
\newcommand{\rating}[2][0.75ex]{%
  \pgfmathsetmacro\th{asin(#2/50-1)}% (theta angle of polar coordinates)
    \tikz{%
      \fill[black] (\th:#1) arc (\th:-180-\th:#1) -- cycle;
      \draw[black, thin, radius=#1] (0,0) circle;
    }%
}.
%%%%%%%%%%%%%%%%%%%%%%%%%%%%%%%%%%%%%%%%%%%%%%%%%%%%%%%%%%%%%%


%%%%% Override Energy Economics macros
\renewcommand{\emph}{\textit}


%%%%% Units

\newcommand{\kWhr}{kW$\cdot$hr}
\newcommand{\lmhr}{lm$\cdot$hr}
\newcommand{\passkm}{pass$\cdot$km}
\newcommand{\Whr}{W$\cdot$hr}


%%%%% Decorations for symbols

\newcommand{\rate}[1]{\dot{#1}}                    % Rate of a quantity

% Create "after" commands
\newcommand{\orig}[1]{{}{#1}^{\scriptscriptstyle \circ}}
\newcommand{\aempl}[1]{{#1}^*}
\newcommand{\asub}[1]{\hat{#1}}
\newcommand{\ainc}[1]{\bar{#1}}
\newcommand{\amacro}[1]{\tilde{#1}}

% Create the "before" commands
\newcommand{\bempl}[1]{\orig{#1}}
\newcommand{\bsub}[1]{\aempl{#1}}
\newcommand{\binc}[1]{\asub{#1}}
\newcommand{\bmacro}[1]{\ainc{#1}}

% Decoration combinations
% Rates after
\newcommand{\rorig}[1]{\orig{\rate{#1}}}
\newcommand{\raempl}[1]{\aempl{\rate{#1}}}
\newcommand{\rasub}[1]{\asub{\rate{#1}}}
\newcommand{\rainc}[1]{\ainc{\rate{#1}}}
\newcommand{\ramacro}[1]{\amacro{\rate{#1}}}

% Rates before
\newcommand{\rbempl}[1]{\rorig{#1}}
\newcommand{\rbsub}[1]{\raempl{#1}}
\newcommand{\rbinc}[1]{\rasub{#1}}
\newcommand{\rbmacro}[1]{\rainc{#1}}

%%%%% Subscript kerning

\newcommand{\om}{O\!M}
\newcommand{\omd}{O\!M\!d}
\newcommand{\md}{md}
\newcommand{\macro}{macr\!o}
\newcommand{\life}{li\!f\!e}
\newcommand{\EEU}{E\!EU}

%%%%% Expression kerning

\newcommand{\MPC}{M\!PC}


%%%%% Convenient symbols

\newcommand{\Sdot}{\rate{S}_{dev}}
\newcommand{\Mdothatprime}{\rbinc{M}^\prime}


%%%%% Elasticities and income shares

\newcommand{\eqspsUC}{\varepsilon_{\rate{q}_s\!,p_s}}
\newcommand{\eqopsUC}{\varepsilon_{\rate{q}_o\!,p_s}}
\newcommand{\eqspsC}{\varepsilon_{\rate{q}_s\!,p_s\!,c}}
\newcommand{\eqopsC}{\varepsilon_{\rate{q}_o\!,p_s\!,c}}

% originally
\newcommand{\eqspsUCorig}{\varepsilon^\circ_{\rate{q}_s\!,p_s}}
\newcommand{\eqopsUCorig}{\varepsilon^\circ_{\rate{q}_o\!,p_s}}
\newcommand{\eqspsCorig}{\varepsilon^\circ_{\rate{q}_s\!,p_s\!,c}}
\newcommand{\eqopsCorig}{\varepsilon^\circ_{\rate{q}_o\!,p_s\!,c}}
% With hats
\newcommand{\eqspsUChat}{\hat{\varepsilon}_{\rate{q}_s\!,p_s}}
\newcommand{\eqopsUChat}{\hat{\varepsilon}_{\rate{q}_o\!,p_s}}
\newcommand{\eqspsChat}{\hat{\varepsilon}_{\rate{q}_s\!,p_s\!,c}}
\newcommand{\eqopsChat}{\hat{\varepsilon}_{\rate{q}_o\!,p_s\!,c}}

\newcommand{\eqsM}{\varepsilon_{\rate{q}_s\!,\rate{M}}}
\newcommand{\eqoM}{\varepsilon_{\rate{q}_o\!,\rate{M}}}

\newcommand{\fCs}{\bempl{f}_{\rate{C}_s}}
\newcommand{\fCshat}{\asub{f}_{\rate{C}_s}}

\newcommand{\eQEpE}{\varepsilon_{\rate{Q}_E,p_E}}


%%%%% Colors

% Original spectrum colours
% \colorlet{emplcolor}{red!25!white}
% \colorlet{subcolor}{orange!25!white}
% \colorlet{inccolor}{green!25!white}
% \colorlet{macrocolor}{blue!25!white}

% New Viridis "plasma" colours
\definecolor{emplcolor}{HTML}{150789}
\definecolor{subcolor}{HTML}{99149F}
\definecolor{inccolor}{HTML}{E76F5A}
\definecolor{macrocolor}{HTML}{F7E225}



%%%%% Coloration of text background

%
% Inline color box around text
% Arguments:
%   [#1]: background color for the box
%   {#2}: text inside the box
%
\newtcbox{\inlinebox}[1][]{on line, 
colback=#1,
colframe=#1,
before upper={\rule[-2pt]{0pt}{10pt}},
boxrule=1pt,
boxsep=0pt,
left=3pt,
right=3pt,
top=2pt,
bottom=2pt}


%%%%% Colored phrases

% Emplacement effect
\newcommand{\empleffect}{\inlinebox[emplcolor]{\textcolor{white}{emplacement effect}}}
\newcommand{\empleffectadj}{\inlinebox[emplcolor]{\textcolor{white}{emplacement-effect}}}
\newcommand{\Empleffect}{\inlinebox[emplcolor]{\textcolor{white}{Emplacement effect}}}
\newcommand{\EmplEffect}{\inlinebox[emplcolor]{\textcolor{white}{Emplacement Effect}}}

% Substitution effect
\newcommand{\subeffect}{\inlinebox[subcolor]{\textcolor{white}{substitution effect}}}
\newcommand{\subeffectadj}{\inlinebox[subcolor]{\textcolor{white}{substitution-effect}}}
\newcommand{\Subeffect}{\inlinebox[subcolor]{\textcolor{white}{Substitution effect}}}
\newcommand{\SubEffect}{\inlinebox[subcolor]{\textcolor{white}{Substitution Effect}}}

% Income effect
\newcommand{\inceffect}{\inlinebox[inccolor]{\textcolor{black}{income effect}}}
\newcommand{\inceffectadj}{\inlinebox[inccolor]{\textcolor{black}{income-effect}}}
\newcommand{\Inceffect}{\inlinebox[inccolor]{\textcolor{black}{Income effect}}}
\newcommand{\IncEffect}{\inlinebox[inccolor]{\textcolor{black}{Income Effect}}}

% Macro effect
\newcommand{\macroeffect}{\inlinebox[macrocolor]{\textcolor{black}{macro effect}}}
\newcommand{\macroeffectadj}{\inlinebox[macrocolor]{\textcolor{black}{macro-effect}}}
\newcommand{\Macroeffect}{\inlinebox[macrocolor]{\textcolor{black}{Macro effect}}}
\newcommand{\MacroEffect}{\inlinebox[macrocolor]{\textcolor{black}{Macro Effect}}}


%%%%% minipage for assumptions and constraints tables
% Arguments:
%   #1: Width (multiple of \linewidth
%   #2: text inside the minipage
%
\newcommand{\mptable}[2]{\begin{minipage}{#1\linewidth} \useshortskip{} \begin{equation} #2 \end{equation} \end{minipage}}


%%%%% Oft-used references

\newcommand{\Ba}[1]{\citeauthor[#1]{Borenstein:2015aa}}
\newcommand{\Bapp}[1]{\citeauthor[#1]{Borenstein:2015aa}'s \citeyearpar{Borenstein:2015aa}}
\newcommand{\Bp}[1]{\citep[#1]{Borenstein:2015aa}}
\newcommand{\Bt}[1]{\citet[#1]{Borenstein:2015aa}}

\newcommand{\Ta}[1]{\citeauthor[#1]{Thomas:2013aa}}
\newcommand{\Tapp}[1]{\citeauthor[#1]{Thomas:2013aa}'s \citeyearpar{Thomas:2013aa}}
\newcommand{\Tp}[1]{\citep[#1]{Thomas:2013aa, Thomas:2013ab}}
\newcommand{\Tpone}[1]{\citep[#1]{Thomas:2013aa}}
\newcommand{\Tptwo}[1]{\citep[#1]{Thomas:2013ab}}
\newcommand{\Tt}[1]{\citet[#1]{Thomas:2013aa, Thomas:2013ab}}
\newcommand{\Ttone}[1]{\citet[#1]{Thomas:2013aa}}
\newcommand{\Tttwo}[1]{\citet[#1]{Thomas:2013ab}}


%%%%% Derivation pages

% Column widths
\newcommand{\derivtextsize}{\footnotesize}
\newcommand{\derivpageleftcolwidth}{0.11\textwidth}
\newcommand{\derivpageenergycolwidth}{0.6\textwidth}
\newcommand{\derivpagefinancialcolwidth}{0.6\textwidth}

% Horizontal rule between sections of derivations

\newcommand{\sectionsep}{\noindent\rule{1.4\textwidth}{0.4pt}}


%
% Derivation section
% Arguments:
%   #1: accounting stage (original, prime, etc.)
%   #2: energy column
%   #3: financial column
%
\newcommand{\derivsection}[3]{%

\derivtextsize{}

\begin{minipage}[t]{\derivpageleftcolwidth}
~\\#1
\end{minipage}
%
%
%
\begin{minipage}[t]{\derivpageenergycolwidth}
#2
\end{minipage}
%
~
%
\begin{minipage}[t]{\derivpagefinancialcolwidth}
#3
\end{minipage}

\normalsize{}

}


%
% Derivation page header
% Arguments:
%   #1: Effect type header text (e.g., Emplacement Effect)
%
\newcommand{\derivheader}[1]{

\begin{center}
  #1
\end{center}

\derivsection{}
{\begin{center}\emph{Energy analysis}\end{center}}
{\begin{center}\emph{Financial analysis}\end{center}}

}


% Equations

% Efficiency ratios
\newcommand{\etaratioinline}{\amacro{\eta}/\bempl{\eta}}
\newcommand{\etaratiostacked}{\frac{\amacro{\eta}}{\bempl{\eta}}}

% Derivative with respect to efficiency ratio
\newcommand{\dbydetaeta}{\frac{\mathrm{d}}{\mathrm{d}(\etaratioinline{})}}


% Original
\newcommand{\Eacctorig}{\rbempl{E} = \rbempl{E}_s + \rbempl{E}_{emb} + (\rbempl{C}_{\omd} + \rbempl{C}_o) I_E}
\newcommand{\Macctorig}{\rate{M} = p_E \rbempl{E}_s + \bempl{R}_\alpha \rbempl{C}_{cap} + \rbempl{C}_{\omd} + \rbempl{C}_o + \rbempl{N}}

% Before emplacement effect (same as original)
\newcommand{\Eacctbempl}{\Eacctorig}      
\newcommand{\Macctbempl}{\Macctorig}      

% After emplacement effect
\newcommand{\Eacctaempl}{\raempl{E} = \raempl{E}_s + \raempl{E}_{emb} + (\raempl{C}_{\omd} + \raempl{C}_o) I_E}                  
\newcommand{\Macctaempl}{\rate{M} = p_E \raempl{E}_s + \aempl{R}_\alpha \raempl{C}_{cap} + \raempl{C}_{\omd} + \raempl{C}_o + \raempl{N}}         

% Before substitution effect (same as after emplacement effect)
\newcommand{\Eacctbsub}{\Eacctaempl}
\newcommand{\Macctbsub}{\Macctaempl}

% After substitution effect
\newcommand{\Eacctasub}{\rasub{E} = \rasub{E}_s + \rasub{E}_{emb} + (\rasub{C}_{\omd} + \rasub{C}_o) I_E}
\newcommand{\Macctasub}{\rate{M} = p_E \rasub{E}_s + \asub{R}_\alpha \rasub{C}_{cap} + \rasub{C}_{\omd} + \rasub{C}_o + \rasub{N}}

% Before income effect (same as after substitution effect)
\newcommand{\Eacctbinc}{\Eacctasub}
\newcommand{\Macctbinc}{\Macctasub}

% After income effect
\newcommand{\Eacctainc}{\rainc{E} = \rainc{E}_s + \rainc{E}_{emb} + (\rainc{C}_{\omd} + \rainc{C}_o) I_E}
\newcommand{\Macctainc}{\rate{M} = p_E \rainc{E}_s + \ainc{R}_\alpha \rainc{C}_{cap} + \rainc{C}_{\omd} + \rainc{C}_o + \rainc{N}}

% Embodied energy rebound
\newcommand{\Reembeqn}{\frac{\left( \frac{\aempl{E}_{emb}}{\bempl{E}_{emb}}
  \frac{\bempl{t}_{\life}}{\aempl{t}_{\life}} - 1 \right) \rbempl{E}_{emb}}{\Sdot}}
  
% Ops, Maintenance, and disposal energy rebound
\newcommand{\ReOMdeqn}{\frac{\left( \frac{\raempl{C}_{\omd}}{\rbempl{C}_{\omd}} - 1 \right) \rbempl{C}_{\omd} I_E}{\Sdot}}

% Equation for S_dot_dev
% \newcommand{\Sdoteqn}{\left( \etaratiostacked - 1 \right)\!\etaratiostacked \rbempl{E}_s}
\newcommand{\Sdoteqn}{\left( \etaratiostacked - 1 \right)\! 
                            \frac{\bempl{\eta}}{\amacro{\eta}} \rbempl{E}_s}

% Equation for Re_dsub
\newcommand{\Redsubeqn}{\frac{\left( \etaratiostacked \right)^{-\eqspsUC} - 1}
                        {\etaratiostacked - 1}}
                        
% Equation for Re_isub
\newcommand{\Reisubeqn}{\frac{{\left( \etaratiostacked  \right)}
                          ^{-\eqopsC} - 1}{\etaratiostacked - 1} \; 
                          \etaratiostacked \; 
                          \frac{\rbempl{C}_o I_E}{\rbempl{E}_s}}
                          
% CES utility equation
\newcommand{\cesutility}{\left[ \fCs \left( \frac{\rate{q}_s}{\rbempl{q}_s} \right)^\rho 
        + (1-\fCs) \left( \frac{\rate{C}_o}{\rbempl{C}_o} \right)^\rho  \right]^{(1/\rho)}}
        
% Equation for q_s_hat/q_s_orig
\newcommand{\qssolution}{\left\{ \fCs + (1-\fCs)
      \left[ \left(  \frac{1-\fCs}{\fCs}  \right) \frac{\amacro{p}_s \rbempl{q}_s}{\rbempl{C}_o}  \right]
                                                  ^{\rho / (1 - \rho)} \right\} ^ {-1/\rho}}

% Equation for C_o_hat/C_o_orig
\newcommand{\Cosolution}{ \left( 1 + \fCs \left\{ \left[ \left( \frac{1-\fCs}{\fCs} \right)
          \frac{\amacro{p}_s \rbempl{q}_s}{\rbempl{C}_o} \right] ^{\rho/(\rho - 1)} - 1 \right\} \right)^{-1/\rho}}

% Equation for Re_dsub for the CES utility model
\newcommand{\RedsubCES}{\frac{\qssolution{} - 1}{\etaratiostacked{} - 1}}

% Equation for Re_isub for the CES utility model
\newcommand{\ReisubCES}{\frac{\Cosolution{} - 1}{\etaratiostacked{} - 1}
                         \etaratiostacked \; 
                          \frac{\rbempl{C}_o I_E}{\rbempl{E}_s}}


% Equation for Re_dinc, approximate method
\newcommand{\Redinceqnapprox}{\frac{ \left( 1 + \frac{\rbinc{N}}{\Mdothatprime} \right) ^{\eqsM} - 1}
              { \etaratiostacked - 1 } \left( \etaratiostacked \right)^{-\eqspsC}}

% Equation for Re_dinc, exact method
\newcommand{\Redinceqnexact}{\frac{ \left( 1 + \frac{\rbinc{N}}{\Mdothatprime} \right) ^{\eqsM} - 1}
              { \etaratiostacked - 1 } \qssolution{} }

% Equation for Re_cap
\newcommand{\Recapeqn}{\frac{\Delta \aempl{(R_\alpha \rate{C}_{cap})} I_E}{\Sdot}}

% Equation for Re_iinc, approximate method
\newcommand{\Reiinceqnapprox}{\frac{\left( 1 + \frac{\rbinc{N}}{\Mdothatprime} \right)^{\eqoM} - 1}{\etaratiostacked - 1} 
              \left( \etaratiostacked \right)^{1 - \eqopsC}
              \frac{\rbempl{C}_o I_E}{\rbempl{E}_s}}

% Equation for Re_iinc, exact method
% \newcommand{\Reiinceqnexact}{\frac{\left( 1 + \frac{\rbinc{N}}{\Mdothatprime} \right)^{\eqoM} - 1}{\etaratiostacked - 1} 
%               \left( \etaratiostacked \right)
%               \left( \frac{\rasub{C}_o}{\rorig{C}_o} \right) 
%               \frac{\rbempl{C}_o I_E}{\rbempl{E}_s}}

\newcommand{\Reiinceqnexact}{\frac{\left( 1 + \frac{\rbinc{N}}{\Mdothatprime} \right)^{\eqoM} - 1}{\etaratiostacked - 1} 
              \left( \etaratiostacked \right)
              \frac{\rbempl{C}_o I_E}{\rbempl{E}_s}
              \Cosolution{}}



% Equation for Re_d (total direct rebound)
\newcommand{\Redeqn}{\frac{ \left( \etaratiostacked \right)^{-\eqspsC}
             \left( 1 + \frac{\rasub{N}}{\rbempl{M}} \right)^{\eqsM}   - 1}
         {\etaratiostacked - 1}}

% Equation for Re_macro
% \newcommand{\Remacroeqn}{k (p_E I_E - Re_{cap} - Re_{\md} - p_E I_E Re_{dsub} - Re_{isub})}
\newcommand{\Remacroeqn}{k (p_E I_E - Re_{cap} - Re_{\omd})}

% Equation for Re_tot
% \newcommand{\Retoteqn}{&Re_{emb} - k Re_{cap} + (1-k) Re_{\md}         \nonumber \\
%                        &+ (1 - k p_E I_E) Re_{dsub} + (1 - k) Re_{isub}   \nonumber \\
%                        &+ Re_{dinc} + Re_{iinc} +  k p_E I_E}
% \newcommand{\Retoteqn}{&Re_{emb} + k (p_E I_E - Re_{cap}) + (1-k) Re_{\md}   \nonumber \\
%                        &+ Re_{dsub} + Re_{isub}                              \nonumber \\
%                        &+ Re_{dinc} + Re_{iinc}}
\newcommand{\Retoteqn}{Re_{emb} + k (p_E I_E - Re_{cap}) + (1-k) Re_{\omd} + Re_{dsub} + Re_{isub} + Re_{dinc} + Re_{iinc}}
                 

%%%% Income preference equations

% Equation for energy service income preferences
\newcommand{\incprefseqn}{\frac{\rainc{q}_s}{\rbinc{q}_s} = \left( 1 + \frac{\rbinc{N}}{\Mdothatprime}  \right) ^{\eqsM}}

% Equation for other goods income preferences
\newcommand{\incprefoeqn}{\frac{\rainc{q}_o}{\rbinc{q}_o} = \left( 1 + \frac{\rbinc{N}}{\Mdothatprime}  \right) ^{\eqoM}}

% Equation for effective income
% \newcommand{\effinceqn}{\Mdothatprime \equiv \rbempl{M} - \rbempl{C}_{cap} - \rbempl{C}_{\md} 
%                         - \rate{G} + p_E \Delta \rbinc{E}_s + \Delta \rbinc{C}_o}
\newcommand{\effinceqn}{\Mdothatprime \equiv \rate{M} - \aempl{R}_\alpha \raempl{C}_{cap} - \raempl{C}_{\omd} - \rasub{N}}


%%%% Budget constraint symbols and equation

\newcommand{\tlife}{t_{li\!f\!e}}
\newcommand{\oneyr}{1\,\mathrm{yr}}
\newcommand{\twoyr}{2\,\mathrm{yr}}
\newcommand{\itlife}{i\,t_{li\!f\!e}}
\newcommand{\iyr}{i\,\mathrm{yr}}

\newcommand{\budgetconstraint}{\rate{M} - \orig{R}_\alpha \rorig{C}_{cap} - \rorig{C}_{\omd} = \orig{p}_E \frac{\rorig{q}_s}{\orig{\eta}} + p_o \rorig{q}_o}


%%%% Proof characters
% Equal sign with question mark above
\DeclareRobustCommand{\questionequal}{\stackrel{?}{=}}




% Segments and lines
% Arguments:
%   #1: left character
%   #2: line color
%   #3: line thickness (e.g., 0.1 mm)
%   #4: right character
% Note that \raisebox{0.9 mm} moves the line up from the baseline.
% Also, \line(1,0){12} gives a horizontal line with length "12" (unknown units!)
% (1, 0) is the slope (1 unit to right, 0 units up).
\newcommand{\seg}[4]{#1\linethickness{#3}\raisebox{0.88 mm}{\textcolor{#2}{\line(1,0){12}}}#4}

% Construction lines
\newcommand{\iicirc}{\seg{$\bempl{\text{i}}$}{black}{0.3 mm}{$\,\bempl{\text{i}}$}}
\newcommand{\iibar}{\seg{$\bmacro{\text{i}}$}{black}{0.3 mm}{$\,\bmacro{\text{i}}$}}
\newcommand{\rr}{\seg{r}{black}{0.1 mm}{r}}
\newcommand{\circcirc}{\seg{$\circ$}{black}{0.1 mm}{$\circ$}}
\newcommand{\starstar}{\seg{$*$}{black}{0.1 mm}{$*$}}
\newcommand{\hathat}{\seg{$\wedge$}{black}{0.1 mm}{$\wedge$}}
\newcommand{\barbar}{\seg{$-\,$}{black}{0.1 mm}{$\, -$}}

% Line segments
\newcommand{\circa}{\seg{$\circ$}{emplcolor}{0.6 mm}{$a$}}
% \newcommand{\ab}{\seg{a}{emplcolor}{0.6 mm}{b}}
\newcommand{\ab}{$a$\tikz[baseline=-0.6ex]\draw [line width=0.6mm,dotted,emplcolor] (0,0) -- (0.45,0);$b$}
\newcommand{\bstar}{\seg{$b\,$}{emplcolor}{0.6 mm}{$*$}}
\newcommand{\starc}{\seg{$*$}{subcolor}{0.6 mm}{$\,c$}}
\newcommand{\chat}{\seg{$c\,$}{subcolor}{0.6 mm}{$\wedge$}}
\newcommand{\hatd}{\seg{$\wedge$}{inccolor}{0.6 mm}{$\,d$}}
\newcommand{\dbar}{\seg{$d$}{inccolor}{0.6 mm}{$\,-$}}
\newcommand{\hatbar}{\seg{$\wedge$}{inccolor}{0.6 mm}{$\,-$}}
\newcommand{\bartilde}{\seg{$- \,$}{macrocolor}{0.6 mm}{$\, \sim$}}


% Rotated text for tables
% See
% https://tex.stackexchange.com/questions/98388/how-to-make-table-with-rotated-table-headers-in-latex/98439#98439
% for details.
\newcommand{\rot}{\rotatebox{90}}


% A "rating" command for filled circles with tikz.
% See 
% https://tex.stackexchange.com/questions/194955/get-partly-filled-circle-symbol-scale-linearly-with-parameter
% for details.
\newcommand{\rating}[2][0.75ex]{%
  \pgfmathsetmacro\th{asin(#2/50-1)}% (theta angle of polar coordinates)
    \tikz{%
      \fill[black] (\th:#1) arc (\th:-180-\th:#1) -- cycle;
      \draw[black, thin, radius=#1] (0,0) circle;
    }%
}.
%%%%%%%%%%%%%%%%%%%%%%%%%%%%%%%%%%%%%%%%%%%%%%%%%%%%%%%%%%%%%%


%%%%% Override Energy Economics macros
\renewcommand{\emph}{\textit}


%%%%% Units

\newcommand{\kWhr}{kW$\cdot$hr}
\newcommand{\lmhr}{lm$\cdot$hr}
\newcommand{\passkm}{pass$\cdot$km}
\newcommand{\Whr}{W$\cdot$hr}


%%%%% Decorations for symbols

\newcommand{\rate}[1]{\dot{#1}}                    % Rate of a quantity

% Create "after" commands
\newcommand{\orig}[1]{{}{#1}^{\scriptscriptstyle \circ}}
\newcommand{\aempl}[1]{{#1}^*}
\newcommand{\asub}[1]{\hat{#1}}
\newcommand{\ainc}[1]{\bar{#1}}
\newcommand{\amacro}[1]{\tilde{#1}}

% Create the "before" commands
\newcommand{\bempl}[1]{\orig{#1}}
\newcommand{\bsub}[1]{\aempl{#1}}
\newcommand{\binc}[1]{\asub{#1}}
\newcommand{\bmacro}[1]{\ainc{#1}}

% Decoration combinations
% Rates after
\newcommand{\rorig}[1]{\orig{\rate{#1}}}
\newcommand{\raempl}[1]{\aempl{\rate{#1}}}
\newcommand{\rasub}[1]{\asub{\rate{#1}}}
\newcommand{\rainc}[1]{\ainc{\rate{#1}}}
\newcommand{\ramacro}[1]{\amacro{\rate{#1}}}

% Rates before
\newcommand{\rbempl}[1]{\rorig{#1}}
\newcommand{\rbsub}[1]{\raempl{#1}}
\newcommand{\rbinc}[1]{\rasub{#1}}
\newcommand{\rbmacro}[1]{\rainc{#1}}

%%%%% Subscript kerning

\newcommand{\om}{O\!M}
\newcommand{\omd}{O\!M\!d}
\newcommand{\md}{md}
\newcommand{\macro}{macr\!o}
\newcommand{\life}{li\!f\!e}
\newcommand{\EEU}{E\!EU}

%%%%% Expression kerning

\newcommand{\MPC}{M\!PC}


%%%%% Convenient symbols

\newcommand{\Sdot}{\rate{S}_{dev}}
\newcommand{\Mdothatprime}{\rbinc{M}^\prime}


%%%%% Elasticities and income shares

\newcommand{\eqspsUC}{\varepsilon_{\rate{q}_s\!,p_s}}
\newcommand{\eqopsUC}{\varepsilon_{\rate{q}_o\!,p_s}}
\newcommand{\eqspsC}{\varepsilon_{\rate{q}_s\!,p_s\!,c}}
\newcommand{\eqopsC}{\varepsilon_{\rate{q}_o\!,p_s\!,c}}

% originally
\newcommand{\eqspsUCorig}{\varepsilon^\circ_{\rate{q}_s\!,p_s}}
\newcommand{\eqopsUCorig}{\varepsilon^\circ_{\rate{q}_o\!,p_s}}
\newcommand{\eqspsCorig}{\varepsilon^\circ_{\rate{q}_s\!,p_s\!,c}}
\newcommand{\eqopsCorig}{\varepsilon^\circ_{\rate{q}_o\!,p_s\!,c}}
% With hats
\newcommand{\eqspsUChat}{\hat{\varepsilon}_{\rate{q}_s\!,p_s}}
\newcommand{\eqopsUChat}{\hat{\varepsilon}_{\rate{q}_o\!,p_s}}
\newcommand{\eqspsChat}{\hat{\varepsilon}_{\rate{q}_s\!,p_s\!,c}}
\newcommand{\eqopsChat}{\hat{\varepsilon}_{\rate{q}_o\!,p_s\!,c}}

\newcommand{\eqsM}{\varepsilon_{\rate{q}_s\!,\rate{M}}}
\newcommand{\eqoM}{\varepsilon_{\rate{q}_o\!,\rate{M}}}

\newcommand{\fCs}{\bempl{f}_{\rate{C}_s}}
\newcommand{\fCshat}{\asub{f}_{\rate{C}_s}}

\newcommand{\eQEpE}{\varepsilon_{\rate{Q}_E,p_E}}


%%%%% Colors

% Original spectrum colours
% \colorlet{emplcolor}{red!25!white}
% \colorlet{subcolor}{orange!25!white}
% \colorlet{inccolor}{green!25!white}
% \colorlet{macrocolor}{blue!25!white}

% New Viridis "plasma" colours
\definecolor{emplcolor}{HTML}{150789}
\definecolor{subcolor}{HTML}{99149F}
\definecolor{inccolor}{HTML}{E76F5A}
\definecolor{macrocolor}{HTML}{F7E225}



%%%%% Coloration of text background

%
% Inline color box around text
% Arguments:
%   [#1]: background color for the box
%   {#2}: text inside the box
%
\newtcbox{\inlinebox}[1][]{on line, 
colback=#1,
colframe=#1,
before upper={\rule[-2pt]{0pt}{10pt}},
boxrule=1pt,
boxsep=0pt,
left=3pt,
right=3pt,
top=2pt,
bottom=2pt}


%%%%% Colored phrases

% Emplacement effect
\newcommand{\empleffect}{\inlinebox[emplcolor]{\textcolor{white}{emplacement effect}}}
\newcommand{\empleffectadj}{\inlinebox[emplcolor]{\textcolor{white}{emplacement-effect}}}
\newcommand{\Empleffect}{\inlinebox[emplcolor]{\textcolor{white}{Emplacement effect}}}
\newcommand{\EmplEffect}{\inlinebox[emplcolor]{\textcolor{white}{Emplacement Effect}}}

% Substitution effect
\newcommand{\subeffect}{\inlinebox[subcolor]{\textcolor{white}{substitution effect}}}
\newcommand{\subeffectadj}{\inlinebox[subcolor]{\textcolor{white}{substitution-effect}}}
\newcommand{\Subeffect}{\inlinebox[subcolor]{\textcolor{white}{Substitution effect}}}
\newcommand{\SubEffect}{\inlinebox[subcolor]{\textcolor{white}{Substitution Effect}}}

% Income effect
\newcommand{\inceffect}{\inlinebox[inccolor]{\textcolor{black}{income effect}}}
\newcommand{\inceffectadj}{\inlinebox[inccolor]{\textcolor{black}{income-effect}}}
\newcommand{\Inceffect}{\inlinebox[inccolor]{\textcolor{black}{Income effect}}}
\newcommand{\IncEffect}{\inlinebox[inccolor]{\textcolor{black}{Income Effect}}}

% Macro effect
\newcommand{\macroeffect}{\inlinebox[macrocolor]{\textcolor{black}{macro effect}}}
\newcommand{\macroeffectadj}{\inlinebox[macrocolor]{\textcolor{black}{macro-effect}}}
\newcommand{\Macroeffect}{\inlinebox[macrocolor]{\textcolor{black}{Macro effect}}}
\newcommand{\MacroEffect}{\inlinebox[macrocolor]{\textcolor{black}{Macro Effect}}}


%%%%% minipage for assumptions and constraints tables
% Arguments:
%   #1: Width (multiple of \linewidth
%   #2: text inside the minipage
%
\newcommand{\mptable}[2]{\begin{minipage}{#1\linewidth} \useshortskip{} \begin{equation} #2 \end{equation} \end{minipage}}


%%%%% Oft-used references

\newcommand{\Ba}[1]{\citeauthor[#1]{Borenstein:2015aa}}
\newcommand{\Bapp}[1]{\citeauthor[#1]{Borenstein:2015aa}'s \citeyearpar{Borenstein:2015aa}}
\newcommand{\Bp}[1]{\citep[#1]{Borenstein:2015aa}}
\newcommand{\Bt}[1]{\citet[#1]{Borenstein:2015aa}}

\newcommand{\Ta}[1]{\citeauthor[#1]{Thomas:2013aa}}
\newcommand{\Tapp}[1]{\citeauthor[#1]{Thomas:2013aa}'s \citeyearpar{Thomas:2013aa}}
\newcommand{\Tp}[1]{\citep[#1]{Thomas:2013aa, Thomas:2013ab}}
\newcommand{\Tpone}[1]{\citep[#1]{Thomas:2013aa}}
\newcommand{\Tptwo}[1]{\citep[#1]{Thomas:2013ab}}
\newcommand{\Tt}[1]{\citet[#1]{Thomas:2013aa, Thomas:2013ab}}
\newcommand{\Ttone}[1]{\citet[#1]{Thomas:2013aa}}
\newcommand{\Tttwo}[1]{\citet[#1]{Thomas:2013ab}}


%%%%% Derivation pages

% Column widths
\newcommand{\derivtextsize}{\footnotesize}
\newcommand{\derivpageleftcolwidth}{0.11\textwidth}
\newcommand{\derivpageenergycolwidth}{0.6\textwidth}
\newcommand{\derivpagefinancialcolwidth}{0.6\textwidth}

% Horizontal rule between sections of derivations

\newcommand{\sectionsep}{\noindent\rule{1.4\textwidth}{0.4pt}}


%
% Derivation section
% Arguments:
%   #1: accounting stage (original, prime, etc.)
%   #2: energy column
%   #3: financial column
%
\newcommand{\derivsection}[3]{%

\derivtextsize{}

\begin{minipage}[t]{\derivpageleftcolwidth}
~\\#1
\end{minipage}
%
%
%
\begin{minipage}[t]{\derivpageenergycolwidth}
#2
\end{minipage}
%
~
%
\begin{minipage}[t]{\derivpagefinancialcolwidth}
#3
\end{minipage}

\normalsize{}

}


%
% Derivation page header
% Arguments:
%   #1: Effect type header text (e.g., Emplacement Effect)
%
\newcommand{\derivheader}[1]{

\begin{center}
  #1
\end{center}

\derivsection{}
{\begin{center}\emph{Energy analysis}\end{center}}
{\begin{center}\emph{Financial analysis}\end{center}}

}


% Equations

% Efficiency ratios
\newcommand{\etaratioinline}{\amacro{\eta}/\bempl{\eta}}
\newcommand{\etaratiostacked}{\frac{\amacro{\eta}}{\bempl{\eta}}}

% Derivative with respect to efficiency ratio
\newcommand{\dbydetaeta}{\frac{\mathrm{d}}{\mathrm{d}(\etaratioinline{})}}


% Original
\newcommand{\Eacctorig}{\rbempl{E} = \rbempl{E}_s + \rbempl{E}_{emb} + (\rbempl{C}_{\omd} + \rbempl{C}_o) I_E}
\newcommand{\Macctorig}{\rate{M} = p_E \rbempl{E}_s + \bempl{R}_\alpha \rbempl{C}_{cap} + \rbempl{C}_{\omd} + \rbempl{C}_o + \rbempl{N}}

% Before emplacement effect (same as original)
\newcommand{\Eacctbempl}{\Eacctorig}      
\newcommand{\Macctbempl}{\Macctorig}      

% After emplacement effect
\newcommand{\Eacctaempl}{\raempl{E} = \raempl{E}_s + \raempl{E}_{emb} + (\raempl{C}_{\omd} + \raempl{C}_o) I_E}                  
\newcommand{\Macctaempl}{\rate{M} = p_E \raempl{E}_s + \aempl{R}_\alpha \raempl{C}_{cap} + \raempl{C}_{\omd} + \raempl{C}_o + \raempl{N}}         

% Before substitution effect (same as after emplacement effect)
\newcommand{\Eacctbsub}{\Eacctaempl}
\newcommand{\Macctbsub}{\Macctaempl}

% After substitution effect
\newcommand{\Eacctasub}{\rasub{E} = \rasub{E}_s + \rasub{E}_{emb} + (\rasub{C}_{\omd} + \rasub{C}_o) I_E}
\newcommand{\Macctasub}{\rate{M} = p_E \rasub{E}_s + \asub{R}_\alpha \rasub{C}_{cap} + \rasub{C}_{\omd} + \rasub{C}_o + \rasub{N}}

% Before income effect (same as after substitution effect)
\newcommand{\Eacctbinc}{\Eacctasub}
\newcommand{\Macctbinc}{\Macctasub}

% After income effect
\newcommand{\Eacctainc}{\rainc{E} = \rainc{E}_s + \rainc{E}_{emb} + (\rainc{C}_{\omd} + \rainc{C}_o) I_E}
\newcommand{\Macctainc}{\rate{M} = p_E \rainc{E}_s + \ainc{R}_\alpha \rainc{C}_{cap} + \rainc{C}_{\omd} + \rainc{C}_o + \rainc{N}}

% Embodied energy rebound
\newcommand{\Reembeqn}{\frac{\left( \frac{\aempl{E}_{emb}}{\bempl{E}_{emb}}
  \frac{\bempl{t}_{\life}}{\aempl{t}_{\life}} - 1 \right) \rbempl{E}_{emb}}{\Sdot}}
  
% Ops, Maintenance, and disposal energy rebound
\newcommand{\ReOMdeqn}{\frac{\left( \frac{\raempl{C}_{\omd}}{\rbempl{C}_{\omd}} - 1 \right) \rbempl{C}_{\omd} I_E}{\Sdot}}

% Equation for S_dot_dev
% \newcommand{\Sdoteqn}{\left( \etaratiostacked - 1 \right)\!\etaratiostacked \rbempl{E}_s}
\newcommand{\Sdoteqn}{\left( \etaratiostacked - 1 \right)\! 
                            \frac{\bempl{\eta}}{\amacro{\eta}} \rbempl{E}_s}

% Equation for Re_dsub
\newcommand{\Redsubeqn}{\frac{\left( \etaratiostacked \right)^{-\eqspsUC} - 1}
                        {\etaratiostacked - 1}}
                        
% Equation for Re_isub
\newcommand{\Reisubeqn}{\frac{{\left( \etaratiostacked  \right)}
                          ^{-\eqopsC} - 1}{\etaratiostacked - 1} \; 
                          \etaratiostacked \; 
                          \frac{\rbempl{C}_o I_E}{\rbempl{E}_s}}
                          
% CES utility equation
\newcommand{\cesutility}{\left[ \fCs \left( \frac{\rate{q}_s}{\rbempl{q}_s} \right)^\rho 
        + (1-\fCs) \left( \frac{\rate{C}_o}{\rbempl{C}_o} \right)^\rho  \right]^{(1/\rho)}}
        
% Equation for q_s_hat/q_s_orig
\newcommand{\qssolution}{\left\{ \fCs + (1-\fCs)
      \left[ \left(  \frac{1-\fCs}{\fCs}  \right) \frac{\amacro{p}_s \rbempl{q}_s}{\rbempl{C}_o}  \right]
                                                  ^{\rho / (1 - \rho)} \right\} ^ {-1/\rho}}

% Equation for C_o_hat/C_o_orig
\newcommand{\Cosolution}{ \left( 1 + \fCs \left\{ \left[ \left( \frac{1-\fCs}{\fCs} \right)
          \frac{\amacro{p}_s \rbempl{q}_s}{\rbempl{C}_o} \right] ^{\rho/(\rho - 1)} - 1 \right\} \right)^{-1/\rho}}

% Equation for Re_dsub for the CES utility model
\newcommand{\RedsubCES}{\frac{\qssolution{} - 1}{\etaratiostacked{} - 1}}

% Equation for Re_isub for the CES utility model
\newcommand{\ReisubCES}{\frac{\Cosolution{} - 1}{\etaratiostacked{} - 1}
                         \etaratiostacked \; 
                          \frac{\rbempl{C}_o I_E}{\rbempl{E}_s}}


% Equation for Re_dinc, approximate method
\newcommand{\Redinceqnapprox}{\frac{ \left( 1 + \frac{\rbinc{N}}{\Mdothatprime} \right) ^{\eqsM} - 1}
              { \etaratiostacked - 1 } \left( \etaratiostacked \right)^{-\eqspsC}}

% Equation for Re_dinc, exact method
\newcommand{\Redinceqnexact}{\frac{ \left( 1 + \frac{\rbinc{N}}{\Mdothatprime} \right) ^{\eqsM} - 1}
              { \etaratiostacked - 1 } \qssolution{} }

% Equation for Re_cap
\newcommand{\Recapeqn}{\frac{\Delta \aempl{(R_\alpha \rate{C}_{cap})} I_E}{\Sdot}}

% Equation for Re_iinc, approximate method
\newcommand{\Reiinceqnapprox}{\frac{\left( 1 + \frac{\rbinc{N}}{\Mdothatprime} \right)^{\eqoM} - 1}{\etaratiostacked - 1} 
              \left( \etaratiostacked \right)^{1 - \eqopsC}
              \frac{\rbempl{C}_o I_E}{\rbempl{E}_s}}

% Equation for Re_iinc, exact method
% \newcommand{\Reiinceqnexact}{\frac{\left( 1 + \frac{\rbinc{N}}{\Mdothatprime} \right)^{\eqoM} - 1}{\etaratiostacked - 1} 
%               \left( \etaratiostacked \right)
%               \left( \frac{\rasub{C}_o}{\rorig{C}_o} \right) 
%               \frac{\rbempl{C}_o I_E}{\rbempl{E}_s}}

\newcommand{\Reiinceqnexact}{\frac{\left( 1 + \frac{\rbinc{N}}{\Mdothatprime} \right)^{\eqoM} - 1}{\etaratiostacked - 1} 
              \left( \etaratiostacked \right)
              \frac{\rbempl{C}_o I_E}{\rbempl{E}_s}
              \Cosolution{}}



% Equation for Re_d (total direct rebound)
\newcommand{\Redeqn}{\frac{ \left( \etaratiostacked \right)^{-\eqspsC}
             \left( 1 + \frac{\rasub{N}}{\rbempl{M}} \right)^{\eqsM}   - 1}
         {\etaratiostacked - 1}}

% Equation for Re_macro
% \newcommand{\Remacroeqn}{k (p_E I_E - Re_{cap} - Re_{\md} - p_E I_E Re_{dsub} - Re_{isub})}
\newcommand{\Remacroeqn}{k (p_E I_E - Re_{cap} - Re_{\omd})}

% Equation for Re_tot
% \newcommand{\Retoteqn}{&Re_{emb} - k Re_{cap} + (1-k) Re_{\md}         \nonumber \\
%                        &+ (1 - k p_E I_E) Re_{dsub} + (1 - k) Re_{isub}   \nonumber \\
%                        &+ Re_{dinc} + Re_{iinc} +  k p_E I_E}
% \newcommand{\Retoteqn}{&Re_{emb} + k (p_E I_E - Re_{cap}) + (1-k) Re_{\md}   \nonumber \\
%                        &+ Re_{dsub} + Re_{isub}                              \nonumber \\
%                        &+ Re_{dinc} + Re_{iinc}}
\newcommand{\Retoteqn}{Re_{emb} + k (p_E I_E - Re_{cap}) + (1-k) Re_{\omd} + Re_{dsub} + Re_{isub} + Re_{dinc} + Re_{iinc}}
                 

%%%% Income preference equations

% Equation for energy service income preferences
\newcommand{\incprefseqn}{\frac{\rainc{q}_s}{\rbinc{q}_s} = \left( 1 + \frac{\rbinc{N}}{\Mdothatprime}  \right) ^{\eqsM}}

% Equation for other goods income preferences
\newcommand{\incprefoeqn}{\frac{\rainc{q}_o}{\rbinc{q}_o} = \left( 1 + \frac{\rbinc{N}}{\Mdothatprime}  \right) ^{\eqoM}}

% Equation for effective income
% \newcommand{\effinceqn}{\Mdothatprime \equiv \rbempl{M} - \rbempl{C}_{cap} - \rbempl{C}_{\md} 
%                         - \rate{G} + p_E \Delta \rbinc{E}_s + \Delta \rbinc{C}_o}
\newcommand{\effinceqn}{\Mdothatprime \equiv \rate{M} - \aempl{R}_\alpha \raempl{C}_{cap} - \raempl{C}_{\omd} - \rasub{N}}


%%%% Budget constraint symbols and equation

\newcommand{\tlife}{t_{li\!f\!e}}
\newcommand{\oneyr}{1\,\mathrm{yr}}
\newcommand{\twoyr}{2\,\mathrm{yr}}
\newcommand{\itlife}{i\,t_{li\!f\!e}}
\newcommand{\iyr}{i\,\mathrm{yr}}

\newcommand{\budgetconstraint}{\rate{M} - \orig{R}_\alpha \rorig{C}_{cap} - \rorig{C}_{\omd} = \orig{p}_E \frac{\rorig{q}_s}{\orig{\eta}} + p_o \rorig{q}_o}


%%%% Proof characters
% Equal sign with question mark above
\DeclareRobustCommand{\questionequal}{\stackrel{?}{=}}




% Segments and lines
% Arguments:
%   #1: left character
%   #2: line color
%   #3: line thickness (e.g., 0.1 mm)
%   #4: right character
% Note that \raisebox{0.9 mm} moves the line up from the baseline.
% Also, \line(1,0){12} gives a horizontal line with length "12" (unknown units!)
% (1, 0) is the slope (1 unit to right, 0 units up).
\newcommand{\seg}[4]{#1\linethickness{#3}\raisebox{0.88 mm}{\textcolor{#2}{\line(1,0){12}}}#4}

% Construction lines
\newcommand{\iicirc}{\seg{$\bempl{\text{i}}$}{black}{0.3 mm}{$\,\bempl{\text{i}}$}}
\newcommand{\iibar}{\seg{$\bmacro{\text{i}}$}{black}{0.3 mm}{$\,\bmacro{\text{i}}$}}
\newcommand{\rr}{\seg{r}{black}{0.1 mm}{r}}
\newcommand{\circcirc}{\seg{$\circ$}{black}{0.1 mm}{$\circ$}}
\newcommand{\starstar}{\seg{$*$}{black}{0.1 mm}{$*$}}
\newcommand{\hathat}{\seg{$\wedge$}{black}{0.1 mm}{$\wedge$}}
\newcommand{\barbar}{\seg{$-\,$}{black}{0.1 mm}{$\, -$}}

% Line segments
\newcommand{\circa}{\seg{$\circ$}{emplcolor}{0.6 mm}{$a$}}
% \newcommand{\ab}{\seg{a}{emplcolor}{0.6 mm}{b}}
\newcommand{\ab}{$a$\tikz[baseline=-0.6ex]\draw [line width=0.6mm,dotted,emplcolor] (0,0) -- (0.45,0);$b$}
\newcommand{\bstar}{\seg{$b\,$}{emplcolor}{0.6 mm}{$*$}}
\newcommand{\starc}{\seg{$*$}{subcolor}{0.6 mm}{$\,c$}}
\newcommand{\chat}{\seg{$c\,$}{subcolor}{0.6 mm}{$\wedge$}}
\newcommand{\hatd}{\seg{$\wedge$}{inccolor}{0.6 mm}{$\,d$}}
\newcommand{\dbar}{\seg{$d$}{inccolor}{0.6 mm}{$\,-$}}
\newcommand{\hatbar}{\seg{$\wedge$}{inccolor}{0.6 mm}{$\,-$}}
\newcommand{\bartilde}{\seg{$- \,$}{macrocolor}{0.6 mm}{$\, \sim$}}


% Rotated text for tables
% See
% https://tex.stackexchange.com/questions/98388/how-to-make-table-with-rotated-table-headers-in-latex/98439#98439
% for details.
\newcommand{\rot}{\rotatebox{90}}


% A "rating" command for filled circles with tikz.
% See 
% https://tex.stackexchange.com/questions/194955/get-partly-filled-circle-symbol-scale-linearly-with-parameter
% for details.
\newcommand{\rating}[2][0.75ex]{%
  \pgfmathsetmacro\th{asin(#2/50-1)}% (theta angle of polar coordinates)
    \tikz{%
      \fill[black] (\th:#1) arc (\th:-180-\th:#1) -- cycle;
      \draw[black, thin, radius=#1] (0,0) circle;
    }%
}.
%%%%%%%%%%%%%%%%%%%%%%%%%%%%%%%%%%%%%%%%%%%%%%%%%%%%%%%%%%%%%%


%%%%% Override Energy Economics macros
\renewcommand{\emph}{\textit}



%%%%% Decorations for symbols

\newcommand{\rate}[1]{\dot{#1}}                    % Rate of a quantity

% Create "after" commands
\newcommand{\orig}[1]{{}{#1}^{\scriptscriptstyle \circ}}
\newcommand{\aempl}[1]{{#1}^*}
\newcommand{\asub}[1]{\hat{#1}}
\newcommand{\ainc}[1]{\bar{#1}}
\newcommand{\amacro}[1]{\tilde{#1}}

% Create the "before" commands
\newcommand{\bempl}[1]{\orig{#1}}
\newcommand{\bsub}[1]{\aempl{#1}}
\newcommand{\binc}[1]{\asub{#1}}
\newcommand{\bmacro}[1]{\ainc{#1}}

% Decoration combinations
% Rates after
\newcommand{\rorig}[1]{\orig{\rate{#1}}}
\newcommand{\raempl}[1]{\aempl{\rate{#1}}}
\newcommand{\rasub}[1]{\asub{\rate{#1}}}
\newcommand{\rainc}[1]{\ainc{\rate{#1}}}
\newcommand{\ramacro}[1]{\amacro{\rate{#1}}}

% Rates before
\newcommand{\rbempl}[1]{\rorig{#1}}
\newcommand{\rbsub}[1]{\raempl{#1}}
\newcommand{\rbinc}[1]{\rasub{#1}}
\newcommand{\rbmacro}[1]{\rainc{#1}}

%%%%% Subscript kerning

% \newcommand{\OM}{O\!M}
% \newcommand{\md}{O\!M\!d}
\newcommand{\md}{md}
\newcommand{\macro}{macr\!o}
\newcommand{\life}{li\!f\!e}

%%%%% Expression kerning

\newcommand{\MPC}{M\!PC}


%%%%% Convenient symbols

\newcommand{\Sdot}{\rate{S}_{dev}}
\newcommand{\Mdothatprime}{\rbinc{M}^\prime}


%%%%% Elasticities and income shares

\newcommand{\eqspsUC}{\varepsilon_{\rate{q}_s\!,p_s}}
\newcommand{\eqopsUC}{\varepsilon_{\rate{q}_o\!,p_s}}
\newcommand{\eqspsC}{\varepsilon_{\rate{q}_s\!,p_s\!,c}}
\newcommand{\eqopsC}{\varepsilon_{\rate{q}_o\!,p_s\!,c}}
\newcommand{\eqsM}{\varepsilon_{\rate{q}_s\!,\rate{M}}}
\newcommand{\eqoM}{\varepsilon_{\rate{q}_o\!,\rate{M}}}
\newcommand{\fCs}{\bempl{f}_{\rate{C}_s}}


%%%%% Colors

% Original spectrum colours
% \colorlet{emplcolor}{red!25!white}
% \colorlet{subcolor}{orange!25!white}
% \colorlet{inccolor}{green!25!white}
% \colorlet{macrocolor}{blue!25!white}

% New Viridis "plasma" colours
\definecolor{emplcolor}{HTML}{150789}
\definecolor{subcolor}{HTML}{99149F}
\definecolor{inccolor}{HTML}{E76F5A}
\definecolor{macrocolor}{HTML}{F7E225}



%%%%% Coloration of text background

%
% Inline color box around text
% Arguments:
%   [#1]: background color for the box
%   {#2}: text inside the box
%
\newtcbox{\inlinebox}[1][]{on line, 
colback=#1,
colframe=#1,
before upper={\rule[-2pt]{0pt}{10pt}},
boxrule=1pt,
boxsep=0pt,
left=3pt,
right=3pt,
top=2pt,
bottom=2pt}


%%%%% Colored phrases

% Emplacement effect
\newcommand{\empleffect}{\inlinebox[emplcolor]{\textcolor{white}{emplacement effect}}}
\newcommand{\empleffectadj}{\inlinebox[emplcolor]{\textcolor{white}{emplacement-effect}}}
\newcommand{\Empleffect}{\inlinebox[emplcolor]{\textcolor{white}{Emplacement effect}}}
\newcommand{\EmplEffect}{\inlinebox[emplcolor]{\textcolor{white}{Emplacement Effect}}}

% Substitution effect
\newcommand{\subeffect}{\inlinebox[subcolor]{\textcolor{white}{substitution effect}}}
\newcommand{\subeffectadj}{\inlinebox[subcolor]{\textcolor{white}{substitution-effect}}}
\newcommand{\Subeffect}{\inlinebox[subcolor]{\textcolor{white}{Substitution effect}}}
\newcommand{\SubEffect}{\inlinebox[subcolor]{\textcolor{white}{Substitution Effect}}}

% Income effect
\newcommand{\inceffect}{\inlinebox[inccolor]{\textcolor{black}{income effect}}}
\newcommand{\inceffectadj}{\inlinebox[inccolor]{\textcolor{black}{income-effect}}}
\newcommand{\Inceffect}{\inlinebox[inccolor]{\textcolor{black}{Income effect}}}
\newcommand{\IncEffect}{\inlinebox[inccolor]{\textcolor{black}{Income Effect}}}

% Macro effect
\newcommand{\macroeffect}{\inlinebox[macrocolor]{\textcolor{black}{macro effect}}}
\newcommand{\macroeffectadj}{\inlinebox[macrocolor]{\textcolor{black}{macro-effect}}}
\newcommand{\Macroeffect}{\inlinebox[macrocolor]{\textcolor{black}{Macro effect}}}
\newcommand{\MacroEffect}{\inlinebox[macrocolor]{\textcolor{black}{Macro Effect}}}


%%%%% minipage for assumptions and constraints tables
% Arguments:
%   #1: Width (multiple of \linewidth
%   #2: text inside the minipage
%
\newcommand{\mptable}[2]{\begin{minipage}{#1\linewidth} \useshortskip{} \begin{equation} #2 \end{equation} \end{minipage}}


%%%%% Oft-used references

\newcommand{\Ba}[1]{\citeauthor[#1]{Borenstein:2015aa}}
\newcommand{\Bapp}[1]{\citeauthor[#1]{Borenstein:2015aa}'s \citeyearpar{Borenstein:2015aa}}
\newcommand{\Bp}[1]{\citep[#1]{Borenstein:2015aa}}
\newcommand{\Bt}[1]{\citet[#1]{Borenstein:2015aa}}

\newcommand{\Ta}[1]{\citeauthor[#1]{Thomas:2013aa}}
\newcommand{\Tapp}[1]{\citeauthor[#1]{Thomas:2013aa}'s \citeyearpar{Thomas:2013aa}}
\newcommand{\Tp}[1]{\citep[#1]{Thomas:2013aa, Thomas:2013ab}}
\newcommand{\Tpone}[1]{\citep[#1]{Thomas:2013aa}}
\newcommand{\Tptwo}[1]{\citep[#1]{Thomas:2013ab}}
\newcommand{\Tt}[1]{\citet[#1]{Thomas:2013aa, Thomas:2013ab}}
\newcommand{\Ttone}[1]{\citet[#1]{Thomas:2013aa}}
\newcommand{\Tttwo}[1]{\citet[#1]{Thomas:2013ab}}


%%%%% Derivation pages

% Column widths
\newcommand{\derivtextsize}{\footnotesize}
\newcommand{\derivpageleftcolwidth}{0.11\textwidth}
\newcommand{\derivpageenergycolwidth}{0.6\textwidth}
\newcommand{\derivpagefinancialcolwidth}{0.6\textwidth}

% Horizontal rule between sections of derivations

\newcommand{\sectionsep}{\noindent\rule{1.4\textwidth}{0.4pt}}


%
% Derivation section
% Arguments:
%   #1: accounting stage (original, prime, etc.)
%   #2: energy column
%   #3: financial column
%
\newcommand{\derivsection}[3]{%

\derivtextsize{}

\begin{minipage}[t]{\derivpageleftcolwidth}
~\\#1
\end{minipage}
%
%
%
\begin{minipage}[t]{\derivpageenergycolwidth}
#2
\end{minipage}
%
~
%
\begin{minipage}[t]{\derivpagefinancialcolwidth}
#3
\end{minipage}

\normalsize{}

}


%
% Derivation page header
% Arguments:
%   #1: Effect type header text (e.g., Emplacement Effect)
%
\newcommand{\derivheader}[1]{

\begin{center}
  #1
\end{center}

\derivsection{}
{\begin{center}\emph{Energy analysis}\end{center}}
{\begin{center}\emph{Financial analysis}\end{center}}

}


% Equations

% Efficiency ratios
\newcommand{\etaratioinline}{\amacro{\eta}/\bempl{\eta}}
\newcommand{\etaratiostacked}{\frac{\amacro{\eta}}{\bempl{\eta}}}

% Derivative with respect to efficiency ratio
\newcommand{\dbydetaeta}{\frac{\mathrm{d}}{\mathrm{d}(\etaratioinline{})}}


% Original
\newcommand{\Eacctorig}{\rbempl{E} = \rbempl{E}_s + \rbempl{E}_{emb} + (\rbempl{C}_{\md} + \rbempl{C}_o) I_E}
\newcommand{\Macctorig}{\rbempl{M} = p_E \rbempl{E}_s + \rbempl{C}_{cap} + \rbempl{C}_{\md} + \rbempl{C}_o + \rbempl{N}}

% Before emplacement effect (same as original)
\newcommand{\Eacctbempl}{\Eacctorig}      
\newcommand{\Macctbempl}{\Macctorig}      

% After emplacement effect
\newcommand{\Eacctaempl}{\raempl{E} = \raempl{E}_s + \raempl{E}_{emb} + (\raempl{C}_{\md} + \raempl{C}_o) I_E}                  
\newcommand{\Macctaempl}{\raempl{M} = p_E \raempl{E}_s + \raempl{C}_{cap} + \raempl{C}_{\md} + \raempl{C}_o + \raempl{N}}         

% Before substitution effect (same as after emplacement effect)
\newcommand{\Eacctbsub}{\Eacctaempl}
\newcommand{\Macctbsub}{\Macctaempl}

% After substitution effect
\newcommand{\Eacctasub}{\rasub{E} = \rasub{E}_s + \rasub{E}_{emb} + (\rasub{C}_{\md} + \rasub{C}_o) I_E}
\newcommand{\Macctasub}{\rasub{M} = p_E \rasub{E}_s + \rasub{C}_{cap} + \rasub{C}_{\md} + \rasub{C}_o + \rasub{N}}

% Before income effect (same as after substitution effect)
\newcommand{\Eacctbinc}{\Eacctasub}
\newcommand{\Macctbinc}{\Macctasub}

% After income effect
\newcommand{\Eacctainc}{\rainc{E} = \rainc{E}_s + \rainc{E}_{emb} + (\rainc{C}_{\md} + \rainc{C}_o) I_E}
\newcommand{\Macctainc}{\rainc{M} = p_E \rainc{E}_s + \rainc{C}_{cap} + \rainc{C}_{\md} + \rainc{C}_o + \rainc{N}}

% Embodied energy rebound
\newcommand{\Reembeqn}{\frac{\left( \frac{\aempl{E}_{emb}}{\bempl{E}_{emb}}
  \frac{\bempl{t}_{\life}}{\aempl{t}_{\life}} - 1 \right) \rbempl{E}_{emb}}{\Sdot}}
  
% Ops, Maintenance, and disposal energy rebound
\newcommand{\ReOMdeqn}{\frac{\left( \frac{\raempl{C}_{\md}}{\rbempl{C}_{\md}} - 1 \right) \rbempl{C}_{\md} I_E}{\Sdot}}

% Equation for S_dot_dev
% \newcommand{\Sdoteqn}{\left( \etaratiostacked - 1 \right)\!\etaratiostacked \rbempl{E}_s}
\newcommand{\Sdoteqn}{\left( \etaratiostacked - 1 \right)\! 
                            \frac{\bempl{\eta}}{\amacro{\eta}} \rbempl{E}_s}

% Equation for Re_dsub
\newcommand{\Redsubeqn}{\frac{\left( \etaratiostacked \right)^{-\eqspsC} - 1}
                        {\etaratiostacked - 1}}
                        
% Equation for Re_isub
\newcommand{\Reisubeqn}{\frac{{\left( \etaratiostacked  \right)}
                          ^{-\eqopsC} - 1}{\etaratiostacked - 1} \; 
                          \etaratiostacked \; 
                          \frac{\rbempl{C}_o I_E}{\rbempl{E}_s}}
                          
% CES utility equation
\newcommand{\cesutility}{\left[ \fCs \left( \frac{\rate{q}_s}{\rbempl{q}_s} \right)^\rho 
        + (1-\fCs) \left( \frac{\rate{C}_o}{\rbempl{C}_o} \right)^\rho  \right]^{(1/\rho)}}
        
% Equation for q_s_hat/q_s_orig
\newcommand{\qssolution}{\left\{ \fCs + (1-\fCs)
      \left[ \left(  \frac{1-\fCs}{\fCs}  \right) \frac{\amacro{p}_s \rbempl{q}_s}{\rbempl{C}_o}  \right]
                                                  ^{\rho / (1 - \rho)} \right\} ^ {-1/\rho}}

% Equation for C_o_hat/C_o_orig
\newcommand{\Cosolution}{ \left( 1 + \fCs \left\{ \left[ \left( \frac{1-\fCs}{\fCs} \right)
          \frac{\amacro{p}_s \rbempl{q}_s}{\rbempl{C}_o} \right] ^{\rho/(\rho - 1)} - 1 \right\} \right)^{-1/\rho}}

% Equation for Re_dsub for the CES utility model
\newcommand{\RedsubCES}{\frac{\qssolution{} - 1}{\etaratiostacked{} - 1}}

% Equation for Re_isub for the CES utility model
\newcommand{\ReisubCES}{\frac{\Cosolution{} - 1}{\etaratiostacked{} - 1}
                         \etaratiostacked \; 
                          \frac{\rbempl{C}_o I_E}{\rbempl{E}_s}}


% Equation for Re_dinc, approximate method
\newcommand{\Redinceqnapprox}{\frac{ \left( 1 + \frac{\rbinc{N}}{\Mdothatprime} \right) ^{\eqsM} - 1}
              { \etaratiostacked - 1 } \left( \etaratiostacked \right)^{-\eqspsC}}

% Equation for Re_dinc, exact method
\newcommand{\Redinceqnexact}{\frac{ \left( 1 + \frac{\rbinc{N}}{\Mdothatprime} \right) ^{\eqsM} - 1}
              { \etaratiostacked - 1 } \qssolution{} }

% Equation for Re_cap
\newcommand{\Recapeqn}{\frac{\Delta \raempl{C}_{cap} I_E}{\Sdot}}

% Equation for Re_iinc, approximate method
\newcommand{\Reiinceqnapprox}{\frac{\left( 1 + \frac{\rbinc{N}}{\Mdothatprime} \right)^{\eqoM} - 1}{\etaratiostacked - 1} 
              \left( \etaratiostacked \right)^{1 - \eqopsC}
              \frac{\rbempl{C}_o I_E}{\rbempl{E}_s}}

% Equation for Re_iinc, exact method
% \newcommand{\Reiinceqnexact}{\frac{\left( 1 + \frac{\rbinc{N}}{\Mdothatprime} \right)^{\eqoM} - 1}{\etaratiostacked - 1} 
%               \left( \etaratiostacked \right)
%               \left( \frac{\rasub{C}_o}{\rorig{C}_o} \right) 
%               \frac{\rbempl{C}_o I_E}{\rbempl{E}_s}}

\newcommand{\Reiinceqnexact}{\frac{\left( 1 + \frac{\rbinc{N}}{\Mdothatprime} \right)^{\eqoM} - 1}{\etaratiostacked - 1} 
              \left( \etaratiostacked \right)
              \frac{\rbempl{C}_o I_E}{\rbempl{E}_s}
              \Cosolution{}}



% Equation for Re_d (total direct rebound)
\newcommand{\Redeqn}{\frac{ \left( \etaratiostacked \right)^{-\eqspsC}
             \left( 1 + \frac{\rasub{N}}{\rbempl{M}} \right)^{\eqsM}   - 1}
         {\etaratiostacked - 1}}

% Equation for Re_macro
% \newcommand{\Remacroeqn}{k (p_E I_E - Re_{cap} - Re_{\md} - p_E I_E Re_{dsub} - Re_{isub})}
\newcommand{\Remacroeqn}{k (p_E I_E - Re_{cap} - Re_{\md})}

% Equation for Re_tot
% \newcommand{\Retoteqn}{&Re_{emb} - k Re_{cap} + (1-k) Re_{\md}         \nonumber \\
%                        &+ (1 - k p_E I_E) Re_{dsub} + (1 - k) Re_{isub}   \nonumber \\
%                        &+ Re_{dinc} + Re_{iinc} +  k p_E I_E}
\newcommand{\Retoteqn}{&Re_{emb} + k (p_E I_E - Re_{cap}) + (1-k) Re_{\md}   \nonumber \\
                       &+ Re_{dsub} + Re_{isub}                              \nonumber \\
                       &+ Re_{dinc} + Re_{iinc}}
                 

%%%% Income preference equations

% Equation for energy service income preferences
\newcommand{\incprefseqn}{\frac{\rainc{q}_s}{\rbinc{q}_s} = \left( 1 + \frac{\rbinc{N}}{\Mdothatprime}  \right) ^{\eqsM}}

% Equation for other goods income preferences
\newcommand{\incprefoeqn}{\frac{\rainc{q}_o}{\rbinc{q}_o} = \left( 1 + \frac{\rbinc{N}}{\Mdothatprime}  \right) ^{\eqoM}}

% Equation for effective income
% \newcommand{\effinceqn}{\Mdothatprime \equiv \rbempl{M} - \rbempl{C}_{cap} - \rbempl{C}_{\md} 
%                         - \rate{G} + p_E \Delta \rbinc{E}_s + \Delta \rbinc{C}_o}
\newcommand{\effinceqn}{\Mdothatprime \equiv \rorig{M} - \raempl{C}_{cap} - \raempl{C}_{\md} - \rasub{N}}
                        



% Segments and lines
% Arguments:
%   #1: left character
%   #2: line color
%   #3: line thickness (e.g., 0.1 mm)
%   #4: right character
% Note that \raisebox{0.9 mm} moves the line up from the baseline.
% Also, \line(1,0){12} gives a horizontal line with length "12" (unknown units!)
% (1, 0) is the slope (1 unit to right, 0 units up).
\newcommand{\seg}[4]{#1\linethickness{#3}\raisebox{0.88 mm}{\textcolor{#2}{\line(1,0){12}}}#4}

% Construction lines
\newcommand{\iicirc}{\seg{$\bempl{\text{i}}$}{black}{0.3 mm}{$\,\bempl{\text{i}}$}}
\newcommand{\iibar}{\seg{$\bmacro{\text{i}}$}{black}{0.3 mm}{$\,\bmacro{\text{i}}$}}
\newcommand{\rr}{\seg{r}{black}{0.1 mm}{r}}
\newcommand{\circcirc}{\seg{$\circ$}{black}{0.1 mm}{$\circ$}}
\newcommand{\starstar}{\seg{$*$}{black}{0.1 mm}{$*$}}
\newcommand{\hathat}{\seg{$\wedge$}{black}{0.1 mm}{$\wedge$}}
\newcommand{\barbar}{\seg{$-\,$}{black}{0.1 mm}{$\, -$}}

% Line segments
\newcommand{\circa}{\seg{$\circ$}{emplcolor}{0.6 mm}{$a$}}
% \newcommand{\ab}{\seg{a}{emplcolor}{0.6 mm}{b}}
\newcommand{\ab}{$a$\tikz[baseline=-0.6ex]\draw [line width=0.6mm,dotted,emplcolor] (0,0) -- (0.45,0);$b$}
\newcommand{\bstar}{\seg{$b\,$}{emplcolor}{0.6 mm}{$*$}}
\newcommand{\starc}{\seg{$*$}{subcolor}{0.6 mm}{$\,c$}}
\newcommand{\chat}{\seg{$c\,$}{subcolor}{0.6 mm}{$\wedge$}}
\newcommand{\hatd}{\seg{$\wedge$}{inccolor}{0.6 mm}{$\,d$}}
\newcommand{\dbar}{\seg{$d$}{inccolor}{0.6 mm}{$\,-$}}
\newcommand{\hatbar}{\seg{$\wedge$}{inccolor}{0.6 mm}{$\,-$}}
\newcommand{\bartilde}{\seg{$- \,$}{macrocolor}{0.6 mm}{$\, \sim$}}


% Rotated text for tables
% See
% https://tex.stackexchange.com/questions/98388/how-to-make-table-with-rotated-table-headers-in-latex/98439#98439
% for details.
\newcommand{\rot}{\rotatebox{90}}


% A "rating" command for filled circles with tikz.
% See 
% https://tex.stackexchange.com/questions/194955/get-partly-filled-circle-symbol-scale-linearly-with-parameter
% for details.
\newcommand{\rating}[2][0.75ex]{%
  \pgfmathsetmacro\th{asin(#2/50-1)}% (theta angle of polar coordinates)
    \tikz{%
      \fill[black] (\th:#1) arc (\th:-180-\th:#1) -- cycle;
      \draw[black, thin, radius=#1] (0,0) circle;
    }%
}
