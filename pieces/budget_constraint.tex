% The next command tells RStudio to do "Compile PDF" on HSB.Rnw,
% instead of this file, thereby eliminating the need to switch back to HSB.Rnw 
% before building the paper.
%!TEX root = ../HSB_framework.Rnw

We assume the device owner has four expense categories 
related to the device:
capital cost ($C_{cap}$), 
energy service cost ($C_s$), 
operations and maintenance cost ($C_{\om}$), and
disposal cost ($C_d$).
We count one expense category for all other
goods and services ($C_g$),
one category for annual income ($M$), and 
net savings ($N$), 
the difference between income and expenses.
Capital ($cap$) and disposal ($d$) costs are applied
at the beginning ($\alpha$) and end ($\omega$), respectively, 
of the device lifetime ($t_{life}$).
All other budget categories are 
applied at the beginning of each year.
A budget can be constructed for the device owner for each stage
of Figure~\ref{fig:flowchart}, 
leading to a different budget 
before emplacement ($\circ$), 
after emplacement ($*$), 
after the substitution effect ($\wedge$), 
after the income effect ($-$), and 
after the macro effect ($\sim$).
When needed, 
the different budgets can be distinguished by symbol decorations
shown in Table~\ref{tab:decorations}.
We allow the device owner to purchase the device 
with a loan and assume a real discount rate $r$.
For a device not purchased on credit,
$r = 0$ applies.
The device owner may save (with real discount rate $r$)
to pay for future disposal costs.

Each budget category is analyzed in perpetuity
to allow comparisons at different rebound stages 
($\circ$, $*$, etc.)
where the device lifetime ($\tlife$) may be different.
The present value ($P$) of each expense category is obtained
with an infinite sum for three cases.

First, the present value ($P_{cap}$) of the capital cost ($C_{cap}$) 
is given by the infinite sum 

\begin{align}
\footnotesize{}
  P_{cap} &= C_{cap} + \frac{C_{cap}}{(1 + r)^{\tlife}} + \frac{C_{cap}}{(1 + r)^{2 \tlife}} +
             \ldots + \frac{C_{cap}}{(1 + r)^{\itlife}} + \ldots 
           = C_{cap} \sum_{i = 0}^{\infty} \frac{1}{(1 + r)^{\itlife}} \nonumber \\
          &= \phi_{\tlife} C_{cap} \; ,
\end{align}
%
where $\phi_t \equiv \frac{(1 + r)^t}{(1 + r)^t - 1}$.

Second, the present value of all yearly expenses or income
can be given by similar equations.
For the example of the present value ($P_s$)
of annual energy services costs ($C_s$),
we have

\begin{align}
\footnotesize{}
  P_s &= C_s + \frac{C_s}{(1 + r)^{\oneyr}} + \frac{C_s}{(1 + r)^{\twoyr}} + 
         \ldots + \frac{C_s}{(1 + r)^{\iyr}} + \ldots 
       = C_s \sum_{i = 0}^{\infty} \frac{1}{(1 + r)^{\iyr}} \nonumber \\
      &= \phi_{\oneyr} C_s \; .
\end{align}
%
Equations for 
the present value of 
annual operations and maintenance costs ($P_{\om}$ and $C_{\om}$), 
annual other goods costs ($P_g$ and $C_g$), 
annual income ($P_M$ and $M$), and 
annual net savings ($P_N$ and $N$)
are identical except for the subscripts.

Finally, the present value ($P_d$) of disposal costs ($C_d$) is given by

\begin{align}
\footnotesize{}
  P_d &= \frac{C_d}{(1 + r)^{\tlife}} + \frac{C_d}{(1 + r)^{2 \tlife}} + 
         \ldots + \frac{C_d}{(1 + r)^{\itlife}} + \ldots 
       = C_d \sum_{i = 1}^{\infty} \frac{1}{(1 + r)^{\itlife}} \nonumber \\
      &= \gamma_{\tlife} C_d \; ,
\end{align}
%
where 
$\gamma_t \equiv \frac{1}{(1 + r)^t - 1}$.



% \begin{align} \label{eq:present_values}
% \footnotesize{}
%   P_{cap} &= C_{cap} + \frac{C_{cap}}{(1 + r)^{\tlife}} + \frac{C_{cap}}{(1 + r)^{2 \tlife}} +
%              \ldots + \frac{C_{cap}}{(1 + r)^{\itlife}} + \ldots 
%            = C_{cap} \sum_{i = 0}^{\infty} \frac{1}{(1 + r)^{\itlife}} \nonumber \\
%           &= \phi_{\tlife} C_{cap} \\
% %
%   P_s &= C_s + \frac{C_s}{(1 + r)^{\oneyr}} + \frac{C_s}{(1 + r)^{\twoyr}} + 
%          \ldots + \frac{C_s}{(1 + r)^{\iyr}} + \ldots 
%        = C_s \sum_{i = 0}^{\infty} \frac{1}{(1 + r)^{\iyr}} \nonumber \\
%       &= \phi_{\oneyr} C_s \\
% %
%   P_{\om} &= C_{\om} + \frac{C_{\om}}{(1 + r)^{\oneyr}} + \frac{C_{\om}}{(1 + r)^{\twoyr}} + 
%              \ldots + \frac{C_{\om}}{(1 + r)^{\iyr}} + \ldots 
%            = C_{\om} \sum_{i = 0}^{\infty} \frac{1}{(1 + r)^{\iyr}} \nonumber \\
%           &= \phi_{\oneyr} C_{\om} \\
% %
%   P_d &= \frac{C_d}{(1 + r)^{\tlife}} + \frac{C_d}{(1 + r)^{2 \tlife}} + 
%          \ldots + \frac{C_d}{(1 + r)^{\itlife}} + \ldots 
%        = C_d \sum_{i = 1}^{\infty} \frac{1}{(1 + r)^{\itlife}} \nonumber \\
%       &= \gamma_{\tlife} C_d \\
% %
%   P_g &= C_g + \frac{C_g}{(1 + r)^{\oneyr}} + \frac{C_g}{(1 + r)^{\twoyr}} + 
%          \ldots + \frac{C_g}{(1 + r)^{\iyr}} + \ldots 
%        = C_g \sum_{i = 0}^{\infty} \frac{1}{(1 + r)^{\iyr}} \nonumber \\
%       &= \phi_{\oneyr} C_g \\
% %
%   P_M &= M + \frac{M}{(1 + r)^{\oneyr}} + \frac{M}{(1 + r)^{\twoyr}} + 
%          \ldots + \frac{M}{(1 + r)^{\iyr}} + \ldots
%        = M \sum_{i = 0}^{\infty} \frac{1}{(1 + r)^{\iyr}} \nonumber \\
%       &= \phi_{\oneyr} M \\
% %
%   P_N &= N + \frac{N}{(1 + r)^{\oneyr}} + \frac{N}{(1 + r)^{\twoyr}} + 
%          \ldots + \frac{N}{(1 + r)^{\iyr}} + \ldots
%        = N \sum_{i = 0}^{\infty} \frac{1}{(1 + r)^{\iyr}} \nonumber \\
%       &= \phi_{\oneyr} N
% \end{align}
% %
% where $\phi_t \equiv \frac{(1 + r)^t}{(1 + r)^t - 1}$
% and $\gamma_t \equiv \frac{1}{(1 + r)^t - 1}$.

For simplicity, we desire annual values ($A$)
with equivalent present value for each cost category. 
Using the capital cost to illustrate, 
we begin with the present value equivalence of the infinite series and 
annual costs:

\begin{equation}
  P_{cap} = P_{A_{cap}} \; .
\end{equation}
%
Substituting expressions for present values ($P$) gives

\begin{equation}
  \phi_{t_{life}} C_{cap} = \phi_{1\,\mathrm{yr}} A_{cap} \; .
\end{equation}
%
Rearranging gives

\begin{equation}
  A_{cap} = \frac{\phi_{\tlife}}{\phi_{\oneyr}} C_{cap} \; .
\end{equation}
%
Further, we desire annualized rates defined as
$\rate{A} \equiv A/\oneyr$ such that
$\rate{A}_{cap} = A_{cap}/\oneyr$
and 
$\rate{C}_{cap} \equiv C_{cap}/\tlife$.
Solving for $A_{cap}$ and $C_{cap}$ and substituting gives

\begin{equation}
  \rate{A}_{cap} (\oneyr) = \frac{\phi_{\tlife}}{\phi_{\oneyr}} \rate{C}_{cap} \tlife \; .
\end{equation}
%
Defining $\tau_\alpha \equiv \frac{\phi_{\tlife}}{\phi_{\oneyr}} \frac{\tlife}{\oneyr}$ 
(with subscript $\alpha$ indicating payments at the beginning of each 
device lifetime) gives

\begin{equation}
  \rate{A}_{cap} = \tau_\alpha \rate{C}_{cap} \; .
\end{equation}

Similar derivations can be employed for all other budget categories
to obtain

\begin{align}
  \rate{A}_s &= \rate{C}_s \\
  \rate{A}_{\om} &= \rate{C}_{\om} \\
  \rate{A}_d &= \tau_\omega \rate{C}_d \\
  \rate{A}_g &= \rate{C}_g \\
  \rate{A}_N &= \rate{N} \\
  \rate{A}_M &= \rate{M} \; ,
\end{align}
%
where 
$\tau_\omega \equiv \frac{\gamma_{\tlife}}{\phi_{\oneyr}} \frac{\tlife}{\oneyr}$
(with subscript $\omega$ indicating payments 
at the end of each device lifetime), and 
$\rate{C}_d \equiv C_d / t_{\life}$, the annualized disposal cost 
without discounting.

The budget constraint expressed in annualized present-value equivalent terms is 
%
\begin{equation}
  \rate{A}_M = \rate{A}_{cap} + 
               \rate{A}_s + 
               \rate{A}_{\om} + 
               \rate{A}_d + 
               \rate{A}_g + 
               \rate{A}_N \; .
\end{equation}
%
Substituting cost rates gives

\begin{equation}
  \rate{M} = \tau_\alpha \rate{C}_{cap} + 
             \rate{C}_s + 
             \rate{C}_{\om} + 
             \tau_\omega \rate{C}_d + 
             \rate{C}_g + 
             \rate{N} \; .
\end{equation}
%
Substituting the product of energy price ($p_E$) and
the rate of energy consumption
(given by the ratio of the rate of energy service consumption
and efficiency, $\rate{q}_s / \eta$),
the product of price ($p_g$) and
the rate ($\rate{q}_g$) of other goods consumption,
$\rorig{C}_{\omd} \equiv \rorig{C}_{\om} + \orig{\tau}_\omega \rorig{C}_d$,
and after some rearranging gives the budget constraint
used in Eq.~(\ref{eq:budget_constraint}):

\begin{equation}
  \rate{M} - \tau_\alpha \rate{C}_{cap} - \rate{C}_{\omd} = p_E \frac{\rate{q}_s}{\eta} + p_g \rate{q}_g + \rate{N} \; .
\end{equation}

The term $\tau_\alpha$ represents the additional cost of annual interest
payments when the device is purchased with a loan.
When $r > 0$, $\tau_\alpha > 1$.
When $r = 0$, $\tau_\alpha = 1$, as proved below
(Section~\ref{sec:proof_R_alpha}).
The term $\tau_\omega$ represents the reduction of disposal costs 
if the device owner pays for disposal costs with money
invested annually assuming real discount rate $r$. 
When $r > 0$, $0 < \tau_\omega < 1$.
When $r = 0$, $\tau_\omega = 1$, as proved below
(Section~\ref{sec:proof_R_omega}).


%------------------------------
\subsubsection{Proof: $\tau_\alpha = 1$ when $r = 0$}
\label{sec:proof_R_alpha}
%------------------------------

We expect that $\tau_\alpha = 1$ when $r = 0$.
However, direct substitution of $r = 0$ into the expression
for $\tau_\alpha$ gives $\frac{0}{0}$, 
so we rather assess
$\lim_{r \to 0^+} \tau_\alpha \questionequal 1$.

Substituting for $\tau_\alpha$ gives

\begin{equation}
  \lim_{r \to 0^+} \left( \frac{\phi_{\tlife}}{\phi_{\oneyr}} \frac{\tlife}{\oneyr} \right) 
  \questionequal 1 \; .
\end{equation}
%
Substituting for $\phi$ terms gives

\begin{equation}
  \lim_{r \to 0^+} \left[ \frac{\frac{(1 + r)^{\tlife}}{(1 + r)^{\tlife} - 1}}{\frac{(1 + r)^{\oneyr}}{(1 + r)^{\oneyr} - 1}} \cdot \frac{\tlife}{\oneyr} \right] \questionequal 1 \; .
\end{equation}
%
Distributing double-fractions gives

\begin{equation}
  \lim_{r \to 0^+} \left[
  \frac{(1 + r)^{\tlife}}{(1 + r)^{\oneyr}} \cdot
  \frac{(1 + r)^{\oneyr} - 1}{(1 + r)^{\tlife} - 1} \cdot
  \frac{\tlife}{\oneyr}
  \right] \questionequal 1 \; .
\end{equation}
%
Multiplying terms in numerator and demoninator gives

\begin{equation}
  \lim_{r \to 0^+} \left\{
  \frac{\left[(1 + r)^{\tlife} (1 + r)^{\oneyr} - (1 + r)^{\tlife} \right] \frac{\tlife}{\oneyr}}{(1 + r)^{\tlife} (1 + r)^{\oneyr} - (1 + r)^{\oneyr}}
  \right\} \questionequal 1 \; .
\end{equation}
%
Applying L'H\^{o}pital's rule gives

\begin{equation}
  \lim_{r \to 0^+} \left(
  \frac{\frac{\partial}{\partial r} \left\{ \left[(1 + r)^{\tlife} (1 + r)^{\oneyr} - (1 + r)^{\tlife} \right] \frac{\tlife}{\oneyr} \right\}}
  {\frac{\partial}{\partial r} \left[ (1 + r)^{\tlife} (1 + r)^{\oneyr} - (1 + r)^{\oneyr}\right]}
  \right) \questionequal 1 \; .
\end{equation}
%
Applying the chain rule repeatedly gives

\begin{equation}
  \lim_{r \to 0^+} \left(
  \frac{\frac{\tlife}{\oneyr} 
      \left\{ \frac{\partial}{\partial r} \left[  (1 + r)^{\tlife} (1 + r)^{\oneyr} \right]  - 
              \frac{\partial}{\partial r} \left[  (1 + r)^{\tlife} \right] 
      \right\}}
  {\frac{\partial}{\partial r} \left[    (1 + r)^{\tlife} (1 + r)^{\oneyr} \right]  - 
              \frac{\partial}{\partial r} \left[  (1 + r)^{\oneyr} \right] }
  \right) \questionequal 1 \; .
\end{equation}

Several intermediate results are helpful.

\begin{equation}
  \lim_{r \to 0^+} \left\{\frac{\partial}{\partial r} \left[ (1 + r)^{\tlife} \right] \right\} = \tlife
\end{equation}

\begin{equation}
  \lim_{r \to 0^+} \left\{\frac{\partial}{\partial r} \left[ (1 + r)^{\oneyr} \right] \right\} = \oneyr
\end{equation}

\begin{equation}
  \lim_{r \to 0^+} \left\{\frac{\partial}{\partial r} \left[ (1 + r)^{\tlife} (1 + r)^{\oneyr} \right] \right\} = \tlife (1 + r)^{\oneyr}  + \oneyr (1 + r)^{\tlife} 
\end{equation}

Substituting the intermediate results gives

\begin{equation}
  \lim_{r \to 0^+} \left\{
                         \frac{\frac{\tlife}{\oneyr}
                         \left[ (1 + r)^{\oneyr} (\tlife) + 
                         (1 + r)^{\tlife} (\oneyr) - 
                         \tlife \right]}
                        {(1 + r)^{\oneyr} (\tlife) + 
                         (1 + r)^{\tlife} (\oneyr) - 
                         \oneyr}
                    \right\} \questionequal 1 \; .
\end{equation}
%
Setting $r = 0$ in the remaining terms gives

\begin{equation}
  \frac{\frac{\tlife}{\oneyr}
              \left[ (1) (\tlife) + 
                     (1) (\oneyr) - 
                     \tlife
              \right]}
       {(1) (\tlife) + 
        (1) (\oneyr) - 
        \oneyr} \questionequal 1 \; .
\end{equation}
%
Simplifying gives

\begin{align}
  \frac{\left( \frac{\tlife}{\oneyr} \right) (\oneyr)}{\tlife} &\questionequal 1 \\
  1 &\stackrel{\checkmark}{=} 1 \; ,
\end{align}
%
thereby completing the proof with the expected result. 



%------------------------------
\subsubsection{Proof: $\tau_\omega = 1$ when $r = 0$}
\label{sec:proof_R_omega}
%------------------------------

We expect that $\tau_\omega = 1$ when $r = 0$.
However, direct substitution of $r = 0$ into the expression
for $\tau_\omega$ gives $\frac{0}{0}$, 
so we rather assess
$\lim_{r \to 0^+} \tau_\omega \questionequal 1$.

Substituting for $\tau_\omega$ gives

\begin{equation}
  \lim_{r \to 0^+} \left( \frac{\gamma_{\tlife}}{\phi_{\oneyr}} \frac{\tlife}{\oneyr} \right) \questionequal 1 \; .
\end{equation}
%
Substituting for $\gamma$ and $\phi$ terms gives

\begin{equation}
  \lim_{r \to 0^+} \left[ \frac{\frac{1}{(1 + r)^{\tlife} - 1}}{\frac{(1 + r)^{\oneyr}}{(1 + r)^{\oneyr} - 1}} \frac{\tlife}{\oneyr} \right] \questionequal 1 \; .
\end{equation}
%
Distributing double-fractions gives

\begin{equation}
  \lim_{r \to 0^+} \left[
  \frac{1}{(1 + r)^{\oneyr}} \cdot
  \frac{(1 + r)^{\oneyr} - 1}{(1 + r)^{\tlife} - 1} \cdot
  \frac{\tlife}{\oneyr}
  \right] \questionequal 1 \; .
\end{equation}
%
Multiplying terms in numerator and demoninator gives

\begin{equation}
  \lim_{r \to 0^+} 
  \left\{
    \frac{\left[ (1 + r)^{\oneyr} - 1 \right] \left( \frac{\tlife}{\oneyr} \right)}
    {(1 + r)^{\tlife} (1 + r)^{\oneyr} - (1 + r)^{\oneyr}} 
  \right\} \questionequal 1 \; .
\end{equation}
%
Applying L'H\^{o}pital's rule gives

\begin{equation}
  \lim_{r \to 0^+} 
  \left\{
    \frac{\frac{\tlife}{\oneyr}
      \frac{\partial}{\partial r} 
        \left[ (1 + r)^{\oneyr} - 1 \right]}
    {\frac{\partial}{\partial r} \left[ (1 + r)^{\tlife} (1 + r)^{\oneyr} \right] - 
     \frac{\partial}{\partial r} \left[ (1 + r)^{\oneyr} \right]}
  \right\} \questionequal 1 \; .
\end{equation}
%
Applying the intermediate results from Section~\ref{sec:proof_R_alpha} yields

\begin{equation}
  \lim_{r \to 0^+} 
  \left[
    \frac{\left(\frac{\tlife}{\oneyr}\right) (\oneyr)}
         {(1 + r)^{\oneyr} (\tlife) + 
          (1 + r)^{\tlife} (\oneyr) - 
          \oneyr}
  \right] \questionequal 1 \; .
\end{equation}
%
Setting $r = 0$ in the remaining terms gives

\begin{equation}
  \frac{\left( \frac{\tlife}{\oneyr} \right) (\oneyr)}
  {(1) (\tlife) + (1) \oneyr - \oneyr} \questionequal 1 \; .
\end{equation}
%
Simplifying the denominator gives

\begin{align}
  \frac{\left( \frac{\tlife}{\oneyr} \right) (\oneyr)}
       {\tlife} \questionequal 1 \\
       1 \stackrel{\checkmark}{=} 1 \; ,
\end{align}
%
thereby completing the proof with the expected result.
