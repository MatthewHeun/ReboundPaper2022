% The next command tells RStudio to do "Compile PDF" on HSB.Rnw,
% instead of this file, thereby eliminating the need to switch back to HSB.Rnw 
% before building the paper.
%!TEX root = ../HSB_framework.Rnw

We assume the device owner has four expense categories 
related to the device:
capital cost ($C_{cap}$), 
energy service cost ($C_s$), 
operations and maintenance cost ($C_{om}$), and
end of life disposal cost ($C_d$).
We count one expense category for all other
goods and services ($C_o$),
one category for annual income ($M$), and 
a way to account net savings ($N$), 
the difference between income and expenses.
Capital ($cap$) and disposal ($d$) costs are applied
at the beginning and end, respectively, 
of the device lifetime ($t_{life}$).
All other budget categories are 
applied at the beginning of each year.
A budget can be constructed for the device owner for each stage
of Figure~\ref{fig:flowchart}, 
leading to a different budget 
before emplacement ($\circ$), 
after emplacement ($*$), 
after the substitution effect ($\wedge$), 
after the income effect ($-$), and 
after the macro effect ($\sim$).
The different budgets are distinguished by symbol decorations.
We allow the device owner to purchase the device 
with a loan with real interest rate $r$.
For device not purchased on credit,
$r = 0$ can be assumed.

Each budget category is analyzed in perpetuity
to allow comparisons at different rebound stages 
($\circ$, $*$, etc.).
The present value ($P$) of each expense category is obtained
with an infinite sum as follows.


\begin{equation} \label{eq:present_values}
  P_{C_{cap}} = C_{cap} + \frac{C_{cap}}{(1 + r)^{t_{life}}} + \frac{C_{cap}}{(1 + r)^{2 t_{life}}} + \ldots \\
  P_{C_s} = C_s} + \frac{C_s}{(1 + r)^{\mathrm{1\,yr}}} + \frac{C_s}{(1 + r)^{2 t_{life}}} + \ldots 
\end{equation}
%



