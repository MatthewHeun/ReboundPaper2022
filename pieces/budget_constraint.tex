% The next command tells RStudio to do "Compile PDF" on HSB.Rnw,
% instead of this file, thereby eliminating the need to switch back to HSB.Rnw 
% before building the paper.
%!TEX root = ../HSB_framework.Rnw

We assume the device owner has four expense categories 
related to the device:
capital cost ($C_{cap}$), 
energy service cost ($C_s$), 
operations and maintenance cost ($C_{om}$), and
end of life disposal cost ($C_d$).
We count one expense category for all other
goods and services ($C_o$),
one category for annual income ($M$), and 
a way to account net savings ($N$), 
the difference between income and expenses.
Capital ($cap$) and disposal ($d$) costs are applied
at the beginning and end, respectively, 
of the device lifetime ($t_{life}$).
All other budget categories are 
applied at the beginning of each year.
A budget can be constructed for the device owner for each stage
of Figure~\ref{fig:flowchart}, 
leading to a different budget 
before emplacement ($\circ$), 
after emplacement ($*$), 
after the substitution effect ($\wedge$), 
after the income effect ($-$), and 
after the macro effect ($\sim$).
The different budgets are distinguished by symbol decorations.
We allow the device owner to purchase the device 
with a loan with real interest rate $r$.
For device not purchased on credit,
$r = 0$ can be assumed.

Each budget category is analyzed in perpetuity
to allow comparisons at different rebound stages 
($\circ$, $*$, etc.).
The present value ($P$) of each expense category is obtained
with an infinite sum as follows

\begin{align} \label{eq:present_values}
  P_{cap} &= C_{cap} + \frac{C_{cap}}{(1 + r)^{\tlife}} + \frac{C_{cap}}{(1 + r)^{2 \tlife}} + \ldots 
  = C_{cap} \sum_{k = 0}^{\infty} \frac{1}{(1 + r)^{k \tlife}} 
  = \phi_{\tlife} C_{cap} \\
%
  P_s &= C_s + \frac{C_s}{(1 + r)^{\oneyr}} + \frac{C_s}{(1 + r)^{\twoyr}} + \ldots
  = C_s \sum_{k = 0}^{\infty} \frac{1}{(1 + r)^{\kyr}}
  = \phi_{\oneyr} C_s \\
%
  P_{om} &= C_{om} + \frac{C_{om}}{(1 + r)^{\oneyr}} +  \frac{C_{om}}{(1 + r)^{\twoyr}} + \ldots
  = C_{om} \sum_{k = 0}^{\infty} \frac{1}{(1 + r)^{\kyr}}
  = \phi_{\oneyr} C_{om} \\
%
  P_d &= \frac{C_d}{(1 + r)^{\tlife}} + \frac{C_d}{(1 + r)^{2 \tlife}} + \ldots 
  = C_d \sum_{k = 1}^{\infty} \frac{1}{(1 + r)^{k \tlife}} 
  = \gamma_{\tlife} C_d \\
%
  P_o &= C_o + \frac{C_o}{(1 + r)^{\oneyr}} + \frac{C_o}{(1 + r)^{\twoyr}} + \ldots
  = C_o \sum_{k = 0}^{\infty} \frac{1}{(1 + r)^{\kyr}}
  = \phi_{\oneyr} C_o \\
%
  P_M &= M + \frac{M}{(1 + r)^{\oneyr}} + \frac{M}{(1 + r)^{\twoyr}} + \ldots
  = M \sum_{k = 0}^{\infty} \frac{1}{(1 + r)^{\kyr}}
  = \phi_{\oneyr} M \\
%
  P_N &= N + \frac{N}{(1 + r)^{\oneyr}} + \frac{N}{(1 + r)^{\twoyr}} + \ldots
  = N \sum_{k = 0}^{\infty} \frac{1}{(1 + r)^{\kyr}}
  = \phi_{\oneyr} N
\end{align}
%
where $\phi_t \equiv \frac{(1 + r)^t}{(1 + r)^t - 1}$
and $\gamma \equiv \frac{1}{(1 + r)^t - 1}$.

For simplicity, we desire annual values ($A$)
with equivalent present value for each cost category. 
Using the capital costs to illustrate, 
we begin with the equivalence of the infinite series and 
annual costs.

\begin{equation}
  P_{cap} = P_{A_{cap}}
\end{equation}
%
Substituting expressions for present values ($P$) gives

\begin{equation}
  \phi_{t_{life}} C_{cap} = \phi_{1\,\mathrm{yr}} C_{cap} \, .
\end{equation}
%
Rearranging gives

\begin{equation}
  A_{cap} = \frac{\phi_{\tlife}}{\phi_{\oneyr}} C_{cap} \, .
\end{equation}
%
Further, we desire annualized rates defined as
$\rate{A} \equiv A/\oneyr$ and 
$\rate{C}_{cap} \equiv C_{cap}/\tlife$.
Solving for $A$ and $C_{cap}$ and substituting gives

\begin{equation}
  \rate{A} (\oneyr) = \frac{\phi_{\tlife}}{\phi_{\oneyr}} \rate{C}_{cap} \tlife \, .
\end{equation}
%
Defining $R_{\tlife,b} \equiv \frac{\phi_{\tlife}}{\phi_{\oneyr}} \frac{\tlife}{\oneyr}$ gives

\begin{equation}
  \rate{A}_{cap} = R_{\tlife,b} \rate{C}_{cap} \, .
\end{equation}

Similar derivations can be employed for all budget categories.

\begin{align}
  \rate{A}_s &= \rate{C}_s \\
  \rate{A}_{om} &= \rate{C}_{om} \\
  \rate{A}_d &= R_{\tlife,e} \rate{C}_d \\
  \rate{A}_o &= \rate{C}_o \\
  \rate{A}_M &= \rate{M} \\
  \rate{A}_N &= \rate{N}
\end{align}
%
where $R_{\tlife,e} \equiv \frac{\gamma_{\tlife}}{\phi_{\oneyr}} \frac{\tlife}{\oneyr}$.

The budget constraint with annualized terms is given as
%
\begin{equation}
  \rate{A}_M = \rate{A}_{cap} + 
               \rate{A}_s + 
               \rate{A}_{om} + 
               \rate{A}_d + 
               \rate{A}_o + 
               \rate{A}_N \, .
\end{equation}
%
Substituting cost terms gives

\begin{equation}
  \rate{M} = R_{\tlife,b} \rate{C}_{cap} + 
             \rate{C}_s + 
             \rate{C}_{om} + 
             R_{\tlife,e} \rate{C}_d + 
             \rate{C}_o + 
             \rate{N} \, .
\end{equation}
%
Substituting $\rate{C}_s = p_s \rate{q}_s$ and 
$\rate{C}_o = p_o \rate{q}_o$ and rearranging gives
the budget constraint used in this paper.

\begin{equation}
  \budgetconstraint
\end{equation}

The term $R_{\tlife,b}$ represents the additional cost of interest
payments when the device is purchased with a loan.
When $r > 0$, $R_{\tlife,b} > 1$.
When $r = 0$, $R_{\tlife,b} = 1$, as proved below.

The term $R_{\tlife,e}$ represents the reduction of disposal costs 
if the device owner pays for disposal costs with money
invested at real interest rate $r$. 
When $r > 0$, $0 < R_{\tlife,e} < 1$.
When $r = 0$, $R_{\tlife,e} = 0$, as proved below.

%------------------------------
\subsubsection{Proof that $R_{\tlife,b} = 1$ when $r = 0$}
%------------------------------




%------------------------------
\subsubsection{Proof that $R_{\tlife,e} = 1$ when $r = 0$}
%------------------------------
