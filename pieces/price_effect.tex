% The next command tells RStudio to do "Compile PDF" on HSB_framework.Rnw,
% instead of this file, thereby eliminating the need to switch back to HSB_framework.Rnw 
% before building the paper.
%!TEX root = ../HSB_framework.Rnw


**** Price effect derivation here. ****

**** Serves as an example of the expandability 
of the framework presented herein. ****


The energy price effect is caused by a reduction 
in energy price ($p_E$)
that can occur when implementation of an
energy efficiency upgrade (EEU)
leads to an economy-wide reduction in energy demand.
Reduced demand leads to a lower energy price ($p_E$), 
thus the device owner spends less on energy purchases
to run the device.
The device owner's additional freed cash
is assumed to be spent on other goods and services
with energy implications 
at the energy intensity of the economy ($I_E$).
This appendix derives an expression for an
energy price-effect rebound (Eq.~\ref{eq:Re_pE})
shown in Section~\ref{sec:notes_price_effect}.
This derivation and our assessment of the energy price effect
illustrates the flexibility and extinsibility of the framework,

The derivation begins with an equation for the new economy-wide 
demand for energy ($\rainc{Q}_E$) after the EEU:

\begin{equation}
  \rainc{Q}_E = \rbempl{Q}_E - f_{\EEU} N_{dev} \rbempl{E}_s \left( 1 - \frac{\rainc{E}_s}{\rbempl{E}_s} \right) \, ,
\end{equation}
%
where 
$\rate{Q}_E$ is the rate of economy-wide demand for energy in MJ/year,
$f_{\EEU}$ is the fraction of devices upgraded across the economy
(i.e., the penetration of the EEU), 
$N_{dev}$ is the number of devices in service, and
$\rate{E}_s$ is the rate of energy consumption by a single device in MJ/dev$\cdot$year.
The decorations ``$\circ$'' and ``$-$'' have the usual meanings
provided in Fig.~\ref{fig:flowchart}, namely that
``$\circ$'' indicates the original, pre-EEU device, and 
``$-$'' indicates conditions after 
emplacement, substitution, and income
adjustments.
The ratio between 
new ($\rainc{Q}_E$) and
pre-EEU ($\rbempl{Q}_E$) 
demand for energy is given by

\begin{equation}
  \frac{\rainc{Q}_E}{\rbempl{Q}_E} =
        \frac{\rbempl{Q}_E - f_{\EEU} N_{dev} \rbempl{E}_s \left( 1 - \frac{\rainc{E}_s}{\rbempl{E}_s}  \right)}
        {\rbempl{Q}_E} \, .
\end{equation}
%
Simplifying gives

\begin{equation} \label{eq:QE_ratio}
  \frac{\rainc{Q}_E}{\rbempl{Q}_E} =
        1 - f_{\EEU} \frac{N_{dev} \rbempl{E}_s}{\rbempl{Q}_E} \left( 1 - \frac{\rainc{E}_s}{\rbempl{E}_s}  \right) \, .
\end{equation}
%
Note that the group $\frac{N_{dev} \rorig{E}_s}{\rorig{Q}_E}$
is the original (pre-EEU) fraction of all energy production
(of the kind used by the device)
consumed by all such devices throughout the economy.

The relationship between energy price ($p_E$) and 
economy-wide energy demand ($\rate{Q}_E$) 
can be given by an elasticity relationship

\begin{equation}
  \frac{\rainc{Q}_E}{\rorig{Q}_E} =
        \left( \frac{\ainc{p}_E}{\orig{p}_E} \right) ^ {\eQEpE} \, ,
\end{equation}
%
where $\eQEpE$ is the energy price ($p_e$) elasticity 
of economy-wide energy consumption ($\rate{Q}_E$).

To assess the effect of demand reduction 
($\rbempl{Q}_E > \rainc{Q}_E$)
on price
($\orig{p}_E > \ainc{p}_E$),
we solve for $\frac{\ainc{p}_E}{\orig{p}_E}$ 
to obtain

\begin{equation}
  \frac{\ainc{p}_E}{\orig{p}_E} = 
        \left( \frac{\rainc{Q}_E}{\rorig{Q}_E} \right)  ^ {\frac{1}{\eQEpE}} \, .
\end{equation}
%
Substituting Eq.~(\ref{eq:QE_ratio}) gives

\begin{equation}
  \frac{\ainc{p}_E}{\orig{p}_E} = 
        \left[ 1 - f_{\EEU} \frac{N_{dev} \rbempl{E}_s}{\rbempl{Q}_E} \left( 1 - \frac{\rainc{E}_s}{\rbempl{E}_s}  \right) \right]  ^ {\frac{1}{\eQEpE}} \, .
\end{equation}

The energy price reduction ($\orig{p}_E > \ainc{p}_E$)
leads to additional freed cash for the device owner in the amount of
$\rainc{E}_s (\orig{p}_E - \ainc{p}_E)$.
The energy implications of spending the additional freed cash
on other goods and services is
$\rainc{E}_s (\orig{p}_E - \ainc{p}_E) I_E$, 
another energy takeback rate.
By Eq.~(\ref{eq:Re_takeback}), 
rebound associated with this takeback 
can be written as

\begin{equation}
  Re_{E_p} = \frac{\rainc{E}_s (\orig{p}_E - \ainc{p}_E) I_E}{\Sdot} \, .
\end{equation}
%
Rearranging gives

\begin{equation}
  Re_{p_E} = \frac{\left(1 - \frac{\ainc{p}_E}{\bempl{p}_E}\right) \orig{p}_E \rainc{E}_s I_E}{\Sdot} \, ,  \tag{\ref{eq:Re_pE}}
\end{equation}
%
as shown in Section~\ref{sec:notes_price_effect}, 
thus completing the derivation.



