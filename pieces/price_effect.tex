% The next command tells RStudio to do "Compile PDF" on HSB_framework.Rnw,
% instead of this file, thereby eliminating the need to switch back to HSB_framework.Rnw 
% before building the paper.
%!TEX root = ../HSB_framework.Rnw


The energy price effect is caused by a reduction 
in energy price ($p_E$)
that can occur when widespread implementation 
of an energy efficiency upgrade (EEU)
leads to an economy-wide reduction in energy demand.
Reduced demand leads via demand-supply rebalancing to a lower energy price ($p_E$).
Thus the device owner spends less on energy purchases
to operate the device.
The device owner's additional freed cash
is assumed to be spent on other goods and services
with energy implications 
at the energy intensity of the economy ($I_E$).
This appendix derives an expression for an
energy price-effect rebound (Eq.~\ref{eq:Re_pE})
shown in Section~\ref{sec:notes_price_effect}.
This derivation and our assessment of the magnitude of the energy price effect
in Part~II
illustrate the flexibility and extinsibility of the framework.

The derivation begins with an equation for the new economy-wide 
demand for energy ($\rainc{Q}_E$) after the EEU:

\begin{equation}
  \rainc{Q}_E = \rbempl{Q}_E - f_{\EEU} N_{dev} \rbempl{E}_s \left( 1 - \frac{\rainc{E}_s}{\rbempl{E}_s} \right) \, ,
\end{equation}
%
where
$\rate{Q}_E$ is the rate of economy-wide demand for energy in MJ/year,
$f_{\EEU}$ is the fraction of devices upgraded across the economy
(i.e., the penetration of the EEU),
$N_{dev}$ is the number of devices in service, and
$\rate{E}_s$ is the rate of energy consumption by a single device in MJ/dev$\cdot$year.
The decorations ``$\circ$'' and ``$-$'' have the usual meanings
provided in Fig.~\ref{fig:flowchart}, namely that
``$\circ$'' indicates the original, pre-EEU device, and
``$-$'' indicates conditions for the device owner after
emplacement, substitution, and income
adjustments.
The ratio between
new ($\rainc{Q}_E$) and
pre-EEU ($\rbempl{Q}_E$)
energy demand is given by

\begin{equation}
  \frac{\rainc{Q}_E}{\rbempl{Q}_E} =
        \frac{\rbempl{Q}_E - f_{\EEU} N_{dev} \rbempl{E}_s \left( 1 - \frac{\rainc{E}_s}{\rbempl{E}_s}  \right)}
        {\rbempl{Q}_E} \; .
\end{equation}
%
Simplifying gives

\begin{equation} \label{eq:QE_ratio}
  \frac{\rainc{Q}_E}{\rbempl{Q}_E} =
        1 - f_{\EEU} \frac{N_{dev} \rbempl{E}_s}{\rbempl{Q}_E} \left( 1 - \frac{\rainc{E}_s}{\rbempl{E}_s}  \right) \; .
\end{equation}
%
Note that the group $\frac{N_{dev} \rorig{E}_s}{\rorig{Q}_E}$
is the original (pre-EEU) fraction of all energy production
(of the kind used by the device)
consumed by all such devices throughout the economy.

The relationship between energy price ($p_E$) and
economy-wide energy demand ($\rate{Q}_E$)
can be given by an elasticity relationship

% Note to self: Do not include spaces around the "^"
% character in the next equation.
% Those spaces are not parsed correctly by latexdiff.
\begin{equation}
  \frac{\rainc{Q}_E}{\rorig{Q}_E} = 
          \left( \frac{\ainc{p}_E}{\orig{p}_E} \right)^{\eQEpE} \, ,
\end{equation}
%
where $\eQEpE$ is the energy price ($p_e$) elasticity
of economy-wide energy consumption ($\rate{Q}_E$).
To assess the effect of demand reduction
($\rbempl{Q}_E > \rainc{Q}_E$)
on price
($\orig{p}_E > \ainc{p}_E$),
we solve for $\frac{\ainc{p}_E}{\orig{p}_E}$
to obtain

% Note to self: Do not include spaces around the "^"
% character in the next equation.
% Those spaces are not parsed correctly by latexdiff.
\begin{equation}
  \frac{\ainc{p}_E}{\orig{p}_E} =
        \left( \frac{\rainc{Q}_E}{\rorig{Q}_E} \right)^{\frac{1}{\eQEpE}} \; .
\end{equation}
%
Substituting Eq.~(\ref{eq:QE_ratio}) gives

% Note to self: Do not include spaces around the "^"
% character in the next equation.
% Those spaces are not parsed correctly by latexdiff.
\begin{equation} \label{eq:price_effect_price_ratio}
  \frac{\ainc{p}_E}{\orig{p}_E} =
        \left[ 1 - f_{\EEU} \frac{N_{dev} \rbempl{E}_s}{\rbempl{Q}_E} \left( 1 - \frac{\rainc{E}_s}{\rbempl{E}_s}  \right) \right]^{\frac{1}{\eQEpE}} \; .
\end{equation}

The energy price reduction ($\orig{p}_E > \ainc{p}_E$)
and the reduction in energy consumed to provide the 
energy service ($\rorig{E}_s > \rainc{E}_s$)
lead to additional freed cash for the device owner at a rate of

\begin{equation}
  \rorig{E}_H \orig{p}_E - [\rorig{E}_H - (\rorig{E}_s - \rainc{E}_s)] \ainc{p}_E \, ,
\end{equation}
%
where $\rorig{E}_H$ is the rate at which the consumer's household 
consumes the final energy carrier that supplies
the energy service
(gasoline a car and
electricity for an electric lamp) 
prior to the EEU.
Rearrangement of terms gives

\begin{equation}
  (\orig{p}_E - \ainc{p}_E) \rorig{E}_H + \ainc{p}_E (\rorig{E}_s - \rainc{E}_s) \, .
\end{equation}

Further regrouping gives

\begin{equation}
  \left[ \frac{\orig{p}_E - \ainc{p}_E}{\orig{p}_E}
         + \frac{\rorig{E}_s - \rainc{E}_s}{\rorig{E}_H} 
              \left( \frac{\ainc{p}_E}{\orig{p}_E} \right) \right] 
                             \orig{p}_E \rorig{E}_H \, ,
\end{equation}
%
where the first fraction in brackets is the fractional reduction 
of energy price, 
the second fraction in brackets is the fraction of
EEU energy savings relative to
the total household energy consumption, and 
the price ratio is give by Eq.~(\ref{eq:price_effect_price_ratio}) above.

The energy implications of spending the additional freed cash
on other goods and services is

\begin{equation}
  \left[ \frac{\orig{p}_E - \ainc{p}_E}{\orig{p}_E}
         + \frac{\rorig{E}_s - \rainc{E}_s}{\rorig{E}_H} 
              \left( \frac{\ainc{p}_E}{\orig{p}_E} \right) \right] 
                             \orig{p}_E \rorig{E}_H I_E \, ,
\end{equation}
%
another energy takeback rate.
By Eq.~(\ref{eq:Re_takeback}),
rebound associated with this energy takeback
can be written as

\begin{equation}
  Re_{p_E} = \frac{\left[ \frac{\orig{p}_E - \ainc{p}_E}{\orig{p}_E}
         + \frac{\rorig{E}_s - \rainc{E}_s}{\rorig{E}_H} 
              \left( \frac{\ainc{p}_E}{\orig{p}_E} \right) \right] 
                             \orig{p}_E \rorig{E}_H I_E}{\Sdot} \, , \tag{\ref{eq:Re_pE}}
\end{equation}
%
as shown in Section~\ref{sec:notes_price_effect},
thus completing the derivation.



