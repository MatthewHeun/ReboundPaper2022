% The next command tells RStudio to do "Compile PDF" on HSB.Rnw,
% instead of this file, thereby eliminating the need to switch back to HSB.Rnw 
% before building the paper.
%!TEX root = ../HSB_framework.Rnw

% This file contains the derivation for substitution effect rebound and net savings.

\begin{landscape}

\linespread{1}

%%%%%%%%%%%%%%%%%%%%%%%%%%%%%%%%%%%%%%%%%
%%%%%%%%%% Substitution Effect %%%%%%%%%%
%%%%%%%%%%%%%%%%%%%%%%%%%%%%%%%%%%%%%%%%%
\derivheader{\refstepcounter{table} Table~\thetable \label{tab:subeffect}. \bf{\SubEffect}}

\sectionsep{}

%%%%%%%%%% Before Substitution Effect %%%%%%%%%%
\derivsection{before ($*$)}
{
% Before substitution energy
\begin{equation}
  \Eacctbsub{} \tag{\ref{eq:E_acct_aemp}}
\end{equation}
}
{
% Before substitution financial
\begin{equation}
  \Macctbsub{} \tag{\ref{eq:M_acct_aemp}}
\end{equation}
}

\sectionsep{}


%%%%%%%%%% After Substitution Effect %%%%%%%%%%
\derivsection{after ($\wedge$)}
{
% After substitution energy
\begin{equation} \label{eq:E_acct_asub}
  \Eacctasub{}
\end{equation}
}
{
% After substitution financial
\begin{equation} \label{eq:M_acct_asub}
  \Macctasub{}
\end{equation}
}

\sectionsep{}

%%%%%%%%%% Derivations for Substitution Effect %%%%%%%%%%
\derivsection{}
% Substitution effect: energy differences
{
%%%%%%%%%% Energy Substitution Effect %%%%%%%%%%
~
  
Take differences to obtain the change in energy consumption, \\
$\Delta \rasub{E} \equiv \rasub{E} - \rbsub{E}$.
%
\begin{equation}
  \Delta \rasub{E} = \Delta \rasub{E}_s 
                      + \cancelto{0}{\Delta \rasub{E}_{emb}} 
                      + (\cancelto{0}{\Delta \rasub{C}_{\omd}} + \Delta \rasub{C}_g) I_E
\end{equation}
%
Thus, 
%
\begin{equation}
  \Delta \rasub{E} = \Delta \rasub{E}_s + \Delta \rasub{C}_g I_E \; .
\end{equation}
%
All terms are energy takeback rates.
Divide by $\Sdot$
to create rebound terms.
%
\begin{equation}
    \frac{\Delta \rasub{E}}{\Sdot} = \frac{\Delta \rasub{E}_s}{\Sdot} + \frac{\Delta \rasub{C}_g I_E}{\Sdot}
\end{equation}
%
Define 
$Re_{sub} \equiv \frac{\Delta \rasub{E}}{\Sdot}$, 
$Re_{dsub} \equiv \frac{\Delta \rasub{E}_s}{\Sdot}$, and
$Re_{isub} \equiv \frac{\Delta \rasub{C}_g I_E}{\Sdot}$,
such that
%
\begin{equation} \label{eq:Re_sub_def}
  Re_{sub} = Re_{dsub} + Re_{isub} \; .
\end{equation}

}
{
%%%%%%%%%% Financial Substitution Effect %%%%%%%%%%
~
  
Use the monetary constraint ($\rate{M}$) to obtain

\begin{align}
  p_E \raempl{E}_s &+ \cancel{\aempl{R}_\alpha \raempl{C}_{cap}} + \cancel{\raempl{C}_{\omd}} + \raempl{C}_g + \raempl{N} \nonumber \\
                   &= p_E \rasub{E}_s + \cancel{\asub{R}_\alpha \rasub{C}_{cap}} + \cancel{\rasub{C}_{\omd}} + \rasub{C}_g + \rasub{N} \; .
\end{align}
%
For the substitution effect, there is no change in capital or operations, maintenance, and disposal costs
($\bsub{R}_\alpha \rbsub{C}_{cap} = \asub{R}_\alpha \rasub{C}_{cap}$ and 
$\rbsub{C}_{\omd} = \rasub{C}_{\omd}$).
Solving for $\Delta \rasub{N} \equiv \rasub{N} - \rbsub{N}$ gives
%
\begin{equation} \label{eq:N_dot_hat_eqn}
  \Delta \rasub{N} = - p_E \Delta \rasub{E}_s - \Delta \rasub{C}_g \; .
\end{equation}
%
The substitution effect adjusts net savings relative to $\rbsub{N}$
by $\Delta \rasub{N}$.
Thus, $\rasub{N} = \rbsub{N} + \Delta \rasub{N}$.
Substituting Eqs.~(\ref{eq:N_dot_star_empl}), (\ref{eq:G_dot}), and~(\ref{eq:N_dot_hat_eqn})
yields
%
\begin{equation} \label{eq:N_dot_after_sub}
  \rasub{N} = \rate{G} - \Delta \bsub{(R_\alpha \rate{C}_{cap})} - \Delta \rbsub{C}_{\omd} - p_E \Delta \rasub{E}_s - \Delta \rasub{C}_g \; .
\end{equation}
%
}

\end{landscape}
