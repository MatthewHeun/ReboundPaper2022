% The next command tells RStudio to do "Compile PDF" on HSB_results.Rnw,
% instead of this file, thereby eliminating the need to switch back to HSB_results.Rnw 
% before building the paper.
%!TEX root = ../HSB_results.Rnw

Rebound planes show the impact of direct and indirect rebound effects
for energy, expenditure, and consumption aspects.
Rebound planes for the car example can be found in 
Figs.~\ref{fig:CarEnergyGraph}--\ref{fig:CarConsGraph}.
Rebound planes for the lamp example can be found in
Figs.~\ref{fig:LampEnergyGraph}--\ref{fig:LampConsGraph}.

This appendix shows the mathematical details of rebound planes,
specifically derivations of equations for lines and curves 
shown in Table~\ref{tab:lines_and_curves}.
The lines and curves enable construction of numerically precise and accurate
paths in rebound planes
as shown in Figs.~\ref{fig:CarEnergyGraph}--\ref{fig:LampConsGraph}.

\begin{table}
\footnotesize
\centering
\caption{Lines and curves for rebound planes.}
\label{tab:lines_and_curves}
\begin{tabular}{rl}
\toprule
Rebound plane                & Lines and curves                        \\ 
\midrule
\multirow{2}{*}{Energy}      & Constant total energy consumption lines \\
                             & 0\% and 100\% rebound lines             \\
\midrule
Expenditure                  & Constant expenditure lines              \\
\midrule
\multirow{3}{*}{Consumption} & Constant expenditure lines              \\
                             & Rays from origin to $\wedge$ point      \\
                             & Indifference curves                     \\
\bottomrule
\end{tabular}
\end{table}


%++++++++++++++++++++++++++++++
\subsection{Energy planes}
\label{sec:energy_path_graph_details}
%++++++++++++++++++++++++++++++

The energy plane shows direct (on the $x$-axis) and indirect (on the $y$-axis)
energy consumption associated with the energy conversion device 
and the device user.
Lines of total energy consumption isoquants provide a 
scale for total rebound.
For example, the 0\% and 100\% rebound lines are constant total energy consumption
lines which pass through the original point ($\circ$) and
the post-direct-emplacement-effect point ($a$) 
in the energy plane.

The equation of a constant total energy consumption line is derived from 

\begin{equation}
  \rate{E}_{tot} = \rate{E}_{dir} + \rate{E}_{indir}
\end{equation}
%
at any rebound stage.
Direct energy consumption is energy consumed by the energy conversion device
($\rate{E}_s$), and 
indirect energy consumption is the sum of embodied energy, 
energy associated with maintenanace and disposal, and energy associated 
with expenditures on other goods
($\rate{E}_{emb} + (\rate{C}_{\omd} + \rate{C}_g) I_E$).

For the energy plane, 
direct energy consumption is placed on the $x$-axis and 
indirect energy consumption is placed on the $y$-axis.
To derive the equation of a constant energy consumption line, 
we first rearrange to put the $y$ coordinate on the left of the equation:

\begin{equation}
  \rate{E}_{indir} = - \rate{E}_{dir} + \rate{E}_{tot} \; .
\end{equation}
%
Next, we substitute $y$ for $\rate{E}_{indir}$,
$x$ for $\rate{E}_{dir}$, and 
$\rate{E}_s + \rate{E}_{emb} + (\rate{C}_{\omd} + \rate{C}_g) I_E$ for $\rate{E}_{tot}$
to obtain

\begin{equation}
  y = -x + \rate{E}_s + \rate{E}_{emb} + (\rate{C}_{\omd} + \rate{C}_g) I_E \; ,
\end{equation}
%
where all of $\rate{E}_s$, $\rate{E}_{emb}$, $\rate{C}_{\omd}$, and $\rate{C}_g$
apply at the same rebound stage.

The constant total energy consumption line 
that passes through the original point ($\circ$)
shows 100\% rebound:

\begin{equation}
  y = -x + \rbempl{E}_s + \rbempl{E}_{emb} + (\rbempl{C}_{\omd} + \rbempl{C}_g) I_E \; .
\end{equation}

The 0\% rebound line is the constant total energy consumption line 
that accounts for expected energy savings ($\Sdot$) only:

\begin{equation}
  y = -x + (\rbempl{E}_s - \Sdot)
          + \rbempl{E}_{emb} + (\rbempl{C}_{\omd} + \rbempl{C}_g) I_E \; .
\end{equation}
%
The above line passes through the $a$ point in the energy plane.


%++++++++++++++++++++++++++++++
\subsection{Expenditure planes}
\label{sec:expenditure_path_graph_details}
%++++++++++++++++++++++++++++++

The expenditure plane shows direct (on the $x$-axis) and indirect (on the $y$-axis)
expenses associated with the energy conversion device 
and the device user.
Lines of constant expenditure are important, 
because they provide budget constraints for the device user.

The equation of a constant total expenditure line is derived from 
the budget constraint

\begin{equation}
  \rate{C}_{tot} = \rate{C}_{dir} + \rate{C}_{indir}
\end{equation}
%
at any rebound stage.
In the expenditure plane,
indirect expenditures are placed on the $y$-axis
and direct expenditures on energy for the energy conversion device are place on the $x$-axis.
Direct expenditure is the cost of energy consumed by the energy conversion device
($\rate{C}_s = p_E \rate{E}_s$), and 
indirect expenses are the sum of capital costs,
operations, maintenanace, and disposal costs, and 
expenditures on other goods
($R_\alpha \rate{C}_{cap} + \rate{C}_{\omd} + \rate{C}_g$).
Rearranging to put the $y$-axis variable on the left side of the equation gives

\begin{equation}
  \rate{C}_{indir} = - \rate{C}_{dir} + \rate{C}_{tot} \; .
\end{equation}

Substituting $y$ for $\rate{C}_{indir}$, 
$x$ for $\rate{C}_{dir}$, and 
$\rate{C}_s + R_\alpha \rate{C}_{cap} + \rate{C}_{\omd} + \rate{C}_g$ for $\rate{C}_{tot}$
gives

\begin{equation}
  y = -x + \rate{C}_s + R_\alpha \rate{C}_{cap} + \rate{C}_{\omd} + \rate{C}_g \; ,
\end{equation}
%
where all of $\rate{C}_s$, $R_\alpha$, $\rate{C}_{cap}$, $\rate{C}_{\omd}$, and $\rate{C}_g$
apply at the same rebound stage.

The constant total expenditure line 
that passes through the original point ($\circ$)
shows the budget constraint for the device user:

\begin{equation}
  y = -x + \rbempl{C}_s + \bempl{R}_\alpha \rbempl{C}_{cap} + \rbempl{C}_{\omd} + \rbempl{C}_g \; ,
\end{equation}
%
into which Eq.~(\ref{PartI-eq:M_acct_orig}) of Part~I can be substituted with 
$\rorig{C}_s = p_E \rorig{E}_s$ and 
$\rorig{N} = 0$ to obtain

\begin{equation}
  y = -x + \rate{M} \; .
\end{equation}

The constant total expenditure line 
that accounts for expected energy savings ($\Sdot$) 
and freed cash ($\rate{G} = p_E \Sdot$) only 
is given by:

\begin{equation}
  y = -x + (\rbempl{C}_s - \rate{G}) + \bempl{R}_\alpha \rbempl{C}_{cap} + \rbempl{C}_{\omd} + \rbempl{C}_g \; ,
\end{equation}
%
or

\begin{equation}
  y = -x + \rate{M} - \rate{G}\; .
\end{equation}
%
The line given by the equation above
passes through the $a$ point in the expenditure plane.


%++++++++++++++++++++++++++++++
\subsection{Consumption planes}
\label{sec:cons_path_graph_details}
%++++++++++++++++++++++++++++++

The consumption plane shows expenditures in 
the $\rate{C}_g/\rbempl{C}_g$ vs.\ $\rate{q}_s/\rbempl{q}_s$ plane,
according to the utility model.
(See Appendix~\ref{PartI-sec:utility_and_elasticities} of Part~I.)
Consumption planes include 
%
\begin{enumerate*}[label={(\roman*)}]
	
  \item constant expenditure lines given prices,
  
  \item a ray from the origin through the $\wedge$ point, and 
  
  \item indifference curves.
    
\end{enumerate*}
%
Derivations for each are shown in the following subsections.


%------------------------------
\subsubsection{Constant expenditure lines} 
\label{sec:pref_graph_constant_expenditure_lines}
%------------------------------

There are four constant expenditure lines in the consumption planes of
Figs.~\ref{fig:CarConsGraph} and \ref{fig:LampConsGraph}.
The constant expenditure lines pass through 
the original point (line \circcirc{}), 
the post-emplacement point (line \starstar{}), 
the post-substitution point (line \hathat{}), and 
the post-income point (line \barbar{}).
Similar to the expenditure plane, 
lines of constant expenditure in the consumption plane
are derived from the budget constraint of the device user
at each of the four points.

Prior to the EEU, the budget constraint is given by Eq.~(\ref{PartI-eq:M_acct_orig}) of Part~I.
Substituting $\orig{p}_s \rorig{q}_s$ for $p_E \rorig{E}_s$ and 
recognizing that there is no net savings before the EEU
($\rorig{N} = 0$) gives

\begin{equation}
  \rate{M} = \orig{p}_s \rorig{q}_s + \orig{R}_\alpha \rorig{C}_{cap} + \rorig{C}_{\omd} + \rorig{C}_g \; .
\end{equation}

To create the line of constant expenditure in the consumption plane, 
we allow $\rorig{q}_s$ and $\rorig{C}_g$ to vary in a compensatory manner:
when one increases, the other must decrease.
To show that variation along the constant expenditure line, 
we remove the notation that ties $\rorig{q}_s$ and $\rorig{C}_g$
to the original point ($\circ$) to obtain

\begin{equation}
  \rate{M} = \bempl{p}_s \rate{q}_s + \bempl{R}_\alpha \rbempl{C}_{cap} + \rbempl{C}_{\omd} + \rate{C}_g \; , 
\end{equation}
%
where all of $\rate{M}$, $\bempl{p}_s$, $\bempl{R}_\alpha \rbempl{C}_{cap}$, and $\rbempl{C}_{\omd}$
apply at the same rebound stage, 
namely the original point ($\circ$) in this instance.

To derive the equation of the line representing the original budget constraint 
in $\rate{C}_g/\rbempl{C}_g$ vs.\ $\rate{q}_s/\rbempl{q}_s$ space
(the \circcirc{} line through the $\circ$ point
in consumption planes), 
we solve for $\rate{C}_g$ to obtain

\begin{equation}
  \rate{C}_g = - \bempl{p}_s \rate{q}_s + \rate{M} - \bempl{R}_\alpha \rbempl{C}_{cap} - \rbempl{C}_{\omd} \; .
\end{equation}
%
Multiplying judiciously by $\rbempl{C}_g/\rbempl{C}_g$ and $\rbempl{q}_s/\rbempl{q}_s$ gives

\begin{equation}
  \frac{\rate{C}_g}{\rbempl{C}_g} \rbempl{C}_g
       = - \bempl{p}_s \frac{\rate{q}_s}{\rbempl{q}_s} \rbempl{q}_s 
         + \rate{M} - \bempl{R}_\alpha \rbempl{C}_{cap} - \rbempl{C}_{\omd} \; .
\end{equation}
%
Dividing both sides by $\rbempl{C}_g$ yields

\begin{equation}
  \frac{\rate{C}_g}{\rbempl{C}_g}
       = - \frac{\bempl{p}_s \rbempl{q}_s}{\rbempl{C}_g} \frac{\rate{q}_s}{\rbempl{q}_s}
         + \frac{1}{\rbempl{C}_g} (\rate{M} - \bempl{R}_\alpha \rbempl{C}_{cap} - \rbempl{C}_{\omd}) \; .
\end{equation}
%
Noting that  
$\rate{q}_s/\rbempl{q}_s$ and 
$\rate{C}_g/\rbempl{C}_g$ are
the $x$-axis and $y$-axis, respectively,
of the consumption plane gives

\begin{equation} \label{eq:orig_line_prefs}
  y = - \frac{\bempl{p}_s \rbempl{q}_s}{\rbempl{C}_g} x
         + \frac{1}{\rbempl{C}_g} (\rate{M} - \bempl{R}_\alpha \rbempl{C}_{cap} - \rbempl{C}_{\omd}) \; .
\end{equation}

A similar procedure can be employed to derive the equation of the
\starstar{} line through the $*$ point
after the emplacement effect.
The starting point is the budget constraint at the $*$ point
(Eq.~(\ref{PartI-eq:M_acct_aemp}) of Part~I)
with 
$\aempl{p}_s \rate{q}_s$ replacing $p_E \raempl{E}_s$ and
$\rate{C}_g$ replacing $\raempl{C}_g$.

\begin{equation}
  \rate{M} = \aempl{p}_s \rate{q}_s + \aempl{R}_\alpha \raempl{C}_{cap} + \raempl{C}_{\omd} + \rate{C}_g + \raempl{N}
\end{equation}
%
Substituting Eq.~(\ref{PartI-eq:N_dot_star_empl}) of Part~I for $\raempl{N}$,
substituting Eq.~(\ref{PartI-eq:G_dot}) of Part~I to obtain $\rate{G}$,
multiplying judiciously by $\rbempl{C}_g/\rbempl{C}_g$ and $\rbempl{q}_s/\rbempl{q}_s$, 
rearranging, and noting that 
$\rate{q}_s/\rbempl{q}_s$ is the $x$-axis and 
$\rate{C}_g/\rbempl{C}_g$ is the $y$-axis gives

\begin{equation} \label{eq:star_line_prefs}
  y = - \frac{\aempl{p}_s \rbempl{q}_s}{\rbempl{C}_g} x
         + \frac{1}{\rbempl{C}_g} (\rate{M} - \bempl{R}_\alpha \rbempl{C}_{cap} - \rbempl{C}_{\omd} - \rate{G}) \; .
\end{equation}
%
Note that the slope of Eq.~(\ref{eq:star_line_prefs}) is less negative
than the slope of Eq.~(\ref{eq:orig_line_prefs}), 
because $\amacro{p}_s < \bempl{p}_s$.
The $y$-intercept of Eq.~(\ref{eq:star_line_prefs}) is less than the 
$y$-intercept of Eq.~(\ref{eq:orig_line_prefs}),
reflecting freed cash.
Both effects are seen in
the consumption planes
(Figs.~\ref{fig:CarConsGraph} and \ref{fig:LampConsGraph}).
The \circcirc{} and \starstar{} lines intersect at the coincident $\circ$ and $*$ points.

A similar derivation process can be used to find the equation of 
the line representing the budget constraint
after the substitution effect (the \hathat{} line through the $\wedge$ point).
The starting point is Eq.~(\ref{PartI-eq:M_acct_asub}) of Part~I, and 
the equation for the constant expenditure line is

\begin{equation} \label{eq:hat_line_prefs}
  y = - \frac{\aempl{p}_s \rbempl{q}_s}{\rbempl{C}_g} x
         + \frac{1}{\rbempl{C}_g} (\rate{M} - \bempl{R}_\alpha \rbempl{C}_{cap} - \rbempl{C}_{\omd} 
                                   - \rate{G} + \aempl{p}_s \Delta \rasub{q}_s + \Delta \rasub{C}_g) \; .
\end{equation}

Note that the \hathat{} line (Eq.~(\ref{eq:hat_line_prefs})) has the same slope as 
the \starstar{} line (Eq.~(\ref{eq:star_line_prefs}))
but a lower $y$-intercept.

Finally, the corresponding derivation
for the equation of the constant expenditure line through the 
$-$ point (line \barbar{}) starts with Eq.~(\ref{PartI-eq:M_acct_ainc}) of Part~I 
and comes to

\begin{equation} 
  y = - \frac{\aempl{p}_s \rbempl{q}_s}{\rbempl{C}_g} x
        + \frac{1}{\rbempl{C}_g} (\rate{M} - \bempl{R}_\alpha \rbempl{C}_{cap} - \rbempl{C}_{\omd}
                                   - \Delta \aempl{(R_\alpha \rate{C}_{cap})} - \Delta \raempl{C}_{\omd}) \; .
\end{equation}

Simplification of $\Delta$ terms gives

\begin{equation} \label{eq:bar_line_prefs}
  y = - \frac{\amacro{p}_s \rbempl{q}_s}{\rbempl{C}_g} x
        + \frac{1}{\rbempl{C}_g} (\rate{M} - \aempl{R}_\alpha \raempl{C}_{cap} - \raempl{C}_{\omd}) \; .
\end{equation}
%


%------------------------------
\subsubsection{Ray from the origin to the $\wedge$ point} 
\label{sec:pref_graph_ray}
%------------------------------

In the consumption plane,
the ray from the origin to the $\wedge$ point 
(line \rr{})
defines the path along which the income effect
(lines \hatd{} and \dbar{})
operates.
The ray from the origin to the $\wedge$ point
has slope $(\rasub{C}_g/\rbempl{C}_g) / (\rasub{q}_s/\rbempl{q}_s)$
and a $y$-intercept of 0.
Therefore, the equation of line \rr{} is

\begin{equation} \label{eq:ray_cons}
  y = \frac{\rasub{C}_g/\rbempl{C}_g}{\rasub{q}_s/\rbempl{q}_s} \, x \; .
\end{equation}


%------------------------------
\subsubsection{Indifference curves} 
\label{sec:cons_graph_indifference_curves}
%------------------------------

In the consumption plane,
indifference curves represent lines of constant utility
for the energy conversion device user.
In the consumption plane
($\rate{C}_g/\rbempl{C}_g$ vs.\ $\rate{q}_s/\rbempl{q}_s$), 
any indifference curve 
is given by 
Eq.~(\ref{PartI-eq:utility_Cg_form}) of Part~I
with $\fCs$ replacing the share parameter $a$, 
as shown in Appendix~\ref{PartI-sec:utility_and_elasticities} of Part~I.
Recognizing that 
$\rate{C}_g/\rbempl{C}_g$ is on the $y$-axis and 
$\rate{q}_s/\rbempl{q}_s$ is on the $x$-axis
leads to substitution of 
$y$ for $\rate{C}_g/\rbempl{C}_g$ and 
$x$ for $\rate{q}_s/\rbempl{q}_s$ to obtain

\begin{equation} \label{eq:utility_line_form}
  y = \left[ \frac{1}{1 - \fCs} \left( \frac{\rate{u}}{\rbempl{u}} \right)^\rho 
            - \frac{\fCs}{1 - \fCs} (x)^\rho \right]^{(1/\rho)} \; .
\end{equation}

At any point on the 
$\rate{C}_g/\rbempl{C}_g$ vs.\ $\rate{q}_s/\rbempl{q}_s$ plane,
namely ($\rate{q}_{s,1}/\rbempl{q}_s$, $\rate{C}_{g,1}/\rbempl{C}_g$),
indexed utility ($\rate{u}_1/\rbempl{u}$) is given by Eq.~(\ref{PartI-eq:ces_utility}) of Part~I as

\begin{equation} \label{eq:utility_ratio_1_point}
  \frac{\rate{u}_1}{\rbempl{u}} =
        \left[ \fCs \left( \frac{\rate{q}_{s,1}}{\rbempl{q}_s} \right)^\rho
        + (1-\fCs) \left( \frac{\rate{C}_{g,1}}{\rbempl{C}_g} \right)^\rho  \right]^{(1/\rho)} \; .
\end{equation}
%
Substituting Eq.~(\ref{eq:utility_ratio_1_point}) into Eq.~(\ref{eq:utility_line_form})
for $\rate{u}/\rorig{u}$
and simplifying exponents gives

\begin{equation}
  y = \left\{ \frac{1}{1 - \fCs} \left[ \fCs \left( \frac{\rate{q}_{s,1}}{\rbempl{q}_s} \right)^\rho 
        + (1-\fCs) \left( \frac{\rate{C}_{g,1}}{\rbempl{C}_g} \right)^\rho   \right] 
            - \frac{\fCs}{1 - \fCs} (x)^\rho \right\}^{(1/\rho)}  .
\end{equation}
%
Simplifying further yields
the equation of an indifference curve passing through point 
($\rate{q}_{s,1}/\rbempl{q}_s$, $\rate{C}_{g,1}/\rbempl{C}_g$):

\begin{equation} \label{eq:indiff_curve_eqn}
  y = \left\{ \left( \frac{\fCs}{1 - \fCs} \right) \left[ \left( \frac{\rate{q}_{s,1}}{\rbempl{q}_s} \right)^\rho 
                                                          - (x)^\rho  \right]
        + \left( \frac{\rate{C}_{g
        ,1}}{\rbempl{C}_g} \right)^\rho \right\}^{(1/\rho)} \; .
\end{equation}
%
Note that if $x$ is $\rate{q}_{s,1}/\rbempl{q}_s$,
$y$ becomes $\rate{C}_{g,1}/\rbempl{C}_g$,
as expected.
