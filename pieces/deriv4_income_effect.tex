% The next command tells RStudio to do "Compile PDF" on HSB.Rnw,
% instead of this file, thereby eliminating the need to switch back to HSB.Rnw 
% before building the paper.
%!TEX root = ../HSB_framework.Rnw

% This file contains the derivation for income effect rebound and net savings.

\begin{landscape}

\linespread{1}

%%%%%%%%%%%%%%%%%%%%%%%%%%%%%%%%%%%
%%%%%%%%%% Income Effect %%%%%%%%%%
%%%%%%%%%%%%%%%%%%%%%%%%%%%%%%%%%%%
\derivheader{\refstepcounter{table} Table~\thetable \label{tab:inceffect}. \bf{\IncEffect}}

\sectionsep{}

%%%%%%%%%% Before Income Effect %%%%%%%%%%
\derivsection{before ($\wedge$)}
{
% Before income energy
\begin{equation}
  \Eacctbinc{} \tag{\ref{eq:E_acct_asub}}
\end{equation}
}
{
% Before income financial
\begin{equation}
  \Macctbinc{} \tag{\ref{eq:M_acct_asub}}
\end{equation}
}

\sectionsep{}

%%%%%%%%%% After Income Effect %%%%%%%%%%
\derivsection{after ($-$)}
{
% After income energy
\begin{equation} \label{eq:E_acct_ainc}
\Eacctainc{}
\end{equation}
}
{
% After income financial
\begin{equation} \label{eq:M_acct_ainc}
\Macctainc{}
\end{equation}
}

\sectionsep{}

%%%%%%%%%%% Derivations for Income Effect %%%%%%%%%%
\derivsection{}
% Income effect: energy differences
{
%%%%%%%%%% Energy Income Effect %%%%%%%%%%
~

Take differences to obtain the change in energy consumption, $\Delta \rainc{E} \equiv \rainc{E} - \rbinc{E}$.
%
\begin{equation}
  \Delta \rainc{E} = \Delta \rainc{E}_s 
                     + \cancelto{0}{\Delta \rainc{E}_{emb}}
                     + (\cancelto{0}{\Delta \rainc{C}_{\md}} + \Delta \rainc{C}_o) I_E
\end{equation}
%
Thus, 
%
\begin{equation}
  \Delta \rainc{E} = \Delta \rainc{E}_s + \Delta \rainc{C}_o I_E
\end{equation}
%
All terms are energy takeback rates.
Divide by $\Sdot$
to create rebound terms.
%
\begin{equation}
  \frac{\Delta \rainc{E}}{\Sdot} = \frac{\Delta \rainc{E}_s}{\Sdot} + \frac{\Delta \rainc{C}_o I_E}{\Sdot}
\end{equation}
%
Define 
$Re_{inc} \equiv \frac{\Delta \rainc{E}}{\Sdot}$, 
$Re_{dinc} \equiv \frac{\Delta \rainc{E}_s}{\Sdot}$, and 
$Re_{iinc} \equiv \frac{\Delta \rainc{C}_o I_E}{\Sdot}$,
such that
%
\begin{equation} \label{eq:Re_inc_def}
  Re_{inc} = Re_{dinc} + Re_{iinc} \; .
\end{equation}
%
}
{
%%%%%%%%%% Financial Income Effect %%%%%%%%%%
~

Use the monetary constraint ($\rbinc{M} = \rainc{M}$) to obtain

\begin{align}
  p_E \rasub{E}_s &+ \cancel{\rasub{C}_{cap}} + \cancel{\rasub{C}_{\md}} + \rasub{C}_o + \rasub{N} \nonumber \\
                  &= p_E \rainc{E}_s + \cancel{\rainc{C}_{cap}} + \cancel{\rainc{C}_{\md}} + \rainc{C}_o + \cancelto{0}{\rainc{N}} \; .
\end{align}
%
For the income effect, there is no change in capital or maintainance and disposal costs
($\rasub{C}_{cap} = \rbsub{C}_{cap}$ and $\rasub{C}_{\md} = \rbsub{C}_{\md}$).
Notably, $\rainc{N} = 0$,
because it is assumed that all net monetary savings ($\rbinc{N}$) are spent on
more energy service ($\rainc{E}_s > \rbinc{E}_s$)
and
additional purchases in the economy ($\rainc{C}_o > \rbinc{C}_o$).
Solving for $\rbinc{N}$ gives 
%
\begin{equation} \label{eq:inc_budget_constraint}
  \rbinc{N} = p_E \Delta \rainc{E}_s + \Delta \rainc{C}_o \; ,
\end{equation}
%
the budget constraint for the income effect.
By construction, 
Eq.~(\ref{eq:inc_budget_constraint}) ensures
spending of net savings~($\rbinc{N}$) on
%
\begin{enumerate*}[label={(\roman*)}]
	
  \item additional energy services~($\Delta \rainc{E}_s$) and
  
  \item additional purchases of other goods in the economy~($\Delta \rainc{C}_o$) only.
    
\end{enumerate*}
}
\end{landscape}
