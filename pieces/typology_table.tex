% The next command tells RStudio to do "Compile PDF" on HSB.Rnw,
% instead of this file, thereby eliminating the need to switch back to HSB.Rnw 
% before building the paper.
%!TEX root = ../HSB_framework.Rnw

% \begin{landscape}
% \begin{table}
% \footnotesize
% \begin{center}
% \caption{Rebound typology.}
% \label{tab:rebound_typology}
% \begin{tabular}{ r l l }
% \toprule
% Category & Mechanism & Source \\ 
% \midrule
% Direct rebound effect & Income effect       & Changes in the consumption of the energy service, owing to the increase \\
% (partial equilibrium) & (consumers)         & in real income stimulated by the EEU \\
%                       \cmidrule{2-3}
%                       & Substitution effect & Changes in the consumption of the energy service, owing to a fall in its \\
%                       & (consumers)         & effective price relative to other commodities (holding utility constant) \\
%                       \cmidrule{2-3}
%                       & Output effect       & Changes in the consumption of the energy service owing to the increase \\
%                       & (producers)         & in output stimulated by the EEU \\
%                       \cmidrule{2-3}
%                       & Substitution effect & Changes in the consumption of the energy service, owing to a fall in its \\
%                       & (producers)         & effective price relative to other inputs (holding output constant) \\
% \midrule
% Indirect rebound effect & Income effect     & Changes in the consumption of other commodities, owing to the increase in \\
% (partial equilibrium)   & (consumers)       & real income stimulated by the EEU \\
%                         \cmidrule{2-3}
%                         & Substitution effect & Changes in the consumption of other commodities, owing to an increase in \\
%                         & (consumers)         & their effective price relative to the energy service (holding utility constant) \\
%                         \cmidrule{2-3}
%                         & Output effect       & Changes in the consumption of other inputs owing to the increase in \\
%                         & (producers)         & output stimulated by the EEU \\
%                         \cmidrule{2-3}
%                         & Substitution effect & Changes in the consumption of other inputs, owing to an increase in their \\
%                         & (producers)         & effective price relative to the energy service (holding output constant) \\
% \midrule
% Macroeconomic rebound effect & Energy market effect & Changes in energy consumption following changes in energy prices \\
% (general equilibrium)        &                      & (leftward shift of the demand curve for energy) \\
%                              \cmidrule{2-3}
%                              & Composition effect   & Changes in energy consumption following structural change in the economy \\
%                              &                      & with energy-intensive sectors and goods benefiting more \\
%                              \cmidrule{2-3}
%                              & Growth effect        & Changes in energy consumption following investment and increased output \\
%                              &                      & stimulated by the EEU \\
%                              \cmidrule{2-3}
%                              & Scale effect         & Changes in energy consumption following reductions in the price of goods and \\
%                              &                      & services stimulated by increased output of those goods and services \\
%                              \cmidrule{2-3}
%                              & Labour supply effect & Changes in energy consumption following increases in real wages \\
%                              &                      & stimulated by the EEU \\
%                              \cmidrule{2-3}
%                              & Disinvestment effect & Changes in energy consumption following disinvestment in the energy supply \\
%                              &                      & sectors in response to lower energy prices \\
% \bottomrule
% \end{tabular}
% \end{center}
% \end{table}
% \end{landscape}


\begin{table}
\footnotesize
\begin{center}
\caption{Rebound typology. Comparison to \citet{Sorrell:2009cg}, \citet{Jenkins2011}, 
                           \citet{Thomas:2013aa,Thomas:2013ab}, and \citet{Walnum2014} \emph{in italics}.}
\label{tab:rebound_typology}
\begin{tabular}{ r l l }
\toprule
                                   & \multicolumn{2}{c}{Location} \\ 
                                   & \multicolumn{1}{c}{Direct}                  & \multicolumn{1}{c}{Indirect} \\
                                   \cmidrule{2-3}
\textbf{Microeconomic rebound}     & \textbf{Emplacement effect} ($Re_{dempl}$)  & \textbf{Emplacement effect} ($Re_{iempl}$) \\
These mechanisms occur             & Accounts for performance of the             & Differential lifecycle energy effects \\
at the single device level         & Energy Efficeincy Upgrade (EEU) only.       & (versus counterfactual) of the EEU, \\
within a static economy            & No behavior changes occur. The direct       & i.e., embodied energy (emb), and imp- \\
based on responses to the          & energy effect of emplacement of the EEU     & lied energy demand from maintenance \\
reduction in implicit price        & is expected device-level energy savings.    & and disposal (md). (\emph{Other authors} \\
of an energy service.              & By definition, there is no rebound from     & \emph{include embodied effects (emb) but} \\
                                   & direct emplacement effects. The direct      & \emph{not effects associated with md.} \\
                                   & emplacement effect is also known as         & \\
                                   & expected energy savings.                    & \\
                                   \cmidrule{2-3}
                                   & \textbf{Substitution effect} ($Re_{dsub}$)  & \textbf{Substitution effect} ($Re_{isub}$) \\
                                   & Spending of freed cash on more of the       & Decreased spending on other goods  \\
                                   & energy service. (\emph{Same as other authors.})    & and services. (\emph{Other authors typically} \\
                                   &                                             & \emph{include indirect substitution effects} \\
                                   &                                             & \emph{within respending and reinvestment} \\ 
                                   &                                             & \emph{effects.}) \\ 
                                   \cmidrule{2-3}
                                   & \textbf{Income effect} ($Re_{dinc}$)        & \textbf{Income effect} ($Re_{iinc}$) \\
                                   & Spending of freed cash on more of the       & Increased spending on other goods and \\
                                   & energy service. (\emph{Same as other authors.}) & services. \emph{Other authors typically} \\
                                   &                                             & \emph{include indirect income effects within} \\ 
                                   &                                             & \emph{re-spending and re-investment effects.}) \\
                                   \cmidrule{2-3}
\textbf{Macroeconomic rebound}     & \textbf{Macroeconomic effect} ($Re_{macro}$)  & \\
These mechanisms originate         & \multicolumn{2}{l}{Comprised of numerous components including: energy market effect, composition} \\
from the dynamic response          & \multicolumn{2}{l}{effect, growth effect, scale effect, labor supply effect, and disinvestment effect.} \\
of the economy to reach            & \multicolumn{2}{l}{(\emph{We have close alignment with other authors in that we (a)~fold the scale and}} \\
a stable equilibrium               & \multicolumn{2}{l}{\emph{investment/disinvestment effects into an economic growth effect and (b)~include}} \\
(between supply and                & \multicolumn{2}{l}{\emph{labor market effects within the indirect factors of production response.}} \\
demand for goods and               & & \\
energy services). These            & & \\
mechanisms combine                 & & \\
various short and long             & & \\
run effects.                       & & \\
\bottomrule
\end{tabular}
\end{center}
\end{table}



