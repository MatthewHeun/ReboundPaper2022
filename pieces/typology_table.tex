% The next command tells RStudio to do "Compile PDF" on HSB.Rnw,
% instead of this file, thereby eliminating the need to switch back to HSB.Rnw 
% before building the paper.
%!TEX root = ../HSB_framework.Rnw



\begin{table}
\footnotesize
\begin{center}
\caption{Rebound typology. Comparison to \citet{Sorrell:2009cg}, \citet{Jenkins2011}, 
                           \citet{Thomas:2013aa,Thomas:2013ab}, and \citet{Walnum2014} in \emph{italics}.}
\label{tab:rebound_typology}
\begin{tabular}{ r l l }
\toprule
                                   & \multicolumn{1}{c}{\textbf{Direct rebound}} & \multicolumn{1}{c}{\textbf{Indirect rebound}} \\
\midrule
\textbf{Microeconomic rebound}     & \textbf{Emplacement effect} ($Re_{dempl}$)  & \textbf{Emplacement effect} ($Re_{iempl}$) \\
These mechanisms occur             & Accounts for performance of the             & Differential non-operational energy  \\
at the single device/owner         & Energy Efficiency Upgrade (EEU) only.       & effects (versus counterfactual) of the  \\
level within a static economy      & No behavior changes occur. The direct       & EEU, via (i) the embodied energy  \\
based on responses to the          & energy effect of emplacement of the EEU     & associated with the manufacturing  \\
reduction in implicit price        & is expected device-level energy savings.    & phase and (ii) the implied energy \\
of an energy service.              & By definition, there is no rebound from     & demand from maintenance and  \\
                                   & direct emplacement effects                  & disposal ($Re_{\md}$).  \\
                                   & ($Re_{dempl} \equiv 0$).                    & \emph{Other authors include embodied}  \\
                                   &                                             & \emph{effects but not effects associated} \\
                                   &                                             & \emph{with maintenance and disposal.} \\
                                   \cmidrule{2-3}
                                   & \textbf{Substitution effect} ($Re_{dsub}$)  & \textbf{Substitution effect} ($Re_{isub}$) \\
                                   & Change in preference toward the energy      & Change in preference away from  \\
                                   & service relative to other goods as a        & other goods relative to the energy  \\
                                   & result of the EEU. Excludes by              & service as a result of the EEU.  \\
                                   & definition the effects of freed cash        & Excludes by definition the effects \\ 
                                   & (income effects). Results in greater        & of freed cash (income effects).     \\ 
                                   & consumption of the energy service.          & Results in reduced consumption of   \\
                                   & \emph{Same as other authors.}               & other goods and services. \emph{Same as} \\
                                   &                                             & \emph{other authors}. \\ 
                                   \cmidrule{2-3}
                                   & \textbf{Income effect} ($Re_{dinc}$)        & \textbf{Income effect} ($Re_{iinc}$) \\
                                   & Spending of some of the freed cash          & Spending of some of the freed cash on  \\
                                   & on the upgraded device. Results in          & other goods and services. Results in \\
                                   & greater consumption of the energy           & greater consumption of other goods  \\ 
                                   & service. \emph{Same as other authors.}      & and services. \emph{Other authors typically} \\
                                   &                                             & \emph{include indirect income effects within} \\
                                   &                                             & \emph{respending (consumer-sided) and} \\
                                   &                                             & \emph{reinvestment (producer-sided) rebound} \\
                                   &                                             & \emph{effects.}     \\ 
\midrule
\textbf{Macroeconomic rebound}     &                                             & \textbf{Macroeconomic effect} ($Re_{macro}$) \\
These mechanisms originate         &                                             & Comprised of numerous components \\
from the dynamic response          &                                             & including: energy market effect, \\
of the economy to reach a          &                                             & composition effect, growth effect, \\
stable equilibrium (between        &                                             & scale effect, labor supply effect, and \\
supply and demand for              &                                             & investment/disinvestment effect. \emph{We} \\
good and energy services).         &                                             & \emph{have close alignment with other} \\
These mechanisms combine           &                                             & \emph{authors in that (i)~we fold scale and} \\
various short and long run         &                                             & \emph{investment/disinvestment effects} \\
effects.                           &                                             & \emph{into an economic growth effect and} \\
                                   &                                             & \emph{(ii)~we include labor market effects } \\ 
                                   &                                             & \emph{within the indirect factors of } \\
                                   &                                             & \emph{production response.} \\
\bottomrule
\end{tabular}
\end{center}
\end{table}



\emph{re-spending and reinvestment effects.}