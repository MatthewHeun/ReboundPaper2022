% The next command tells RStudio to do "Compile PDF" on HSB.Rnw,
% instead of this file, thereby eliminating the need to switch back to HSB.Rnw 
% before building the paper.
%!TEX root = ../HSB_framework.Rnw



\begin{table}
\footnotesize
\begin{center}
\caption{Rebound typology. Comparison to \citet{Sorrell:2009cg}, \citet{Jenkins2011}, 
                           \citet{Thomas:2013aa,Thomas:2013ab}, and \citet{Walnum2014} in \emph{italics}.}
\label{tab:rebound_typology}
\begin{tabular}{ r l l }
\toprule
                                   & \multicolumn{2}{c}{Location} \\ 
                                   & \multicolumn{1}{c}{Direct}                  & \multicolumn{1}{c}{Indirect} \\
                                   \cmidrule{2-3}
\textbf{Microeconomic rebound}     & \textbf{Emplacement effect} ($Re_{dempl}$)  & \textbf{Emplacement effect} ($Re_{iempl}$) \\
These mechanisms occur             & Accounts for performance of the             & Differential lifecycle energy effects \\
at the single device level         & Energy Efficiency Upgrade (EEU) only.       & (versus counterfactual) of the EEU, \\
within a static economy            & No behavior changes occur. The direct       & i.e., embodied energy ($Re_{emb}$), and \\
based on responses to the          & energy effect of emplacement of the EEU     & implied energy demand from \\
reduction in implicit price        & is expected device-level energy savings.    & maintenance and disposal ($Re_{\md}$). \\
of an energy service.              & By definition, there is no rebound from     & \emph{Other authors include embodied} \\
                                   & direct emplacement effects                  & \emph{effects but not effects associated} \\
                                   & ($Re_{dempl} \equiv 0$).                    & \emph{with maintenance and disposal}. \\
                                   \cmidrule{2-3}
                                   & \textbf{Substitution effect} ($Re_{dsub}$)  & \textbf{Substitution effect} ($Re_{isub}$) \\
                                   & Spending of freed cash on more of the       & Decreased spending on other goods  \\
                                   & energy service. \emph{Same as other authors.}  & and services. \emph{Other authors typically} \\
                                   &                                             & \emph{include indirect substitution effects} \\
                                   &                                             & \emph{within re-spending and reinvestment} \\ 
                                   &                                             & \emph{effects.} \\ 
                                   \cmidrule{2-3}
                                   & \textbf{Income effect} ($Re_{dinc}$)        & \textbf{Income effect} ($Re_{iinc}$) \\
                                   & Spending of freed cash on more of the       & Increased spending on other goods and \\
                                   & energy service. \emph{Same as other authors.} & services. \emph{Other authors typically} \\
                                   &                                             & \emph{include indirect income effects within} \\ 
                                   &                                             & \emph{re-spending and reinvestment effects.} \\
                                   \cmidrule{2-3}
\textbf{Macroeconomic rebound}     &                                             & \textbf{Macroeconomic effect} ($Re_{macro}$) \\
These mechanisms originate         &                                             & Comprised of numerous components \\
from the dynamic response          &                                             & including: energy market effect, \\
of the economy to reach a          &                                             & composition effect, growth effect, \\
stable equilibrium (between        &                                             & scale effect, labor supply effect, and \\
supply and demand for              &                                             & disinvestment effect. \emph{We have close} \\
good and energy services).         &                                             & \emph{alignment with other authors in that} \\
These mechanisms combine           &                                             & \emph{(i)~we fold scale and investment/} \\
various short and long run         &                                             & \emph{disinvestment effects into an} \\
effects.                           &                                             & \emph{economic growth effect and (ii)~we} \\
                                   &                                             & \emph{include labor market effects within} \\ 
                                   &                                             & \emph{the indirect factors of production} \\
                                   &                                             & \emph{response.} \\
\bottomrule
\end{tabular}
\end{center}
\end{table}

