% The next command tells RStudio to do "Compile PDF" on HSB.Rnw,
% instead of this file, thereby eliminating the need to switch back to HSB.Rnw 
% before building the paper.
%!TEX root = ../HSB_framework.Rnw



\begin{table}
\footnotesize
\begin{center}
\caption{Rebound typology for our framework.}
\label{tab:rebound_typology}
\begin{tabular}{ r l l }
\toprule
                                   & \multicolumn{1}{c}{\textbf{Direct rebound}}  & \multicolumn{1}{c}{\textbf{Indirect rebound}} \\
                                   & \multicolumn{1}{c}{($Re_{dir}$)}             & \multicolumn{1}{c}{($Re_{indir}$)}            \\
\midrule
\textbf{Microeconomic rebound}     & \textbf{Emplacement effect} ($Re_{dempl}$)   & \textbf{Emplacement effect} ($Re_{iempl}$) \\
\multicolumn{1}{c}{($Re_{micro}$)} & Accounts for performance of the              & Differential non-operational energy  \\
These mechanisms occur             & Energy Efficiency Upgrade (EEU) only.        & effects (versus counterfactual) of the  \\
at the single device/owner         & No behavior changes occur. The direct        & EEU, via (i) the embodied energy  \\
level within a static economy      & energy effect of emplacement of the EEU      & associated with the manufacturing  \\
based on responses to the          & is expected device-level energy savings.     & phase ($Re_{emb}$) and (ii) the implied  \\
reduction in implicit price        & By definition, there is no rebound from      & energy demand from maintenance   \\
of an energy service.              & direct emplacement effects                   & and disposal ($Re_{\md}$). $Re_{iempl}$ can  \\
                                   & ($Re_{dempl} \equiv 0$).                     & be $> 0$ or $< 0$, depending on the   \\
                                   &                                              & characteristics of the EEU.        \\
                                   \cmidrule{2-3}
                                   & \textbf{Substitution effect} ($Re_{dsub}$)   & \textbf{Substitution effect} ($Re_{isub}$) \\
                                   & Change in preference toward the energy       & Change in preference away from  \\
                                   & service relative to other goods as a         & other goods relative to the energy  \\
                                   & result of the EEU. Excludes by               & service as a result of the EEU.  \\
                                   & definition the effects of freed cash         & Excludes by definition the effects \\ 
                                   & (income effects). $Re_{dsub} > 0$ is typical & of freed cash (income effects).     \\ 
                                   & due to greater consumption of the            & $Re_{isub} < 0$ is typical due to reduced    \\
                                   & energy service.                              & consumption of other goods and   \\
                                   &                                              & services. \\
                                   \cmidrule{2-3}
                                   & \textbf{Income effect} ($Re_{dinc}$)         & \textbf{Income effect} ($Re_{iinc}$) \\
                                   & Spending of some of the freed cash to        & Spending of some of the freed cash   \\
                                   & obtain more of the energy service.           & on other goods and services.   \\
                                   & $Re_{dinc} > 0$ is typical due to increased  & $Re_{iinc} > 0$ is typical due to increased    \\ 
                                   & consumption of the energy service.           & consumption of other goods and   \\
                                   &                                              & services. \\
\midrule
\textbf{Macroeconomic rebound}     &                                              & \textbf{Macroeconomic effect} ($Re_{macro}$) \\
\multicolumn{1}{c}{($Re_{macro}$)} &                                              & Increased energy consumption in the  \\
These mechanisms originate         &                                              & broader macroeconomic system, i.e.\  \\
from the dynamic response          &                                              & beyond responses at the micro-  \\
of the economy to reach a          &                                              & economic (device/owner) level. \\
stable equilibrium (between        &                                              & $Re_{macro} > 0$ is typical, due to   \\
supply and demand for              &                                              & spending of freed cash (at the micro-\\
energy services and other          &                                              & economic level) causing greater con- \\
goods). These mechanisms           &                                              & sumption in the wider economy. \\
combine various short and          &                                              &  \\
long run effects.                  &                                              &  \\
\bottomrule
\end{tabular}
\end{center}
\end{table}