% The next command tells RStudio to do "Compile PDF" on HSB.Rnw,
% instead of this file, thereby eliminating the need to switch back to HSB.Rnw 
% before building the paper.
%!TEX root = ../HSB_framework.Rnw


This appendix provides a detailed derivation of 
the analytical framework,
beginning with the budget constraint for the device owner.


%++++++++++++++++++++++++++++++
\subsection{Budget constraint}
\label{sec:budget_constraint}
%++++++++++++++++++++++++++++++

% The next command tells RStudio to do "Compile PDF" on HSB.Rnw,
% instead of this file, thereby eliminating the need to switch back to HSB.Rnw 
% before building the paper.
%!TEX root = ../HSB_framework.Rnw

We assume the device owner has four expense categories 
related to the device:
capital cost ($C_{cap}$), 
energy service cost ($C_s$), 
operations and maintenance cost ($C_{om}$), and
disposal cost ($C_d$).
We count one expense category for all other
goods and services ($C_o$),
one category for annual income ($M$), and 
net savings ($N$), 
the difference between income and expenses.
Capital ($cap$) and disposal ($d$) costs are applied
at the beginning ($\alpha$) and end ($\omega$), respectively, 
of the device lifetime ($t_{life}$).
All other budget categories are 
applied at the beginning of each year.
A budget can be constructed for the device owner for each stage
of Figure~\ref{fig:flowchart}, 
leading to a different budget 
before emplacement ($\circ$), 
after emplacement ($*$), 
after the substitution effect ($\wedge$), 
after the income effect ($-$), and 
after the macro effect ($\sim$).
When needed, 
the different budgets can be distinguished by symbol decorations
shown in Table~\ref{tab:decorations}.
We allow the device owner to purchase the device 
with a loan with real interest rate $r$.
For a device not purchased on credit,
$r = 0$ applies.

Each budget category is analyzed in perpetuity
to allow comparisons at different rebound stages 
($\circ$, $*$, etc.).
The present value ($P$) of each expense category is obtained
with an infinite sum as follows

\begin{align} \label{eq:present_values}
  P_{cap} &= C_{cap} + \frac{C_{cap}}{(1 + r)^{\tlife}} + \frac{C_{cap}}{(1 + r)^{2 \tlife}} + \ldots 
  = C_{cap} \sum_{k = 0}^{\infty} \frac{1}{(1 + r)^{k \tlife}} 
  = \phi_{\tlife} C_{cap} \\
%
  P_s &= C_s + \frac{C_s}{(1 + r)^{\oneyr}} + \frac{C_s}{(1 + r)^{\twoyr}} + \ldots
  = C_s \sum_{k = 0}^{\infty} \frac{1}{(1 + r)^{\kyr}}
  = \phi_{\oneyr} C_s \\
%
  P_{om} &= C_{om} + \frac{C_{om}}{(1 + r)^{\oneyr}} +  \frac{C_{om}}{(1 + r)^{\twoyr}} + \ldots
  = C_{om} \sum_{k = 0}^{\infty} \frac{1}{(1 + r)^{\kyr}}
  = \phi_{\oneyr} C_{om} \\
%
  P_d &= \frac{C_d}{(1 + r)^{\tlife}} + \frac{C_d}{(1 + r)^{2 \tlife}} + \ldots 
  = C_d \sum_{k = 1}^{\infty} \frac{1}{(1 + r)^{k \tlife}} 
  = \gamma_{\tlife} C_d \\
%
  P_o &= C_o + \frac{C_o}{(1 + r)^{\oneyr}} + \frac{C_o}{(1 + r)^{\twoyr}} + \ldots
  = C_o \sum_{k = 0}^{\infty} \frac{1}{(1 + r)^{\kyr}}
  = \phi_{\oneyr} C_o \\
%
  P_M &= M + \frac{M}{(1 + r)^{\oneyr}} + \frac{M}{(1 + r)^{\twoyr}} + \ldots
  = M \sum_{k = 0}^{\infty} \frac{1}{(1 + r)^{\kyr}}
  = \phi_{\oneyr} M \\
%
  P_N &= N + \frac{N}{(1 + r)^{\oneyr}} + \frac{N}{(1 + r)^{\twoyr}} + \ldots
  = N \sum_{k = 0}^{\infty} \frac{1}{(1 + r)^{\kyr}}
  = \phi_{\oneyr} N
\end{align}
%
where $\phi_t \equiv \frac{(1 + r)^t}{(1 + r)^t - 1}$
and $\gamma_t \equiv \frac{1}{(1 + r)^t - 1}$.

For simplicity, we desire annual values ($A$)
with equivalent present value for each cost category. 
Using the capital cost to illustrate, 
we begin with the present value equivalence of the infinite series and 
annual costs.

\begin{equation}
  P_{cap} = P_{A_{cap}}
\end{equation}
%
Substituting expressions for present values ($P$) gives

\begin{equation}
  \phi_{t_{life}} C_{cap} = \phi_{1\,\mathrm{yr}} A_{cap} \, .
\end{equation}
%
Rearranging gives

\begin{equation}
  A_{cap} = \frac{\phi_{\tlife}}{\phi_{\oneyr}} C_{cap} \, .
\end{equation}
%
Further, we desire annualized rates defined as
$\rate{A} \equiv A/\oneyr$ such that
$\rate{A}_{cap} = A_{cap}/\oneyr$
and 
$\rate{C}_{cap} \equiv C_{cap}/\tlife$.
Solving for $A_{cap}$ and $C_{cap}$ and substituting gives

\begin{equation}
  \rate{A}_{cap} (\oneyr) = \frac{\phi_{\tlife}}{\phi_{\oneyr}} \rate{C}_{cap} \tlife \, .
\end{equation}
%
Defining $R_\alpha \equiv \frac{\phi_{\tlife}}{\phi_{\oneyr}} \frac{\tlife}{\oneyr}$ 
(with subscript $\alpha$ indicating payments at the beginning of each 
device lifetime) gives

\begin{equation}
  \rate{A}_{cap} = R_\alpha \rate{C}_{cap} \, .
\end{equation}

Similar derivations can be employed for all other budget categories.

\begin{align}
  \rate{A}_s &= \rate{C}_s \\
  \rate{A}_{om} &= \rate{C}_{om} \\
  \rate{A}_d &= R_\omega \rate{C}_d \\
  \rate{A}_o &= \rate{C}_o \\
  \rate{A}_M &= \rate{M} \\
  \rate{A}_N &= \rate{N}
\end{align}
%
where 
$R_\omega \equiv \frac{\gamma_{\tlife}}{\phi_{\oneyr}} \frac{\tlife}{\oneyr}$
(with subscript $\omega$ indicating paymets 
at the end of each device lifetime), and 
$\rate{C}_d \equiv C_d / t_{\life}$, the annualized disposal cost 
without discounting.

The budget constraint with annualized terms is given as
%
\begin{equation}
  \rate{A}_M = \rate{A}_{cap} + 
               \rate{A}_s + 
               \rate{A}_{om} + 
               \rate{A}_d + 
               \rate{A}_o + 
               \rate{A}_N \, .
\end{equation}
%
Substituting cost terms gives

\begin{equation}
  \rate{M} = R_\alpha \rate{C}_{cap} + 
             \rate{C}_s + 
             \rate{C}_{om} + 
             R_\omega \rate{C}_d + 
             \rate{C}_o + 
             \rate{N} \, .
\end{equation}
%
Substituting $\rate{C}_s = p_s \rate{q}_s$ and 
$\rate{C}_o = p_o \rate{q}_o$ and rearranging gives
the budget constraint used in this paper.

\begin{equation}
  \budgetconstraint
\end{equation}

The term $R_\alpha$ represents the additional cost of annual interest
payments when the device is purchased with a loan.
When $r > 0$, $R_\alpha > 1$.
When $r = 0$, $R_\alpha = 1$, as proved below
(Section~\ref{sec:proof_R_alpha}).

The term $R_\omega$ represents the reduction of disposal costs 
if the device owner pays for disposal costs with money
invested annually at real interest rate $r$. 
When $r > 0$, $0 < R_\omega < 1$.
When $r = 0$, $R_\omega = 0$, as proved below
(Section~\ref{sec:proof_R_omega}).


%------------------------------
\subsubsection{Proof: $R_\alpha = 1$ when $r = 0$}
\label{sec:proof_R_alpha}
%------------------------------

We expect that $R_\alpha = 1$ when $r = 0$.
However, direct substitution of $r = 0$ into the expression
for $R_\alpha$ gives $\frac{0}{0}$, 
so we rather assess
$\lim_{r \to 0^+} R_\alpha \questionequal 1$.

Substituting for $R_\alpha$ gives

\begin{equation}
  \lim_{r \to 0^+} \left( \frac{\phi_{\tlife}}{\phi_{\oneyr}} \frac{\tlife}{\oneyr} \right) 
  \questionequal 1\, .
\end{equation}
%
Substituting for $\phi$ terms gives

\begin{equation}
  \lim_{r \to 0^+} \left[ \frac{\frac{(1 + r)^{\tlife}}{(1 + r)^{\tlife} - 1}}{\frac{(1 + r)^{\oneyr}}{(1 + r)^{\oneyr} - 1}} \cdot \frac{\tlife}{\oneyr} \right] \questionequal 1 \, .
\end{equation}
%
Distributing double-fractions gives

\begin{equation}
  \lim_{r \to 0^+} \left[
  \frac{(1 + r)^{\tlife}}{(1 + r)^{\oneyr}} \cdot
  \frac{(1 + r)^{\oneyr} - 1}{(1 + r)^{\tlife} - 1} \cdot
  \frac{\tlife}{\oneyr}
  \right] \questionequal 1 \, .
\end{equation}
%
Multiplying terms in numerator and demoninator gives

\begin{equation}
  \lim_{r \to 0^+} \left\{
  \frac{\left[(1 + r)^{\tlife} (1 + r)^{\oneyr} - (1 + r)^{\tlife} \right] \frac{\tlife}{\oneyr}}{(1 + r)^{\tlife} (1 + r)^{\oneyr} - (1 + r)^{\oneyr}}
  \right\} \questionequal 1 \, .
\end{equation}
%
Applying L'H\^{o}pital's rule gives

\begin{equation}
  \lim_{r \to 0^+} \left(
  \frac{\frac{\partial}{\partial r} \left\{ \left[(1 + r)^{\tlife} (1 + r)^{\oneyr} - (1 + r)^{\tlife} \right] \frac{\tlife}{\oneyr} \right\}}
  {\frac{\partial}{\partial r} \left[ (1 + r)^{\tlife} (1 + r)^{\oneyr} - (1 + r)^{\oneyr}\right]}
  \right) \questionequal 1 \, .
\end{equation}
%
Applying the chain rule repeatedly gives

\begin{equation}
  \lim_{r \to 0^+} \left(
  \frac{\frac{\tlife}{\oneyr} 
      \left\{ \frac{\partial}{\partial r} \left[  (1 + r)^{\tlife} (1 + r)^{\oneyr} \right]  - 
              \frac{\partial}{\partial r} \left[  (1 + r)^{\tlife} \right] 
      \right\}}
  {\frac{\partial}{\partial r} \left[    (1 + r)^{\tlife} (1 + r)^{\oneyr} \right]  - 
              \frac{\partial}{\partial r} \left[  (1 + r)^{\oneyr} \right] }
  \right) \questionequal 1 \, .
\end{equation}

Several intermediate results are helpful.

\begin{equation}
  \lim_{r \to 0^+} \left\{\frac{\partial}{\partial r} \left[ (1 + r)^{\tlife} \right] \right\} = \tlife
\end{equation}

\begin{equation}
  \lim_{r \to 0^+} \left\{\frac{\partial}{\partial r} \left[ (1 + r)^{\oneyr} \right] \right\} = \oneyr
\end{equation}

\begin{equation}
  \lim_{r \to 0^+} \left\{\frac{\partial}{\partial r} \left[ (1 + r)^{\tlife} (1 + r)^{\oneyr} \right] \right\} = \tlife (1 + r)^{\oneyr}  + \oneyr (1 + r)^{\tlife} 
\end{equation}

Substituting the intermediate results gives

\begin{equation}
  \lim_{r \to 0^+} \left\{
                         \frac{\frac{\tlife}{\oneyr}
                         \left[ (1 + r)^{\oneyr} (\tlife) + 
                         (1 + r)^{\tlife} (\oneyr) - 
                         \tlife \right]}
                        {(1 + r)^{\oneyr} (\tlife) + 
                         (1 + r)^{\tlife} (\oneyr) - 
                         \oneyr}
                    \right\} \questionequal 1 \, .
\end{equation}
%
Setting $r = 0$ in the remaining terms gives

\begin{equation}
  \frac{\frac{\tlife}{\oneyr}
              \left[ (1) (\tlife) + 
                     (1) (\oneyr) - 
                     \tlife
              \right]}
       {(1) (\tlife) + 
        (1) (\oneyr) - 
        \oneyr} \questionequal 1 \, .
\end{equation}
%
Simplifying gives

\begin{align}
  \frac{\left( \frac{\tlife}{\oneyr} \right) (\oneyr)}{\tlife} &\questionequal 1 \\
  1 &\stackrel{\checkmark}{=} 1
\end{align}
%
as expected.



%------------------------------
\subsubsection{Proof: $R_\omega = 1$ when $r = 0$}
\label{sec:proof_R_omega}
%------------------------------

We expect that $R_\omega = 1$ when $r = 0$.
However, direct substitution of $r = 0$ into the expression
for $R_\omega$ gives $\frac{0}{0}$, 
so we rather assess
$\lim_{r \to 0^+} R_\omega \questionequal 1$.

Substituting for $R_\omega$ gives

\begin{equation}
  \lim_{r \to 0^+} \left( \frac{\gamma_{\tlife}}{\phi_{\oneyr}} \frac{\tlife}{\oneyr} \right) \questionequal 1 \, .
\end{equation}
%
Substituting for $\gamma$ and $\phi$ terms gives

\begin{equation}
  \lim_{r \to 0^+} \left[ \frac{\frac{1}{(1 + r)^{\tlife} - 1}}{\frac{(1 + r)^{\oneyr}}{(1 + r)^{\oneyr} - 1}} \frac{\tlife}{\oneyr} \right] \questionequal 1 \, .
\end{equation}
%
Distributing double-fractions gives

\begin{equation}
  \lim_{r \to 0^+} \left[
  \frac{1}{(1 + r)^{\oneyr}} \cdot
  \frac{(1 + r)^{\oneyr} - 1}{(1 + r)^{\tlife} - 1} \cdot
  \frac{\tlife}{\oneyr}
  \right] \questionequal 1 \, .
\end{equation}
%
Multiplying terms in numerator and demoninator gives

\begin{equation}
  \lim_{r \to 0^+} 
  \left\{
    \frac{\left[ (1 + r)^{\oneyr} - 1 \right] \left( \frac{\tlife}{\oneyr} \right)}
    {(1 + r)^{\tlife} (1 + r)^{\oneyr} - (1 + r)^{\oneyr}} 
  \right\} \questionequal 1 \, .
\end{equation}
%
Applying L'H\^{o}pital's rule gives

\begin{equation}
  \lim_{r \to 0^+} 
  \left\{
    \frac{\frac{\tlife}{\oneyr}
      \frac{\partial}{\partial r} 
        \left[ (1 + r)^{\oneyr} - 1 \right]}
    {\frac{\partial}{\partial r} \left[ (1 + r)^{\tlife} (1 + r)^{\oneyr} \right] - 
     \frac{\partial}{\partial r} \left[ (1 + r)^{\oneyr} \right]}
  \right\} \questionequal 1 \, .
\end{equation}
%
Applying the intermediate results from Section~\ref{sec:proof_R_alpha} yields

\begin{equation}
  \lim_{r \to 0^+} 
  \left[
    \frac{\left(\frac{\tlife}{\oneyr}\right) (\oneyr)}
         {(1 + r)^{\oneyr} (\tlife) + 
          (1 + r)^{\tlife} (\oneyr) - 
          \oneyr}
  \right] \questionequal 1 \, .
\end{equation}
%
Setting $r = 0$ in the remaining terms gives

\begin{equation}
  \frac{\left( \frac{\tlife}{\oneyr} \right) (\oneyr)}
  {(1) (\tlife) + (1) \oneyr - \oneyr} \questionequal 1 \, .
\end{equation}
%
Simplifying the denominator gives

\begin{align}
  \frac{\left( \frac{\tlife}{\oneyr} \right) (\oneyr)}
       {\tlife} \questionequal 1 \\
       1 \stackrel{\checkmark}{=} 1
\end{align}
%
as expected.






%++++++++++++++++++++++++++++++
\subsection{Relationships for rebound effects}
\label{sec:relationships_for_stages}
%++++++++++++++++++++++++++++++

For each energy rebound effect in Fig.~\ref{fig:flowchart},
energy and financial analysis must be performed.
The purposes of the analyses are to determine for each effect
%
\begin{enumerate*}[label={(\roman*)}]

  \item an expression for energy rebound~($Re$) for the effect and

  \item an equation for net savings~($\rate{N}$) remaining after the effect.

\end{enumerate*}

Analysis of each rebound effect
involves a set of assumptions and constraints
as shown in Table~\ref{tab:analysis_assumptions}.
In Table~\ref{tab:analysis_assumptions},
relationships for emplacement effect
embodied energy rates ($\rbempl{E}_{emb}$ and $\raempl{E}_{emb}$),
capital expenditure rates ($\rbempl{C}_{cap}$ and $\raempl{C}_{cap}$), and
operations, maintenance, and disposal
expenditure rates ($\rbempl{C}_{\omd}$ and $\raempl{C}_{\omd}$)
are typical, and
inequalities could switch direction for a specific EEU.
Macro effect relationships are given for a single device only.
If the EEU is deployed at scale across the economy,
the energy service consumption rate~($\ramacro{q}_s$),
device energy consumption rate~($\ramacro{E}_s$),
embodied energy rate~($\ramacro{E}_{emb}$),
capital expenditure rate~($\ramacro{C}_{cap}$), and
operations, maintenance, and disposal expenditure rate~($\ramacro{C}_{\omd}$)
will all increase in proportion to the number of devices emplaced.

% The next command tells RStudio to do "Compile PDF" on HSB.Rnw,
% instead of this file, thereby eliminating the need to switch back to HSB.Rnw 
% before building the paper.
%!TEX root = ../HSB_framework.Rnw

\begin{landscape}

\begin{table}
\footnotesize
\centering
\caption{Assumptions and constraints for analysis of rebound effects.}
\label{tab:analysis_assumptions}

\begin{tabular}{r c c c c c}
\toprule
Parameter & \EmplEffect{} & \SubEffect & \IncEffect & \MacroEffect \\
\midrule
Energy price                      & $\bempl{p}_E  = \aempl{p}_E$         
                                  & $\bsub{p}_E   = \asub{p}_E$ 
                                  & $\binc{p}_E   = \ainc{p}_E$ 
                                  & $\bmacro{p}_E = \amacro{p}_E$ \\
%
Energy service efficiency         & $\bempl{\eta}  < \aempl{\eta}$         
                                  & $\bsub{\eta}   = \asub{\eta}$ 
                                  & $\binc{\eta}   = \ainc{\eta}$ 
                                  & $\bmacro{\eta} = \amacro{\eta}$ \\
%
Energy service price              & $\bempl{p}_s  > \aempl{p}_s$          
                                  & $\bsub{p}_s   = \asub{p}_s$ 
                                  & $\binc{p}_s   = \ainc{p}_s$  
                                  & $\bmacro{p}_s = \amacro{p}_s$ \\
%
Other goods price                 & $\bempl{p}_o  = \aempl{p}_o$          
                                  & $\bsub{p}_o   = \asub{p}_o$ 
                                  & $\binc{p}_o   = \ainc{p}_o$  
                                  & $\bmacro{p}_o = \amacro{p}_o$ \\
%
Energy service consumption rate   & $\rbempl{q}_s  = \raempl{q}_s$         
                                  & $\rbsub{q}_s   < \rasub{q}_s$ 
                                  & $\rbinc{q}_s   < \rainc{q}_s$ 
                                  & $\rbmacro{q}_s = \ramacro{q}_s$ \\
%
Other goods consumption rate      & $\rbempl{q}_o  = \raempl{q}_o$         
                                  & $\rbsub{q}_o   > \rasub{q}_o$ 
                                  & $\rbinc{q}_o   < \rainc{q}_o$ 
                                  & $\rbmacro{q}_o = \ramacro{q}_o$ \\
%
Device energy consumption rate    & $\rbempl{E}_s  > \raempl{E}_s$
                                  & $\rbsub{E}_s   < \rasub{E}_s$ 
                                  & $\rbinc{E}_s   < \rainc{E}_s$ 
                                  & $\rbmacro{E}_s = \ramacro{E}_s$ \\
%
Embodied energy rate              & $\rbempl{E}_{emb}  < \raempl{E}_{emb}$ 
                                  & $\rbsub{E}_{emb}   = \rasub{E}_{emb}$ 
                                  & $\rbinc{E}_{emb}   = \rainc{E}_{emb}$ 
                                  & $\rbmacro{E}_{emb} = \ramacro{E}_{emb}$ \\
%
Device lifetime                   & $\bempl{t}_{\life}  < \aempl{t}_{\life}$ 
                                  & $\bsub{t}_{\life}   = \asub{t}_{\life}$ 
                                  & $\binc{t}_{\life}   = \ainc{t}_{\life}$ 
                                  & $\bmacro{t}_{\life} = \amacro{t}_{\life}$ \\
%
Beginning-of-life discount factor & $\bempl{R}_\alpha  < \aempl{R}_\alpha$ 
                                  & $\bsub{R}_\alpha   = \asub{R}_\alpha$ 
                                  & $\binc{R}_\alpha   = \ainc{R}_\alpha$ 
                                  & $\bmacro{R}_\alpha = \amacro{R}_\alpha$ \\
%
End-of-life discount factor       & $\bempl{R}_\omega  > \aempl{R}_\omega$
                                  & $\bsub{R}_\omega   = \asub{R}_\omega$
                                  & $\binc{R}_\omega   = \ainc{R}_\omega$
                                  & $\bmacro{R}_\omega = \amacro{R}_\omega$ \\
%
Capital expenditure rate          & $\rbempl{C}_{cap}  < \raempl{C}_{cap}$ 
                                  & $\rbsub{C}_{cap}   = \rasub{C}_{cap}$ 
                                  & $\rbinc{C}_{cap}   = \rainc{C}_{cap}$ 
                                  & $\rbmacro{C}_{cap} = \ramacro{C}_{cap}$ \\
%
Ops., maint., and disp.\ expenditure rate & $\rbempl{C}_{\omd} < \raempl{C}_{\omd}$ 
                                  & $\rbsub{C}_{\omd}   = \rasub{C}_{\omd}$ 
                                  & $\rbinc{C}_{\omd}   = \rainc{C}_{\omd}$ 
                                  & $\rbmacro{C}_{\omd} = \ramacro{C}_{\omd}$ \\
%
Energy service expenditure rate   & $\rbempl{C}_s  > \raempl{C}_s$
                                  & $\rbsub{C}_s   < \rasub{C}_s$ 
                                  & $\rbinc{C}_s   < \rainc{C}_s$ 
                                  & $\rbmacro{C}_s  = \ramacro{C}_s$ \\
%
Other goods expenditure rate      & $\rbempl{C}_o  = \raempl{C}_o$         
                                  & $\rbsub{C}_o   > \rasub{C}_o$ 
                                  & $\rbinc{C}_o   < \rainc{C}_o$ 
                                  & $\rbmacro{C}_o  = \ramacro{C}_o$ \\
%
Income                            & $\rbempl{M} = \raempl{M}$         
                                  & $\rbsub{M}  = \rasub{M}$ 
                                  & $\rbinc{M}  = \rainc{M}$ 
                                  & $\rbmacro{M} = \ramacro{M}$  \\
%
Net savings                       & 0 = $\rbempl{N} <   \raempl{N}$         
                                  & $\rbsub{N}      <   \rasub{N}$ 
                                  & $\rbinc{N}      >   \rainc{N} = 0$ 
                                  & $\rbmacro{N}    =   \ramacro{N} = 0$  \\
\bottomrule
\end{tabular}


\end{table}

\end{landscape}



%++++++++++++++++++++++++++++++
\subsection{Derivations}
\label{sec:derivations}
%++++++++++++++++++++++++++++++

Derivations for rebound definitions and net savings equations
are presented in Tables~\ref{tab:empleffect}--\ref{tab:macroeffect},
one for each rebound effect in Fig.~\ref{fig:flowchart}.
Energy and financial analyses are shown side by side, because
each informs the other.

Several terms in Tables~\ref{tab:empleffect}--\ref{tab:macroeffect}
are zeroed, e.g.\ $\cancelto{0}{\Delta \raempl{C}_g}$.
These zeroes can be traced back to Table~\ref{tab:analysis_assumptions}.
Table~\ref{tab:zeroed_terms} highlights the equations
in Table~\ref{tab:analysis_assumptions}
that justify zeroing each term.

\begin{table}
\footnotesize
\centering % Centered table
\caption{Justification for zeroed terms in Tables~\ref{tab:empleffect}--\ref{tab:macroeffect}.}
\begin{tabular}{r l}
  \toprule
  Zeroed term & Justification (from Table~\ref{tab:analysis_assumptions}). \\
  \midrule
  $\cancelto{0}{\Delta \raempl{C}_g}$    & $\rorig{C}_g = \raempl{C}_g$ ($\rate{C}_g$ unchanged across emplacement effect.) \\
  $\cancelto{0}{\rorig{N}}$              & $0 = \rorig{N}$ (Net savings are zero prior to the EEU.) \\
  $\cancelto{0}{\Delta \rasub{E}_{emb}}$ & $\rbsub{E}_{emb} = \rasub{E}_{emb}$ ($\rate{E}_{emb}$ unchanged across substitution effect.) \\
  $\cancelto{0}{\Delta \rasub{C}_{\omd}}$ & $\rbsub{C}_{\omd} = \rasub{C}_{\omd}$ ($\rate{C}_{\omd}$ unchanged across substitution effect.) \\
  $\cancelto{0}{\Delta \rainc{E}_{emb}}$ & $\rbinc{E}_{emb} = \rainc{E}_{emb}$ ($\rate{E}_{emb}$ unchanged across income effect.) \\
  $\cancelto{0}{\Delta \rainc{C}_{\omd}}$ & $\rbinc{C}_{\omd} = \rainc{C}_{\omd}$ ($\rate{C}_{\omd}$ unchanged across income effect.) \\
  $\cancelto{0}{\rainc{N}}$              & $\rainc{N} = 0$ (All net savings are spent in the income effect.) \\
  \bottomrule
\end{tabular}
\label{tab:zeroed_terms}
\end{table}



% Derivation tables

% The next command tells RStudio to do "Compile PDF" on HSB.Rnw,
% instead of this file, thereby eliminating the need to switch back to HSB.Rnw 
% before building the paper.
%!TEX root = ../HSB_framework.Rnw

% This file contains the derivation for emplacement effect rebound terms and net savings.

\newgeometry{left=1in, right=1in, top=1in, bottom=1in}  

\begin{landscape}

\linespread{1}

%%%%%%%%%%%%%%%%%%%%%%%%%%%%%%%%%%%%%%%%
%%%%%%%%%% Emplacement Effect %%%%%%%%%%
%%%%%%%%%%%%%%%%%%%%%%%%%%%%%%%%%%%%%%%%
\derivheader{\refstepcounter{table} Table~\thetable \label{tab:empleffect}. \bf{\EmplEffect}}

\sectionsep{}

%%%%%%%%%% Before Emplacement Effect %%%%%%%%%%
\derivsection{before ($\circ$)}
{
% Original energy
\begin{equation} \label{eq:E_acct_orig}
  \Eacctorig{}
\end{equation}
}
{
% Original financial
\begin{equation} \label{eq:M_acct_orig}
  \Macctorig{}
\end{equation}
}

\sectionsep{}

%%%%%%%%%% After Emplacement Effect %%%%%%%%%%
\derivsection{after ($*$)}
{
% After device energy
\begin{equation} \label{eq:E_acct_aemp}
  \Eacctaempl{}
\end{equation}
}
{
% After device financial
\begin{equation} \label{eq:M_acct_aemp}
  \Macctaempl{}
\end{equation}
}

\sectionsep{}

%%%%%%%%%% Derivations for Emplacement Effect %%%%%%%%%%
\derivsection{}
% Emplacement effect: energy differences
{
%%%%%%%%%% Energy Emplacement Effect %%%%%%%%%%
~

Note: $\rate{C}_{\omd} \equiv \rate{C}_{\om} + R_{\omega} \rate{C}_d$.

Take differences to obtain the change in energy consumption, \\
$\Delta \raempl{E} \equiv \raempl{E} - \rbempl{E}$.
%
\begin{equation}
  \Delta \raempl{E} = \Delta \raempl{E}_s
                      + \Delta \raempl{E}_{emb}
                      + (\Delta \raempl{C}_{\omd}
                      + \cancelto{0}{\Delta \raempl{C}_g}) I_E
\end{equation}
%
Thus, 
%
\begin{equation}
\Delta \raempl{E} = \Delta \raempl{E}_s + \Delta \raempl{E}_{emb} + \Delta \raempl{C}_{\omd} I_E \; .
\end{equation}
%
Define
%
\begin{equation} \label{eq:Sdot_def}
\Sdot \equiv -\Delta \raempl{E}_s
\end{equation}
%
(Also see Eqs.~(\ref{eq:S_dot_def}) and~(\ref{eq:Sdot})). 
Use Eq.~(\ref{eq:Re_def}) to obtain
%
\begin{equation}
Re_{empl} = 1 - \frac{-\Delta \raempl{E}}{\Sdot} 
          = 1 - \frac{-\Delta \raempl{E}_s}{\Sdot} 
              - \frac{-\Delta \raempl{E}_{emb}}{\Sdot}
              - \frac{-\Delta \raempl{C}_{\omd} I_E}{\Sdot} \; .
\end{equation}
%
Define $Re_{dempl} \equiv 1 - \frac{-\Delta \raempl{E}_s}{\Sdot} (= 0)$, 
$Re_{iempl} \equiv Re_{emb} + Re_{\omd}$, 
$Re_{emb} \equiv \frac{\Delta \raempl{E}_{emb}}{\Sdot}$,
$Re_{\omd} \equiv \frac{\Delta \raempl{C}_{\omd} I_E}{\Sdot}$, 
$Re_{\omd} = Re_{\om} + Re_d$,
$Re_{\om} \equiv \frac{\Delta \raempl{C}_{\om} I_E}{\Sdot}$, and 
$Re_d \equiv \frac{\Delta \aempl{(R_\omega \rate{C}_d)} I_E}{\Sdot}$
such that
%
\begin{equation} \label{eq:Re_empl_def}
Re_{empl} = Re_{dempl} + Re_{iempl} \; .
\end{equation}
}
{
%%%%%%%%%% Financial Emplacement Effect %%%%%%%%%%
~
    
Use the monetary constraint ($\rate{M}$)
and constant spending on other items ($\rbempl{C}_g = \raempl{C}_g$) to cancel terms to obtain

\begin{align}
  p_E \rbempl{E}_s &+ \bempl{\tau}_\alpha \rbempl{C}_{cap} + \rbempl{C}_{\omd} + \cancel{\rbempl{C}_g} + \cancelto{0}{\rbempl{N}} \nonumber \\
                   &= p_E \raempl{E}_s + \aempl{\tau}_\alpha \raempl{C}_{cap} + \raempl{C}_{\omd} + \cancel{\raempl{C}_g}  + \raempl{N} \; .
\end{align}
%
Solving for $\Delta \raempl{N} \equiv \raempl{N} - \cancelto{0}{\rbempl{N}}$ gives 
%
\begin{equation}
  \Delta \raempl{N} = p_E(\rbempl{E}_s - \raempl{E}_s) 
                      + \bempl{\tau}_\alpha \rbempl{C}_{cap} - \aempl{\tau}_\alpha \raempl{C}_{cap}
                      + \rbempl{C}_{\omd} - \raempl{C}_{\omd} \; .
\end{equation}
%
Rewriting with $\Delta$ terms gives
%
\begin{equation}
  \Delta \raempl{N} = - p_E \Delta \raempl{E}_s - \Delta \aempl{(R_\alpha \rate{C}_{cap})} - \Delta \raempl{C}_{\omd} \; .
\end{equation}
%
Substituting Eq.~(\ref{eq:Sdot_def}) gives
%
\begin{equation} \label{eq:N_dot_star_empl}
  \Delta \raempl{N} = \raempl{N} = p_E \Sdot - \Delta \aempl{(R_\alpha \rate{C}_{cap})} - \Delta \raempl{C}_{\omd} \; .
\end{equation}
%
Freed cash ($\rate{G}$) resulting from the EEU, 
before any energy takeback, is given by 
%
\begin{equation} \label{eq:G_dot}
  \rate{G} = p_E \Sdot \; .
\end{equation}

Note that Eq.~(\ref{eq:M_acct_orig}) and $\rbempl{N} = 0$ can be used to calculate $\rbempl{C}_g$ as
%
\begin{equation} \label{eq:C_dot_o}
  \rbempl{C}_g = \rate{M} - p_E \rbempl{E}_s - \bempl{\tau}_\alpha \rbempl{C}_{cap} - \rbempl{C}_{\omd} \; .
\end{equation}
%

}
\end{landscape}

\restoregeometry




% The next command tells RStudio to do "Compile PDF" on HSB.Rnw,
% instead of this file, thereby eliminating the need to switch back to HSB.Rnw 
% before building the paper.
%!TEX root = ../HSB_framework.Rnw

% This file contains the derivation for substitution effect rebound and net savings.

\begin{landscape}

\linespread{1}

%%%%%%%%%%%%%%%%%%%%%%%%%%%%%%%%%%%%%%%%%
%%%%%%%%%% Substitution Effect %%%%%%%%%%
%%%%%%%%%%%%%%%%%%%%%%%%%%%%%%%%%%%%%%%%%
\derivheader{\refstepcounter{table} Table~\thetable \label{tab:subeffect}. \bf{\SubEffect}}

\sectionsep{}

%%%%%%%%%% Before Substitution Effect %%%%%%%%%%
\derivsection{before ($*$)}
{
% Before substitution energy
\begin{equation}
  \Eacctbsub{} \tag{\ref{eq:E_acct_aemp}}
\end{equation}
}
{
% Before substitution financial
\begin{equation}
  \Macctbsub{} \tag{\ref{eq:M_acct_aemp}}
\end{equation}
}

\sectionsep{}


%%%%%%%%%% After Substitution Effect %%%%%%%%%%
\derivsection{after ($\wedge$)}
{
% After substitution energy
\begin{equation} \label{eq:E_acct_asub}
  \Eacctasub{}
\end{equation}
}
{
% After substitution financial
\begin{equation} \label{eq:M_acct_asub}
  \Macctasub{}
\end{equation}
}

\sectionsep{}

%%%%%%%%%% Derivations for Substitution Effect %%%%%%%%%%
\derivsection{}
% Substitution effect: energy differences
{
%%%%%%%%%% Energy Substitution Effect %%%%%%%%%%
~
  
Take differences to obtain the change in energy consumption, $\Delta \rasub{E} \equiv \rasub{E} - \rbsub{E}$.
%
\begin{equation}
  \Delta \rasub{E} = \Delta \rasub{E}_s 
                      + \cancelto{0}{\Delta \rasub{E}_{emb}} 
                      + (\cancelto{0}{\Delta \rasub{C}_{\md}} + \Delta \rasub{C}_o) I_E
\end{equation}
%
Thus, 
%
\begin{equation}
  \Delta \rasub{E} = \Delta \rasub{E}_s + \Delta \rasub{C}_o I_E \; .
\end{equation}
%
All terms are energy takeback rates.
Divide by $\Sdot$
to create rebound terms.
%
\begin{equation}
    \frac{\Delta \rasub{E}}{\Sdot} = \frac{\Delta \rasub{E}_s}{\Sdot} + \frac{\Delta \rasub{C}_o I_E}{\Sdot}
\end{equation}
%
Define 
$Re_{sub} \equiv \frac{\Delta \rasub{E}}{\Sdot}$, 
$Re_{dsub} \equiv \frac{\Delta \rasub{E}_s}{\Sdot}$, and
$Re_{isub} \equiv \frac{\Delta \rasub{C}_o I_E}{\Sdot}$,
such that
%
\begin{equation} \label{eq:Re_sub_def}
  Re_{sub} = Re_{dsub} + Re_{isub} \; .
\end{equation}

}
{
%%%%%%%%%% Financial Substitution Effect %%%%%%%%%%
~
  
Use the monetary constraint ($\rbsub{M} = \rasub{M}$) to obtain

\begin{align}
  p_E \raempl{E}_s &+ \cancel{\raempl{C}_{cap}} + \cancel{\raempl{C}_{\md}} + \raempl{C}_o + \raempl{N} \nonumber \\
                   &= p_E \rasub{E}_s + \cancel{\rasub{C}_{cap}} + \cancel{\rasub{C}_{\md}} + \rasub{C}_o + \rasub{N} \; .
\end{align}
%
For the substitution effect, there is no change in capital or maintenance and disposal costs
($\rasub{C}_{cap} = \rbsub{C}_{cap}$ and $\rasub{C}_{\md} = \rbsub{C}_{\md}$).
Solving for $\Delta \rasub{N} \equiv \rasub{N} - \rbsub{N}$ gives
%
\begin{equation} \label{eq:N_dot_hat_eqn}
  \Delta \rasub{N} = - p_E \Delta \rasub{E}_s - \Delta \rasub{C}_o \; .
\end{equation}
%
The substitution effect adjusts net savings relative to $\rbsub{N}$
by $\Delta \rasub{N}$.
Thus, $\rasub{N} = \rbsub{N} + \Delta \rasub{N}$.
Substituting Eqs.~(\ref{eq:N_dot_star_empl}), (\ref{eq:G_dot}), and~(\ref{eq:N_dot_hat_eqn})
yields
%
\begin{equation} \label{eq:N_dot_after_sub}
  \rasub{N} = \rate{G} - \Delta \rbsub{C}_{cap} - \Delta \rbsub{C}_{\md} - p_E \Delta \rasub{E}_s - \Delta \rasub{C}_o \; .
\end{equation}
%
}

\end{landscape}



% The next command tells RStudio to do "Compile PDF" on HSB.Rnw,
% instead of this file, thereby eliminating the need to switch back to HSB.Rnw 
% before building the paper.
%!TEX root = ../HSB_framework.Rnw

% This file contains the derivation for income effect rebound and net savings.

\begin{landscape}

\linespread{1}

%%%%%%%%%%%%%%%%%%%%%%%%%%%%%%%%%%%
%%%%%%%%%% Income Effect %%%%%%%%%%
%%%%%%%%%%%%%%%%%%%%%%%%%%%%%%%%%%%
\derivheader{\refstepcounter{table} Table~\thetable \label{tab:inceffect}. \bf{\IncEffect}}

\sectionsep{}

%%%%%%%%%% Before Income Effect %%%%%%%%%%
\derivsection{before ($\wedge$)}
{
% Before income energy
\begin{equation}
  \Eacctbinc{} \tag{\ref{eq:E_acct_asub}}
\end{equation}
}
{
% Before income financial
\begin{equation}
  \Macctbinc{} \tag{\ref{eq:M_acct_asub}}
\end{equation}
}

\sectionsep{}

%%%%%%%%%% After Income Effect %%%%%%%%%%
\derivsection{after ($-$)}
{
% After income energy
\begin{equation} \label{eq:E_acct_ainc}
\Eacctainc{}
\end{equation}
}
{
% After income financial
\begin{equation} \label{eq:M_acct_ainc}
\Macctainc{}
\end{equation}
}

\sectionsep{}

%%%%%%%%%%% Derivations for Income Effect %%%%%%%%%%
\derivsection{}
% Income effect: energy differences
{
%%%%%%%%%% Energy Income Effect %%%%%%%%%%
~

Take differences to obtain the change in energy consumption, \\
$\Delta \rainc{E} \equiv \rainc{E} - \rbinc{E}$.
%
\begin{equation}
  \Delta \rainc{E} = \Delta \rainc{E}_s 
                     + \cancelto{0}{\Delta \rainc{E}_{emb}}
                     + (\cancelto{0}{\Delta \rainc{C}_{\omd}} + \Delta \rainc{C}_o) I_E
\end{equation}
%
Thus, 
%
\begin{equation}
  \Delta \rainc{E} = \Delta \rainc{E}_s + \Delta \rainc{C}_o I_E
\end{equation}
%
All terms are energy takeback rates.
Divide by $\Sdot$
to create rebound terms.
%
\begin{equation}
  \frac{\Delta \rainc{E}}{\Sdot} = \frac{\Delta \rainc{E}_s}{\Sdot} + \frac{\Delta \rainc{C}_o I_E}{\Sdot}
\end{equation}
%
Define 
$Re_{inc} \equiv \frac{\Delta \rainc{E}}{\Sdot}$, 
$Re_{dinc} \equiv \frac{\Delta \rainc{E}_s}{\Sdot}$, and 
$Re_{iinc} \equiv \frac{\Delta \rainc{C}_o I_E}{\Sdot}$,
such that
%
\begin{equation} \label{eq:Re_inc_def}
  Re_{inc} = Re_{dinc} + Re_{iinc} \; .
\end{equation}
%
}
{
%%%%%%%%%% Financial Income Effect %%%%%%%%%%
~

Use the monetary constraint ($\rate{M}$) to obtain

\begin{align}
  p_E \rasub{E}_s &+ \cancel{\asub{R}_\alpha \rasub{C}_{cap}} + \cancel{\rasub{C}_{\omd}} + \rasub{C}_o + \rasub{N} \nonumber \\
                  &= p_E \rainc{E}_s + \cancel{\ainc{R}_\alpha \rainc{C}_{cap}} + \cancel{\rainc{C}_{\omd}} + \rainc{C}_o + \cancelto{0}{\rainc{N}} \; .
\end{align}
%
For the income effect, there is no change in capital or maintainance, opoerations, and disposal costs
($\binc{R}_\alpha \rbinc{C}_{cap} = \ainc{R}_\alpha \rainc{C}_{cap}$ and
$\rbinc{C}_{\omd} = \rainc{C}_{\omd}$).
Notably, $\rainc{N} = 0$,
because it is assumed that all net monetary savings 
after the substitution effect ($\rasub{N}$) are spent on
more energy service ($\rbinc{E}_s < \rainc{E}_s$)
and
additional purchases in the economy ($\rbinc{C}_o < \rainc{C}_o$).
Solving for $\rbinc{N}$ gives 
%
\begin{equation} \label{eq:inc_budget_constraint}
  \rbinc{N} = p_E \Delta \rainc{E}_s + \Delta \rainc{C}_o \; ,
\end{equation}
%
the budget constraint for the income effect.
By construction, 
Eq.~(\ref{eq:inc_budget_constraint}) ensures
spending of net savings~($\rbinc{N}$) on
%
\begin{enumerate*}[label={(\roman*)}]
	
  \item additional energy services~($\Delta \rainc{E}_s$) and
  
  \item additional purchases of other goods in the economy~($\Delta \rainc{C}_o$) only.
    
\end{enumerate*}
}
\end{landscape}



% The next command tells RStudio to do "Compile PDF" on HSB.Rnw,
% instead of this file, thereby eliminating the need to switch back to HSB.Rnw 
% before building the paper.
%!TEX root = ../HSB_framework.Rnw

% This file contains the derivation for macro effect rebound.

\begin{landscape}

\linespread{1}

%%%%%%%%%%%%%%%%%%%%%%%%%%%%%%%%%%
%%%%%%%%%% Macro Effect %%%%%%%%%%
%%%%%%%%%%%%%%%%%%%%%%%%%%%%%%%%%%
\derivheader{\refstepcounter{table} Table~\thetable \label{tab:macroeffect}. \bf{\MacroEffect}}

\sectionsep{}

%%%%%%%%%% Before Macro Effect %%%%%%%%%%
\derivsection{before ($-$)}
{
% Before macro energy
\begin{equation}
  \rbmacro{E}
\end{equation}
}
{
% Before macro financial
}

\sectionsep{}

%%%%%%%%%% After Macro Effect %%%%%%%%%%
\derivsection{after ($\sim$)}
{
% After macro energy
\begin{equation}
\ramacro{E}
\end{equation}
}
{
% After macro financial
}

\sectionsep{}

%%%%%%%%%%% Derivations for macro Effect %%%%%%%%%%
\derivsection{}
% Macro effect: energy differences
{
%%%%%%%%%% Energy Macro Effect %%%%%%%%%%
~

Take differences to obtain the change in energy consumption,
%
\begin{equation}
  \Delta \ramacro{E} \equiv \ramacro{E} - \rbmacro{E} \; .
\end{equation}
%
The energy change due to the macro effect ($\Delta \ramacro{E}$) 
is a scalar multiple ($k$) of net savings ($\raempl{N}$), 
assumed to be spent at the energy intensity of the economy ($I_E$).
%
\begin{equation}
  \Delta \ramacro{E} = k \raempl{N} I_E
\end{equation}
%
All terms are energy takeback rates.
Divide by $\Sdot$
to create rebound terms.
%
\begin{equation}
  \frac{\Delta \ramacro{E}}{\Sdot} = \frac{k \raempl{N} I_E}{\Sdot}
\end{equation}
%
Define 
$Re_{\macro} \equiv \frac{\Delta \ramacro{E}}{\Sdot}$, 
such that
%
\begin{equation}
  Re_{\macro} = \frac{k \raempl{N} I_E}{\Sdot} \; . \tag{\ref{eq:Re_macro_def}}
\end{equation}
%
}
{
%%%%%%%%%% Financial Macro Effect %%%%%%%%%%
~
\centering

N/A
}
\end{landscape}



%++++++++++++++++++++++++++++++
\subsection{Rebound expressions}
\label{sec:rebound_expressions}
%++++++++++++++++++++++++++++++

All that remains is to determine expressions for each rebound effect.
We begin with the device-level expected energy savings rate~($\Sdot$), which
appears in the denominator of all rebound expressions.


%------------------------------
\subsubsection{Expected energy savings ($\Sdot$)}
\label{sec:Sdot}
%------------------------------

$\Sdot$ is the reduction of energy consumption rate
by the device due to the EEU.
No other effects are considered.

\begin{equation} \label{eq:S_dot_def}
  \Sdot \equiv \rbempl{E}_s - \raempl{E}_s
\end{equation}
%
The final energy consumption rates ($\rbempl{E}_s$ and $\raempl{E}_s$)
can be written as Eq.~(\ref{eq:typ_qs_eta_Edot}) in the forms
$\rbempl{E}_s = \rbempl{q}_s/\bempl{\eta}$ and
$\raempl{E}_s = \raempl{q}_s/\aempl{\eta}$.

\begin{equation}
  \Sdot = \frac{\rbempl{q}_s}{\bempl{\eta}} - \frac{\raempl{q}_s}{\aempl{\eta}}
\end{equation}
%
With reference to Table~\ref{tab:analysis_assumptions},
we use $\raempl{q}_s = \rbempl{q}_s$ to obtain

\begin{equation}
  \Sdot = \frac{\rbempl{q}_s}{\bempl{\eta}} - \frac{\rbempl{q}_s}{\aempl{\eta}} \; .
\end{equation}
%
When the EEU increases efficiency such that
$\bempl{\eta} < \aempl{\eta}$,
expected energy savings grows ($\Sdot > 0$)
as the rate of final energy consumption declines,
as expected.
As $\aempl{\eta} \rightarrow \infty$,
all final energy consumption is eliminated ($\raempl{E}_s \rightarrow 0$), and
$\Sdot = \rbempl{q}_s/\bempl{\eta} = \rbempl{E}_s$.
(Of course, $\aempl{\eta} \rightarrow \infty$ is impossible.
See \citet{Paoli:2020aa} for a recent discussion of upper limits to device efficiencies.)

After rearrangement and using $\rbempl{E}_s = \rbempl{q}_s/\bempl{\eta}$,
we obtain a convenient form

\begin{equation}
  \Sdot = \Sdoteqn \; .  \tag{\ref{eq:Sdot}}
\end{equation}


%------------------------------
\subsubsection{\Empleffect{}}
\label{sec:Re_emp}
%------------------------------

The emplacement effect accounts for performance of the EEU only.
No behavior changes occur.
The direct emplacement effect of the EEU is device energy savings and energy cost savings.
The indirect emplacement effects of the EEU produce changes in the embodied energy rate and
the maintenance and disposal expenditure rates.
By definition, the direct emplacement effect has no rebound.
However, indirect emplacement effects may cause energy rebound.
Both direct and indirect emplacement effects are discussed below.


%..............................
\paragraph{Direct emplacement effect rebound expression ($Re_{dempl}$)}
\label{sec:Re_dempl}
%..............................

As shown in Table~\ref{tab:empleffect},
the direct rebound from the emplacement effect is
$Re_{dempl} \equiv 0$.
This result is expected,
because in the absence of
embodied energy, maintenance and disposal cost, or behavioral changes,
there is no takeback of energy savings
at the upgraded device.


%..............................
\paragraph{Indirect emplacement effect rebound expression ($Re_{iempl}$)}
\label{sec:Re_iempl}
%..............................

Indirect emplacement rebound effects
can occur at any point in the life cycle of an energy conversion device,
from manufacturing and distribution
to the use phase (maintenance),
and finally to disposal.
For simplicity, we group maintenance with disposal to form
two distinct indirect emplacement rebound effects:
%
\begin{enumerate*}[label={(\roman*)}]

  \item an embodied energy effect ($Re_{emb}$) and

  \item a maintenance and disposal effect ($Re_{\md}$).

\end{enumerate*}


%..............................
\paragraph{Embodied energy effect rebound expression ($Re_{emb}$)}
\label{sec:Re_emb}
%..............................

The first component of indirect emplacement effect rebound
involves embodied energy.
We define embodied energy consistent with the energy analysis literature
to be the sum of all final energy consumed
in the production of the energy conversion device.
The EEU
causes the embodied final energy of the device to change
from $\rbempl{E}_{emb}$ to $\raempl{E}_{emb}$.

Energy is embodied in the device within manufacturing and distribution supply chains
prior to consumer acquisition of the device.
For simplicity, we spread all embodied energy
over the lifetime of the device,
an equal amount assigned to each period.

Thus, we allocate embodied energy over the life of the original and upgraded devices
($\bempl{t}_{\life}$ and $\aempl{t}_{\life}$, respectively)
without discounting
to obtain embodied energy rates, such that
$\rbempl{E}_{emb} = \bempl{E}_{emb} / \bempl{t}_{\life}$
and
$\raempl{E}_{emb} = \aempl{E}_{emb} / \aempl{t}_{\life}$.
The change in embodied final energy due to the EEU (expressed as a rate) is given by
$\raempl{E}_{emb} - \rbempl{E}_{emb}$.
After substitution and algebraic rearrangement,
the change in embodied energy rate due to the EEU can be expressed as
$[ (\aempl{E}_{emb}/\bempl{E}_{emb})(\bempl{t}_{\life}/\aempl{t}_{\life}) - 1 ] \rbempl{E}_{emb}$,
a term that represents energy savings taken back due to embodied energy effects.
Thus, Eq.~(\ref{eq:Re_takeback}) can be employed to write embodied energy rebound as
%
\begin{equation}
  Re_{emb} = \Reembeqn{} \; . \tag{\ref{eq:Re_emb}}
\end{equation}

Embodied energy rebound can be either positive or negative, depending on
the sign of the term
$(\aempl{E}_{emb}/\bempl{E}_{emb})(\bempl{t}_{\life}/\aempl{t}_{\life}) - 1$.
Rising energy efficiency can be associated with increased device complexity
and more embodied energy,
such that $\aempl{E}_{emb} > \bempl{E}_{emb}$ and $Re_{emb} > 0$.
However, if the upgraded device has longer life than the original device
($\aempl{t}_{\life} > \bempl{t}_{\life}$),
$\raempl{E}_{emb} - \rbempl{E}_{emb}$ can be negative,
meaning that the upgraded device has a lower embodied energy rate than the original device.


%..............................
\paragraph{Operations, maintenance, and disposal effect rebound expression ($Re_{\omd}$)}
\label{sec:Re_OMd}
%..............................

In addition to embodied energy effects,
indirect emplacement rebound
can be associated with energy demanded by
operations, maintenance, and disposal expenditures.
We apply discounting to end-of-life disposal expenditures
to form expenditure rates such that
$\rbempl{C}_{\omd} = \rbempl{C}_{\om} + \bempl{R}_\omega \rbempl{C}_d$
and
$\raempl{C}_{\omd} = \raempl{C}_{\om} + \aempl{R}_\omega \raempl{C}_d$,
with $\rate{C}_d \equiv C_d / t_{\life}$.
(For details, see Appendix~\ref{sec:budget_constraint}.)

We assume, for simplicity, that operations, maintenance, and disposal expenditures
indicate energy consumption
elsewhere in the economy at its energy intensity~($I_E$).
Therefore, the change in energy consumption rate caused by a change in
operations, maintenance, and disposal expenditures
is given by $\Delta \raempl{C}_{\omd} I_E$.
This term is an energy takeback rate,
so maintenance and disposal rebound is given by

\begin{equation} \label{eq:Re_md_def}
  Re_{\omd} = \frac{\Delta \raempl{C}_{\omd} I_E}{\Sdot} \; ,
\end{equation}
%
as shown in Table~\ref{tab:empleffect}.
Slight rearrangement gives

\begin{equation}
  Re_{\omd} = \ReOMdeqn{} \; . \tag{\ref{eq:Re_md}}
\end{equation}

Rebound from operations, maintenance, and disposal can be positive or negative,
depending on the sign of the term $\raempl{C}_{\omd}/\rbempl{C}_{\omd} - 1$.


%------------------------------
\subsubsection{\Subeffect{}}
\label{sec:Re_sub}
%------------------------------

This section derives expressions for substitution effect rebound.
Two terms comprise substitution effect rebound,
direct substitution rebound ($Re_{dsub}$) and
indirect substitution rebound ($Re_{isub}$).
Assuming that conditions after the emplacement effect ($*$) are known,
both the
rate of energy service consumption ($\rasub{q}_s$) and
the rate of other goods consumption ($\rasub{C}_g$)
must be determined as a result of the substitution effect
(the $\wedge$ point).

The EEU's energy efficiency increase
($\bempl{\eta} < \amacro{\eta}$)
causes the price of the energy service provided by the device to fall
($\bempl{p}_s > \amacro{p}_s$).
The substitution effect quantifies the amount by which
the device user, in response,
increases the consumption rate of the energy service
($\rbsub{q}_s < \rasub{q}_s$)
and decreases the consumption rate of other goods
($\rbsub{q}_g > \rasub{q}_g$).

The increase in consumption of the energy service
substitutes for consumption of other goods in the economy,
subject to a utility constraint.
The reduction in spending on other goods in the economy
is captured by indirect substitution rebound~($Re_{isub}$).

We begin by deriving an expression for direct and indirect
substitution effect rebound ($Re_{dsub}$ and $Re_{isub}$, respectively).
Thereafter, we develop
a constant price elasticity (CPE) utility model and
a constant elasticity of substitution (CES) utility model
for determining the
post-substitution point ($\rasub{q}_s$ and $\rasub{C}_g$).


%..............................
\paragraph{Direct substitution effect rebound expression}
\label{sec:Redsub_expression}
%..............................

Direct substitution effect rebound ($Re_{dsub}$) is given by

\begin{equation}
  Re_{dsub} = \frac{\Delta \rasub{E}_s}{\Sdot} = \frac{\rasub{E}_s - \rbsub{E}_s}{\Sdot} \; . \tag{\ref{eq:Re_dsub_def}}
\end{equation}
%
Substituting the typical relationship of Eq.~(\ref{eq:typ_qs_eta_Edot})
in the form $\rate{E}_s = \rate{q}_s/\eta$ gives

\begin{equation}
  Re_{dsub} = \frac{\frac{\rasub{q}_s}{\asub{\eta}} - \frac{\rbsub{q}_s}{\bsub{\eta}}}{\Sdot} \; .
\end{equation}
%
Realizing that $\bsub{\eta} = \asub{\eta}$ and rearranging produces

\begin{equation}
  Re_{dsub} = \frac{ \left( \frac{\rasub{q}_s}{\rbempl{q}_s}
                    - \frac{\rbsub{q}_s}{\rbempl{q}_s}  \right) \frac{\rbempl{q}_s}{\aempl{\eta}} }{\Sdot} \; .
\end{equation}
%
Recognizing that the rate of energy service consumption ($\rate{q}_s$)
is unchanged across the emplacement effect leads to $\rbsub{q}_s/\rbempl{q}_s = 1$.
Furthermore, $\rbempl{q}_s/\aempl{\eta}
            = (\rbempl{q}_s/\bempl{\eta}) (\bempl{\eta}/\aempl{\eta})
            = \rbempl{E}_s (\bempl{\eta}/\aempl{\eta})$,
such that

\begin{equation}
  Re_{dsub} = \left( \frac{\rasub{q}_s}{\rbempl{q}_s} - 1  \right)
              \frac{\rbempl{E}_s \frac{\bempl{\eta}}{\aempl{\eta}}}{\Sdot} \; .
\end{equation}
%
Substituting Eq.~(\ref{eq:Sdot}) for $\Sdot$ and rearranging gives

\begin{equation}
  Re_{dsub} = \frac{\frac{\rasub{q}_s}{\rbempl{q}_s} - 1}{\frac{\aempl{\eta}}{\bempl{\eta}} - 1}
              \left( \frac{\cancel{\rbempl{E}_s} \cancel{\frac{\bempl{\eta}}{\aempl{\eta}}}    }{  \cancel{\frac{\bempl{\eta}}{\aempl{\eta}}}    \cancel{\rbempl{E}_s}} \right) \; .
\end{equation}
%
Canceling terms yields

\begin{equation}
  Re_{dsub} = \frac{\frac{\rasub{q}_s}{\rbempl{q}_s} - 1}{\frac{\aempl{\eta}}{\bempl{\eta}} - 1} \; .
                                                               \tag{\ref{eq:Re_dsub_prelim}}
\end{equation}
%
Eq.~(\ref{eq:Re_dsub_prelim}) is the basis for developing
expressions for $Re_{dsub}$
under both the CPE and the CES utility models.


%..............................
\paragraph{Indirect substitution effect rebound expression}
\label{sec:Reisub_expression}
%..............................

Indirect substitution effect rebound ($Re_{isub}$) is given by

\begin{equation}
  Re_{isub} = \frac{\Delta \rasub{C}_g I_E}{\Sdot} = \frac{(\rasub{C}_g - \rbsub{C}_g) I_E}{\Sdot} \; . \tag{\ref{eq:Re_isub_def}}
\end{equation}
%
Rearranging gives

\begin{equation}
  Re_{isub} = \frac{\left(\frac{\rasub{C}_g}{\rbempl{C}_g} - \frac{\rbsub{C}_g}{\rbempl{C}_g}  \right) \rbempl{C}_g I_E }{\Sdot} \; .
\end{equation}
%
Recognizing that expenditures on other goods are constant across the emplacement effect gives
$\raempl{C}_g/\rbempl{C}_g = 1$ and

\begin{equation}
  Re_{isub} = \left(\frac{\rasub{C}_g}{\rbempl{C}_g} - 1  \right) \frac{\rbempl{C}_g I_E }{\Sdot} \; .
\end{equation}
%
Substituting Eq.~($\ref{eq:Sdot}$) for $\Sdot$ and rearranging gives

\begin{equation}
  Re_{isub} = \frac{\frac{\rasub{C}_g}{\rorig{C}_g} - 1}{\etaratiostacked - 1} \;
                          \etaratiostacked \;
                          \frac{\rbempl{C}_g I_E}{\rbempl{E}_s} \; .
                              \tag{\ref{eq:Re_isub_prelim}}
\end{equation}
%
Eq.~(\ref{eq:Re_isub_prelim}) is the basis for
developing expressions for $Re_{isub}$ under both the
CPE and the CES utility models.

Determining the post-substitution effect conditions
requires reference to a consumer utility model.
We first show the CPE utility model, often used in the literature.
Second, we use a constant elasticity of substitution (CES)
utility model.
The CES utility model is used for nearly all calculations and graphs
in this paper.


%..............................
\paragraph{Constant price elasticity (CPE) utility model}
\label{sec:Resub_approximate_method}
%..............................

In the literature, a constant price elasticity (CPE) utility model
has been used to determine
conditions after the substitution effect ($\wedge$)
\citep[p.~17, footnote~43]{Borenstein:2015aa}.
However, the CPE model does not produce precisely
utility-preserving preferences,
thus it cannot calculate the actual substitution effect.
We discuss the CPE utility model here for
comparison purposes only.

\citeauthor{Borenstein:2015aa}'s CPE utility model
uses the reduced form relationship between
energy service price ($p_s$) and
energy service consumption rate ($\rate{q}_s$),
namely the observed, uncompensated own price elasticity of energy service demand~($\eqspsUC$), such that

\begin{equation}
  \frac{\rasub{q}_s}{\rbsub{q}_s} = \left( \frac{\aempl{p}_s}{\bempl{p}_s} \right)^{\eqspsUC} \; .
\end{equation}
%
Note that the uncompensated own price elasticity
of energy service demand ($\eqspsUC$)
is assumed constant in the CPE utility model.
A negative value for the
uncompensated own price elasticity of energy service demand
is expected ($\eqspsUC < 0$),
such that when the energy service price decreases
($\bempl{p}_s > \aempl{p}_s $),
the rate of energy service consumption increases
($\rbsub{q}_s < \rasub{q}_s$).

Substituting Eq.~(\ref{eq:ps_pE_eta}) in the form
$\bempl{p}_s = \bempl{p}_E/\bempl{\eta}$ and
$\aempl{p}_s = \bempl{p}_E/\aempl{\eta}$
and noting that $\rbempl{q}_s = \raempl{q}_s$ gives

\begin{equation} \label{eq:approx_qshat_over_qsorig}
  \frac{\rasub{q}_s}{\rbempl{q}_s} = \left( \frac{\aempl{\eta}}{\bempl{\eta}} \right)^{-\eqspsUC} \; .
\end{equation}
%
Again, note that the compensated own price elasticity of energy service demand
is negative ($\eqspsUC < 0$), so that
as energy service efficiency increases
($\bempl{\eta} < \aempl{\eta}$),
the energy service consumption rate increases
($\rorig{q}_s = \rbsub{q}_s < \rasub{q}_s$)
as well.

Substituting Eq.~(\ref{eq:approx_qshat_over_qsorig}) into Eq.~(\ref{eq:Re_dsub_prelim})
yields the CPE model's expression for direct substitution rebound

\begin{equation} \label{eq:Re_dsub_approx}
  Re_{dsub} = \Redsubeqn \; ,
\end{equation}
%
such that, e.g.,
$\eqspsUC = -0.2$ and $\aempl{\eta}/\bempl{\eta} = 2$
yields $Re_{dsub} = 0.15$.

As long as $\eqspsUC \in (-1, 0)$,
the CPE utility model indicates that
direct substitution rebound will be below 1.
At $\eqspsUC = 1$,
the effect would be the same as the Cobb-Douglas utility model
(see footnote \ref{fn:other_utility_models}) and
the sum of substitution and income rebound effects
would be exactly 100\%.

To quantify the substitution effect on other purchases
in the CPE utility model,
expenditure on other goods is reduced
by the same dollar amount
as expenditure on the energy service increased
due to the direct substitution effect:
expenditure is held constant.
Thus,

\begin{equation}
  \Delta \rasub{C}_g = - \Delta \rasub{C}_s \; .
\end{equation}
%
The advantage of this approach is that
no cross price elasticity is needed.
The disadvantage is that it does not adhere
to the definition of the substitution effect,
which assumes that utility, not expenditure, is held constant.

Solving for $\rasub{C}_g/\rbsub{C}_g$,
substituting an expression for
the change in expenditure on the energy service ($\Delta \rasub{C}_s$),
namely

\begin{equation}
  \Delta \rasub{C}_s = \frac{p_E \left( \rasub{q}_s - \rbsub{q}_s \right)}{\aempl{\eta}} \; ,
\end{equation}
%
and
substituting Eq.~(\ref{eq:approx_qshat_over_qsorig}) gives

\begin{equation} \label{eq:C_dot_g_ratio_CPE}
  \frac{\rasub{C}_g}{\rbsub{C}_g} = 1 - \frac{p_E \rbsub{q}_s}{\bsub{\eta}  \rbsub{C}_g} \left[ \left( \etaratiostacked \right)^{-\eqspsUC} - 1 \right] \; .
\end{equation}
%
Substituting Eq.~(\ref{eq:C_dot_g_ratio_CPE})
into Eq.~(\ref{eq:Re_isub_prelim}) gives

\begin{equation}
  Re_{isub} = - \frac{\frac{p_E \rbsub{q}_s}{\aempl{\eta} \rbsub{C}_g}
                   \left[ \left( \etaratiostacked \right)^{-\eqspsUC} - 1   \right] }
                {\etaratiostacked - 1}
                \etaratiostacked
                \frac{\rorig{C}_g I_E}{\rorig{E}_s} \; .
\end{equation}
%
Rearranging and substituting Eq.~(\ref{eq:Re_dsub_approx}) gives
the expression for indirect substitution rebound under the
CPE utility model.

\begin{equation} \label{eq:Re_isub_cpe}
  Re_{isub} = - \frac{\rbsub{q}_s \rorig{C}_g p_E I_E}{\orig{\eta} \rbsub{C}_g \rorig{E}_s} Re_{dsub}
\end{equation}

Because %
\begin{enumerate*}[label={(\roman*)}]

  \item the compensated cross price elasticity
        of other goods consumption is positive ($\eqgpsC > 0$),
        i.e., we exclude Giffen goods
        \citep{Spiegel:1994aa}
        whose consumption declines
        as their price declines and

  \item the energy service efficiency ratio is greater than 1
        ($\bempl{\eta} < \amacro{\eta}$),

\end{enumerate*}
%
direct substitution rebound will be positive always
($Re_{dsub} > 0$) and
indirect substitution rebound will be negative always
($Re_{isub} < 0$),
as expected, under the CPE utility model.
Negative rebound indicates that indirect substitution effects
reduce the energy takeback rate by direct substitution effects.


%..............................
\paragraph{CES utility model}
\label{sec:Resub_exact_method}
%..............................

The CPE utility model assumes that
the compensated own price elasticity of energy service demand ($\eqspsC$)
is constant along an indifference curve, an
assumption that holds only
for infinitesimally small energy service price changes
($\Delta \aempl{p}_s \equiv \aempl{p}_s - \bempl{p}_s \approx 0$).
The CPE utility model provides
reasonable approximations for a 1--2\% change in energy efficiency.
However, in the case of an energy efficiency upgrade (EEU),
the energy service price change is neither infinitesimal nor confined
to single-digit percentages.
Rather,
$\Delta \aempl{p}_s$ is finite and may be very large in percentage terms.

To determine the new consumption bundle after the substitution effect
($\rasub{q}_s$ and $\rasub{C}_g$)
and,
ultimately, to quantify the direct and indirect substitution rebound effects
($Re_{dsub}$ and $Re_{isub}$) exactly,
we remove the restriction that energy service price elasticity
($\eqspsUC$)
must be constant along an indifference curve
(as in the CPE utility model).
Instead, we require constancy of only
the elasticity of substitution ($\sigma$) between
the consumption rate of the energy service ($\rate{q}_s$)
and the expenditure rate for other goods ($\rate{C}_g$)
across the substitution effect.
Thus, we employ a CES utility model in our framework.
Figs.~\ref{PartII-fig:CarConsGraph}
and~\ref{PartII-fig:LampConsGraph} in Part~II
(especially segments \starc{} and \chat{})
illustrates features
of the CES utility model for determining the new consumption bundle.

Two equations are helpful for this analysis.
First, the slope at any point on indifference curve
(the \iicirc{} curve in
Figs.~\ref{PartII-fig:CarConsGraph}
and~\ref{PartII-fig:LampConsGraph}
of Part~II)
is given by Eq.~(\ref{eq:slope_indifference_curve}) with
$\rate{u}/\rbempl{u} = 1$ and
the share parameter ($a$) replaced by $\fCs$,
as discussed in Appendix~\ref{sec:utility_and_elasticities}.

\begin{align} \label{eq:slope_indifference_curve_u1}
  \frac{\partial (\rate{C}_g/\rbempl{C}_g)}{\partial (\rate{q}_s/\rbempl{q}_s)} =&
        -\frac{\fCs}{1 - \fCs} \left( \frac{\rate{q}_s}{\rbempl{q}_s} \right)^{(\rho -1)} \nonumber  \\
        &\times \left[ \left( \frac{1}{1 - \fCs} \right)
                          \left( \frac{\rate{q}}{\rbempl{q}_s} \right)^\rho \right]^{(1 - \rho)/\rho} \; .
\end{align}
%
Second, the equation of the pre-substitution-effect expenditure line
(\starstar{} in
Figs.~\ref{PartII-fig:CarConsGraph}
and~\ref{PartII-fig:LampConsGraph}
of Part~II) is

\begin{equation} \label{eq:starstar_line}
  \frac{\rate{C}_g}{\rbempl{C}_g} =
      -\frac{\aempl{p}_s \rbempl{q}_s}
            {\rbempl{C}_g}
        \left(  \frac{\rate{q}_s}{\rbempl{q}_s} \right)
      + \frac{1}{\rbempl{C}_g}
        (\rate{M} - \bempl{\tau}_\alpha \rbempl{C}_{cap} - \rbempl{C}_{\omd} - \rate{G}) \; .
\end{equation}

To find the rate of energy service consumption after the substitution effect
($\rasub{q}_s$), we set the slope of the
expenditure line (Eq.~(\ref{eq:starstar_line})
and line \starstar{} in
Figs.~\ref{PartII-fig:CarConsGraph}
and~\ref{PartII-fig:LampConsGraph} of Part~II)
equal to the slope of the
indifference curve
(\iicirc{} in
Figs.~\ref{PartII-fig:CarConsGraph}
and~\ref{PartII-fig:LampConsGraph} of Part~II)
at the original utility rate of $\rate{u}/\rbempl{u} = 1$ (Eq.~(\ref{eq:slope_indifference_curve_u1})).

\begin{equation}
  -\frac{\aempl{p}_s \rbempl{q}_s}
        {\rbempl{C}_g} =
    -\frac{\fCs}{1 - \fCs} \left( \frac{\rate{q}_s}{\rbempl{q}_s} \right)^{(\rho -1)} % \nonumber  \\
        \left[ \left( \frac{1}{1 - \fCs} \right)
                - \left( \frac{\fCs}{1 - \fCs} \right)
                          \left( \frac{\rate{q}}{\rbempl{q}_s} \right)^\rho \right]^{(1 - \rho)/\rho}
\end{equation}

Solving for $\rate{q}_s/\rbempl{q}_s$ gives $\rasub{q}_s/\rbempl{q}_s$ as

\begin{equation}
  \frac{\rasub{q}_s}{\rbempl{q}_s} = \qssolution{} \; . \tag{\ref{eq:q_s_solution}}
\end{equation}

Eq.~(\ref{eq:q_s_solution}) can be substituted directly
into Eq.~(\ref{eq:Re_dsub_prelim})
to obtain an estimate for direct substitution rebound ($Re_{dsub}$)
via the CES utility model.

\begin{equation}
  Re_{dsub} = \RedsubCES{} \tag{\ref{eq:Re_dsub_CES}}
\end{equation}

The rate of other goods consumption after the substitution effect ($\rasub{C}_g$)
can be found by substituting Eq.~(\ref{eq:q_s_solution}) and
$\rate{u}/\rbempl{u} = 1$
into the functional form of the CES utility model (Eq.~(\ref{eq:utility_Cg_form}))
to obtain

\begin{equation}
  \frac{\rasub{C}_g}{\rbempl{C}_g} = \left( \left( \frac{1}{1-\fCs} \right)
                                     - \left( \frac{\fCs}{1-\fCs} \right)
              \left\{ \fCs + (1-\fCs)
                  \left[ \left( \frac{1-\fCs}{\fCs} \right) \frac{\aempl{p}_s \rbempl{q}_s}{\rbempl{C}_g}   \right]
                      ^{\frac{\rho}{1-\rho}} \right\} ^{-1} \right) ^{1/\rho} \; .
\end{equation}

Simplifying gives

\begin{equation}
  \frac{\rasub{C}_g}{\rbempl{C}_g} = \Cgsolution{} \; . \tag{\ref{eq:C_g_solution}}
\end{equation}

Eq.~(\ref{eq:C_g_solution}) can be substituted into Eq.~(\ref{eq:Re_isub_prelim})
to obtain an expression for indirect substitution rebound ($Re_{isub}$)
via the CES utility model.

\begin{equation}
  Re_{isub} = \ReisubCES{} \tag{\ref{eq:Re_isub_CES}}
\end{equation}


%------------------------------
\subsubsection{\Inceffect{}}
\label{sec:Re_inc}
%------------------------------

Rebound from the income effect rebound quantifies the rate of additional energy demand
that arises because the user of the energy conversion device spends net
savings from the EEU.
The income rate of the device user is $\rate{M}$,
which remains unchanged across the rebound effects.
Freed cash from the EEU is given by Eq.~(\ref{eq:G_dot})
as $\rate{G} = p_E \Sdot$.
In combination, the emplacement effect and
the substitution effect leave the device user with
\emph{net} savings ($\rasub{N}$) from the EEU,
as shown in Eq.~(\ref{eq:N_dot_after_sub}).
Derivations of expressions for
freed cash from the emplacement effect ($\rate{G}$) and
net savings after the substitution effect ($\rasub{N}$)
are presented in Tables~\ref{tab:empleffect} and~\ref{tab:subeffect}.

In this framework, all net savings ($\rasub{N}$) are spent on either
%
\begin{enumerate*}[label={(\roman*)}]

  \item additional energy service
        ($\rbinc{q}_s < \rainc{q}_s$) or

  \item additional other goods
        ($\rbinc{q}_g < \rainc{q}_g$).

\end{enumerate*}
%
The income elasticity of energy service demand and
the income elasticity of other goods demand
($\eqsM$ and $\eqgM$, respectively)
quantify the income preferences of the device user according to the following expressions:

\begin{equation}
  \incprefseqn \tag{\ref{eq:qsrat_eqsM}}
\end{equation}
%
and

\begin{equation}
  \incprefoeqn \; , \tag{\ref{eq:qgrat_eqgM}}
\end{equation}
%
where effective income ($\Mdothatprime$) is

\begin{equation}
  \effinceqn{} \; . \tag{\ref{eq:effective_income}}
\end{equation}
%
Homotheticity means that $\eqsM = 1$ and $\eqgM = 1$.

The budget constraint across the income effect (Eq.~(\ref{eq:inc_budget_constraint}))
ensures that all net savings available after the substitution effect ($\rasub{N}$)
is re-spent across the income effect,
such that $\rainc{N} = 0$.
Appendix~\ref{sec:proof_income_elasticities} proves that
the income preference equations (Eqs.~(\ref{eq:qsrat_eqsM}) and~(\ref{eq:qgrat_eqgM}))
satsify the budget constraint (Eq.~(\ref{eq:inc_budget_constraint})).

The purpose of this section is derivation of expressions for
%
\begin{enumerate*}[label={(\roman*)}]

  \item direct income rebound~($Re_{dinc}$)
        arising from increased consumption of the energy service
        ($\rbinc{q}_s < \rainc{q}_s$) and

  \item indirect income rebound~($Re_{iinc}$)
        arising from increased consumption of other goods
        ($\rbinc{q}_g < \rainc{q}_g$).

\end{enumerate*}

But first, we derive an expression
for device energy consumption rate prior to the income effect
($\rbinc{E}_s$).
This expression will be helpful later.

%..............................
\paragraph{Derivation of expression for $\rbinc{E}_s$}
\label{sec:E_dot_s_hat_expression}
%..............................

An expression for $\rbinc{E}_s$ that will be helpful later
begins with

\begin{equation}
  \rbinc{E}_s = \left( \frac{\rasub{E}_s}{\rbsub{E}_s} \right)
                \left( \frac{\raempl{E}_s}{\rbempl{E}_s} \right)
                \rbempl{E}_s \; .
\end{equation}
%
Substituting Eq.~(\ref{eq:typ_qs_eta_Edot}) and noting efficiency ($\eta$)
equalities from Table~\ref{tab:analysis_assumptions} gives

\begin{equation}
  \rbinc{E}_s = \left( \frac{\rasub{q}_s / \cancel{\asub{\eta}}}{\rbsub{q}_s / \cancel{\aempl{\eta}}} \right)
                \left( \frac{\raempl{q}_s / \aempl{\eta}}{\rbempl{q}_s / \bempl{\eta}} \right)
                \rbempl{E}_s \; .
\end{equation}
%
Canceling terms yields

\begin{equation}
  \rbinc{E}_s = \left( \frac{\rasub{q}_s}{\rbsub{q}_s} \right)
                \left( \frac{\cancel{\raempl{q}_s}}{\cancel{\rbempl{q}_s}} \right)
                \left( \frac{\bempl{\eta}}{\aempl{\eta}}  \right)
                \rbempl{E}_s \; .
\end{equation}
%
Noting energy service consumption rate equalities from Table~\ref{tab:analysis_assumptions}
($\raempl{q}_s = \rbempl{q}_s$) gives

\begin{equation} \label{eq:E_dot_s_hat}
  \rbinc{E}_s = \frac{\rasub{q}_s}{\rbsub{q}_s}
                \frac{\bempl{\eta}}{\aempl{\eta}}
                \rbempl{E}_s \; .
\end{equation}

The next step is to develop an expression for $Re_{dinc}$
using the income preference for energy service consumption.


%..............................
\paragraph{Derivation of expression for $Re_{dinc}$}
\label{sec:Re_dinc}
%..............................

As shown in Table~\ref{tab:inceffect}, direct income rebound is defined as

\begin{equation}
  Re_{dinc} \equiv \frac{\Delta \rainc{E}_s}{\Sdot} \; . \tag{\ref{eq:Re_dinc_def}}
\end{equation}
%
Expanding the difference and rearranging gives

\begin{equation}
  Re_{dinc} = \frac{\rainc{E}_s - \rbinc{E}_s}{\Sdot} \; ,
\end{equation}
%
and

\begin{equation}
  Re_{dinc} = \frac{\left( \frac{\rainc{E}_s}{\rbinc{E}_s} - 1  \right) \rbinc{E}_s}{\Sdot} \; .
\end{equation}
%
Substituting Eq.~(\ref{eq:typ_qs_eta_Edot}) as
$\rainc{E}_s = \rainc{q}_s / \ainc{\eta}$ and
$\rbinc{E}_s = \rbinc{q}_s / \binc{\eta}$ gives

\begin{equation}
  Re_{dinc} = \frac{\left( \frac{\rainc{q}_s / \cancel{\ainc{\eta}}}{\rbinc{q}_s / \cancel{\binc{\eta}}} - 1  \right) \rbinc{E}_s}
              {\Sdot} \; .
\end{equation}
%
Eliminating terms and substituting Eq.~(\ref{eq:Sdot}) for $\rate{S}_{dev}$ and
Eq.~(\ref{eq:qsrat_eqsM}) for $\rainc{q}_s / \rbinc{q}_s$ gives

\begin{equation}
  Re_{dinc} = \frac{\left[ \left( 1 + \frac{\rbinc{N}}{\Mdothatprime} \right) ^{\eqsM} - 1  \right] \rbinc{E}_s}
              {\Sdoteqn} \; .
\end{equation}
%
Substituting Eq.~(\ref{eq:E_dot_s_hat}) for $\rbinc{E}_s$ gives

\begin{equation}
  Re_{dinc} = \frac{\left[ \left( 1 + \frac{\rbinc{N}}{\Mdothatprime} \right) ^{\eqsM} - 1  \right]
                  \frac{\rasub{q}_s}{\rbsub{q}_s}
                \cancel{\frac{\bempl{\eta}}{\aempl{\eta}}}
                \cancel{\rbempl{E}_s}}
              {\left( \etaratiostacked - 1 \right)\!
                  \cancel{\frac{\bempl{\eta}}{\aempl{\eta}}} \cancel{\rbempl{E}_s}} \; .
\end{equation}
%
Eliminating terms, recognizing that
$\rorig{q}_s = \raempl{q}_s$, and substituting Eq.~(\ref{eq:q_s_solution}),
which assumes the CES utility model,
gives

% \begin{equation}
%   Re_{dinc} = \Redinceqnexact{} \; . \tag{\ref{eq:Re_dinc}}
% \end{equation}
\begin{align}
  Re_{dinc} = \Redinceqnexact{} \; . \tag{\ref{eq:Re_dinc}}
\end{align}
%
If there is no net savings ($\rasub{N} = 0$),
direct income effect rebound is zero ($Re_{dinc} = 0$), as expected.

The next step is to develop an expression for $Re_{iinc}$
using the income preference for other goods consumption.


%..............................
\paragraph{Derivation of expression for $Re_{iinc}$}
\label{sec:Re_iinc}
%..............................

As shown in Table~\ref{tab:inceffect}, indirect income rebound is defined as

\begin{equation}
  Re_{iinc} \equiv \frac{\Delta \rainc{C}_g I_E}{\Sdot} \; . \tag{\ref{eq:Re_iinc_def}}
\end{equation}
%
Expanding the difference and rearranging gives

\begin{equation}
  Re_{iinc} = \frac{(\rainc{C}_g - \rbinc{C}_g) I_E}{\Sdot} \; ,
\end{equation}
%
and

\begin{equation}
  Re_{iinc} = \frac{\left( \frac{\rainc{C}_g}{\rbinc{C}_g} - 1  \right) \rbinc{C}_g I_E}{\Sdot} \; .
\end{equation}
%
Substituting $\rainc{C}_g = p_g \rainc{q}_g$ and $\rbinc{C}_g = p_g \rbinc{q}_g$ and
cancelling terms gives

\begin{equation}
  Re_{iinc} = \frac{\left( \frac{\rainc{q}_g}{\rbinc{q}_g} - 1  \right) \rbinc{C}_g I_E}{\Sdot} \; .
\end{equation}
%
Substituting the income preference equation for other goods consumption (Eq.~(\ref{eq:qgrat_eqgM})
for $\rainc{q}_g / \rbinc{q}_g$
and Eq.~(\ref{eq:Sdot}) for $\rate{S}_{dev}$ yields

\begin{equation}
  Re_{iinc} = \frac{\left[ \left( 1 + \frac{\rbinc{N}}{\Mdothatprime} \right)^{\eqgM} - 1  \right]
              \rbinc{C}_g I_E}{\Sdoteqn} \; .
\end{equation}
%
Sutstituting $(\rasub{C}_g/\rorig{C}_g) \rorig{C}_g$ for $\rasub{C}_g$,
recognizing that $\rbsub{C}_g = \rbempl{C}_g$, and simplifying gives

\begin{equation}
  Re_{iinc} = \frac{\left( 1 + \frac{\rbinc{N}}{\Mdothatprime} \right)^{\eqgM} - 1}{\etaratiostacked - 1}
              \left( \etaratiostacked \right)
              \frac{\rbempl{C}_g I_E}{\rbempl{E}_s}
              \left( \frac{\rasub{C}_g}{\rorig{C}_g} \right) \; .
\end{equation}

Substituting Eq.~(\ref{eq:C_g_solution})
for $\rasub{C}_g / \rorig{C}_g$,
thereby assuming the CES utility model,
gives the final form
of the indirect income rebound expression:

\begin{align}
  Re_{iinc} = \Reiinceqnexact{} \; . \tag{\ref{eq:Re_iinc}}
\end{align}
%
If there is no net savings ($\rasub{N} = 0$),
indirect income effect rebound is zero ($Re_{iinc} = 0$), as expected.


%..............................
\paragraph{Income effect rebound under the CPE utility model}
\label{sec:income_effect_CPE}
%..............................

Following \citet{Borenstein:2015aa},
under CPE utility model
all freed cash is spent on other goods,
as in the fully satiated case discussed
in Section~\ref{sec:inc_effect_main_paper}.
However, because the substitution effect
under the CPE utility model
does not alter freed cash,
the income effect
involves the product of
the energy intensity of the economy ($I_E$)
and $\rbsub{N}$
(instead of $\rasub{N}$).


%------------------------------
\subsubsection{\Macroeffect{}}
\label{sec:Re_macro}
%------------------------------

Macro rebound~($Re_{\macro}$) is given by Eq.~(\ref{eq:Re_macro_def}).
Substituting Eq.~(\ref{eq:N_dot_star_empl}) for net savings~($\raempl{N}$) gives

\begin{equation}
  % Re_{\macro} = \frac{k \rate{G} I_E}{\Sdot}
  %                               - \frac{k \Delta \raempl{C}_{cap} I_E}{\Sdot}
  %                               - \frac{k \Delta \raempl{C}_{\md} I_E}{\Sdot}
  %                               - \frac{k p_E I_E \Delta \rasub{E}_s}{\Sdot}
  %                               - \frac{k \Delta \rasub{C}_g I_E}{\Sdot} \; .
  Re_{\macro} = \frac{k (p_E \Sdot - \Delta \aempl{(R_\alpha \rate{C}_{cap})} - \Delta \raempl{C}_{\omd}) I_E}{\Sdot} \; .
\end{equation}
%
Separating terms gives

\begin{equation}
  Re_{\macro} = \frac{k p_E \cancel{\Sdot} I_E}{\cancel{\Sdot}}
                                - \frac{k \Delta \aempl{(R_\alpha  \rate{C}_{cap})} I_E}{\Sdot}
                                - \frac{k \Delta \raempl{C}_{\omd} I_E}{\Sdot} \; .
\end{equation}
%
Canceling terms, substituting Eq.~(\ref{eq:Re_md_def}) to obtain $Re_{\omd}$, and
defining $Re_{cap}$ as

\begin{equation} \label{eq:Re_cap}
  Re_{cap} \equiv \Recapeqn{}
\end{equation}
%
gives

\begin{equation}
  Re_{\macro} = \Remacroeqn{} \; . \tag{\ref{eq:Re_macro}}
\end{equation}


%------------------------------
\subsubsection{Rebound sum}
\label{sec:total_rebound}
%------------------------------

The sum of the four rebound effects is

\begin{equation}
  Re_{tot} = Re_{empl} + Re_{sub} + Re_{inc} + Re_{\macro} \; .
\end{equation}
%
Substituting Eqs.~(\ref{eq:Re_empl_def}), (\ref{eq:Re_sub_def}), and~(\ref{eq:Re_inc_def}) gives

\begin{align}
  Re_{tot} = \; &Re_{emb} + Re_{\omd}      & \mathrm{\empleffect} \nonumber \\
                &+ Re_{dsub} + Re_{isub}   & \mathrm{\subeffect}  \nonumber \\
                &+ Re_{dinc} + Re_{iinc}   & \mathrm{\inceffect}  \nonumber \\
                &+ Re_{\macro}      & \mathrm{\macroeffect}
\end{align}
%
Macro effect rebound~($Re_{\macro}$, Eq.~(\ref{eq:Re_macro}))
can be expressed in terms of other rebound effects.
Substituting Eq.~(\ref{eq:Re_macro}) gives

\begin{align}
  Re_{tot} = \; &Re_{emb} + Re_{\omd}      & \mathrm{\empleffect}       \nonumber \\
                &+ Re_{dsub} + Re_{isub}   & \mathrm{\subeffect}        \nonumber \\
                &+ Re_{dinc} + Re_{iinc}   & \mathrm{\inceffect}        \nonumber \\
                &+ k p_E I_E - k Re_{cap} - k Re_{\omd} \; .  & \mathrm{\macroeffect}
\end{align}
%
Rearranging distributes macro effect terms
to emplacement and substitution effect terms.
This last rearrangement gives the final expression for total rebound.

\begin{align}
  Re_{tot} = \; \Retoteqn{} \tag{\ref{eq:Re_tot}}
\end{align}

Eq.~(\ref{eq:Re_tot}) shows that determining seven rebound values,

\begin{itemize}

  \item $Re_{emb}$ (Eq.~(\ref{eq:Re_emb})),

  \item $Re_{cap}$ (Eq.~(\ref{eq:Re_cap})),

  \item $Re_{\omd}$ (Eq.~(\ref{eq:Re_md})),

  \item $Re_{dsub}$ (Eq.~(\ref{eq:Re_dsub_CES})),

  \item $Re_{isub}$ (Eq.~(\ref{eq:Re_isub_CES})),

  \item $Re_{dinc}$ (Eq.~(\ref{eq:Re_dinc})), and

  \item $Re_{iinc}$ (Eq.~(\ref{eq:Re_iinc})),

\end{itemize}
%
is sufficient to calculate total rebound,
provided that
the macro factor~($k$),
the price of energy~($p_E$), and
the energy intensity of the economy~($I_E$)
are known.
