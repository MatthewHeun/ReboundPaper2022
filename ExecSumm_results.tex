\documentclass[12pt]{article}    % For submission

\usepackage{authblk}            % For a nice author block
\usepackage[inline]{enumitem}   % For inline enumeration
\usepackage[letterpaper, left=1in, right=1in, top=1in, bottom=1in, footskip=.25in]{geometry} % For better margins.


\title{Executive summary for \\
  Energy, expenditure, and consumption \\
  aspects of rebound,\\
          Part II: Applications}
\author[1,*]{Matthew Kuperus Heun}
\author[2]{Gregor Semieniuk}
\author[3]{Paul E.\ Brockway}
\affil[1]{Engineering Department, Calvin University, 3201 Burton St. SE, Grand Rapids, MI, 49546}
\affil[2]{Political Economy Research Institute and 
  Department of Economics,
  UMass Amherst}
\affil[3]{Sustainability Research Institute, 
  School of Earth and Environment,
  University of Leeds}
\affil[*]{\normalfont{Corresponding author: \texttt{mkh2@calvin.edu}}}
\renewcommand\Affilfont{\itshape\small}

\date{} % Kill the date

\begin{document}

\maketitle


%++++++++++++++++++++++++++++++
\subsection*{Motivations underlying the research}
\label{sec:motivations}
%++++++++++++++++++++++++++++++

Widespread implementation of energy efficiency
is a key greenhouse gas emissions mitigation measure, 
but rebound can ``take back'' energy savings.
However, the absence of solid analytical foundations hinders
empirical determination of the size of rebound.
A new clarity is needed, one that is built upon solid analytical frameworks
involving both economics and energy analysis.
In particular, a way to visualize energy, expenditure, and consumption 
aspects of rebound will be helpful for communication of energy analysis
findings to broader audiences. 
Software tools for rebound analysis will be a welcome addition
to energy analysis.
And demonstration of the rigorous analytical framework 
developed in Part~I via detailed numerical examples
will be a benefit to energy analysts.  


%++++++++++++++++++++++++++++++
\subsection*{Short account of the research performed}
\label{sec:account}
%++++++++++++++++++++++++++++++

In this paper (Part~II of a two-part paper),
we develop and demonstrate
energy, expenditure, and consumption rebound planes, 
a novel, mutually consistent, and numerically precise
way to visualize and illustrate those three aspects of rebound.
Further, we perform the first calibration of the macro factor
for macroeconomic rebound, finding $k \approx 3$. 
We apply
the rigorous analytical framework developed in Part~I 
and the visualization advances described in Part~II 
to calculate and show total rebound for two case studies: 
energy efficiency upgrades of
a car (47\%) and
an electric lamp (67\%).
Comparison of our rebound values to previous values is provided.
Finally, we provide information about new open source software tools for calculating
rebound magnitudes using the framework.


%++++++++++++++++++++++++++++++
\subsection*{Main conclusions}
\label{sec:conclusions}
%++++++++++++++++++++++++++++++

From the application of the rebound framework in Part~II, 
we draw two important conclusions.
First, the car and lamp examples show that
        the framework enables
        quantification of rebound magnitudes at microeconomic and macroeconomic levels, including 
        energy, expenditure, and consumption aspects of 
        direct and indirect rebound 
        for emplacement, substitution, income, and macro effects.
Second, the examples show that magnitudes of all rebound effects
        vary with the type of EEU performed.


%++++++++++++++++++++++++++++++
\subsection*{Potential benefits, applications and policy implications}
\label{sec:benefits}
%++++++++++++++++++++++++++++++

There are two implications of this work for energy analysis and policy.
First, 
the magnitude of every rebound effect is different between the car and lamp examples.
The implication is that every EEU needs to be analyzed separately. 
Values for rebound effects  
for one EEU should never be assumed to apply to a different EEU.
Second, 
one cannot know \emph{a-priori} which rebound effects
will be large and which will be small
for a given EEU.
Furthermore, some rebound effects are dependent upon economic parameters,
such as energy intensity of the economy.
Thus, it is important to calculate the magnitude of all rebound effects 
for each EEU in each economy.


\end{document}